
\chapter{\label{chap:determinant}Diagonal Entries of Hermite Normal Form
and Determinant}

In this Chapter, we consider the problem of computing a determinant
of a nonsingular input matrix $\mathbf{F}\in\mathbb{K}\left[x\right]^{n\times n}$
with column degrees bounded by a shift $\vec{s}$. The computation
is done by using the column basis and kernel basis computation to
compute the diagonal entries of the Hermite form of $\mathbf{F}$,
and then multiply these diagonal entries.

Consider unimodularly transforming $\mathbf{F}$ to 
\begin{equation}
\mathbf{F}\mathbf{U}=\mathbf{G}=\begin{bmatrix}\mathbf{G}_{1} & 0\\
* & \mathbf{G}_{2}
\end{bmatrix},\label{eq:step1HermiteDiagonal}
\end{equation}


 After this unimodular transformation, which eliminated the top
right block of $\mathbf{G}$ , the matrix is now closer to the Hermite
normal form of $\mathbf{F}$. This procedure can then be applied recursively
to $\mathbf{G}_{1}$ and $\mathbf{G}_{2}$, until the matrices reaching
dimension 1, which then gives the diagonal entries of the Hermite
normal form of $\mathbf{F}$.

Although this procedure correctly computes the diagonal entries of
the Hermite normal form of $\mathbf{F}$, a major problem is that
the degree of the unimodular $\mathbf{U}$ can be too large for $\mathbf{U}$
to be efficiently computed. However, with the tools we have developed
in the earlier chapters, we can efficiently compute $\mathbf{G}_{1}$
and $\mathbf{G}_{2}$ without computing $\mathbf{U}$.

If we separate $\mathbf{F}$ to $\mathbf{F}=\begin{bmatrix}\mathbf{F}_{U}\\
\mathbf{F}_{D}
\end{bmatrix}$, each has full-rank as $\mathbf{F}$ is assumed to be nonsingular,
and also separate $\mathbf{U}$ to $\mathbf{U}=\begin{bmatrix}\mathbf{U}_{L} & \mathbf{U}_{R}\end{bmatrix}$,
where the column dimension of $\mathbf{U}_{L}$ matches the row dimension
of $\mathbf{F}_{U}$, then 
\[
\mathbf{F}\mathbf{U}=\begin{bmatrix}\mathbf{F}_{U}\\
\mathbf{F}_{D}
\end{bmatrix}\begin{bmatrix}\mathbf{U}_{L} & \mathbf{U}_{R}\end{bmatrix}=\mathbf{G}=\begin{bmatrix}\mathbf{G}_{1} & 0\\
* & \mathbf{G}_{2}
\end{bmatrix}.
\]
 Notice that the matrix $\mathbf{G}_{1}$ is nonsingular and is therefore
just a column basis of $\mathbf{F}_{U}$, and can be efficiently computed
using \prettyref{alg:colBasis}. To compute $\mathbf{G}_{2}=\mathbf{F}_{D}\mathbf{U}_{R}$,
notice that the matrix $\mathbf{U}_{R}$ is a right kernel basis\textbf{
}of $\mathbf{F}$, which makes the top right block of $\mathbf{G}$
zero. As we have seen from \prettyref{lem:unimodular_kernel_columnBasis},
the kernel basis $\mathbf{U}_{R}$ can be replaced by any other kernel
basis of $\mathbf{F}$ to give another unimodular matrix that also
works. 
\begin{lem}
\label{lem:oneStepHermiteDiagonal}Given a polynomial matrix $\mathbf{F}=\begin{bmatrix}\mathbf{F}_{U}\\
\mathbf{F}_{D}
\end{bmatrix}$. If $\mathbf{G}_{1}$ is a column basis of $\mathbf{F}_{U}$ and
$\mathbf{N}$ is a kernel basis of $\mathbf{F}_{U}$, then there is
a unimodular matrix $\mathbf{U}=\left[*,\mathbf{N}\right]$ such that
\[
\mathbf{F}\mathbf{U}=\begin{bmatrix}\mathbf{G}_{1}\\
* & \mathbf{G}_{2}
\end{bmatrix},
\]
 where $\mathbf{G}_{2}=\mathbf{F}_{D}\mathbf{N}$.  If $\mathbf{F}$
is square nonsingular, then $\mathbf{G}_{1}$ and $\mathbf{G}_{2}$
are also square nonsingular.
\end{lem}
Note that we do not compute the blocks represented by the symbol $*$,
which may have very large degrees and cannot be computed efficiently.

\prettyref{lem:oneStepHermiteDiagonal} allows us to compute $\mathbf{G}_{1}$
and $\mathbf{G}_{2}$ independently without computing the unimodular
matrix. $\mathbf{G}_{1}$ can be computed using the method from \prettyref{chap:Matrix-GCD},
while the kernel basis computation from \prettyref{chap:NullspaceBasis}
can be used to compute a kernel basis $\mathbf{N}_{1}$ of $\mathbf{F}_{U}$,
which can then be used to compute $\mathbf{G}_{2}=\mathbf{F}_{D}\mathbf{N}_{1}$.

After $\mathbf{G}_{1}$ and $\mathbf{G}_{2}$ are computed, we can
repeat the same process on each of these two matrices, which now have
lower dimensions, until the dimension becomes one. This procedure
of computing the diagonal entries gives \prettyref{alg:hermiteDiagonal}

\begin{algorithm}[t]
\caption{$\hermiteDiagonal(\mathbf{F})$}
\label{alg:hermiteDiagonal}

\begin{algorithmic}
[1]\REQUIRE{$\mathbf{F}\in\mathbb{K}\left[x\right]^{n\times n}$ is nonsingular.
}

\ENSURE{$\mathbf{d}\in\mathbb{K}\left[x\right]^{n}$ a list of diagonal entries
of the Hermite normal form of $\mathbf{F}$.}

\STATE{$\begin{bmatrix}\mathbf{F}_{U}\\
\mathbf{F}_{D}
\end{bmatrix}:=\mathbf{F}$, with $\mathbf{F}_{u}$ consists of the top $\left\lceil n/2\right\rceil $
rows of $\mathbf{F}$;}



\STATE{\textbf{if }$n=1$ \textbf{then} \textbf{return} $\mathbf{F}$; \textbf{endif};}

\STATE{$\mathbf{G}_{1}:=\colBasis(\mathbf{F}_{u})$;}



\STATE{$\mathbf{N}:=\mnb(\mathbf{F}_{u},\cdeg\mathbf{F})$;}

\STATE{$\mathbf{G}_{2}:=\mathbf{F}_{d}\mathbf{N}$;}

\STATE{$\mathbf{d}_{1}:=\hermiteDiagonal(\mathbf{G}_{1});$$\mathbf{d}_{2}:=\hermiteDiagonal(\mathbf{G}_{2});$}

\STATE{\textbf{return} $\left[\mathbf{d}_{1},\mathbf{d}_{2}\right]$;}
\end{algorithmic}
\end{algorithm}




\section{Computational Cost}

Let us look at the computational cost of \prettyref{alg:hermiteDiagonal}. 
\begin{thm}
\prettyref{alg:hermiteDiagonal} costs $O^{\sim}\left(n^{\omega-1}\xi\right)$
field operations to compute the diagonal entries for the Hermite normal
form of a nonsingular matrix $\mathbf{F}\in\mathbb{K}\left[x\right]^{n\times n}$,
where $\xi$ is the sum of the column degrees \end{thm}
\begin{proof}
The three main operations are computing a column basis of $\mathbf{F}_{U}$,
computing a kernel basis $\mathbf{N}$ of $\mathbf{F}_{U}$, and the
matrix multiplication $\mathbf{F}_{D}\mathbf{N}$.

For the column basis computation, by \prettyref{thm:columnBasisCost1}
we know that a column basis $\mathbf{G}_{1}$ of $\mathbf{F}_{U}$
can be computed with a cost of $O^{\sim}\left(n^{\omega-1}\xi\right).$
By \prettyref{lem:colBasisDegreeBoundByInputDegrees} the column degrees
of the computed column basis $\mathbf{G}_{1}$ are also bounded by
the original column degrees $\vec{s}$.

For the kernel basis computation, it also costs $O^{\sim}\left(n^{\omega-1}\xi\right)$
to compute a $\vec{s}$-minimal kernel basis $\mathbf{N}$ of $\mathbf{F}_{U}$
from \prettyref{thm:costLowColDimension}. The sum of the $\vec{s}$-column
degrees of the output kernel basis $\mathbf{N}$ is bounded by $\xi$
by \prettyref{thm:boundOfSumOfShiftedDegreesOfKernelBasis}.

Finally for the matrix multiplication $\mathbf{F}_{D}\mathbf{N}$,
since the sum of the column degrees of $\mathbf{F}_{D}$ and the sum
of the $\vec{s}$-column degrees of $\mathbf{N}$ are both bounded
by $\xi$, \prettyref{thm:multiplyUnbalancedMatrices} applies and
the multiplication can be done with a cost of $O^{\sim}\left(n^{\omega-1}\xi\right)$.

Now if we let the cost of \prettyref{alg:hermiteDiagonal} be $g(n,\xi)$
for a input matrix of dimension $n$ with $\xi$ bounding the sum
of its column degrees, then we have the recurrence relation
\[
g(n,\xi)\in O^{\sim}(n^{\omega-1}\xi)+g(\left\lceil n/2\right\rceil ,\xi)+g(\left\lfloor n/2\right\rfloor ,\xi),
\]
 which is the same as in \prettyref{thm:inverseCost} for computing
the matrix inverse. Therefore, we also get $g(n,\xi)=O^{\sim}(n^{\omega-1}\xi)$
as in the inverse computation.\end{proof}
\begin{cor}
The determinant of a nonsingular matrix $\mathbf{F}\in\mathbb{K}\left[x\right]^{n\times n}$
can be computed with a cost of $O^{\sim}(n^{\omega-1}\xi)$ field
operations, where $\xi$ is the minimum of the sum of the column degrees
and the sum of the row degrees of the input matrix.\end{cor}
\begin{proof}
Just use \prettyref{alg:hermiteDiagonal} to compute the diagonal
entries of the Hermite normal form of either $\mathbf{F}$ or $\mathbf{F}^{T}$,
and then multiply the diagonal entries.\end{proof}

