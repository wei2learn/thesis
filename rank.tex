
\chapter{\label{chap:rank}Rank Profile and Rank Sensitive Computation of
Kernel Basis}

In this chapter, we consider the problems of computing the row rank
profile of an input matrix $\mathbf{F}\in\mathbb{K}\left[x\right]^{m\times n}$,
which also immediately gives us the rank. If $n\ge m$, the rank can
already be computed by either kernel basis computation or column basis
computation from the earlier chapters. The column basis computation
(\prettyref{alg:colBasis}) can compute the rank with a cost of $O^{\sim}(m^{\omega-1}\xi)$
field operations, where $\xi$ is the sum of the column degrees of
$\mathbf{F}$. However, we would like to refine this cost to $O^{\sim}\left(mr^{\omega-2}\xi\right)$,
where $r$ is the rank of $\mathbf{F}$. We also would like to compute
a row rank profile with the same cost.

We use the following approach to achieve the desired cost. We first
modify our kernel basis algorithm, \prettyref{alg:minimalNullspaceBasisWithRankProfile},
slightly to allow the rank profile to be computed along with a kernel
basis. Then we do a series of computations with increasing number
of rows from $\mathbf{F}$. For each set of rows we do successive
column basis computation (or order basis computation) as in \prettyref{sec:successiveColBasisComputation}
to reduce the column dimension of the problem, so the modified \prettyref{alg:minimalNullspaceBasisWithRankProfile}
can work efficiently to compute the rank profile of this set of rows.


\section{Rank Profile from Kernel Basis Computation}

Recall that the row rank profile of $\mathbf{F}$ is the lexicographically
smallest list of row indices $\left[i_{1},i_{2},\dots i_{r}\right]$
such that these rows of $\mathbf{F}$ are linearly independent, where
$r$ is the rank of $\mathbf{F}$. Let us see how the row rank profile
can be computed by our kernel basis algorithm.  The following lemma
provides a key to the rank profile computation using \prettyref{alg:minimalNullspaceBasis}.
\begin{lem}
\label{lem:rankProfileIndicator}At any base case of running \prettyref{alg:minimalNullspaceBasis}
on the input matrix\textbf{ $\mathbf{F}$}, we work with an input
matrix $\mathbf{g}$ consisting of a single row. Let $\mathbf{f}$
be the original row in $\mathbf{F}$ corresponding to $\mathbf{g}$
and $\mathbf{F}'$ be the submatrix of $\mathbf{F}$ consists of the
rows above $\mathbf{f}$. Then $\mathbf{g}=0$ if and only if $\mathbf{f}$
is linearly dependent with the rows of $\mathbf{F}'$.\end{lem}
\begin{proof}
When the algorithm has reached the base case involving the single
row matrix $\mathbf{g}$, it has finished processing $\mathbf{F}'$
and has produced a number of order bases and kernel bases from the
earlier subproblems, where the kernel bases computed only involved
all the rows of $\mathbf{F}'$. The matrix $\mathbf{g}$ is the residual
from multiplying $\mathbf{f}$ with these order bases and kernel bases.
Note that such multiplications do not change the linear dependency
of $\mathbf{g}$ with the rows of $\mathbf{F}'$. But if $\mathbf{f}$
is linearly dependent with the rows of $\mathbf{F}'$, the residual
$\mathbf{g}$ becomes 0 after multiplying with kernel bases of the
rows of $\mathbf{F}'$.
\end{proof}
\prettyref{lem:rankProfileIndicator} now allows us to provide a small
modification to \prettyref{alg:minimalNullspaceBasis} to produce
the rank profile of $\mathbf{F}$. The modified algorithm is given
in \prettyref{alg:minimalNullspaceBasisWithRankProfile}. Note that
the rank profile in our algorithm is represented using a list of $n$
indicators that indicate the first $r$ linearly independent rows
of $\mathbf{F}$. At this point, the rank profile of $\mathbf{F}$
still costs the same to compute as a kernel basis of $\mathbf{F}$.
In the following, we see how column basis computation can be used
to improve this.

\begin{algorithm}[t]
\caption{$\mnbrp(\mathbf{F},\vec{s})$}
\label{alg:minimalNullspaceBasisWithRankProfile}

\begin{algorithmic}[1]
\REQUIRE{$\mathbf{F}\in\mathbb{K}\left[x\right]^{m\times n}$, $\vec{s}=[s_{1},\dots,s_{n}]\in\mathbb{Z}^{n}$
with entries arranged in non-decreasing order and bounding the corresponding
column degrees of $\mathbf{F}$. }

\ENSURE{A $\vec{s}$-minimal kernel basis $\mathbf{N}$ of $\mathbf{F}$
and the row rank profile of $\mathbf{F}$ given by a list binary indicators
$\bar{e}=\left[e_{1},\dots,e_{m}\right]$, with $1$'s indicating
the columns in the rank profile.}

\STATE{$\xi:=\sum_{i=1}^{n}s_{i}$; $\rho:=\sum_{i=n-m+1}^{n}s_{i};$ $s:=\rho/m$; }

\STATE{$\left[\mathbf{P},\vec{b}\right]:=\mab\left(\mathbf{F},3s,\vec{s}\right)$,
a $\left(\mathbf{F},3s,\vec{s}\right)$-basis with the columns of
$\mathbf{P}$ and the entries of is $\vec{s}$-column degrees $\vec{b}$
arranged so that the entries of $\vec{b}$ are in non-decreasing order;}

\STATE{$\left[\mathbf{P}_{1},\mathbf{P}_{2}\right]:=\mathbf{P}$ where $\mathbf{P}_{1}$
consists of all columns $\mathbf{p}$ of $\mathbf{P}$ satisfying
$\mathbf{F}\mathbf{p}=0$;}

\IF{$m=1$}

\ifbody{\IF{$\mathbf{F}=0$ }

\ifbody{\RETURN $\mathbf{P}_{1},[0]$

\ELSE{}

\RETURN $\mathbf{P}_{1},[1]$}

\ELSE{}

\STATE{ $\vec{t}:=\deg_{\vec{s}}\mathbf{P}_{2}-\left[3s,3s,\dots,3s\right];$}

\STATE{$\mathbf{G}:=\mathbf{F}\mathbf{P}_{2}/x^{3s}$;}

\STATE{$\left[\mathbf{G}_{1}^{T},\mathbf{G}_{2}^{T}\right]^{T}:=\mathbf{G}$,
with $\mathbf{G}_{1}$ having $\left\lfloor m/2\right\rfloor $ rows
and $\mathbf{G}_{2}$ having $\left\lceil m/2\right\rceil $ rows;}

\STATE{$\mathbf{N}_{1},\bar{e}_{1}:=\mnb\left(\mathbf{G}_{1},\vec{t}\right);$ }

\STATE{$\mathbf{N}_{2},\bar{e}_{2}:=\mnb\left(\mathbf{G}_{2}\mathbf{N}_{1},\cdeg_{\vec{t}}\mathbf{N}_{1}\right);$}

\STATE{$\mathbf{Q}:=\mathbf{N}_{1}\mathbf{N}_{2}$;}

\RETURN $\left[\mathbf{P}_{1},\mathbf{P}_{2}\mathbf{Q}\right],\left[\bar{e}_{1},\bar{e}_{2}\right]$}
\end{algorithmic}
\end{algorithm}




\section{Successive Rank Profile Computation}

To compute the rank and rank profile of $\mathbf{F}$ in a rank-sensitive
way, we do a series of computations with a set of increasing number
of rows from $\mathbf{F}$.

We start with the first nonzero row of $\mathbf{F}$, which is the
first row in the rank-profile. Suppose we have found the indices
$\bar{j}=\left[j_{1},\dots,j_{k}\right]$ in the rank profile. To
find the next linearly independent rows, we work with the matrix $\mathbf{G}=\mathbf{F}_{\left[\bar{j},j_{k+1}\dots,j_{k}+i\right]}$,
the matrix consists of the $k$ linearly independent rows indexed
by $\bar{j}$ and the next $k$ rows. We compute a column basis $\mathbf{T}$
of this matrix, which has the same rank profile as $\mathbf{G}$.
Now we can use \prettyref{alg:minimalNullspaceBasisWithRankProfile}
to compute the rank profile of $\mathbf{T}$, which gives more indices
for the rank profile of $\mathbf{F}$. We repeat this procedure until
all rows of $\mathbf{F}$ are processed. This gives us \prettyref{alg:rankProfile}
for computing the rank profile of $\mathbf{F}$.

\begin{algorithm}[t]
\caption{$\rankProfile(\mathbf{F})$: Compute a rank profile of $\mathbf{F}$}
\label{alg:rankProfile}

\begin{algorithmic}
[1]\REQUIRE{$\mathbf{F}\in\mathbb{K}\left[x\right]^{m\times n}$. }

\ENSURE{A row rank profile $\bar{j}=\left[j_{1},\dots,j_{r}\right]\in\mathbb{Z}^{r}$
of $\mathbf{F}$. }

\STATE{$k:=1$; ($k$ keeps track of the current row) }

\STATE{\textbf{while }$k\le m$ \textbf{and }$\mathbf{F}_{k}\ne0$ \textbf{do
$k:=k+1$ end while}; (find the first nonzero row $\mathbf{F}_{k}$)}

\STATE{$\bar{j}=\left[k\right];$ $r:=1$; ($r$ is the current rank)}

\WHILE{$k<m$ \AND{} }

\whilebody{\STATE{$k':=\min\left(m,k+r-1\right)$; (last row in the block)}

\STATE{$\mathbf{G}:=\mathbf{F}_{\left[\bar{j},k\dots,k'\right]};$ (the $r$
linearly independent rows and the next block of rows)}

\STATE{$\mathbf{T}:=\colBasis(\mathbf{G});$}

\STATE{$\mathbf{N},\bar{e}:=\mnbrp(\mathbf{G});$ }

\FOR{$i$ \textbf{from $r+1$ to }$2r$ (convert indicator to indices and
append to $\bar{j}$) }

\forbody{\STATE{\textbf{if} $e_{i}=1$ \textbf{then} $\bar{j}:=\left[\bar{j},e_{i}-r+k-1\right]$
\textbf{end if};}}

\STATE{$k:=k+r$; }}

\RETURN $\bar{j}$;
\end{algorithmic}
\end{algorithm}



The main remaining task is to analyze the computational cost of \prettyref{alg:rankProfile}. 
\begin{thm}
The cost of \prettyref{alg:rankProfile} is $O^{\sim}\left(\xi mr^{\omega-2}\right)$
field operations for computing a rank profile of $\mathbf{F}\in\mathbb{K}\left[x\right]^{m\times n}$.\end{thm}
\begin{proof}
Let $r_{i}$ be the number of rows in the new block considered at
step $i$. Then the cost at step $i$ is $O^{\sim}\left(r_{i}^{\omega-1}\xi\right)$
field operations for the column basis and the kernel basis with rank
profile computation. The total cost is then 
\[
\sum O^{\sim}\left(r_{i}^{\omega-1}\xi\right)=O^{\sim}\left(\xi\sum r_{i}^{\omega-1}\right).
\]
 We also know that $\sum r_{i}=m$, and the maximum of $r_{i}$ is
the rank $r$ of $\mathbf{F}$. Note that 
\[
\sum r_{i}^{\omega-1}\le\frac{m}{r}r^{\omega-1}=mr^{\omega-2}.
\]
 Hence 
\[
\sum O^{\sim}\left(r_{i}^{\omega-1}\xi\right)=O^{\sim}\left(\xi\sum r_{i}^{\omega-1}\right)=O^{\sim}\left(\xi mr^{\omega-2}\right).
\]

\end{proof}

\section{Applications of Rank Profile Computation}


\subsection{\label{sub:removeDimensionAssumption}Remove the assumption $n\ge m$}

In order basis, kernel basis, and column basis computations from previous
chapters, we have assumed that the column dimension $n$ is no less
than the row dimension $m.$ We can now use the rank profile computation
to ensure that this is always the case. For order basis and kernel
basis computation, we can just determine the rank profile $\bar{j}$
of the input matrix $\mathbf{F}$, and then work with just $\mathbf{F}_{\bar{j}}$,
which consists of only $r$ linearly independent rows, as we know
the rank $r$ is always bounded by the column dimension $n$. For
column basis computation, the assumption is only required by the kernel
basis computation used. Therefore removing this assumption from the
kernel basis computation also removes this assumption from the column
basis computation.


\subsection{Rank-sensitive computation of minimal kernel bases}

With the ability to compute a rank profile efficiently, we can now
slightly improve \prettyref{cor:costOfMinimalNullspaceBasis} on the
cost of kernel basis computation with a matrix of degree $d$, by
using only the linearly independent rows from $\mathbf{F}$, hence
reducing the row dimension of the input matrix from $m$ to $r$,
after a cost of $O^{\sim}\left(\xi mr^{\omega-2}\right)=O^{\sim}\left(nmr^{\omega-2}d\right)$
to compute the rank profile. 
\begin{thm}
Given a matrix $\mathbf{F}\in\mathbb{K}\left[x\right]^{m\times n}$with
degree $d$, a minimal kernel basis of $\mathbf{F}$ can be computed
with a cost of $O^{\sim}(nmr^{\omega-2}d+n^{\omega-1}rd)$.
\end{thm}
It looks difficult to further improve this cost by removing the exponent
$\omega$ from the column dimension $n$ if a minimal kernel basis
is required. Since the minimality requires us to work with some matrix
involving all $n$ columns at the same time. 
