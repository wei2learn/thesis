
\chapter{\label{chap:Matrix-inverse}Matrix inverse}

In this chapter, we consider the problem of computing the inverse
of a $n\times n$ polynomial matrix with degree $d$. \citet{jeannerod-villard:05}
gave a deterministic algorithm for this problem that costs $O^{\sim}\left(n^{3}d\right)$
field operations. But their algorithm only works well on input matrices
that are generic with dimension a power of 2. \citet{storjohann:2008}
gave another algorithm with a similar cost, but the algorithm is randomized
Las Vegas. In the following, we show that Jeannerod and Villard's
algorithm can be improved to handle any matrix with a cost of $O^{\sim}\left(n^{3}d\right)$
using new results from this thesis. The algorithm given here is still
deterministic. In fact, an additional improvement of our algorithm
is that it is more general, as its computational cost is given in
terms of the sum of the column (or row) degrees. If $\xi$ is the
minimum of the sum of the column degrees and the sum of the row degrees
of the input matrix, then the inverse can be computed with $O^{\sim}\left(n^{2}\xi\right).$
In the following, we assume without loss of generality that the sum
of the column degrees is the minimum sum.

\begin{comment}
Note that rank sensitive computation for order basis is not as natural,
since for order basis we work with power series in general, whose
rank may never be truly computed. In addition, the computed basis
does not correspond to the rank. For these reasons, we do not pursue
rank sensitive computations for order basis. 
\end{comment}
\begin{algorithm}[t]
\caption{$\inv(\mathbf{F},\vec{s})$}
\label{alg:matrixInverse}

\begin{algorithmic}[1]
\REQUIRE{$\mathbf{F}\in\mathbb{K}\left[x\right]^{n\times n}$; $\vec{s}$ is
initially set to the column degrees of $\mathbf{F}$. It keeps track
of the degrees.}

\ENSURE{$\mathcal{A}=\left[\mathbf{A}_{1},\dots,\mathbf{A}_{\left\lceil \log n\right\rceil }\right],\mathbf{B}$
with $\mathbf{A}_{1},\dots,\mathbf{A}_{\left\lceil \log n\right\rceil },\mathbf{B}\in\mathbb{K}\left[x\right]^{n\times n}$
such that $\mathbf{A}_{1}\dots\mathbf{A}_{\left\lceil \log n\right\rceil }\mathbf{B}^{-1}=\mathbf{F}^{-1}$
if $\mathbf{F}$ is nonsingular, or fail if $\mathbf{F}$ is singular.}

\begin{comment}
\STATE{$\vec{s}:=$column degrees of $\mathbf{F}$;}
\end{comment}


\STATE{$\left[\mathbf{F}_{1}^{T},\mathbf{F}_{2}^{T}\right]:=\mathbf{F}^{T}$
with $\mathbf{F}_{1}$ consists of the top $\left\lceil n/2\right\rceil $
rows of $\mathbf{F}$;}

\STATE{\textbf{if }$\mathbf{F}=0$ \textbf{then} fail \textbf{endif};}

\begin{comment}
\IF{$\mathbf{F}=0$ }

\ifbody{fail; //$\mathbf{F}$ is singular}
\end{comment}


\STATE{\textbf{if }$n=1$ \textbf{then} \textbf{return} $\left\{ 1,\mathbf{F}\right\} $;
\textbf{endif};}

\begin{comment}
\IF{$n=1$ }

\ifbody{\RETURN $1,\mathbf{F}$;}
\end{comment}


\STATE{$\mathbf{N}_{1}:=\mnb(\mathbf{F}_{1},\vec{s})$;$\mathbf{N}_{2}:=\mnb(\mathbf{F}_{2},\vec{s})$;}\label{line:nullspaceBasisComputation}

\begin{comment}
\IF{$\columnDimension(\mathbf{N}_{1})\ne\left\lfloor n\right\rfloor $
\OR{} $\columnDimension(\mathbf{N}_{2})\ne\left\lceil n\right\rceil $ }

\ifbody{ fail; //$\mathbf{F}$ is singular}
\end{comment}


\STATE{\textbf{if }$\columnDimension(\mathbf{N}_{1})\ne\left\lfloor n/2\right\rfloor $
\OR{} $\columnDimension(\mathbf{N}_{2})\ne\left\lceil n/2\right\rceil $
\textbf{then} fail; \textbf{endif};}

\STATE{$\mathbf{R}_{1}:=\mathbf{F}_{1}\mathbf{N}_{2}$;$\mathbf{R}_{2}:=\mathbf{F}_{2}\mathbf{N}_{1}$;}\label{line:multiplyFN}

\STATE{$\left\{ \mathcal{A}^{(1)},\mathbf{H}_{1}\right\} :=\inv(\mathbf{R}_{1},\cdeg_{\vec{s}}\mathbf{N}_{2})$;
$\left\{ \mathcal{A}^{(2)},\mathbf{H}_{2}\right\} :=\inv(\mathbf{R}_{2},\cdeg_{\vec{s}}\mathbf{N}_{1})$;}

\STATE{$\mathcal{A}:=\left[\left[\mathbf{N}_{2},\mathbf{N}_{1}\right],\diag(\mathcal{A}_{1}^{(1)},\mathcal{A}_{1}^{(2)}),\dots,\diag(\mathcal{A}_{\left\lceil \log n\right\rceil -1}^{(1)},\mathcal{A}_{\left\lceil \log n\right\rceil -1}^{(2)})\right]$}

\STATE{\textbf{return} $\left\{ \mathcal{A},\diag\left([\mathbf{H}_{1},\mathbf{H}_{2}]\right)\right\} $;}
\end{algorithmic}
\end{algorithm}



\prettyref{alg:matrixInverse} is a recursive version of the algorithm
from \citet{jeannerod-villard:05}, except that we replace the kernel
basis computation at \prettyref{line:nullspaceBasisComputation} and
the matrix multiplications at \prettyref{line:multiplyFN} with the
new algorithms from this thesis. The algorithm also returns a list
of matrices $\mathbf{A}_{1},\dots,\mathbf{A}_{\left\lceil \log n\right\rceil },\mathbf{B}$
satisfying $\mathbf{A}_{1}\dots\mathbf{A}_{\left\lceil \log n\right\rceil }\mathbf{B}^{-1}=\mathbf{F}^{-1}$,
instead of just two matrices $\mathbf{A},\mathbf{B}$ satisfying $\mathbf{A}\mathbf{B}^{-1}=\mathbf{F}^{-1}$.
We can then compute the product $\mathbf{A}=\mathbf{A}_{1}\dots\mathbf{A}_{\left\lceil \log n\right\rceil }$
with a cost of $O^{\sim}\left(n^{2}\xi\right)$. It is interesting
to note that the output $\mathbf{A}_{1},\dots,\mathbf{A}_{\left\lceil \log n\right\rceil },\mathbf{B}$
takes only $O(n\xi\log n)$ space, but the product $\mathbf{A}=\mathbf{A}_{1}\dots\mathbf{A}_{\left\lceil \log n\right\rceil }$
takes $O(n^{2}\xi)$ space.

Let us first look at the cost of the kernel basis computation and
matrix multiplications, since they dominate the cost of \prettyref{alg:matrixInverse}. 
\begin{lem}
The kernel basis computation at \prettyref{line:nullspaceBasisComputation}
costs $O^{\sim}(n^{\omega-1}\xi)$.\end{lem}
\begin{proof}
Just use the earlier kernel basis algorithm with the shift set to
the column degrees of the input matrix.\end{proof}
\begin{lem}
The multiplications $\mathbf{R}_{1}:=\mathbf{F}_{1}\mathbf{N}_{2}$
and $\mathbf{R}_{2}:=\mathbf{F}_{2}\mathbf{N}_{1}$at \prettyref{line:multiplyFN}
cost $O^{\sim}(n^{\omega-1}\xi)$.\end{lem}
\begin{proof}
From \prettyref{thm:boundOfSumOfShiftedDegreesOfKernelBasis} we know
that the sum of the $\vec{s}$-column degrees of $\mathbf{N}_{1}$
and that of $\mathbf{N}_{2}$ are both bounded by $\xi$. Now \prettyref{thm:multiplyUnbalancedMatrices}
can be applied.\end{proof}
\begin{thm}
\label{thm:inverseCost}\prettyref{alg:matrixInverse} costs $O^{\sim}\left(n^{\omega-1}\xi\right)$
field operations to compute an inverse of a nonsingular matrix $\mathbf{F}\in\mathbb{K}\left[x\right]^{n\times n}$,
where $\xi$ is the minimum of the sum of the column degrees and the
sum of the row degrees of the input matrix.\end{thm}
\begin{proof}
If the sum of the row degrees is smaller, we can just transpose the
matrix. Let the cost be $g(n,\xi)$. Then we have the following recurrence
relation:
\begin{eqnarray*}
g(n,\xi) & \in & O^{\sim}(n^{\omega-1}\xi)+g(\left\lceil n/2\right\rceil ,\xi)+g(\left\lfloor n/2\right\rfloor ,\xi)\\
 & \in & O^{\sim}(n^{\omega-1}\xi)+2g(\left\lceil n/2\right\rceil ,\xi)\\
 & \in & O^{\sim}(n^{\omega-1}\xi).
\end{eqnarray*}
 Note that always rounding up $n/2$ to $\left\lceil n/2\right\rceil $
is no worse than assuming $n$ is a power of $2$. In other words,
the entries in the sequence $\left[\left\lceil n/2\right\rceil ,\left\lceil n/4\right\rceil ,\dots,1\right]$
is no larger than the corresponding entries in the sequence $\left[n'/2,n'/4,\dots,1\right]$,
where $n'$ is the smallest power of $2$ that is no less than $n$,
that is, $n'=2^{\left\lceil \log_{2}n\right\rceil }$. \end{proof}
\begin{lem}
The multiplications $\mathbf{A}=\mathbf{A}_{1}\dots\mathbf{A}_{\left\lceil \log n\right\rceil }$
can be done with a cost of $O^{\sim}(n^{2}\xi)$ .\end{lem}
\begin{proof}
Note that $\mathbf{A}_{i}$ for $i\le\log n$ has $2^{i}$ blocks
on the diagonal. Each block of $\mathbf{A}_{i}$ is used to compute
two corresponding blocks of $\mathbf{A}_{i+1}$. Let us first look
at $\mathbf{A}_{1}=\left[\mathbf{N}_{2},\mathbf{N}_{1}\right]$ and
\[
\mathbf{A}_{2}=\begin{bmatrix}\mathbf{N}'_{2} & \mathbf{N}'_{1}\\
 &  & \mathbf{M}'_{2} & \mathbf{M}'_{1}
\end{bmatrix},
\]
 where $\mathbf{N}'_{1},$ $\mathbf{N}'_{2}$ are the kernel bases
of the submatrices $\mathbf{F}'_{1},$ $\mathbf{F}'_{2}$ contained
in 
\[
\mathbf{R}_{1}=\begin{bmatrix}\mathbf{F}'_{1}\\
\mathbf{F}'_{2}
\end{bmatrix}=\mathbf{F}_{1}\mathbf{N}_{2}.
\]
 When multiplying $\mathbf{A}_{1}$ and $\mathbf{A}_{2}$, the submatrix
$\mathbf{N}_{2}$ of $\mathbf{A}_{1}$ is multiplied with the block
$\left[\mathbf{N}'_{2},\mathbf{N}'_{1}\right]$ in $\mathbf{A}_{2}$.
Let $\vec{s}'$ be the list of the $\vec{s}$-column degrees of $\mathbf{N}_{2}$,
where $\vec{s}$ is list of the column degrees of the input matrix
$\mathbf{F}$. Then $\sum\vec{s}'\le\sum\vec{s}=\xi$ by \prettyref{thm:boundOfSumOfShiftedDegreesOfKernelBasis}.
From \prettyref{lem:boundOnDegreesOfFA}, we know the column degrees
of $\mathbf{R}_{1}=\mathbf{F}_{1}\mathbf{N}_{2}$ are bounded component-wise
by the $\vec{s}$-column degrees $\vec{s}'$ of $\mathbf{N}_{2}$,
hence the sum of the column degrees of $\mathbf{R}_{1}$ is also bounded
by $\xi$. It follows that the sum of $\vec{s}'$-column degrees of
$\mathbf{N}'_{1}$ and that of $\mathbf{N}'_{2}$ are each bounded
by $\xi$. We can therefore use \prettyref{thm:multiplyUnbalancedMatrices}
to multiply $\mathbf{N}_{2}$ and $\left[\mathbf{N}'_{2},\mathbf{N}'_{1}\right]$
with a cost of $O^{\sim}\left(n^{\omega-1}\xi\right)$. From \prettyref{lem:boundOnDegreesOfFA},
the $\vec{s}$-column degrees of the product $\mathbf{N}_{2}\left[\mathbf{N}'_{2},\mathbf{N}'_{1}\right]$
are bounded by the $\vec{s}'$-column degrees of $\left[\mathbf{N}'_{2},\mathbf{N}'_{1}\right]$,
hence the sum of the $\vec{s}$-column degrees of each column block
in $\mathbf{N}_{2}\left[\mathbf{N}'_{2},\mathbf{N}'_{1}\right]=\left[\mathbf{N}_{2}\mathbf{N}'_{2},\mathbf{N}_{2}\mathbf{N}'_{1}\right]$
is still bounded by $\xi$. The multiplication involving $\mathbf{N}_{1}$
and the second block of $\mathbf{A}_{2}$ is done in the same way
as the multiplication $\mathbf{N}_{2}\left[\mathbf{N}'_{2},\mathbf{N}'_{1}\right]$,
hence the multiplication $\mathbf{A}_{1}\mathbf{A}_{2}$ cost $O^{\sim}\left(n^{\omega-1}\xi\right)$,
with the sum of $\vec{s}$-column degrees of each of the four column
blocks in $\mathbf{A}_{1}\mathbf{A}_{2}=\left[\mathbf{N}_{2}\mathbf{N}'_{2},\mathbf{N}_{2}\mathbf{N}'_{1},\mathbf{N}_{1}\mathbf{M}'_{2},\mathbf{N}_{1}\mathbf{M}'_{1}\right]$
bounded by $\xi$.

Next, we multiply $\mathbf{A}_{1}\mathbf{A}_{2}$ with $\mathbf{A}_{3}$.
The matrix $\mathbf{A}_{3}$ now has four blocks on the diagonal.
Consider $\mathbf{N}_{2}\mathbf{N}'_{2}$ , the first column block
of $\mathbf{A}_{1}\mathbf{A}_{2}$, multiplied with the first block
$\left[\mathbf{N}"_{2},\mathbf{N}"_{1}\right]$ on the diagonal of
$\mathbf{A}_{3}$. Let $\vec{s}"$ be the $\vec{s}'$-column degrees
of $\mathbf{N}'_{2}$, which bound the $\vec{s}$-column degrees of
$\mathbf{N}_{2}\mathbf{N}'_{2}$. Then $\sum\vec{s}"\le\sum\vec{s}'\le\sum\vec{s}=\xi$.
Following the same reasoning as before, the sum of the $\vec{s}"$-column
degrees of $\mathbf{N}"_{2}$ is still bounded by $\xi$. We can therefore
again use \prettyref{thm:multiplyUnbalancedMatrices} to multiply
$\mathbf{N}_{2}\mathbf{N}'_{2}$ and $\mathbf{N}"_{2}$. The multiplication
of the remaining blocks are done in the same way. The product $\mathbf{A}_{1}\mathbf{A}_{2}\mathbf{A}_{3}$
now has 8 column blocks, with the sum of the $\vec{s}$-column degrees
of each column block bounded by $\xi$.

Repeating this process, we multiply $\mathbf{A}_{1}\cdots\mathbf{A}_{i}$
with $\mathbf{A}_{i+1}$ at step $i$ for $i$ from 1 to $\left\lfloor \log n\right\rfloor $.
Each of the $2^{i}$ column blocks of $\mathbf{A}_{1}\cdots\mathbf{A}_{i}$
has dimension $n\times O(n/2^{i})$. Each of the $O(2^{i})$ column
blocks on the diagonal of $\mathbf{A}_{i+1}$ has dimension $O(n/2^{i})\times O(n/2^{i})$.
(Big $O$ notation is used here because $n/2^{i}$ may not be an integer.)
Let $\vec{u}_{j}$ be the shift used to compute the $j$th in $\mathbf{A}_{j+1}$,
then as before, the $\vec{s}$-column degrees of the $j$th column
block in $\mathbf{A}_{1}\cdots\mathbf{A}_{i}$ are bounded by $\vec{u}_{j}$,
with $\sum\vec{u}_{j}\le\xi$. The sum of the $\vec{u}$-column degrees
of the $j$th block in $\mathbf{A}_{j+1}$ is bounded by $2\xi$.
(Each of the left half and the right half has the sum bounded by $\xi$.)
Therefore, multiplying $\mathbf{A}_{1}\cdots\mathbf{A}_{i}$ with
$\mathbf{A}_{i+1}$ cost $O^{\sim}\left(2^{j}2^{j}\left(n/2^{j}\right)^{\omega-1}\xi\right)$.
Take $\omega=3,$ we get $O^{\sim}\left(n^{2}\xi\right)$ as desired.
\end{proof}
Again, it is interesting to note that \prettyref{alg:matrixInverse}
costs only $O^{\sim}\left(n^{\omega-1}\xi\right)$ to compute the
inverse and it represents the inverse with $O\left(n\xi\log n\right)$
space. It is possible that this representation is useful in some applications.
For example, if we wish to multiply another low degree matrix or a
row vector $\mathbf{H}$ by $\mathbf{F}^{-1}$, representing $\mathbf{F}^{-1}=\mathbf{A}\mathbf{B}^{-1}$
requires us to multiply $\mathbf{H}$ with a high degree matrix $\mathbf{A}$.
This can be more expensive than the multiplication using the representation
$\mathbf{F}^{-1}=\mathbf{A}_{1}\mathbf{A}_{2}\cdots\mathbf{A}_{\left\lceil \log n\right\rceil }\mathbf{B}^{-1}$,
then $\mathbf{H}\mathbf{F}^{-1}=\mathbf{H}\mathbf{A}_{1}\mathbf{A}_{2}\cdots\mathbf{A}_{\left\lceil \log n\right\rceil }\mathbf{B}^{-1}$,
which is less expensive. It may be interesting to look for other applications
where this smaller representation is useful.
