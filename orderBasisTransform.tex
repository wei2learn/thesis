
\section{Extending Storjohann's Transformation}

\label{sec:transform}

\begin{comment}
notations: 
\begin{itemize}
\item Matrices

\begin{itemize}
\item Polynomial matrices: bold capital $\mathbf{A,B,F,G,P,R}$

\begin{itemize}
\item $\mathbf{F}$: input matrix,

\begin{itemize}
\item $\bar{\mathbf{F}}$: Storjohann transformed matrix, 
\item $\check{\mathbf{F}}$: the transformation that extends $\bar{\mathbf{F}}$. 
\end{itemize}
\item $\mathbf{G}$: input matrix for the second subproblem. 
\end{itemize}
\item scalar matrices: non-bold Capital, used to distinguish coefficient
matrix 
\item Identity matrix: $\mathbf{I}$ this is to be consistent with polynomial
matrices, as identity is considered as an important element of polynomial
matrices. 
\end{itemize}
\item order

\begin{itemize}
\item $\sigma,\omega$ 
\end{itemize}
\item vectors

\begin{itemize}
\item polynomial vectors: $\mathbf{p,q,r,s,t}$ 
\item $\mathbf{p}$: element in basis $\mathbf{P}$ 
\item $\mathbf{q}:$ some test element in $\left\langle \left(\mathbf{F},\sigma\right)\right\rangle $ 
\item $\bar{\mathbf{t}}$: test element in $\left\langle \left(\check{\mathbf{F}},\vec{\omega}\right)\right\rangle $ 
\end{itemize}
\item shifts:

\begin{itemize}
\item $\vec{e}$: uniform 0 shift used for transformation. 
\item $\vec{s}$: original input shift 
\item $\vec{t}$: output shifted degrees of the current basis, with result
from previous subproblems included. i.e., $\left[\vec{s},0,\dots,0\right]$-degrees
of the current basis. 
\item $\vec{b},\vec{a}$: input shift and output shifted degrees for each
reduced subproblem that calls OrderBasis. In the iterative computational
process, input shift $\vec{b}^{\left(i-1\right)}$ corresponds to
input shift $\vec{a}^{\left(i\right)}$. The high degree entries of
$\vec{a}^{\left(i\right)}$ at iteration $i$ is then used as the
input shift $\vec{b}^{\left(i\right)}$ for iteration $i$. 
\end{itemize}
\item dimension:

\begin{itemize}
\item $m,n$: row, column dimension 
\item $k$: number of columns with $\left[\vec{s},0,\dots,0\right]$-degrees
less than degree bound 
\item $l$: number of block rows 
\end{itemize}
\item degrees, order:

\begin{itemize}
\item $\sigma$: order 
\item $\delta$: degree bound 
\item $d$: average degree (=$m\sigma/n$) \end{itemize}
\end{itemize}
\end{comment}


In this section, we introduce a transformation that can be viewed
as an extension of Storjohann's transformation which allows for computation
of a full, rather than partial, order basis. More generally (as discussed
in the next section) this transformation provides a link between two
Storjohann transformed problems constructed using different degree
parameters. For easier understanding, we first focus on a particular
case of this transformation in \prettyref{sub:particularCase} and
then generalize this in \prettyref{sub:generalTransform}.


\subsection{\label{sub:particularCase}A Particular Case}

Consider the problem of computing a $\left(\mathbf{F},\sigma,\vec{s}\right)$-basis.
We assume $\sigma=4\delta$ %
\begin{comment}
(The results below hold for any positive integer great than one, but
4 is used for simplicity. ) 
\end{comment}
for a positive integer $\delta$ and write the input matrix polynomial
as $\mathbf{F}=\mathbf{F}_{0}+\mathbf{F}_{1}x^{\delta}+\mathbf{F}_{2}x^{2\delta}+\mathbf{F}_{3}x^{3\delta}$
with $\deg\mathbf{F}_{i}<\delta$.%
\begin{comment}
again, changed from $\deg\mathbf{F}_{i}\le\delta-1$ to $\deg\mathbf{F}_{i}<\delta$
even though I prefer $\le$ 
\end{comment}
{} In the following, we show that computing a $\left(\mathbf{F},\sigma,\vec{s}\right)$-basis
can be done by computing a $(\mathbf{F}',\vec{\omega},\vec{s'})$-basis
where 
\begin{equation}
\mathbf{F}'=\left[\begin{array}{cc}
\mathbf{F} & \mathbf{0}\\
\mathbf{F}'_{21} & \mathbf{F}'_{22}
\end{array}\right]=\left[\begin{array}{c|cr}
\mathbf{F}_{0}+\mathbf{F}_{1}x^{\delta}+\mathbf{F}_{2}x^{2\delta}+\mathbf{F}_{3}x^{3\delta} & \mathbf{0} & \mathbf{0}\\
\hline \mathbf{F}_{1}+\mathbf{F}_{2}x^{\delta} & \mathbf{I}_{m} & \mathbf{0}\\
\mathbf{F}_{2}+\mathbf{F}_{3}x^{\delta} & \mathbf{0} & \mathbf{I}_{m}
\end{array}\right]\label{eq:extendedSTransformTop}
\end{equation}
 with order $\vec{\omega}=\left[4\delta,\dots,4\delta,2\delta,\dots,2\delta\right]$
(with $m$ $4\delta$'s and $2m$ $2\delta$'s) and degree shift $\vec{s'}=\left[\vec{s},e,\dots,e\right]$
(with $2m$ $e$'s), where $e$ is an integer less than or equal to
$1$. %
\begin{comment}
actually true for $e\le\min\vec{s}+1$. keeping this $e$ helps to
show that this hold true in more general situations, while Storjohann
only used $e=1$. it can be negative as well. 
\end{comment}
{} We set $e$ to $0$ in this paper for simplicity%
\footnote{Storjohann used $e=1$ in \citep{Storjohann:2006}. All results in
this section still hold for any other $e\le1.$%
}. %
\begin{comment}
In fact, it is quite easy to construct a $(\check{\mathbf{F}},\vec{\omega},\vec{s'})$-basis
from a $\left(\mathbf{F},\sigma,\vec{s}\right)$-basis, as we show
later in \prettyref{lem:FtoAbasis}. However, it requires more work
to extract a $\left(\mathbf{F},\sigma,\vec{s}\right)$-basis from
a $(\check{\mathbf{F}},\vec{\omega},\vec{s'})$-basis, which is addressed
eventually in \prettyref{cor:extractingFbasisFromGbasis}. Note that
although constructing a $(\check{\mathbf{F}},\vec{\omega},\vec{s'})$-basis
from a $\left(\mathbf{F},\sigma,\vec{s}\right)$-basis in \prettyref{lem:FtoAbasis}
is the reverse of what we want, this well-formed $(\check{\mathbf{F}},\vec{\omega},\vec{s'})$-basis
restricts the elements of $\langle(\check{\mathbf{F}},\vec{\omega},\vec{s'})\rangle$
to a simple form shown in \prettyref{cor:FtauBasisForm}, which helps
to establish a close correspondence between a $(\check{\mathbf{F}},\vec{\omega},\vec{s'})$-basis
and a $\left(\mathbf{F},\sigma,\vec{s}\right)$-basis in \prettyref{lem:2delta-1Basis},
\prettyref{lem:2deltaBasis}, and \prettyref{thm:mainTheorem}. 
\end{comment}


We first look at the correspondence between the elements of $\left\langle \left(\mathbf{F},\sigma,\vec{s}\right)\right\rangle _{\tau}$
and the elements of $\langle(\mathbf{F}',\vec{\omega},\vec{s'})\rangle_{\tau}$
in \prettyref{lem:qToBqOrder} to \prettyref{lem:bqToqOrder}. The
correspondence between $\left(\mathbf{F},\sigma,\vec{s}\right)$-bases
and $(\mathbf{F}',\vec{\omega},\vec{s'})$-bases is then considered
in \prettyref{cor:FtauBasisForm} to \prettyref{thm:mainTheorem}.

Let 
\[
\mathbf{B}=\left[\begin{array}{c}
\mathbf{I}_{n}\\
x^{-\delta}\mathbf{F}_{0}\\
x^{-2\delta}\left(\mathbf{F}_{0}+\mathbf{F}_{1}x^{\delta}\right)
\end{array}\right].
\]

\begin{lem}
\label{lem:qToBqOrder}If $\mathbf{q}\in\left\langle \left(\mathbf{F},\sigma\right)\right\rangle $,
then $\mathbf{B}\mathbf{q}\in\langle(\mathbf{F}'\vec{,\omega})\rangle$.\end{lem}
\begin{proof}
The lemma follows from 
\[
\mathbf{F}'\mathbf{B}\mathbf{q}=\left[\begin{array}{r}
\mathbf{F}_{0}+\mathbf{F}_{1}x^{\delta}+\mathbf{F}_{2}x^{2\delta}+\mathbf{F}_{3}x^{3\delta}\\
\mathbf{F}_{0}x^{-\delta}+\mathbf{F}_{1}+\mathbf{F}_{2}x^{\delta}\\
\mathbf{F}_{0}x^{-2\delta}+\mathbf{F}_{1}x^{-\delta}+\mathbf{F}_{2}+\mathbf{F}_{3}x^{\delta}
\end{array}\right]\mathbf{q}\equiv\mathbf{0}\mod x^{\vec{\omega}}.
\]
 Note that the bottom rows of $\mathbf{B}$ may not be polynomials.
However, $\mathbf{B}\mathbf{q}$ is a polynomial vector since $\mathbf{q}\in\left\langle \left(\mathbf{F},\sigma\right)\right\rangle $
implies $\mathbf{q}\in\left\langle \left(\mathbf{F}_{0},\delta\right)\right\rangle $
and $\mathbf{q}\in\left\langle \left(\mathbf{F}_{0}+\mathbf{F}_{1}x^{\delta},2\delta\right)\right\rangle $. 
\end{proof}
\noindent The following lemma shows that the condition $e\le1$ forces
$\deg_{\vec{s'}}\mathbf{B}\mathbf{q}$ to be determined by $\mathbf{q}$. 
\begin{lem}
\label{lem:qToBqDegree}If $\mathbf{q}\in\left\langle \left(\mathbf{F},\sigma,\vec{s}\right)\right\rangle _{\tau}$
for any degree bound $\tau\in\mathbb{Z}$, then $\deg_{\vec{s'}}\mathbf{B}\mathbf{q}=\deg_{\vec{s}}\mathbf{q}$.\end{lem}
\begin{proof}
By assumption $s_{i}\ge0$, so $\deg\mathbf{q}\le\deg_{\vec{s}}\mathbf{q}.$
Now consider the degree of the bottom $2m$ entries, $\mathbf{q}_{2},\mathbf{q}_{3}$,
of 
\[
\begin{bmatrix}\mathbf{q}\\
\mathbf{q}_{2}\\
\mathbf{q}_{3}
\end{bmatrix}=\mathbf{B}\mathbf{q}=\left[\begin{array}{r}
\mathbf{q}\\
x^{-\delta}\mathbf{F}_{0}\cdot\mathbf{q}\\
x^{-2\delta}\left(\mathbf{F}_{0}+\mathbf{F}_{1}x^{\delta}\right)\cdot\mathbf{q}
\end{array}\right].
\]
 Our goal is to show $\deg_{\vec{e}}\left[\mathbf{q}_{2}^{T},\mathbf{q}_{3}^{T}\right]^{T}\le\deg_{\vec{s}}\mathbf{q}$.
Note that 
\[
\deg\mathbf{q}_{2}=\deg\left(\mathbf{F}_{0}\mathbf{q}/x^{\delta}\right)\le\deg\mathbf{q}+\delta-1-\delta\le\deg_{\vec{s}}\mathbf{q}-1,
\]
 and similarly $\deg\mathbf{q}_{3}\le\deg_{\vec{s}}\mathbf{q}-1$.
Therefore 
\[
\deg_{\vec{e}}\begin{bmatrix}\mathbf{q}_{2}\\
\mathbf{q}_{3}
\end{bmatrix}=\deg\begin{bmatrix}\mathbf{q}_{2}\\
\mathbf{q}_{3}
\end{bmatrix}+e\le\deg_{\vec{s}}\mathbf{q}-1+e\le\deg_{\vec{s}}\mathbf{q}.
\]
 \end{proof}
\begin{cor}
\label{cor:sToBs}If $\mathbf{q}\in\left\langle \left(\mathbf{F},\sigma,\vec{s}\right)\right\rangle _{\tau}$
for any degree bound $\tau\in\mathbb{Z}$ , then\textup{ }$\mathbf{B}\mathbf{q}\in\langle(\mathbf{F}',\vec{\omega},\vec{s'})\rangle_{\tau}$\textup{.} 
\end{cor}
%\begin{pf}
%This follows from \prettyref{lem:qToBqOrder} and \prettyref{lem:qToBqDegree}.\end{pf}

\begin{cor}
\label{cor:linearCombinationOfFirstnRows}Let $\bar{\mathbf{S}}_{\tau}$
be a $(\mathbf{F}',\vec{\omega},\vec{s'})_{\tau}$-basis and $\mathbf{S}_{\tau}$
be the top $n$ rows of $\bar{\mathbf{S}}_{\tau}$ for any bound $\tau\in\mathbb{Z}$.
Then any $\mathbf{q}\in\left\langle \left(\mathbf{F},\sigma,\vec{s}\right)\right\rangle _{\tau}$
is a linear combination of the columns of $\mathbf{S}_{\tau}$.\end{cor}
\begin{proof}
By \prettyref{cor:sToBs}, $\mathbf{B}\mathbf{q}\in\langle(\mathbf{F}',\vec{\omega},\vec{s'})\rangle_{\tau}$,
and so is a linear combination of columns of $\bar{\mathbf{S}}_{\tau}$.
That is, there exists a polynomial vector $\mathbf{u}$ such that
$\mathbf{B}\mathbf{q}=\bar{\mathbf{S}}_{\tau}\mathbf{u}$. This remains
true if we restrict the equation to the top $n$ rows, that is, $\mathbf{q}=\left[\mathbf{I}_{n},\mathbf{0}\right]\mathbf{B}\mathbf{q}=\left[\mathbf{I}_{n},\mathbf{0}\right]\bar{\mathbf{S}}_{\tau}\mathbf{u}=\mathbf{S}_{\tau}\mathbf{u}$.\end{proof}
\begin{lem}
\label{lem:bqToqOrder}Let $\bar{\mathbf{q}}\in\langle(\mathbf{F}',\vec{\omega},\vec{s'})\rangle_{\tau}$
for any degree bound $\tau\in\mathbb{Z}$, and $\mathbf{q}_{1}$ the
first $n$ entries of $\mathbf{\bar{q}}$. Then $\mathbf{q}_{1}\in\left\langle \left(\mathbf{F},\sigma,\vec{s}\right)\right\rangle _{\tau}$.\end{lem}
\begin{proof}
The top rows of 
\[
\mathbf{F}'\mathbf{q}=\left[\begin{array}{cc}
\mathbf{F} & \mathbf{0}\\
\mathbf{F}'_{21} & \mathbf{F}'_{22}
\end{array}\right]\left[\begin{array}{c}
\mathbf{q}_{1}\\
\mathbf{q}_{2}
\end{array}\right]=\begin{bmatrix}\mathbf{F}\mathbf{q}_{1}\\
\mathbf{F}'_{21}\mathbf{q}_{1}+\mathbf{F}'_{22}\mathbf{q}_{2}
\end{bmatrix}\equiv\mathbf{0}\mod x^{\vec{\omega}}
\]
 give $\mathbf{F}\mathbf{q}_{1}\equiv\mathbf{0}\mod x^{\sigma}$. 
\end{proof}
The next lemma shows a $(\mathbf{F}',\vec{\omega},\vec{s'})$-basis
can be constructed from a $\left(\mathbf{F},\sigma,\vec{s}\right)$-basis.
This $(\mathbf{F}',\vec{\omega},\vec{s'})$-basis has a structure
that restricts the elements of $\langle(\mathbf{F}',\vec{\omega},\vec{s'})\rangle$
to a simple form shown in \prettyref{cor:FtauBasisForm}. This in
turn helps to establish a close correspondence between a $(\mathbf{F}',\vec{\omega},\vec{s'})$-basis
and a $\left(\mathbf{F},\sigma,\vec{s}\right)$-basis in \prettyref{lem:2delta-1Basis},
\prettyref{lem:2deltaBasis}, and \prettyref{thm:mainTheorem}. 
\begin{lem}
\label{lem:FtoAbasis}If $\mathbf{P}$ is a $\left(\mathbf{F},\sigma,\vec{s}\right)$-basis,
then 
\begin{eqnarray*}
\bar{\mathbf{T}} & = & \left[\mathbf{B}\mathbf{P}~\begin{array}{|c}
\mathbf{0}_{n\times2m}\\
x^{2\delta}\mathbf{I}_{2m}
\end{array}\right]=\left[\begin{array}{r|cc}
\mathbf{P} & \mathbf{0}_{n\times m} & \mathbf{0}_{n\times m}\\
\hline x^{-\delta}\mathbf{F}_{0}\cdot\mathbf{P} & x^{2\delta}\mathbf{I}_{m} & \mathbf{0}_{m}\\
x^{-2\delta}\left(\mathbf{F}_{0}+\mathbf{F}_{1}x^{\delta}\right)\cdot\mathbf{P} & \mathbf{0}_{m} & x^{2\delta}\mathbf{I}_{m}
\end{array}\right]
\end{eqnarray*}
 is a $(\mathbf{F}',\vec{\omega},\vec{s'})$-basis.\end{lem}
\begin{proof}
By \prettyref{lem:qToBqOrder}, $\bar{\mathbf{T}}$ has order $(\mathbf{F}',\vec{\omega})$
and is $\vec{s'}$-column reduced since $\mathbf{P}$ dominates the
$\vec{s'}$-degrees of $\bar{\mathbf{T}}$ on the left side by \prettyref{lem:qToBqDegree}.
It remains to show that any $\bar{\mathbf{q}}\in\langle(\mathbf{F}',\vec{\omega},\vec{s'})\rangle$
is a linear combination of the columns of $\mathbf{\bar{\mathbf{T}}}$.

Let $\mathbf{q}$ be the top $n$ entries of $\bar{\mathbf{q}}$.
Then by \prettyref{lem:bqToqOrder}, $\mathbf{q}\in\left\langle \left(\mathbf{F},\sigma,\vec{s}\right)\right\rangle $,
hence is a linear combination of the columns of $\mathbf{P}$, that
is $\mathbf{q}=\mathbf{P}\mathbf{u}$ with $\mathbf{u}=\mathbf{P}^{-1}\mathbf{q}\in\mathbb{K}\left[x\right]^{n\times1}$.
Subtracting the contribution of $\mathbf{P}$ from $\bar{\mathbf{q}}$,
we get 
\[
\mathbf{q}'=\bar{\mathbf{q}}-\mathbf{B}\mathbf{P}\mathbf{u}=\bar{\mathbf{q}}-\mathbf{B}\mathbf{q}=\left[\begin{array}{c}
\mathbf{0}\\
\mathbf{v}
\end{array}\right],
\]
 which is still in $\langle(\mathbf{F}',\vec{\omega},\vec{s'})\rangle$,
that is, 
\[
\mathbf{F}'\mathbf{q}'=\begin{bmatrix}\mathbf{0}\\
\mathbf{I}_{2m}\mathbf{v}
\end{bmatrix}\equiv\mathbf{0}\mod x^{\vec{\omega}}.
\]
 This forces $\mathbf{v}$ to be a linear combination of the columns
of $x^{2\delta}\mathbf{I}_{2m}$, the bottom right submatrix of $\bar{\mathbf{T}}$.
Now $\bar{\mathbf{q}}=\bar{\mathbf{T}}\left[\mathbf{u}^{T},\mathbf{v}^{T}\right]^{T}$
as required.\end{proof}
\begin{cor}
\label{cor:FtauBasisForm}Let $\tau\in\mathbb{Z}$ be any degree bound
and $\mathbf{P}_{\tau}\in\mathbb{K}\left[x\right]^{n\times t}$ be
a $\left(\mathbf{F},\sigma,\vec{s}\right)_{\tau}$-basis. If $\bar{\mathbf{q}}\in\langle(\mathbf{F}',\vec{\omega},\vec{s'})\rangle_{\tau}$
and $\mathbf{q}$ is the top $n$ entries of $\bar{\mathbf{q}}$,
then $\bar{\mathbf{q}}$ must have the form 
\[
\bar{\mathbf{q}}=\mathbf{B}\mathbf{P}_{\tau}\mathbf{u}+x^{2\delta}\begin{bmatrix}\mathbf{0}\\
\mathbf{v}
\end{bmatrix}=\mathbf{B}\mathbf{q}+x^{2\delta}\begin{bmatrix}\mathbf{0}\\
\mathbf{v}
\end{bmatrix}
\]
 for some polynomial vector $\mathbf{u}\in\mathbb{K}\left[x\right]^{t\times1}$
and $\mathbf{v}\in\mathbb{K}\left[x\right]^{2m\times1}$. In particular,
if $\deg_{\vec{s'}}\bar{\mathbf{q}}<2\delta$, then $\bar{\mathbf{q}}=\mathbf{B}\mathbf{P}_{\tau}\mathbf{u}=\mathbf{B}\mathbf{q}$. \end{cor}
\begin{proof}
This follows directly from \prettyref{lem:FtoAbasis} with $\vec{s'}$-degrees
restricted to $\tau$.\end{proof}
\begin{lem}
\label{lem:2delta-1Basis}If $\bar{\mathbf{S}}^{\left(1\right)}$
is a $(\check{\mathbf{F}},\vec{\omega},\vec{s'})_{2\delta-1}$-basis,
then the matrix $\mathbf{S}^{\left(1\right)}$ consisting of its first
$n$ rows is a $\left(\mathbf{F},\sigma,\vec{s}\right)_{2\delta-1}$-basis.\end{lem}
\begin{proof}
By \prettyref{lem:bqToqOrder}, $\mathbf{S}^{\left(1\right)}$ has
order $\left(\mathbf{F},\sigma\right)$. By \prettyref{cor:linearCombinationOfFirstnRows},
any $\mathbf{q}\in\left\langle \left(\mathbf{F},\sigma,\vec{s}\right)\right\rangle _{2\delta-1}$
is a linear combination of $\mathbf{S}^{\left(1\right)}$. It remains
to show that $\mathbf{S}^{\left(1\right)}$ is $\vec{s}$-column reduced.

By \prettyref{cor:FtauBasisForm}, $\bar{\mathbf{S}}^{\left(1\right)}=\mathbf{B}\mathbf{S}^{\left(1\right)}$,
and by \prettyref{lem:bqToqOrder}, the columns of $\mathbf{S}^{\left(1\right)}$
are in $\left\langle \left(\mathbf{F},\sigma,\vec{s}\right)\right\rangle _{2\delta-1}$.
Thus, by \prettyref{lem:qToBqDegree}, $\mathbf{S}^{\left(1\right)}$
determines the $\vec{s'}$-column degrees of $\mathbf{S}^{\left(1\right)}$.
Therefore, $\bar{\mathbf{S}}^{\left(1\right)}$ being $\vec{s'}$-column
reduced implies that $\mathbf{S}^{\left(1\right)}$ is $\vec{s}$-column
reduced.\end{proof}
\begin{lem}
\label{lem:2deltaBasis} Let $\bar{\mathbf{S}}^{\left(12\right)}=[\bar{\mathbf{S}}^{\left(1\right)},\bar{\mathbf{S}}^{\left(2\right)}]$
be a $(\mathbf{F}',\vec{\omega},\vec{s'})_{2\delta}$-basis, with
$\deg_{\vec{s'}}\bar{\mathbf{S}}^{\left(1\right)}\le2\delta-1$ and
$\deg_{\vec{s'}}\bar{\mathbf{S}}^{\left(2\right)}=2\delta$, and $\mathbf{S}^{\left(12\right)},\mathbf{S}^{\left(1\right)},\mathbf{S}^{\left(2\right)}$
the first $n$ rows of $\bar{\mathbf{S}}^{\left(12\right)},\bar{\mathbf{S}}^{\left(1\right)},\bar{\mathbf{S}}^{\left(2\right)}$,
respectively. Let $I$ be the column rank profile (the lexicographically
smallest sequence of column indices that indicates a full column rank
submatrix) of $\mathbf{S}^{\left(12\right)}$. %which contains all columns of $\mathbf{S}^{\left(1\right)}$ by \prettyref{lem:2delta-1Basis}.
Then the submatrix\textbf{ $\mathbf{S}_{I}^{\left(12\right)}$ }comprised
of the columns of $\mathbf{S}^{\left(12\right)}$ indexed by $I$
is a $\left(\mathbf{F},\sigma,\vec{s}\right)_{2\delta}$-basis. \end{lem}
\begin{proof}
Consider doing $\vec{s}$-column reduction on $\mathbf{S}^{\left(12\right)}$.
From \prettyref{lem:2delta-1Basis}, we know that $\mathbf{S}^{\left(1\right)}$
is a $\left(\mathbf{F},\sigma,\vec{s}\right)_{2\delta-1}$-basis.
Therefore, only $\mathbf{S}^{\left(2\right)}$ may be $\vec{s}$-reduced.
If a column $\mathbf{c}$ of $\mathbf{S}^{\left(2\right)}$ can be
further $\vec{s}$-reduced, then it becomes an element of $\left\langle \left(\mathbf{F},\sigma,\vec{s}\right)\right\rangle _{2\delta-1}$,
which is generated by $\mathbf{S}^{\left(1\right)}$. Thus $\mathbf{c}$
must be reduced to zero by $\mathbf{S}^{\left(1\right)}$. The only
nonzero columns of $\mathbf{S}^{\left(12\right)}$ remaining after
$\vec{s}$-column reduction are therefore the columns that cannot
be $\vec{s}$-reduced. Hence $\mathbf{S}^{\left(12\right)}$ $\vec{s}$-reduces
to \textbf{$\mathbf{S}_{I}^{\left(12\right)}$}. In addition, \textbf{$\mathbf{S}_{I}^{\left(12\right)}$}
has order $\left(\mathbf{F},\sigma\right)$ as $\mathbf{S}^{\left(12\right)}$
has order $\left(\mathbf{F},\sigma\right)$ by \prettyref{lem:bqToqOrder}.
From \prettyref{cor:linearCombinationOfFirstnRows} any $\mathbf{q}\in\left\langle \left(\mathbf{F},\sigma,\vec{s}\right)\right\rangle _{2\delta}$
is a linear combination of $\mathbf{S}^{\left(12\right)}$ and hence
is also a linear combination of $\mathbf{S}_{I}^{\left(12\right)}$. 
\end{proof}
To extract $\mathbf{S}_{I}^{\left(12\right)}$ from $\mathbf{S}^{\left(12\right)}$,
note that doing $\vec{s}$-column reduction on $\mathbf{S}^{\left(12\right)}$
is equivalent to the more familiar problem of doing column reduction
on $x^{\vec{s}}\mathbf{S}^{\left(12\right)}$. As $\mathbf{S}^{\left(12\right)}$
$\vec{s}$-column reduces to \textbf{$\mathbf{S}_{I}^{\left(12\right)}$},
this corresponds to determining the column rank profile of the\emph{
leading column coefficient matrix }of\emph{ }\textbf{$x^{\vec{s}}\mathbf{S}^{\left(12\right)}$}%
\begin{comment}
\emph{ }$S^{\left(12\right)}=\lcoeff(x^{\vec{s}}\cdot\mathbf{S}^{\left(12\right)})$ 
\end{comment}
\emph{. }Recall that the leading column coefficient matrix of a matrix
$\mathbf{A}=\left[\mathbf{a}_{1},\dots,\mathbf{a}_{k}\right]$ used
for column reduction is\emph{ 
\begin{eqnarray*}
\lcoeff\left(\mathbf{A}\right) & = & \left[\lcoeff\left(\mathbf{a}_{1}\right),\dots,\lcoeff\left(\mathbf{a}_{k}\right)\right]\\
 & = & \left[\coeff\left(\mathbf{a}_{1},\deg\left(\mathbf{a}_{1}\right)\right),\dots,\coeff\left(\mathbf{a}_{k},\deg\left(\mathbf{a}_{k}\right)\right)\right].
\end{eqnarray*}
 }The column rank profile of $\lcoeff(x^{\vec{s}}\mathbf{S}^{\left(12\right)})$
can be determined by (the transposed version of) LSP factorization
\citep{IbarraMH82}, which factorizes $\lcoeff(x^{\vec{s}}\mathbf{S}^{\left(12\right)})=PSU$
as the product of a permutation matrix $P$, a matrix $S$ with its
nonzero columns forming a lower triangular submatrix, and an upper
triangular matrix $U$ with $1$'s on the diagonal. The indices, $I$,
of the nonzero columns of $S$ then give $\mathbf{S}_{I}^{\left(12\right)}$
in $\mathbf{S}^{\left(12\right)}$. 
\begin{thm}
\label{thm:mainTheorem}Let $\bar{\mathbf{S}}=[\bar{\mathbf{S}}^{\left(12\right)},\bar{\mathbf{S}}^{\left(3\right)}]$
be a $(\mathbf{F}',\vec{\omega},\vec{s'})$-basis, with $\deg_{\vec{s'}}\bar{\mathbf{S}}^{\left(12\right)}\le2\delta$
and $\deg_{\vec{s'}}\bar{\mathbf{S}}^{\left(3\right)}\ge2\delta+1$,
and $\mathbf{S},\mathbf{S}^{\left(12\right)},\mathbf{S}^{\left(3\right)}$
the first $n$ rows of $\bar{\mathbf{S}},\bar{\mathbf{S}}^{\left(12\right)},\bar{\mathbf{S}}^{\left(3\right)}$,
respectively. If $I$ is the column rank profile of $\mathbf{S}^{\left(12\right)}$,
then the submatrix \textbf{ $[\mathbf{S}_{I}^{\left(12\right)},\mathbf{S}^{\left(3\right)}]$
}of $\mathbf{S}$ is a $\left(\mathbf{F},\sigma,\vec{s}\right)$-basis. \end{thm}
\begin{proof}
By \prettyref{lem:bqToqOrder}, $\mathbf{S}$ has order $\left(\mathbf{F},\sigma\right)$,
and so $[\mathbf{S}_{I}^{\left(12\right)},\mathbf{S}^{\left(3\right)}]$
also has order $\left(\mathbf{F},\sigma\right)$. By \prettyref{cor:linearCombinationOfFirstnRows},
any $\mathbf{q}\in\left\langle \left(\mathbf{F},\sigma,\vec{s}\right)\right\rangle $
is a linear combination of the columns of $\mathbf{S}$, and so $\mathbf{q}$
is also a linear combination of the columns of $[\mathbf{S}_{I}^{\left(12\right)},\mathbf{S}^{\left(3\right)}]$.
It only remains to show that $[\mathbf{S}_{I}^{\left(12\right)},\mathbf{S}^{\left(3\right)}]$
is $\vec{s}$-column reduced.

Let $\mathbf{P}$ be a $\left(\mathbf{F},\sigma,\vec{s}\right)$-basis
and $\mathbf{\bar{\mathbf{T}}}$ be the $(\mathbf{F}',\vec{\omega},\vec{s'})$-basis
constructed from $\mathbf{P}$ as in \prettyref{lem:FtoAbasis}. Let
$\bar{\mathbf{T}}^{\left(3\right)}$ be the columns of $\bar{\mathbf{T}}$
with $\vec{s'}$-degrees greater than $2\delta,$ and $\mathbf{P}^{\left(3\right)}$
be the columns of $\mathbf{P}$ with $\vec{s}$-degrees greater than
$2\delta.$ Assume without loss of generality that $\mathbf{S},$
$\mathbf{P}$, and $\bar{\mathbf{T}}$ have their columns sorted according
to their $\vec{s}$-degrees and $\vec{s'}$-degrees, respectively.
Then $\deg_{\vec{s}}\mathbf{S}^{\left(3\right)}\le\deg_{\vec{s'}}\bar{\mathbf{S}}^{\left(3\right)}=\deg_{\vec{s'}}\bar{\mathbf{T}}^{\left(3\right)}=\deg_{\vec{s}}\mathbf{P}^{\left(3\right)}$.
Combining this with the $\vec{s}$-minimality of $\mathbf{S}_{I}^{\left(12\right)}$
from \prettyref{lem:2deltaBasis}, it follows that $\deg_{\vec{s}}[\mathbf{S}_{I}^{\left(12\right)},\mathbf{S}^{\left(3\right)}]\le\deg_{\vec{s}}\mathbf{P}$.
This combined with the fact that $[\mathbf{S}_{I}^{\left(12\right)},\mathbf{S}^{\left(3\right)}]$
still generates $\left\langle \left(\mathbf{F},\sigma,\vec{s}\right)\right\rangle $
implies that $\deg_{\vec{s}}[\mathbf{S}_{I}^{\left(12\right)},\mathbf{S}^{\left(3\right)}]=\deg_{\vec{s}}\mathbf{P}$.
Therefore, $[\mathbf{S}_{I}^{\left(12\right)},\mathbf{S}^{\left(3\right)}]$
is a $\left(\mathbf{F},\sigma,\vec{s}\right)$-basis. \end{proof}
\begin{cor}
\label{cor:extractingFbasisFromGbasis}Let $\bar{\mathbf{S}}$ be
a $(\mathbf{F}',\vec{\omega},\vec{s'})$-basis with its columns sorted
in an increasing order of their $\vec{s'}$ degrees, and $\mathbf{S}$
the first $n$ rows of $\bar{\mathbf{S}}$. If $J$ is the column
rank profile of $\lcoeff(x^{\vec{s}}\mathbf{S})$, then the submatrix
$\mathbf{S}_{J}$ of $\mathbf{S}$ indexed by $J$ is a $\left(\mathbf{F},\sigma,\vec{s}\right)$-basis.\end{cor}
\begin{proof}
This follows directly from \prettyref{thm:mainTheorem}. 
\end{proof}
This rank profile $J$ can be determined by LSP factorization on $\lcoeff(x^{\vec{s}}\cdot\mathbf{S}^{\left(12\right)})$.
%as discussed before.

\begin{example}
\label{exm:auxiliaryTransformation}For the problem in \prettyref{exm:StorjohannTransformation},
$\check{\mathbf{F}}$ is given by{\scriptsize{
\[
\left[{\begin{array}{cccccr}
x+x^{2}+x^{3}+x^{4}+x^{5}+x^{6} & \ 1+x+x^{5}+x^{6}+x^{7}\  & 1+x^{2}+x^{6}+x^{7} & \ 1+x+x^{3}+x^{7}\  & 0\  & 0\\
1+x+x^{2}+x^{3} & x^{3} & 1+x^{2}+x^{3} & x & 1\  & 0\\
1+x+x^{2} & x+x^{2}+x^{3} & 1+x+x^{2}+x^{3} & x^{3} & 0\  & 1
\end{array}}\right]
\]
}}{\scriptsize \par}

and a $\left(\mathbf{F}',\left[8,4,4\right],\vec{0}\right)$-basis
is given as 
\[
\left[\begin{array}{cccc|cc}
~1~ & x & 1 & x^{2} & x^{2}+x^{4} & 1+x^{2}+x^{3}+x^{4}\\
0 & 1 & x^{2}+x^{3} & 0 & x^{3} & 0\\
1 & 1+x & x & x^{3}+x^{4} & 0 & x+x^{2}+x^{3}\\
1 & 0 & 0 & 0 & 0 & 0\\
\hline 0 & 1 & 1+x^{2} & x^{2} & x^{2}+x^{3} & 1+x^{2}+x^{3}+x^{4}\\
0 & 1 & 1 & x^{2}+x^{4} & x^{2}+x^{3} & 1+x^{3}
\end{array}\right].
\]
 Column reduction on the top 4 rows gives the top left $4\times4$
submatrix, which is a \textbf{$(\mathbf{F},8,\vec{0})$}-basis. 
\end{example}
The following two lemmas verify Storjohann's result in the case of
degree parameter $\delta=\sigma/4$. More specifically, we show that
the matrix of the top $n$ rows of a $(\bar{\mathbf{F}},2\delta,\vec{s'})_{\delta-1}$-basis
is a $\left(\mathbf{F},\sigma,\vec{s}\right)_{\delta-1}$-basis, with
the transformed input matrix 
\begin{equation}
\bar{\mathbf{F}}=\left[\begin{array}{l|cc}
\mathbf{F}_{0}+\mathbf{F}_{1}x^{\delta} & \mathbf{0} & \mathbf{0}\\
\hline \mathbf{F}_{1}+\mathbf{F}_{2}x^{\delta} & \mathbf{I}_{m} & \mathbf{0}\\
\mathbf{F}_{2}+\mathbf{F}_{3}x^{\delta} & \mathbf{0} & ~\mathbf{I}_{m}
\end{array}\right]\equiv\mathbf{F}'\mod x^{2\delta}.\label{eq:storjohannTransformation4parts}
\end{equation}

\begin{lem}
\label{lem:A_delta-1Form}If $\bar{\mathbf{q}}\in\langle(\bar{\mathbf{F}},2\delta,\vec{s'})\rangle_{\delta-1}$
and $\mathbf{q}$ denotes the first $n$ entries of $\bar{\mathbf{q}}$,
then $\bar{\mathbf{q}}$ must have the form 
\[
\bar{\mathbf{q}}=\mathbf{B}\mathbf{q}=\left[\begin{array}{r}
\mathbf{q}\\
x^{-\delta}\mathbf{F}_{0}\cdot\mathbf{q}\\
x^{-2\delta}\left(\mathbf{F}_{0}+\mathbf{F}_{1}x^{\delta}\right)\cdot\mathbf{q}
\end{array}\right]
\]
 and $\mathbf{q}\in\left\langle \left(\mathbf{F},\sigma,\vec{s}\right)\right\rangle _{\delta-1}$.\end{lem}
\begin{proof}
Let $\mathbf{q},\mathbf{q}_{2},\mathbf{q}_{3}$ consist of the top
$n$ entries, middle $m$ entries, and bottom $m$ entries, respectively,
of $\mathbf{\bar{\mathbf{q}}}$ so that 
\begin{align}
\bar{\mathbf{F}}\bar{\mathbf{q}} & \equiv\left[\begin{array}{r}
\mathbf{F}_{0}\mathbf{q}+x^{\delta}\mathbf{F}_{1}\mathbf{q}\\
\mathbf{q}_{2}+\mathbf{F}_{1}\mathbf{q}+x^{\delta}\mathbf{F}_{2}\mathbf{q}\\
\mathbf{q}_{3}+\mathbf{F}_{2}\mathbf{q}+x^{\delta}\mathbf{F}_{3}\mathbf{q}
\end{array}\right]\equiv\mathbf{0}\mod x^{2\delta}.\label{eq:Aq}
\end{align}
 From the first and second block rows, we get $\mathbf{F}_{0}\mathbf{q}+x^{\delta}\mathbf{F}_{1}\mathbf{q}\equiv\mathbf{0}\mod x^{2\delta}$
and $\mathbf{q}_{2}+\mathbf{F}_{1}\mathbf{q}\equiv\mathbf{0}\mod x^{\delta}$,
which implies 
\begin{equation}
\mathbf{F}_{0}\mathbf{q}\equiv x^{\delta}\mathbf{q}_{2}\mod x^{2\delta}.\label{eq:q1q2}
\end{equation}
 Similarly, from the second and third rows, we get $\mathbf{q}_{2}+\mathbf{F}_{1}\mathbf{q}+x^{\delta}\mathbf{F}_{2}\mathbf{q}\equiv\mathbf{0}\mod x^{2\delta}$
and $\mathbf{q}_{3}+\mathbf{F}_{2}\mathbf{q}\equiv\mathbf{0}\mod x^{\delta}$,
which implies $\mathbf{q}_{2}+\mathbf{F}_{1}\mathbf{q}\equiv x^{\delta}\mathbf{q}_{3}\mod x^{2\delta}$.

Since $\deg\mathbf{q}\le\deg_{\vec{s}}\mathbf{q}=\delta-1$, we have
$\deg\mathbf{F}_{0}\mathbf{q}\le2\delta-2$, hence from \prettyref{eq:q1q2}
$\deg\mathbf{q}_{2}\le\delta-2$ and $\mathbf{q}_{2}x^{\delta}=\mathbf{F}_{0}\mathbf{q}$.
Similarly, $\deg\mathbf{q}_{3}\le\delta-2$ and $\mathbf{q}_{3}x^{2\delta}=\mathbf{q}_{2}x^{\delta}+\mathbf{F}_{1}\mathbf{q}x^{\delta}=\mathbf{F}_{0}\mathbf{q}+\mathbf{F}_{1}\mathbf{q}x^{\delta}$.
Substituting this to $\mathbf{F}\mathbf{q}=(\mathbf{F}_{0}\mathbf{q}+\mathbf{F}_{1}\mathbf{q}x^{\delta})+(\mathbf{F}_{2}\mathbf{q}x^{2\delta}+\mathbf{F}_{3}\mathbf{q}x^{3\delta})$,
we get $\mathbf{F}\mathbf{q}=\mathbf{q}_{3}x^{2\delta}+(\mathbf{F}_{2}\mathbf{q}x^{2\delta}+\mathbf{F}_{3}\mathbf{q}x^{3\delta})\equiv\mathbf{0}\mod x^{4\delta}$
using the bottom block row of \prettyref{eq:Aq}.\end{proof}
\begin{lem}
\label{lem:delta-1Basis} If $\bar{\mathbf{S}}_{\delta-1}$ is a $(\bar{\mathbf{F}},2\delta,\vec{s'})_{\delta-1}$-basis,
then the matrix of its first $n$ rows, $\mathbf{S}_{\delta-1}$,
is a $\left(\mathbf{F},\sigma,\vec{s}\right)_{\delta-1}$-basis.\end{lem}
\begin{proof}
By \prettyref{lem:A_delta-1Form}, $\mathbf{S}_{\delta-1}$ has order
$\left(\mathbf{F},\sigma\right)$. Following Lemmas \ref{lem:qToBqOrder}
and \ref{lem:qToBqDegree} and Corollaries \ref{cor:sToBs} and \ref{cor:linearCombinationOfFirstnRows}
(replacing $\vec{\omega}$ by $2\delta$), we conclude that any $\mathbf{q}\in\left\langle \left(\mathbf{F},\sigma,\vec{s}\right)\right\rangle _{\delta-1}$
is a linear combination of the columns of $\mathbf{S}_{\delta-1}$.
In addition, since $\bar{\mathbf{S}}_{\delta-1}=\mathbf{B}\mathbf{S}_{\delta-1}$
by \prettyref{lem:A_delta-1Form}, and the columns of $\mathbf{S}_{\delta-1}$
are in $\left\langle \left(\mathbf{F},\sigma,\vec{s}\right)\right\rangle _{\delta-1}$,
it follows from \prettyref{lem:qToBqDegree} that $\mathbf{S}_{\delta-1}$
determines the $\vec{s'}$-column degrees of $\bar{\mathbf{S}}_{\delta-1}$.
Hence $\bar{\mathbf{S}}_{\delta-1}$ $\vec{s'}$-column reduced implies
that $\mathbf{S}_{\delta-1}$ is $\vec{s}$-column reduced. 
\end{proof}

\subsection{\label{sub:generalTransform}More General Results}

Let us now consider an immediate extension of the results in the previous
subsection. Suppose that instead of a $\left(\mathbf{F},\sigma,\vec{s}\right)$-basis
we now want to compute a $(\bar{\mathbf{F}}^{\left(i\right)},2\delta^{\left(i\right)},\vec{s}^{\left(i\right)})$-basis
with a Storjohann transformed input matrix 
\[
\bar{\mathbf{F}}^{\left(i\right)}=\left[\begin{array}{c|cccc}
\mathbf{F}_{0}+\mathbf{F}_{1}x^{\delta^{\left(i\right)}} & \mathbf{0}_{m} & \cdots & \mathbf{\cdots} & \mathbf{0}_{m}\\
\hline \mathbf{F}_{1}+\mathbf{F}_{2}x^{\delta^{\left(i\right)}} & \mathbf{I}_{m}\\
\mathbf{F}_{2}+\mathbf{F}_{3}x^{\delta^{\left(i\right)}} &  & \mathbf{I}_{m}\\
\vdots &  &  & \ddots\\
\mathbf{F}_{l^{\left(i\right)}-1}+\mathbf{F}_{l^{\left(i\right)}}x^{\delta^{\left(i\right)}} &  &  &  & \mathbf{I}_{m}
\end{array}\right]_{ml^{(i)}\times(n+m(l^{(i)}-1))}
\]
 made with degree parameter $\delta^{\left(i\right)}=2^{i}d$ for
some integer $i$ between $2$%
\begin{comment}
the base case is $i=1$ and problem is not to be subdivided 
\end{comment}
{} and $\log\left(\sigma/d\right)-1$, and a shift $\vec{s}^{\left(i\right)}=[\vec{s},0,\dots,0]$
(with $m(l^{\left(i\right)}-1)$ 0's), where $l^{\left(i\right)}=\sigma/\delta^{\left(i\right)}-1$
is the number of block rows%
\footnote{Recall that $d=m\sigma/n$ is the average degree of the input matrix
$\mathbf{F}$ if we treat $\mathbf{F}$ as a square $n\times n$ matrix.
Also, $i$ starts at $2$ because $i=1$ is our base case in the computation
of an order basis, which may become more clear in the next section.
The base case can be computed efficiently using the method of Giorgi
et al. \citeyearpar{Giorgi2003} directly and does not require the
transformation discussed in this section.%
}. To apply a transformation analogous to \prettyref{eq:extendedSTransformTop},
we write each $\mathbf{F}_{j}=\mathbf{F}_{j0}+\mathbf{F}_{j1}\delta^{\left(i-1\right)}$
and set 
\begin{equation}
\mathbf{F}'^{\left(i\right)}=\left[\begin{array}{l|c}
\mathbf{F}_{00}+\mathbf{F}_{01}x^{\delta^{\left(i-1\right)}}+\mathbf{F}_{10}x^{2\delta^{\left(i-1\right)}}+\mathbf{F}_{11}x^{3\delta^{\left(i-1\right)}} & ~~\mathbf{0}~~\\
\hline \mathbf{F}_{01}+\mathbf{F}_{10}x^{\delta^{\left(i-1\right)}}\\
\mathbf{F}_{10}+\mathbf{F}_{11}x^{\delta^{\left(i-1\right)}}+\mathbf{F}_{20}x^{2\delta^{\left(i-1\right)}}+\mathbf{F}_{21}x^{3\delta^{\left(i-1\right)}}\\
\mathbf{F}_{11}+\mathbf{F}_{20}x^{\delta^{\left(i-1\right)}}\\
\vdots & ~~\mathbf{I}~~\\
\mathbf{F}_{\left(l^{\left(i\right)}-1\right)0}+\mathbf{F}_{\left(l^{\left(i\right)}-1\right)1}x^{\delta^{\left(i-1\right)}}+\mathbf{F}_{l^{\left(i\right)}0}x^{2\delta^{\left(i-1\right)}}+\mathbf{F}_{l^{\left(i\right)}1}x^{3\delta^{\left(i-1\right)}}\\
\mathbf{F}_{\left(l^{\left(i\right)}-1\right)1}+\mathbf{F}_{l^{\left(i\right)}0}x^{\delta^{\left(i-1\right)}}\\
\mathbf{F}_{l^{\left(i\right)}0}+\mathbf{F}_{l^{\left(i\right)}1}x^{\delta^{\left(i-1\right)}}
\end{array}\right],\label{eq:extendedStorjohannTransform}
\end{equation}
 %
\begin{comment}
This is not ideal, but no better idea. 
\end{comment}
{} and $\vec{\omega}^{\left(i\right)}=\left[\left[[2\delta^{\left(i\right)}]^{m},[\delta^{\left(i\right)}]^{m}\right]^{l^{\left(i\right)}},[\delta^{\left(i\right)}]^{m}\right]$,
where $\left[\circ\right]^{k}$ represents $\circ$ repeated $k$
times%
\begin{comment}
Not sure if using this notation is a good thing to do, but it saves
space and makes presentation easier 
\end{comment}
. The order entries $2\delta^{\left(i\right)}$, $\delta^{\left(i\right)}$
in $\vec{\omega}^{\left(i\right)}$ correspond to the degree $2\delta^{\left(i\right)}-1$,
degree $\delta^{\left(i\right)}-1$ rows in $\mathbf{F}'^{\left(i\right)}$
respectively. Let 
\[
\mathbf{E}^{\left(i\right)}=\left[\begin{array}{c||cc|cc|cc|cc||cc}
\mathbf{I}_{n} &  &  &  &  &  &  &  &  & \mathbf{0}_{n\times m} & \mathbf{0}_{n\times m}\\
\hline\hline  & \mathbf{0}_{m} & \mathbf{I}_{m} &  &  &  &  &  &  & \ \\
\hline  &  &  & \mathbf{0}_{m} & \mathbf{I}_{m} &  &  &  &  & \ \\
\hline  &  &  &  &  & \ddots & \ddots &  &  & \ \\
\hline  &  &  &  &  &  &  & \mathbf{0}_{m} & \mathbf{I}_{m}
\end{array}\right]
\]
 with $l^{\left(i\right)}-1$ blocks of $\left[\mathbf{0}_{m},\mathbf{I}_{m}\right]$
and hence an overall dimension of $(n+m(l^{\left(i\right)}-1))\times(n+m(l^{\left(i-1\right)}-1))$.
Thus $\mathbf{E}^{\left(i\right)}\mathbf{M}$ picks out from $\mathbf{M}$
the first $n$ rows and the even block rows from the remaining rows
except the last block row for a matrix $\mathbf{M}$ with $n+m(l^{\left(i-1\right)}-1)$
rows. In particular, if $i=\log\left(n/m\right)-1$, then $(\mathbf{F}'^{\left(i\right)},\vec{\omega}^{\left(i\right)},\vec{s}^{\left(i-1\right)})=(\mathbf{F}',\vec{\omega},\vec{s'})$,
which for $d=m\sigma/n$ gives the problem considered earlier in \prettyref{sub:particularCase},
and $\mathbf{E}^{\left(i\right)}=\left[\mathbf{I}_{n},\mathbf{0}_{n\times m},\mathbf{0}_{n\times m}\right]$
is used to select the top $n$ rows of a $(\mathbf{F}',\vec{\omega},\vec{s'})$-basis
for a $\left(\mathbf{F},\sigma,\vec{s}\right)$-basis to be extracted.

We can now state the analog of \prettyref{cor:extractingFbasisFromGbasis}: 
\begin{thm}
\label{thm:extractingOrderBasis}Let $\mathbf{S}'^{\left(i\right)}$
be a $(\mathbf{F}'^{\left(i\right)},\vec{\omega}^{\left(i\right)},\vec{s}^{\left(i-1\right)})$-basis
with its columns sorted in an increasing order of their $\vec{s}^{\left(i-1\right)}$
degrees. Let $\hat{\mathbf{S}}^{\left(i\right)}=\mathbf{E}^{\left(i\right)}\mathbf{S}'^{\left(i\right)}$.
Let $J$ be the column rank profile of $\lcoeff(x^{\vec{s}^{\left(i\right)}}\hat{\mathbf{S}}^{\left(i\right)})$.
Then $\hat{\mathbf{S}}_{J}^{\left(i\right)}$ is a $(\bar{\mathbf{F}}^{\left(i\right)},2\delta^{\left(i\right)},\vec{s}^{\left(i\right)})$-basis.\end{thm}
\begin{proof}
One can follow the same arguments used before from \prettyref{lem:qToBqOrder}
to \prettyref{cor:extractingFbasisFromGbasis}. Alternatively, this
can be derived from \prettyref{cor:extractingFbasisFromGbasis} by
noticing the redundant block rows that can be disregarded after applying
transformation \prettyref{eq:extendedSTransformTop} directly to the
input matrix $\bar{\mathbf{F}}^{\left(i\right)}$. 
\end{proof}
\prettyref{lem:delta-1Basis} can also be extended in the same way
to capture Storjohann's transformation with more general degree parameters: 
\begin{lem}
\label{lem:linkStorjohanTransform}If $\bar{\mathbf{P}}_{1}^{\left(i-1\right)}$
is a $(\bar{\mathbf{F}}^{\left(i-1\right)},2\delta^{\left(i-1\right)},\vec{s}^{\left(i-1\right)})_{\delta^{\left(i-1\right)}-1}$-basis,
then $\mathbf{E}^{\left(i\right)}\bar{\mathbf{P}}_{1}^{\left(i-1\right)}$
is a $(\bar{\mathbf{F}}^{\left(i\right)},2\delta^{\left(i\right)},\vec{s}^{\left(i\right)})_{\delta^{\left(i-1\right)}-1}$-basis
and the matrix consists of the top $n$ rows of $\bar{\mathbf{P}}_{1}^{\left(i-1\right)}$
is a $(\mathbf{F},\sigma,\vec{s})_{\delta^{\left(i-1\right)}-1}$-basis.\end{lem}
\begin{proof}
Again, this can be justified as done in \prettyref{lem:delta-1Basis}.
Alternatively, one can apply Storjohann's transformation with degree
parameter $\delta^{\left(i-1\right)}$ to $\bar{\mathbf{F}}^{\left(i\right)}$
as in \prettyref{eq:storjohannTransformation4parts}. The lemma then
follows from \prettyref{lem:delta-1Basis} after noticing the redundant
block rows that can be disregarded. 
\end{proof}
Notice that if $i=\log\left(n/m\right)-1,$ then \prettyref{thm:extractingOrderBasis}
and \prettyref{lem:linkStorjohanTransform} specialize to \prettyref{cor:extractingFbasisFromGbasis}
and \prettyref{lem:delta-1Basis}. 
