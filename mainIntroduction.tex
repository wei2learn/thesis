
\chapter{Introduction}

In this thesis, we present efficient algorithms for a number of problems
involving matrices of univariate power series or polynomials over
a field. The first problem we consider is the computation of order
bases, which can be viewed as the most fundamental among all the problems
considered in this thesis, since order basis computation is used by
the algorithms for all other problems. The second problem, minimal
kernel basis computation, provides another essential tool used by
the algorithms for the remaining problems, including the computation
of matrix inverse, determinant, column basis, unimodular completion,
and Hermite normal form. The algorithm for kernel basis computation
also immediately allows us to solve linear systems. The algorithm
for column basis also immediately allows us to compute matrix GCDs,
column reduced forms and Popov normal forms for matrices of any dimension. 



Let us first look at order bases and kernel bases in more detail.

Let $\mathbf{F}\in\mathbb{K}\left[\left[x\right]\right]^{m\times n}$
be a matrix of power series over a field $\mathbb{K}$. Given a nonnegative
integer $\sigma$, we say a vector $\mathbf{p}\in\mathbb{K}\left[x\right]^{n\times1}$
of polynomials gives an \emph{order} $\sigma$ approximation of $\mathbf{F}$,
or $\mathbf{p}$ has order $\left(\mathbf{F},\sigma\right)$, if 
\[
\mathbf{F}\cdot\mathbf{p}\equiv\mathbf{0}\mod x^{\sigma},
\]
 that is, the first $\sigma$ terms of $\mathbf{F}\cdot\mathbf{p}$
are zero. Historically such problems date back to their use in Hermite's
proof of the transcendence of $e$ in 1873. In 1893 Pad�, a student
of Hermite, formalized the concepts introduced by Hermite and defined
what is now known as Hermite-Pad� approximants (where $m=1$), Pad�
approximants (where $m=1,n=2$) and simultaneous Pad� approximants
(where $\mathbf{F}$ has a special structure). Such rational approximations
also specified degree constraints on the polynomials $\mathbf{p}$
and had their order conditions related to these degree constraints.
Additional order problems include vector and matrix versions of rational
approximation, partial realizations of matrix sequences and vector
rational reconstruction just to name a few (cf. the references in
\citet{BL1997}). As an example, the factorization of differential
operators algorithm of \citet{vanHoeij} makes use of vector Hermite-Pad�
approximation to reconstruct differential factorizations over rational
functions from factorizations of differential operators over power
series domains.

The set of all such order $\left(\mathbf{F},\sigma\right)$ approximations
forms a module over $\mathbb{K}\left[x\right]$. An {\em order basis}
- or minimal approximant basis or $\sigma$-basis - is a basis of
this module having a type of minimal degree property (called reduced
order basis in \citep{BL1997}). The minimal degree property parameterizes
solutions to an order problem by the degrees of the columns of the
order basis. In the case of rational approximation, order bases can
be viewed as a natural generalization of the Pad� table of a power
series \citep{gravesmorris} since they are able to describe {\em
all} solutions to such problems given particular degree bounds. They
can even be used to show the well known block structure of the Pad�
and related Rational Interpolation tables \citep{BL1997}. Order bases
are used in such diverse applications as the inversion of structured
matrices \citep{La92}, normal forms of matrix polynomials \citep{BLV:1999,BLV:jsc06},
and other important problems in matrix polynomial arithmetic including
matrix inversion, determinant and kernel computation \citep{Giorgi2003,storjohann-villard:2005}.
\begin{comment}
In our case we also allow the minimal degree property to include a
shift $\vec{s}$. Such a shift is important, for example, for matrix
normal form problems \citep{BLV:1999,BLV:jsc06}.
\end{comment}


Kernel bases are closely related to order bases.

For a matrix of polynomials $\mathbf{F}\in\mathbb{K}\left[x\right]^{m\times n}$
with rank $r$. The set

\[
\left\{ \mathbf{p}\in\mathbb{K}\left[x\right]^{n}~|~\mathbf{F}\mathbf{p}=0\right\} ,
\]
 is the (right) kernel of $\mathbf{F}$, which is also a $\mathbb{K}[x]$-module.
It can be generated by a basis -- a kernel basis of $\mathbf{F}$,
that can be represented as a matrix in $\mathbb{K}\left[x\right]^{n\times\left(n-r\right)}$,
with the columns being the basis elements.

Kernel bases of polynomial matrices appear in a large number of applications,
being first used as an algebraic formalism in the area of control
theory \citep{Kucera:1979}. For example, in linear system theory
if a system is represented by a transfer function given in terms of
a left coprime matrix fraction decomposition $\mathbf{T}=\mathbf{D}_{\ell}^{-1}\mathbf{N}_{\ell}$,
with $\mathbf{D}_{\ell}$ and $\mathbf{N}_{\ell}$ polynomial matrices,
then one often wants to find a right coprime matrix fraction representation
$\mathbf{T}=\mathbf{N}_{r}\mathbf{D}_{r}^{-1}$ with $\mathbf{D}_{r}$
and $\mathbf{N}_{r}$ polynomial matrices of appropriate dimensions
\citep{kailath:1980}. This is equivalent to the kernel basis computation
\begin{equation}
[\mathbf{D}_{\ell}~~-\mathbf{N}_{\ell}]\left[\begin{array}{c}
\mathbf{N}_{r}\\
\mathbf{D}_{r}
\end{array}\right]=0.\label{rightfactor}
\end{equation}
 Solving and determining fundamental properties of the basic matrix
equation $\mathbf{A}\mathbf{Z}=\mathbf{B}$ where $\mathbf{A}$ and
$\mathbf{B}$ have polynomial elements can be determined by finding
a complete description (that is, a basis) of the kernel of $[\mathbf{A},-\mathbf{B}]$.
Other examples of the use of kernels and their bases include fault
diagnostics \citep{frisk:phd} and column reduction of matrix polynomials
\citep{BVP:1988,neven:1993}.

% a bit more specific description of our problem.


In most applications one is interested in finding a {\em minimal
kernel basis} of $\mathbf{F}$ in $\mathbb{K}\left[x\right]^{n}$
\citep{forney:1975}. A kernel basis $\mathbf{N}$ of $\mathbf{F}$
is said to be minimal if it has the minimal possible column degrees
among all right kernel bases. This is also often referred to as a
{\em minimal polynomial basis}. Examples where minimality are needed
include the right coprime matrix factorization problem and the problem
of column reducing a polynomial matrix. As an example, finding a basis
for the kernel corresponding to the right matrix fraction problem
(\ref{rightfactor}) finds a matrix fraction while a minimal kernel
basis finds such a fraction in reduced form having a minimal column
degree denominator (needed for example in minimal partial realization
problems). 


\section{Shifted Degrees and Their Sums}

The standard way to measure the size of a matrix is to use its dimension
and its degree. A major complication in many polynomial matrix computation
problems is that the degrees of the intermediate results or the output
can be much larger than the input. This seems to prevent the problems
to be computed efficiently, since the size of the intermediate results
and the size of the output provides lower bounds on the computational
cost of any algorithm.  But it turns out that the degree just may
not be the best choice to be used in these computations. Instead
of the standard degrees, we use the more general shifted degrees to
guide the computations, and use the sum of the shifted degrees to
measure the size of polynomial matrices. We will see that the shifted
degree is in fact a more natural choice, as it guides the computation
so that the sizes of the output and the intermediate results are indeed
bounded by the size as the input for all these problems.  Closely
examination of the shifted degrees reveals new structures of the problems
in this thesis, leading to better understanding of the problems, and
allowing the development of simple and efficient algorithms. 

\begin{comment}
Although the degree is simple and seems natural, 

In such cases, one may wonder why the extra space is needed if we
are just transforming the original input and not creating extra information
to store. One possibilities is that the output contains some redundant
information that makes it easier to use in someway. Another possibility
is that the measurement is flawed, which turns out to be the case
for all the problems we consider in this thesis. By using a better
criterion to measure the sizes, we will see that the sizes of the
output and the intermediate results are indeed bounded by the size
as the input for all these problems. The importance of the criterion
for measuring the sizes cannot be understated, as it provides clean,
elegant structures to the previous messy problems and guides us efficiently
to the output. 

In this thesis, we use the sum of the shifted degrees to measure the
size of polynomial matrices. We introduce the shifted column degrees
in this section. 
\end{comment}


For a column vector $\mathbf{p}=\left[p_{1},\dots,p_{n}\right]^{T}$
of univariate polynomials over a field $\mathbb{K}$, its column degree
is just the maximum of the degrees of the entries of $\mathbf{p}$,
that is, 
\[
\cdeg\mathbf{p}=\max_{1\le i\le n}\deg p_{i}.
\]
The \emph{shifted column degree} generalizes this standard column
degree by taking the maximum after shifting the degrees by a given
integer vector that is known as a \emph{shift}. More specifically,
the shifted column degree of $\mathbf{p}$ with respect to a shift
$\vec{s}=\left[s_{1},\dots,s_{n}\right]\in\mathbb{Z}^{n}$, or the
\emph{$\vec{s}$-column degree} of $\mathbf{p}$ is

\[
\cdeg_{\vec{s}}\mathbf{p}=\max_{1\le i\le n}[\deg p^{\left(i\right)}+s_{i}]=\deg(x^{\vec{s}}\cdot\mathbf{p}),
\]
 where 
\[
x^{\vec{s}}=\diag\left(\left[x^{s_{1}},x^{s_{2}},\dots,x^{s_{n}}\right]\right)=\begin{bmatrix}x^{s_{1}}\\
 & x^{s_{2}}\\
 &  & \ddots\\
 &  &  & x^{s_{1}}
\end{bmatrix}.
\]
For a matrix $\mathbf{P}$, we use $\cdeg\mathbf{P}$ and $\cdeg_{\vec{s}}\mathbf{P}$
to denote respectively the list of its column degrees and the list
of its shifted $\vec{s}$-column degrees. When $\vec{s}=\left[0,\dots,0\right]$,
the shifted column degree specializes to the standard column degree.
The shifted row degree of a row vector \textbf{$\mathbf{q}$ }is defined
in the same way.
\[
\rdeg_{\vec{s}}\mathbf{q}=\max_{1\le i\le n}[\deg p^{\left(i\right)}+s_{i}]=\deg(\mathbf{q}\cdot x^{\vec{s}}).
\]


The shifted degrees have been used previously in polynomial matrix
computations and to generalize matrix normal forms \citep{BLV:jsc06}.
The shifted column degree is equivalent to the notion of \emph{defect}
commonly used in the literature. Our definition of $\vec{s}$-column
degree is a special case of the $\mathbf{H}$-degree from \citep{BL1997},
where in this case $\mathbf{H}=x^{\vec{s}}$.%
\begin{comment}
The shifted column degree of a column polynomial vector $\mathbf{p}$
with shift $\vec{s}=\left[s_{1},\dots,s_{n}\right]\in\mathbb{Z}^{n}$
is given by 
\[
\deg_{\vec{s}}\mathbf{p}=\max_{1\le i\le n}[\deg p^{\left(i\right)}+s_{i}]=\deg(x^{\vec{s}}\cdot\mathbf{p}).
\]
 We call this the \emph{$\vec{s}$-column degree}, or simply the \emph{$\vec{s}$-degree}
of $\mathbf{p}$. A shifted column degree defined this way is equivalent
to the notion of \emph{defect} commonly used in the literature. Our
definition of $\vec{s}$-degree is a special case of the $\mathbf{H}$-degree
from \citep{BL1997}, where in this case $\mathbf{H}=x^{\vec{s}}$.
As in the uniform shift case, we say a matrix is \emph{$\vec{s}$-column
reduced} or \emph{$\vec{s}$-reduced} if its $\vec{s}$-degrees cannot
be decreased by unimodular column operations. More precisely, if $\mathbf{P}$
is a $\vec{s}$-column reduced and $[d_{1},\dots,d_{n}]$ are the
$\vec{s}$-degrees of columns of $\mathbf{P}$ sorted in nondecreasing
order, then $[d_{1},\dots,d_{n}]$ is lexicographically minimal among
all matrices right equivalent to $\mathbf{P}$. Note that a matrix
$\mathbf{P}$ is $\vec{s}$-column reduced if and only if $x^{\vec{s}}\cdot\mathbf{P}$
is column reduced \citep{BLV:1999,BLV:jsc06}.
\end{comment}
{} 


\section{Order Basis Computation}

The first problem considered in this thesis is the efficient computation
of order basis. Algorithms for fast computation of order basis include
that of \citet{BeLa94} which converts the matrix problem into a vector
problem of higher order (which they called the Power Hermite-Pad�
problem). Their divide and conquer algorithm has complexity of $O^{\sim}(n^{2}m\sigma+nm^{2}\sigma)$
field operations. As usual, the soft-$O$ notation $O^{\sim}$ is
simply Big-$O$ with polylogarithmic factors $(\log(nm\sigma))^{O(1)}$
omitted. By working more directly on the input $m\times n$ input
matrix, \citet{Giorgi2003} give a divide and conquer method with
cost $O^{\sim}\left(n^{\omega}\sigma\right)$ arithmetic operations.
Their method is very efficient if $m$ is close to the size of $n$
but can be improved if $m$ is small.

In a novel construction, \citet{Storjohann:2006} effectively reverses
the approach of Beckermann and Labahn. Namely, rather than convert
a high dimension matrix order problem into a lower dimension vector
problem of higher order, Storjohann converts a low dimension problem
to a high dimension problem with lower order. For example, computing
an order basis for a $1\times n$ vector input $\mathbf{f}$ and order
$\sigma$ can be converted to a problem of order basis computation
with an $O\left(n\right)\times O\left(n\right)$ input matrix and
an order $O\left(\left\lceil \sigma/n\right\rceil \right)$. Combining
this conversion with the method of Giorgi et al. can then be used
effectively for problems with small row dimensions to achieve a cost
of $O^{\sim}\left(n^{\omega}\left\lceil m\sigma/n\right\rceil \right)$.

However, while order bases of the original problem can have degree
up to $\sigma$, the nature of Storjohann's conversion limits the
degree of an order basis of the converted problem to $O\left(\left\lceil m\sigma/n\right\rceil \right)$
in order to be computationally efficient. In other words, this approach
does not in general compute a complete order basis. Rather, in order
to achieve efficiency, it only computes a partial order basis containing
basis elements with degrees within $O\left(\left\lceil m\sigma/n\right\rceil \right)$,
referred to by Storjohann as a {\em minbasis}. Fast methods for
computing a minbasis are particularly useful for certain problems,
for example, in the case of inversion of structured block matrices
where one needs only precisely the minbasis \citep{La92}. However,
in other applications, such as those arising in polynomial matrix
arithmetic, one needs a complete basis which specifies all solutions
of a given order, not just those within a particular degree bound
(cf. \citet{BL1997}).

In \chapref{OrderBasis} we present algorithms which compute an entire
order basis with a cost of $O^{\sim}(n^{\omega-1}m\sigma)$ field
operations when $n\in O\left(m\sigma\right)$. The algorithms differ
depending on the nature of the degree shift required for the reduced
order basis. In the first case we use a transformation that can be
considered as an extension of Storjohann's transformation. This new
transformation provides a way to extend the results from one transformed
problem to another transformed problem of a higher degree. This enables
us to use an idea from the null space basis algorithm found in \citep{storjohann-villard:2005}
in order to achieve efficient computation. At each iteration, basis
elements within a specified degree bound are computed via a Storjohann
transformed problem. Then the partial result is used to simplify the
next Storjohann transformed problem of a higher degree, allowing basis
elements within a higher degree bound to be computed efficiently.
This is repeated until all basis elements are computed.

In order to compute an order basis efficiently, the first algorithm
requires that the degree shifts are balanced. In the case where the
shift is not balanced, the row degrees of the basis can also become
unbalanced in addition to the unbalanced column degrees. We give a
second algorithm that balances the high degree rows and uses $O^{\sim}(n^{\omega}\lceil m\sigma/n\rceil)$
field operations when the shift $\vec{s}$ is unbalanced but satisfies
the condition $\sum_{i=1}^{n}(\max(\vec{s})-\vec{s}_{i})\le m\sigma$.
This condition essentially allows us to locate the high degree unbalanced
rows that need to be balanced. %
\begin{comment}
In fact, every problem with any unbalanced shift can be reformulated
to a problem with a shift satisfying this condition if the degrees
of the resulting order basis is known. 
\end{comment}
{} The algorithm converts a problem of unbalanced shift to one with
balanced shift, based on a second idea from \citep{Storjohann:2006}.
Then the first algorithm is used to efficiently compute the elements
of an order basis whose shifted degrees exceed a specified parameter.
The problem is then reduced to one where we remove the computed elements.
This results in a new problem with smaller dimension and higher degree.
The same process is repeated again on this new problem in order to
compute the elements with the next highest shifted degrees.

At the end of \chapref{OrderBasis}, we discuss how the assumption
of $n\in O\left(m\sigma\right)$ can be removed while an order basis
can still be computed with a cost of $O^{\sim}(n^{\omega-1}m\sigma)$
field operations when the shifts are balanced.

Some results on order basis computation have appeared in \citep{za2009,ZL2012}.

\begin{comment}
The remaining paper is structured as follows. Basic definitions and
properties of order bases are given in the next section. \secref{transform}
provides an extension to Storjohann's transformation to allow higher
degree basis elements to be computed. Based on this new transformation,
\secref{Order-Basis-Computation} establishes a link between two Storjohann
transformed problems of different degrees, from which an recursive
method and then an iterative algorithm are derived. The time complexity
is analyzed in the next section. After this, \chapref{Unbalanced-Shift}
describes an algorithm which handles problems with a type of unbalanced
shift. This is followed by a conclusion along with a description for
topics for future research. 
\end{comment}



\section{Kernel Basis Computation}

We are interested in fast computation of minimal kernel bases and
shifted minimal kernel bases in exact environments. Historically computation
of a minimal kernel basis has made use of either matrix pencil or
resultant methods (often called a linearized approaches) or use of
elimination methods for matrix polynomials. Matrix pencil methods
convert a kernel basis computation problem to one of larger matrix
size but having polynomial degree one. In this case a minimal kernel
basis is determined from the computation of the Kronecker canonical
form, with efficient algorithms given by \citet{beelen:1988,varga:1994,Oara:1997}.
The cost of these algorithms is $O(m^{2}nd^{3})$. Resultant methods
convert the kernel basis computation of the matrix polynomial $\mathbf{F}$
into a block Toeplitz kernel problem with much higher dimension with
the resulting complexity again being high. In \citep{storjohann-villard:2005}
the authors give a randomized Las Vegas algorithm for computing a
set of $n-r$ linearly independent elements in the kernel of $\mathbf{F}$
with a cost of $O^{\sim}\left(nmr^{\omega-1}\right)$ where $O^{\sim}$
is the same as Big-$O$ but without log factors and where $\omega$
is the power of fast matrix multiplication. These linearly independent
elements do not in general form a basis for the kernel, as they generally
do not generate all the elements in the kernel. A set of any such
$n-r$ linearly independent elements only form a basis for the $\mathbb{F}\left(x\right)$-vector
space $\left\{ \mathbf{p}\in\mathbb{K}\left(x\right)^{n}~|~\mathbf{F}\mathbf{p}=0\right\} $
when the ring $\mathbb{K}\left[x\right]$ is extended to the field
$\mathbb{K}\left(x\right)$.



In \chapref{NullspaceBasis} we present a deterministic algorithm
for computing a minimal kernel basis with a cost of $O^{\sim}\left(n^{\omega-1}md\right)$
field operations in $\mathbb{K}$. The same algorithm can also compute
a $\vec{s}$-minimal kernel basis of $\mathbf{F}$ with a cost of
$O^{\sim}(n^{\omega}\rho/m)$ if the entries of $\vec{s}$ bound the
corresponding column degrees of $\mathbf{F}$, where $\rho$ is the
sum of the $m$ largest entries of $\vec{s}$. 

% a bit more information about our method


A key component of the algorithm is the computation of order basis%
\begin{comment}
 (also known as minimal approximant basis or $\sigma$-basis) \citep{BeLa94}.
Order bases are efficiently computed using the algorithms from \citet{Giorgi2003}
and \citet{za2009}
\end{comment}
. We use order basis computation to compute a partial kernel basis,
which also reduces the column dimension of the problem. The problem
can then be separated to two subproblems of smaller row dimensions,
which can then be handled in the same way as the original problem.

Some results on kernel basis computation have appeared in \citep{za2012}.

\begin{comment}
% the rest


The remainder of this paper is structured as follows. Basic definitions
and properties of order bases and kernel bases are given in the next
section. The details of our kernel basis computation and a formal
statement of the algorithm can be found in \secref{Nullspace-Basis-Computation}.
A complexity analysis of the algorithm is provided in the following
section. The paper ends with a conclusion and topics for future research. 
\end{comment}



\section{Overview}

The remainder of this thesis is structured as follows. Basic definitions
and properties are given in the next Chapter. The details of our order
basis computation can be found in \chapref{OrderBasis}. Kernel basis
computation is described in \chapref{NullspaceBasis}.  \chapref{Matrix-inverse}
discusses the algorithm for computing matrix inverse. \chapref{Matrix-GCD}
discusses the computation of column bases. Unimodular completion is
then discussed in \chapref{Unimodular-Completion}. Then we look at
determinant computation in \chapref{determinant}, Hermite normal
form computation in \ref{chap:hermite}, and rank profile and rank-sensitive
computations in \ref{chap:rankSensitive}. %
\begin{comment}
The paper ends with a conclusion and topics for future research. 
\end{comment}

