
\chapter{\label{chap:Unimodular-Completion}Unimodular Completion}

Given a matrix $\mathbf{F}\in\mathbb{K}\left[x\right]^{m\times n}$
with $n>m$ and column degrees $\vec{s}$, we consider the problem
of efficiently computing a matrix $\mathbf{G}\in\mathbb{K}\left[x\right]^{(n-m)\times n}$
\begin{comment}
with $\left(-\vec{s}\right)$-minimal rows 
\end{comment}
such that $\begin{bmatrix}\mathbf{F}\\
\mathbf{G}
\end{bmatrix}$ is unimodular.%
\begin{comment}
let us call the product of the nonzero entries of its smith normal
form the \emph{generalized determinant} of $\mathbf{F}$.

Suppose $\mathbf{F}$ is full-rank with column degrees bounded by
the entries of a shift $\vec{s}\in\mathbb{Z}_{\ge0}^{n}$. We consider
the problem of finding a matrix $\mathbf{G}\in\mathbb{K}\left[x\right]^{(n-m)\times n}$
with $\left(-\vec{s}\right)$-minimal rows such that $\left[\mathbf{F}^{T},\mathbf{G}^{T}\right]^{T}$
has the same determinant as the generalized determinant of $\mathbf{F}$.
In the special case where the generalized determinant of $\mathbf{F}$
is $1$, the problem specializes to the standard unimodular completion
problem, where $\left[\mathbf{F}^{T},\mathbf{G}^{T}\right]^{T}$ is
unimodular. Note that the $\left(-\vec{s}\right)$ shift is chosen
to make the degrees consistent with the degrees of the input matrix
$\mathbf{F}$.
\begin{example}
If $\mathbf{F}=\left[1,0\right]$, $\vec{s}=\left[0,0\right]$. Then
the generalized determinant of $\mathbf{F}$ is $1$. A $\left(-\vec{s}\right)$-minimal
unimodular completion of $\mathbf{F}$ is then $\mathbf{G}=\left[0,1\right]$.
A unimodular completion that is not minimal is $\left[x^{9},1\right]$.
If $\mathbf{F}=\left[x,x^{2}\right]$, then a $\left(-\vec{s}\right)$-minimal
completion that maintains the generalized determinant is again $\left[0,1\right]$.\end{example}
\end{comment}
{} Unimodular completion is a useful basic operation in matrix computations
\citep{newman1972}. Our goal is to do this with a cost of $O^{\sim}\left(n^{\omega}s\right)$
field operations, where $s$ is the average of the $m$ largest column
degrees of $\mathbf{F}$.

Before discussing the computation of a unimodular completion, we need
to check the existence of unimodular completion for a given matrix.
In fact, a unimodular completion does not exist for some input matrices,
as in the case of $\mathbf{F}=\left[0,x\right]$. So we need to know
what type of input matrices admit a unimodular completion.
\begin{lem}
\label{lem:unimodularCompletionCondition}A unimodular completion
of $\mathbf{F}$ exists if and only if $\mathbf{F}$ has unimodular
column bases. \end{lem}
\begin{proof}
If $\mathbf{F}$ has a non-unimodular column basis $\mathbf{A}$,
then $\diag\left(\left[\mathbf{A},I\right]\right)$ is always a factor
of $\begin{bmatrix}\mathbf{F}\\
\mathbf{B}
\end{bmatrix}$ for any polynomial matrix $\mathbf{B}$, implying that the matrix
$\begin{bmatrix}\mathbf{F}\\
\mathbf{B}
\end{bmatrix}$ is non-unimodular. On the other hand, if $\mathbf{F}$ has a unimodular
column basis, then there exists a unimodular matrix $\mathbf{U}$
such that $\mathbf{F}\mathbf{U}=\left[I_{m},0\right]$, or $\mathbf{F}=\left[I_{m},0\right]\mathbf{U}^{-1}$
after rearranging, that is, $\mathbf{F}$ must be consists of the
top $m$ rows of $\mathbf{U}^{-1}$. The matrix $\mathbf{U}^{-1}$
is therefore a unimodular completion of the matrix $\mathbf{F}$.
\end{proof}
Since a unimodular completion is only possible for input matrices
with unimodular column bases, we assume for simplicity this is the
case with our input matrix $\mathbf{F}$. This also requires $\mathbf{F}$
to be full rank. For other matrices without unimodular column bases,
our method still works directly to compute a matrix completion for
a right factor of $\mathbf{F}$ that has its column basis factor removed.
In other words, let $\mathbf{F}$ be factored as $\mathbf{F}=\mathbf{T}\mathbf{R}$
as in \prettyref{lem:matrixGCD}, where $\mathbf{T}$ is a column
basis of $\mathbf{F}$ and $\mathbf{R}$ is the remaining right factor,
the our method always works to compute a unimodular completion of
$\mathbf{R}$. In the special case where $\mathbf{T}$ is unimodular,
the unimodular completion computed is also a unimodular completion
of $\mathbf{F}$.

The proof of \prettyref{lem:unimodularCompletionCondition} shows
that a unimodular completion of $\mathbf{F}$ can be obtained from
the unimodular matrix $\mathbf{U}$ that transforms $\mathbf{F}$
to its column bases. However, we may not be able to compute this $\mathbf{U}$
efficiently since its degree might be too large. More specifically,
$\mathbf{U}$ contains a kernel basis of $\mathbf{F}$ that may have
degree $\xi$, while each of the remaining columns of $\mathbf{U}$
may also have degree $\xi$. 

Before discussing the actual matrix completion, let us look at the
operations that reverses the coefficients of a polynomial, the coefficients
of the polynomial entries of a vector, and the coefficients of the
polynomial entries of a polynomial matrix. These operations are needed
in the computation of our matrix completion.


\section{Reversing polynomial coefficients}

First let us look at the operation that reverses the coefficients
of a polynomial.
\begin{defn}
For a polynomial $p=p_{0}+p_{1}x+\dots+p_{u}x^{u}\in\mathbb{K}\left[x\right]$
with degree bounded by $u$, we define the operation 
\[
\rev(p,u)=\left(p(x^{-1})\right)x^{u}=p_{u}+p_{u-1}x+\cdots+p_{1}x^{u-1}+p_{0}x^{u}.
\]

\end{defn}
We now extend this definition to column vectors and row vectors with
shifted degrees.
\begin{defn}
Let $\vec{u}=\left[u_{1},\dots u_{n}\right]\in\mathbb{Z}^{n}$ be
a degree shift, and a column vector $\mathbf{a}\in\mathbb{K}\left[x\right]^{n\times1}$
with $\vec{u}$-column degree bounded by $v$. We define
\[
\colRev(\mathbf{a},\vec{u},v)=x^{-\vec{u}}\left(\mathbf{a}(x^{-1})\right)x^{v}=\begin{bmatrix}\rev(p,v-u_{1})\\
\vdots\\
\rev(p,v-u_{n})
\end{bmatrix}.
\]
Similarly for a row vector $\mathbf{b}\in\mathbb{K}\left[x\right]^{1\times n}$
with $\vec{u}$-row degree bounded by $v$, where $\vec{u}=\left[u_{1},\dots u_{n}\right]\in\mathbb{Z}^{n}$
is a degree shift, we define
\[
\rowRev(\mathbf{b},\vec{u},v)=\colRev(\mathbf{b}^{T},\vec{u},v)^{T}=x^{v}\left(\mathbf{b}(x^{-1})\right)x^{-\vec{u}}.
\]
\end{defn}
\begin{example}
If $\mathbf{f}=\left[10+x,5+x+2x^{2}\right]$, $\vec{u}=\left[-1,-2\right]$,
and $v=0$, then 
\[
\rowRev(\mathbf{f},\vec{u},v)=x^{0}\left[10+x^{-1},5+x^{-1}+2x^{-2}\right]\begin{bmatrix}x\\
 & x^{2}
\end{bmatrix}=\left[10x+1,5x^{2}+x+2\right].
\]

\end{example}
We can extend the reverse operation further to polynomial matrices.
\begin{defn}
Let $\vec{u}=\left[u_{1},\dots u_{n}\right]\in\mathbb{Z}^{n}$ be
a degree shift. Let $\mathbf{A}\in\mathbb{K}\left[x\right]^{n\times k}$
with $\vec{u}$-column degrees bounded component-wise by $\vec{v}=\left[v_{1},\dots,v_{k}\right]$,
we define 
\[
\colRev(\mathbf{A},\vec{u},\vec{v})=x^{-\vec{u}}\left(\mathbf{A}(1/x)\right)x^{\vec{v}}
\]
 Similarly, for $\vec{u}=\left[u_{1},\dots,u_{n}\right]$ and $\mathbf{B}\in\mathbb{K}\left[x\right]^{k\times n}$
with $\vec{u}$-row degrees bounded component-wise by $\vec{v}=\left[v_{1},\dots,v_{k}\right]$,
\[
\rowRev(\mathbf{B},\vec{u},\vec{v})=\colRev(\mathbf{B}^{T},\vec{u},\vec{v})^{T}=x^{\vec{v}}\left(\mathbf{B}(1/x)\right)x^{-\vec{u}}
\]
 Note that we also have $\rowRev(\mathbf{B},\vec{u},\vec{v})=\colRev(\mathbf{B},-\vec{v},-\vec{u})$.
\end{defn}
\begin{comment}
Again, we assume a shift $\vec{s}$ bounds the column degrees of $\mathbf{F}$
component-wise with $\sum\vec{s}=\xi$. First, we compute a $\vec{s}$-minimal
kernel basis $\mathbf{N}$ of $\mathbf{F}$. We then reverse the coefficients
of $\mathbf{N}$ based on its $\vec{s}$-column degrees as follows:
To reverse the coefficients of column $\mathbf{n}$ that has $\vec{s}$-column
degrees bounded by $t$, let
\[
\reverse(\mathbf{n},\vec{s},t)=x^{-\vec{s}}\left(\mathbf{n}(1/x)\right)x^{t}=\begin{bmatrix}x^{-s_{1}}\\
 & \ddots\\
 &  & x^{-s_{n}}
\end{bmatrix}\left(\mathbf{n}(1/x)\right)x^{t}.
\]
 I.e., for the $i$th entry $\mathbf{n}_{i}$ of $\mathbf{n}$, where
$\mathbf{n}_{i}=p_{0}+p_{1}x+\dots+p_{t-s_{i}}x^{t-s_{i}}$, the reversed
$\mathbf{n}$ becomes $\reverse(\mathbf{n},\vec{s},t)=p_{t-s_{i}}+p_{t-s_{i}-1}x+\cdots+p_{1}x^{t-s_{i}-1}+p_{0}x^{t-s_{i}}.$
Each column of $\mathbf{N}$ is reversed in this way to get a new
matrix $\bar{\mathbf{N}}$. That is, for matrix $\mathbf{N}\in\mathbb{K}\left[x\right]^{n\times k}$
with $\vec{s}$-column degrees bounded component-wise by $\vec{t}$,
we define 
\[
\reverse(\mathbf{N},\vec{s},\vec{t})=x^{-\vec{s}}\left(\mathbf{N}(1/x)\right)x^{\vec{t}}=\begin{bmatrix}x^{-s_{1}}\\
 & \ddots\\
 &  & x^{-s_{n}}
\end{bmatrix}\left(\mathbf{N}(1/x)\right)\begin{bmatrix}x^{t_{1}}\\
 & \ddots\\
 &  & x^{t_{k}}
\end{bmatrix}.
\]


We then compute a $\left(\bar{\mathbf{N}}^{T},\sigma,-\vec{s}\right)$-basis
$\bar{\mathbf{P}}'$ with $\sigma$ big enough to contain a complete
kernel basis $\bar{\mathbf{F}}'$ of $\bar{\mathbf{N}}^{T}$. Let
$\mathbf{P}$ and $\mathbf{F'}$ be the matrices $\bar{\mathbf{P}}'$
and $\mathbf{\bar{F}'}$ with coefficients reversed based on their
$(-\vec{s})$-column degrees respectively. Then it is not difficult
to see that $\mathbf{F}'$ is a kernel basis of $\mathbf{N}'$.
\end{comment}


It is useful to note that any degree bound remains the same after
the reverse operations.
\begin{lem}
If $\mathbf{A}\in\mathbb{K}\left[x\right]^{n\times k}$ has $\vec{u}$-column
degrees bounded by the corresponding entries of $\vec{v}$, then $\colRev(\mathbf{A},\vec{u},\vec{v})$
also has $\vec{u}$-column degrees bounded by the corresponding entries
of $\vec{v}$.
\end{lem}
As one would expect, applying two reverse operations gives back the
original input.
\begin{lem}
The following equalities holds:%
\begin{comment}
\begin{eqnarray*}
\rev\left(\rev(p,u),u\right) & = & p\\
\colRev\left(\colRev(\mathbf{a},\vec{u},v),\vec{u},v\right) & = & \mathbf{a}\\
\rowRev\left(\rowRev(\mathbf{b},\vec{u},v),\vec{u},v\right) & = & \mathbf{b}
\end{eqnarray*}
\end{comment}
\begin{eqnarray*}
\colRev\left(\colRev(\mathbf{A},\vec{u},\vec{v}),\vec{u},\vec{v}\right) & = & \mathbf{A}\\
\rowRev\left(\rowRev(\mathbf{B},\vec{u},\vec{v}),\vec{u},\vec{v}\right) & = & \mathbf{B}
\end{eqnarray*}

\end{lem}
Let us look at a degree bound on the product of a row vector and a
column vector, based on their shifted degrees, when opposite shifts
are used.
\begin{lem}
\label{lem:vectorProductBound}If $\mathbf{a}\in\mathbb{K}\left[x\right]^{1\times n}$
and $\mathbf{a}^{T}$ has $\left(-\vec{u}\right)$-column degree bounded
by $\alpha$ (or equivalently, $\mathbf{a}$ has $\left(-\vec{u}\right)$-row
degree bounded by $\alpha$) and $\mathbf{b}\in\mathbb{K}\left[x\right]^{n\times1}$
has $\vec{u}$-column degree bounded by $\beta$, then $\mathbf{a}\mathbf{b}$
has degree bounded by $\alpha+\beta$.\end{lem}
\begin{proof}
Since $\mathbf{a}x^{-\vec{u}}$ has degree bounded by $\alpha$ and
$x^{\vec{u}}\mathbf{b}$ has degree bounded by $\beta$, $\mathbf{a}x^{-\vec{u}}x^{\vec{u}}\mathbf{b}=\mathbf{a}\mathbf{b}$
has degree bounded by $\alpha+\beta$.
\end{proof}
The following lemma shows that the reverse operation and the multiplication
are commutative when we use the opposite shifts.
\begin{lem}
\label{lem:reverseProduct}If $\mathbf{a}\in\mathbb{K}\left[x\right]^{1\times n}$
has $\left(-\vec{u}\right)$-row degree bounded by $\alpha$ and $\mathbf{b}\in\mathbb{K}\left[x\right]^{n\times1}$
has $\vec{u}$-column degree bounded by $\beta$, then 
\[
\rowRev(\mathbf{a},-\vec{u},\alpha)\cdot\colRev(\mathbf{b},\vec{u},\beta)=\rev(\mathbf{a}\mathbf{b},\alpha+\beta).
\]
\end{lem}
\begin{proof}
\ 
\begin{eqnarray*}
 &  & \rowRev(\mathbf{a},-\vec{u},\alpha)\cdot\colRev(\mathbf{b},\vec{u},\beta)\\
 & = & x^{\alpha}\left(\mathbf{a}(1/x)\right)x^{\vec{u}}x^{-\vec{u}}\left(\mathbf{b}(1/x)\right)x^{\beta}\\
 & = & \left(\mathbf{a}(1/x)\right)x^{\vec{u}}x^{-\vec{u}}\left(\mathbf{b}(1/x)\right)x^{\alpha+\beta}\\
 & = & \left(\mathbf{a}(1/x)\right)\left(\mathbf{b}(1/x)\right)x^{\alpha+\beta}\\
 & = & \left(\left(\mathbf{a}\mathbf{b}\right)(1/x)\right)x^{\alpha+\beta}\\
 & = & \rev(\mathbf{a}\mathbf{b},\alpha+\beta)
\end{eqnarray*}
\begin{comment}
\begin{lem}
If $\mathbf{F}\in\mathbb{K}\left[x\right]^{m\times n}$ has $\left(-\vec{s}\right)$-row
degree bounded component-wise by $\vec{a}=\left[a_{1},\dots,a_{m}\right]$
and $\mathbf{G}\in\mathbb{K}\left[x\right]^{n\times k}$ has $\vec{s}$-column
degree bounded component-wise by $\vec{b}=[b_{1},\dots,b_{m}]$, then
\[
\reverse(\mathbf{F},-\vec{s},\vec{a})\cdot\reverse(\mathbf{G},\vec{s},\vec{b})=\reverse(\mathbf{F}\mathbf{G},\vec{c}),
\]
 where 
\[
\vec{c}=\begin{bmatrix}a_{1}+b_{1} & a & \cdots & a_{1}+b_{k}\\
\\
\\
\end{bmatrix}\in\mathbb{Z}^{m\times k}
\]
\end{lem}
\begin{proof}
sd$\reverse(\mathbf{n},\vec{s},t)\cdot\reverse(\mathbf{g},\vec{s},b)$\end{proof}
\end{comment}


We also have the following similar result on the reverse operation
and matrix multiplication\end{proof}
\begin{lem}
\label{lem:reverseMatrixProduct}If $\mathbf{A}\in\mathbb{K}\left[x\right]^{m\times n}$
has $\vec{u}$-column degrees bounded by $\vec{v}$, and $\mathbf{B}\in\mathbb{K}\left[x\right]^{n\times k}$
has $\vec{v}$-column degrees bounded by $\vec{w}$, then 
\[
\colRev(\mathbf{A},\vec{u},\vec{v})\colRev(\mathbf{B},\vec{v},\vec{w})=\colRev(\mathbf{A}\mathbf{B},\vec{u},\vec{w})
\]
 has $\vec{u}$-column degrees bounded by $\vec{w}$. \end{lem}
\begin{proof}
\ 
\begin{eqnarray*}
 &  & \colRev(\mathbf{A},\vec{u},\vec{v})\colRev(\mathbf{B},\vec{v},\vec{w})\\
 & = & x^{-\vec{u}}\left(\mathbf{A}(1/x)\right)x^{\vec{v}}x^{-\vec{v}}\left(\mathbf{B}(1/x)\right)x^{\vec{w}}\\
 & = & x^{-\vec{u}}\left(\mathbf{A}\mathbf{B}\right)(1/x)x^{\vec{w}}.
\end{eqnarray*}
\end{proof}
\begin{lem}
\label{lem:reverseMatrixProduct2}If $\mathbf{A}\in\mathbb{K}\left[x\right]^{m\times n}$
has $\vec{u}$-row degrees bounded by $\vec{v}$, and $\mathbf{B}\in\mathbb{K}\left[x\right]^{n\times k}$
has $-\vec{u}$-column degrees bounded by $\vec{w}$, then 
\[
\rowRev(\mathbf{A},\vec{u},\vec{v})\colRev(\mathbf{B},-\vec{u},\vec{w})=\colRev(\mathbf{A}\mathbf{B},-\vec{v},\vec{w}).
\]
.\end{lem}
\begin{proof}
\begin{eqnarray*}
 &  & \rowRev(\mathbf{A},\vec{u},\vec{v})\colRev(\mathbf{B},-\vec{u},\cdeg_{-\vec{u}}\mathbf{B})\\
 & = & x^{\vec{v}}\left(\mathbf{A}(1/x)\right)x^{-\vec{u}}x^{\vec{u}}\left(\mathbf{B}(1/x)\right)x^{\cdeg_{-\vec{u}}\mathbf{B}}\\
 & = & x^{\vec{v}}\left(\mathbf{A}(1/x)\right)\left(\mathbf{B}(1/x)\right)x^{\cdeg_{-\vec{u}}\mathbf{B}}\\
 & = & x^{\vec{v}}\left(\mathbf{A}\mathbf{B}(1/x)\right)x^{\cdeg_{-\vec{u}}\mathbf{B}}.
\end{eqnarray*}

\end{proof}

\section{Unimodular completion}

In this section, we look at how a unimodular completion can be done
using a combination of kernel basis computations, order basis computations,
and reverse operations. First, we have the following natural relationship
between a kernel basis and the reverse operation.
\begin{lem}
\label{lem:reverseNullspaceBasis}Let $\vec{u}\in\mathbb{Z}^{n}$,
$\mathbf{A}\in\mathbb{K}\left[x\right]^{m\times n}$ with $(-\vec{u})$-row
degrees bounded component-wise by $\vec{a}$, and $\mathbf{A}^{r}=\rowRev\left(\mathbf{A},-\vec{u},\vec{a}\right)$.
Then a matrix $\mathbf{N}\in\mathbb{K}\left[x\right]^{n\times k}$
with $\vec{u}$-column degrees $\vec{b}$ is a $(\mathbf{A},\vec{u})$-kernel
basis %
\begin{comment}
$\vec{u}$-minimal kernel basis of $\mathbf{A}$ 
\end{comment}
if and only if $\mathbf{N}^{r}=\colRev\left(\mathbf{N},\vec{u},\vec{b}\right)$
is a $(\mathbf{A}^{r},\vec{u})$-kernel basis%
\begin{comment}
$\vec{u}$-minimal kernel basis of $\rowRev\left(\mathbf{A},-\vec{u},\vec{a}\right)$
\end{comment}
.\end{lem}
\begin{proof}
If $\mathbf{N}$ is a kernel basis of $\mathbf{A}$, then we know
from \prettyref{lem:reverseProduct} that 
\[
\rowRev\left(\mathbf{A},-\vec{u},\vec{a}\right)\cdot\colRev\left(\mathbf{N},\vec{u},\vec{b}\right)=0,
\]
so $\colRev\left(\mathbf{N},\vec{u},\vec{b}\right)$ is a kernel basis
of $\rowRev\left(\mathbf{A},-\vec{u},\vec{a}\right)$. Suppose $\colRev\left(\mathbf{N},\vec{u},\vec{b}\right)$
is not $\vec{u}$-minimal and we have another kernel basis $\mathbf{M}$
of $\rowRev\left(\mathbf{A},-\vec{u},\vec{a}\right)$ with $\vec{u}$-column
degrees $\vec{c}$ that has some entry lower than the corresponding
entry in $\vec{b}$. Then $\colRev\left(\mathbf{M},\vec{u},\vec{c}\right)$
is also a kernel of $\mathbf{A}$ with lower $\vec{u}$-column degrees
than $\vec{b}$, contradicting the $\vec{u}$-minimality of $\mathbf{N}$.
\end{proof}


The following lemma shows the unimodular equivalence between any matrix
$\mathbf{A}$ that has a unimodular column basis, and a left kernel
basis of any right kernel basis of $\mathbf{A}$.
\begin{lem}
\label{lem:unimodularEquivalenceNullspaceBasisOfNullspaceBasis}Given
a matrix $\mathbf{A}\in\mathbb{K}\left[x\right]^{m\times n}$ with
unimodular column basis. Let $\mathbf{N}\in\mathbb{K}\left[x\right]^{n\times\left(n-m\right)}$
be a right kernel basis of $\mathbf{A}$. Let $\mathbf{B}$ be a left
kernel basis of $\mathbf{N}$. Then $\mathbf{A}=\mathbf{U}\mathbf{B}$
for a unimodular matrix $\mathbf{U}$.\end{lem}
\begin{proof}
This follows from \prettyref{lem:matrixGCD}, which tells us that
$\mathbf{U}$ is just a column basis of $\mathbf{A}$.
\end{proof}
Now let us look at how an order basis can lead to a unimodular matrix. 
\begin{lem}
\label{lem:reverseOrderBasisToUnimodular}Let $\vec{u}=\left[u_{1},\dots u_{n}\right]\in\mathbb{Z}^{n}$
be a degree shift. Any $\left(\mathbf{A},\sigma,\vec{u}\right)$-basis
$\mathbf{P}$ with $\cdeg_{\vec{u}}\mathbf{P}=\vec{v}=\left[v_{1},\dots,v_{k}\right]$
has $\det\left(\mathbf{P}\right)=cx^{\sum\vec{v}-\sum\vec{u}}$ and
$\det\left(\colRev(\mathbf{P},\vec{u},\vec{v})\right)=c$ for some
constant $c\in\mathbb{K}$. In other words, $\colRev(\mathbf{P},\vec{u},\vec{v})$
is unimodular.\end{lem}
\begin{proof}
To see that $\det\left(\mathbf{P}\right)=cx^{\sum\vec{v}-\sum\vec{u}}$,
note that an identity matrix is an $\left(\mathbf{A},0,\vec{u}\right)$-basis,
which has $\vec{u}$-column degrees $\vec{u}$ and determinant $1$.
Then the $\vec{u}$-column degrees only increases by multiplying some
column of $\mathbf{P}$ by $x$. The second property $\det\left(\colRev(\mathbf{P},\vec{u},\vec{v})\right)=c$
follows from the definition 
\[
\colRev(\mathbf{P},\vec{u},\vec{v})=x^{-\vec{u}}\left(\mathbf{P}(1/x)\right)x^{\vec{v}}.
\]

\end{proof}
\prettyref{lem:reverseOrderBasisToUnimodular} suggests that a unimodular
completion of $\mathbf{F}$ can be computed by embedding $\mathbf{F}$
in a reversed order basis, or equivalently, embedding a reversed $\mathbf{F}$
in an order basis. The next question is therefore how to embed a matrix
in an order basis. Recall that \prettyref{lem:nullspaceBasisInOrderBasis}
shows how kernel bases can be embedded in order bases. Therefore,
if we can make the reversed $\mathbf{F}$ a kernel basis of some matrix
$\mathbf{M}$, then there is an order basis of $\mathbf{M}$ that
contains the reversed $\mathbf{F}$. A natural choice for $\mathbf{M}$
is a kernel basis of the reversed $\mathbf{F}$.  We actually have
two choices here. We can either reverse the coefficients of $\mathbf{F}$,
as we do in \prettyref{lem:unimodularComputation} below, or we can
reverse the coefficients of a kernel basis of $\mathbf{F}$.
\begin{lem}
\label{lem:unimodularComputation}Let $\mathbf{F}^{r}=\rowRev\left(\mathbf{F},-\vec{s},0\right)$
and $\mathbf{M}$ be a $(\mathbf{F}^{r},\vec{s})$-kernel basis with
$\cdeg_{\vec{s}}\mathbf{M}=\vec{b}$. Let $\mathbf{P}=\left[\mathbf{P}_{1},\mathbf{P}_{2}\right]$
be a $\left(\mathbf{M}^{T},\vec{b}+1,-\vec{s}\right)$-basis, where
$\mathbf{P}_{1}$ consists of all columns $\mathbf{p}$ with $\cdeg_{-\vec{s}}\mathbf{p}\le0$.
If $\mathbf{P}_{2}^{r}=\colRev\left(\mathbf{P}_{2},-\vec{s},\cdeg_{-\vec{s}}\mathbf{P}_{2}\right)$,
then $\left[\mathbf{F}^{T},\mathbf{P}_{2}^{r}\right]$ is a unimodular
matrix.\end{lem}
\begin{proof}
Let $\mathbf{P}_{1}^{r}=\colRev\left(\mathbf{P}_{1},-\vec{s},\cdeg_{-\vec{s}}\mathbf{P}_{1}\right)$.
We know from \prettyref{lem:reverseOrderBasisToUnimodular} that $\left[\mathbf{P}_{1}^{r},\mathbf{P}_{2}^{r}\right]$
is unimodular. Let $\mathbf{M}^{r}=\colRev\left(\mathbf{M},\vec{s},\vec{b}\right)$.
Then from \prettyref{lem:reverseNullspaceBasis} we know $\mathbf{M}^{r}$
is a $\left(\mathbf{F},\vec{s}\right)$-kernel basis and $\mathbf{P}_{1}^{r}$
is a $\left(\left(\mathbf{M}^{r}\right)^{T},-\vec{s}\right)$-kernel
basis, hence by \prettyref{lem:unimodularEquivalenceNullspaceBasisOfNullspaceBasis}
$\mathbf{F}=\mathbf{U}\left(\mathbf{P}_{1}^{r}\right)^{T}$ for some
unimodular matrix $\mathbf{U}$. Now $\left[\mathbf{F}^{T},\mathbf{P}_{2}^{r}\right]^{T}=\diag\left(\left[\mathbf{U},I\right]\right)\left[\mathbf{P}_{1}^{r},\mathbf{P}_{2}^{r}\right]^{T}$.
\end{proof}
\prettyref{lem:unimodularComputation} provides a way to correctly
compute a unimodular completion of $\mathbf{F}$. To improve the computational
efficiency, we can in fact separate the rows of $\mathbf{M}^{T}$
and just work with one subset of rows at a time. 
\begin{lem}
\label{lem:unimodularComputationByRows}Let the matrix $\mathbf{F}^{r}=\rowRev\left(\mathbf{F},-\vec{s},0\right)$.
Let the matrix $\mathbf{M}$ be a $(\mathbf{F}^{r},\vec{s})$-kernel
basis with $\cdeg_{\vec{s}}\mathbf{M}=\vec{b}$ and be partitioned
as $\mathbf{M}=\left[\mathbf{M}_{1},\mathbf{M}_{2}\right]$. Let $\mathbf{P}_{1}$
be a $\left(\mathbf{M}_{1}^{T},\cdeg_{\vec{s}}\mathbf{M}_{1}+1,-\vec{s}\right)$-basis
and be partitioned as $\mathbf{P}_{1}=\left[\mathbf{N}_{1},\mathbf{Q}_{1}\right]$,
where $\mathbf{N}_{1}$ consists of all columns $\mathbf{p}$ of $\mathbf{P}_{1}$
with $\cdeg_{-\vec{s}}\mathbf{p}\le0$. Let $\vec{t}=\cdeg_{-\vec{s}}\mathbf{N}_{1}$
and $\mathbf{P}_{2}$ be a $\left(\mathbf{M}_{2}^{T}\mathbf{N}_{1},\cdeg_{\vec{s}}\mathbf{M}_{2}+1,\vec{t}\right)$-basis
and be partitioned as $\mathbf{P}_{2}=\left[\mathbf{N}_{2},\mathbf{Q}_{2}\right]$,
where $\mathbf{N}_{2}$ consists of all columns $\mathbf{p}$ of $\mathbf{P}_{2}$
with $\cdeg_{-\vec{t}}\mathbf{p}\le0$. Let $\mathbf{R}=\left[\mathbf{N}_{1}\mathbf{Q}_{2},\mathbf{Q}_{1}\right]$
and $\mathbf{R}^{r}=\colRev\left(\mathbf{R},-\vec{s},\cdeg_{-\vec{s}}\mathbf{R}\right)$.
Then $\left[\mathbf{F}^{T},\mathbf{R}^{r}\right]$ is a unimodular
matrix.\end{lem}
\begin{proof}
We know from \prettyref{lem:reverseOrderBasisToUnimodular} that $\mathbf{P}_{1}^{r}=\colRev\left(\mathbf{P}_{1},-\vec{s},\cdeg_{-\vec{s}}\mathbf{P}_{1}\right)$
and $\mathbf{P}_{2}^{r}=\colRev\left(\mathbf{P}_{1},\vec{t},\cdeg_{\vec{t}}\mathbf{P}_{2}\right)$
are both unimodular. Hence $\mathbf{P}_{1}^{r}\cdot\diag\left(\left[\mathbf{P}_{2}^{r},I\right]\right)=\left[\mathbf{N}_{1}^{r}\mathbf{N}_{2}^{r},\mathbf{N}_{1}^{r}\mathbf{Q}_{2}^{r},\mathbf{Q}_{1}\right]=\left[\mathbf{N}_{1}^{r}\mathbf{N}_{2}^{r},\mathbf{R}^{r}\right]$
is unimodular, where\textbf{ $\mathbf{N}_{1}\mathbf{N}_{2}$ }is a
kernel basis of $\mathbf{M}$. The result follows by the same reasoning
as in \prettyref{lem:unimodularComputation}.
\end{proof}

\section{Efficient Computation}

\prettyref{lem:unimodularComputationByRows} provides a way to correctly
compute a unimodular completion of $\mathbf{F}$. Our next task is
to make sure it can be computed efficiently and analyze its computational
cost. We already know that a $(\mathbf{F}^{r},\vec{s})$-kernel basis
can be computed with a cost of $O^{\sim}\left(n^{\omega}s\right)$.
Therefore, it only remains to check the cost of the order basis computations.
Note that the non-uniform order makes our problem here a little more
difficult. But on the other hand, the output basis has its $-\vec{s}$-column
degrees bounded by $1$, which is a consequence of the fact $\mathbf{M}$
is a $\vec{s}$-minimal kernel basis, as shown in \prettyref{lem:nullspaceOrderbasisDegree}
below. But we first need a few general lemmas on the degree bounds
of order bases and kernel bases.

First, the following lemma is a simple extension of \prettyref{lem:boundOfSumOfShiftedDegreesOfOrderBasis}
for dealing with nonuniform orders.
\begin{lem}
\label{lem:boundOfSumOfShiftedDegreesOfOrderBasisWithNonuniformOrder}Given
an input matrix $\mathbf{A}\in\mathbb{K}^{m\times n}[x]$, a shift
$\vec{u}\in\mathbb{Z}^{n}$, and an order list $\vec{\sigma}\in\mathbb{Z}^{m}$.
Let $\vec{v}$ be the $\vec{u}$-column degrees of a $\left(\mathbf{A},\vec{\sigma},\vec{u}\right)$-basis.
Then $\sum\vec{t}~\le~\sum\vec{s}+\sum\vec{\sigma}$%
\begin{comment}
 and $\max_{i}\left(\vec{t}_{i}-\vec{s}_{i}\right)\le\sigma$
\end{comment}
\textup{}%
\begin{comment}
need to permute the columns to put the pivots on the diagonal.
\end{comment}
. \end{lem}
\begin{proof}
\begin{comment}
For example, for a input matrix with two rows, instead of increasing
the order of both rows to $\left[1,1\right]$ at once, we can work
on the second row first to increase the order to $\left[0,1\right]$,
and then compute to order $\left[1,1\right]$.  
\end{comment}
The sum of the $\vec{s}$-column degrees is $\sum\vec{s}$ at order
$\left[0,\dots,0\right]$, since the identity matrix is a $\left(\mathbf{A},\left[0,\dots,0\right],\vec{s}\right)$-basis.
This sum increases by $1$ for each order increase of each row. The
total number of order increases required for all rows is at most $\sum\vec{\sigma}$.
Note that from \prettyref{thm:combineOrderBases}, we can work with
just one row at a time to increase its order in the order basis computation. 
\end{proof}
The following lemma extends \prettyref{thm:boundOfSumOfShiftedDegreesOfKernelBasis}
to give a bound based on the shifted column degrees or shifted row
degrees, instead of just the column degrees of the input matrix.
\begin{lem}
\label{lem:generaKernelBasisDegreeBound}If $\mathbf{A}\in\mathbb{K}^{m\times n}[x]$
has $\rdeg_{\vec{u}}\mathbf{A}\le\vec{v}$ or equivalently $\cdeg_{-\vec{v}}\mathbf{A}\le-\vec{u}$,
then any $(\mathbf{A},-\vec{u})$-kernel basis has $-\vec{u}$-column
degrees bounded by $\sum\vec{v}-\sum\vec{u}$. \end{lem}
\begin{proof}
Let $\mathbf{P}=\left[\mathbf{B},\bar{\mathbf{B}}\right]$ be a $\left(\mathbf{A},\vec{v}+\left[\sigma,\dots,\sigma\right],-\vec{u}\right)$-basis
containing a kernel basis, $\mathbf{B}$, of $\mathbf{A}$. Then $\sum\cdeg_{-\vec{u}}\mathbf{P}$
is at least $m\sigma+\sum\vec{v}-\sum\vec{u}$. We also know that
$\sum\cdeg_{-\vec{u}}\bar{\mathbf{B}}\ge\sum\cdeg_{-\vec{v}}\mathbf{A}\bar{\mathbf{B}}$,
but $\cdeg\mathbf{A}\bar{\mathbf{B}}\ge\vec{v}+\left[\sigma,\dots,\sigma\right]$
or $\sum\cdeg_{-\vec{v}}\mathbf{A}\bar{\mathbf{B}}\ge m\sigma$, therefore
$\sum\cdeg_{-\vec{u}}\bar{\mathbf{B}}\ge m\sigma$. It follows that
$\sum\cdeg_{-\vec{u}}\mathbf{B}\le m\sigma+\sum\vec{v}-\sum\vec{u}-m\sigma=\sum\vec{v}-\sum\vec{u}$.
\end{proof}
When the matrix $\mathbf{A}$ is also a $\left(\mathbf{B}^{T},\vec{u}\right)$-kernel
basis, as in our case, the bound in fact becomes tight.
\begin{lem}
\label{lem:mutualMinimalNullspaceBasisDegrees}Let $\mathbf{A}\in\mathbb{K}^{m\times n}[x]$
and $\mathbf{B}\in\mathbb{K}^{n\times(n-m)}\left[x\right]$. If $\mathbf{B}$
is a $(\mathbf{A},-\vec{u})$-kernel basis with $\cdeg_{-\vec{u}}\mathbf{B}=\vec{w}$
and $\mathbf{A}^{T}$ is a $\left(\mathbf{B}^{T},\vec{u}\right)$-kernel
basis with $\rdeg_{\vec{u}}\mathbf{A}=\vec{v}$, then $\sum\vec{w}=\sum\vec{v}-\sum\vec{u}$.\end{lem}
\begin{proof}
This follows from \prettyref{lem:generaKernelBasisDegreeBound}, which
gives $\sum\vec{w}\le\sum\vec{v}-\sum\vec{u}$ and also $\sum\vec{v}\le\sum\vec{w}+\sum\vec{u}$
in the reverse direction.
\end{proof}
From \prettyref{lem:nullspaceBasisInOrderBasis}, we know that any
$\left(\mathbf{M}^{T},\vec{b}+1,-\vec{s}\right)$-basis contains a
$\left(\mathbf{M}^{T},-\vec{s}\right)$-kernel basis whose $-\vec{s}$-column
degrees bounded by 0. The following lemma shows that the remaining
part of the $\left(\mathbf{M}^{T},\vec{b}+1,-\vec{s}\right)$-basis
has degrees bounded by 1.
\begin{lem}
\label{lem:nullspaceOrderbasisDegree}Let $\mathbf{F}^{r}=\rowRev\left(\mathbf{F},-\vec{s},0\right)$
and $\mathbf{M}$ be a $(\mathbf{F}^{r},\vec{s})$-kernel basis with
$\cdeg_{\vec{s}}\mathbf{M}=\vec{b}$ as before. Let $\mathbf{P}$
be a $\left(\mathbf{M}^{T},\vec{b}+1,-\vec{s}\right)$-basis. Then
$\cdeg_{-\vec{b}-1}\mathbf{M}^{T}\mathbf{P}_{2}=\left[0,\dots,0\right]$
and $\cdeg_{-\vec{s}}\mathbf{P}_{2}=\left[1,\dots,1\right]$.\end{lem}
\begin{proof}
We already know that $\mathbf{P}$ contains a $\left(\mathbf{M}^{T},-\vec{s}\right)$-kernel
basis. Let this kernel basis be $\mathbf{P}_{1}$ in $\mathbf{P}=\left[\mathbf{P}_{1},\mathbf{P}_{2}\right]$.
We know that $\sum\cdeg_{-\vec{s}}\mathbf{P}=-\sum\vec{s}+\sum\vec{b}+n-m$
and for the kernel basis $\mathbf{P}_{1}$ in $\mathbf{P}$, we know
$\sum\cdeg_{-\vec{s}}\mathbf{P}_{1}=\sum\vec{b}-\sum\vec{s}$ from
\prettyref{lem:mutualMinimalNullspaceBasisDegrees}. Therefore, $\sum\cdeg_{-\vec{s}}\mathbf{P}_{2}=n-m$.
It follows that $\sum\cdeg_{-\vec{b}}\mathbf{M}^{T}\mathbf{P}_{2}\le\sum\cdeg_{-\vec{s}}\mathbf{P}_{2}=n-m$,
or $\sum\cdeg_{-\vec{b}-1}\mathbf{M}^{T}\mathbf{P}_{2}=0$. But since
$\mathbf{P}_{2}$ is nonzero and has order $\left(\mathbf{F},\vec{b}+1\right)$,
we have $\cdeg_{-\vec{b}-1}\mathbf{M}^{T}\mathbf{P}_{2}\ge\left[0,\dots,0\right]$,
implying $\sum\cdeg_{-\vec{b}-1}\mathbf{M}^{T}\mathbf{P}_{2}\ge0$.
It follows that $\sum\cdeg_{-\vec{b}-1}\mathbf{M}^{T}\mathbf{P}_{2}=0$,
hence $\cdeg_{-\vec{b}-1}\mathbf{M}^{T}\mathbf{P}_{2}=\left[0,\dots,0\right]$
or $\cdeg_{-\vec{b}}\mathbf{M}^{T}\mathbf{P}_{2}=\left[1,\dots,1\right]$.
Combining this with $\sum\cdeg_{-\vec{b}}\mathbf{M}^{T}\mathbf{P}_{2}\le\sum\cdeg_{-\vec{s}}\mathbf{P}_{2}=n-m$
we then get $\cdeg_{-\vec{s}}\mathbf{P}_{2}=\left[1,\dots,1\right]$.
\end{proof}
\begin{comment}
\begin{lem}
Let $\mathbf{F}^{r}=\rowRev\left(\mathbf{F},-\vec{s},0\right)$ and
$\mathbf{M}$ be a $(\mathbf{F}^{r},\vec{s})$-kernel basis with $\cdeg_{\vec{s}}\mathbf{M}=\vec{b}$
as before. Let $\mathbf{P}$ be a $\left(\mathbf{M}^{T},\vec{b}+1,-\vec{s}\right)$-basis.
Then $\mathbf{M}^{T}\mathbf{P}$ has row degrees $\vec{b}+[1,\dots,1]$.
In other words, the row degrees are the same as the order.\end{lem}
\begin{proof}
From \prettyref{lem:nullspaceBasisOrderBasisContainsNullspaceBasis},
$\mathbf{P}$ contains a $\left(\mathbf{M}^{T},-\vec{s}\right)$-kernel
basis. Let this kernel basis be $\mathbf{P}_{1}$ in $\mathbf{P}=\left[\mathbf{P}_{1},\mathbf{P}_{2}\right]$.
We know that $\sum\cdeg_{-\vec{s}}\mathbf{P}=-\sum\vec{s}+\sum\vec{b}+n-m$
and for the kernel basis $\mathbf{P}_{1}$ in $\mathbf{P}$, we know
$\sum\cdeg_{-\vec{s}}\mathbf{P}_{1}=\sum\vec{b}-\sum\vec{s}$. Therefore,
$\sum\cdeg_{-\vec{s}}\mathbf{P}_{2}=n-m$. It follows that $\sum\cdeg_{-\vec{b}}\mathbf{M}^{T}\mathbf{P}_{2}\le\sum\cdeg_{-\vec{s}}\mathbf{P}_{2}=n-m$,
or $\sum\cdeg_{-\vec{b}-1}\mathbf{M}^{T}\mathbf{P}_{2}=0$. But since
$\mathbf{P}_{2}$ is nonzero and has order $\left(\mathbf{F},\vec{b}+1\right)$,
we have $\cdeg_{-\vec{b}-1}\mathbf{M}^{T}\mathbf{P}_{2}\ge\left[0,\dots,0\right]$,
implying $\sum\cdeg_{-\vec{b}-1}\mathbf{M}^{T}\mathbf{P}_{2}\ge0$.
It follows that $\sum\cdeg_{-\vec{b}-1}\mathbf{M}^{T}\mathbf{P}_{2}=0$,
hence $\cdeg_{-\vec{b}-1}\mathbf{M}^{T}\mathbf{P}_{2}=\left[0,\dots,0\right]$. 

Since 

or equivalently, $\rdeg\mathbf{M}^{T}\mathbf{P}_{2}=\vec{b}+1$.

We know that $\mathbf{M}^{T}$ has an identity matrix GCD, hence by
\prettyref{lem:CoprimeFullRankConstantCoefficientMatrix} $\mathbf{M}^{T}\mod x$
is full-rank, and by \prettyref{cor:fullRankConstantCoefficientMatrix}
$x^{-\vec{b}-[1,\dots,]}\mathbf{M}^{T}\mathbf{P}\mod x$ is also full-rank.
Now since 
\begin{eqnarray*}
 &  & \rowRev(\mathbf{M}^{T},\vec{s},\vec{b})\colRev(\mathbf{P},-\vec{s},\cdeg_{-\vec{s}}\mathbf{P})\\
 & = & x^{\vec{b}}\left(\mathbf{M}^{T}(1/x)\right)x^{-\vec{s}}x^{\vec{s}}\left(\mathbf{P}(1/x)\right)x^{\cdeg_{-\vec{s}}\mathbf{P}}\\
 & = & x^{\vec{b}}\left(\mathbf{M}^{T}(1/x)\right)\left(\mathbf{P}(1/x)\right)x^{\cdeg_{-\vec{s}}\mathbf{P}}\\
 & = & x^{\vec{b}}\left(\mathbf{M}^{T}\mathbf{P}(1/x)\right)x^{\cdeg_{-\vec{s}}\mathbf{P}}
\end{eqnarray*}


We know that $\mathbf{P}$ contains a $(\mathbf{M}^{T},-\vec{s})$-kernel
basis since any $(\mathbf{M}^{T},-\vec{s})$-kernel basis has $-\vec{s}$-column
degrees bounded by that of $\mathbf{F}^{rT}$, which is the same as
that of $\mathbf{F}^{T}$ and no more than $0$.\end{proof}
\end{comment}


We are now ready to look at the algorithm for computing a $\left(\mathbf{M}^{T},\vec{b}+1,-\vec{s}\right)$-basis,
given in \prettyref{alg:unimodularCompletion}.  We follow the same
process as in \prettyref{sec:computeRightFactor}. We assume without
loss of generality that the rows of $\mathbf{M}^{T}$ are arranged
in decreasing $\vec{s}$-row degrees. We divide $\mathbf{M}^{T}$
into $\log k$ row blocks according to the $\vec{s}$-row degrees
of its rows. Let 
\[
\mathbf{M}^{T}=\left[\mathbf{M}_{1}^{T},\mathbf{M}_{2}^{T},\cdots,\mathbf{M}_{\log k-1}^{T},\mathbf{M}_{\log k}^{T}\right]
\]
 with $\mathbf{M}_{\log k},\mathbf{M}_{\log k-1},\cdots,\mathbf{M}_{2},\mathbf{M}_{1}$
having $\vec{s}$-row degrees in the range 
\[
\left[0,2\xi/k\right],\ (2\xi/k,4\xi/k],\ (4\xi/k,8\xi/k],\ ...,\ (\xi/4,\xi/2],\ (\xi/2,\xi].
\]
 Let $\vec{\sigma}_{i}=\left[\xi/2^{i-1}+1,\dots,\xi/2^{i-1}+1\right]$
with the same dimension as the row dimension of $\mathbf{M}_{i}$.
Let $\vec{\sigma}=\left[\vec{\sigma}_{\log k},\vec{\sigma}_{\log k-1},\dots,\vec{\sigma}_{1}\right]$
be the order in the order basis computation. For simplicity, instead
of using $\mathbf{M}^{T}$ as the input matrix, we use 
\begin{eqnarray*}
\hat{\mathbf{M}} & =\begin{bmatrix}\hat{\mathbf{M}}_{1}\\
\vdots\\
\hat{\mathbf{M}}_{\log k}
\end{bmatrix}= & x^{\vec{\sigma}-\vec{b}-1}\begin{bmatrix}\mathbf{M}_{1}\\
\vdots\\
\mathbf{M}_{\log k}
\end{bmatrix}=x^{\vec{\sigma}-\vec{b}-1}\mathbf{M}
\end{eqnarray*}
 instead, so that a $\left(\hat{\mathbf{M}},\vec{\sigma},-\vec{s}\right)$-basis
is a $\left(\mathbf{M},\vec{b}+1,-\vec{s}\right)$-basis.



\begin{comment}
kjh
\begin{lem}
If $\mathbf{A}$ is $\vec{u}$-row reduced and has an identity matrix
GCD, and if $\mathbf{P}$ is a $\left(\mathbf{A},\vec{v},-\vec{u}\right)$-basis
that contains a complete kernel basis of $\mathbf{A}$, then $\det\mathbf{P}=x^{\sum\vec{v}}$
or equivalently, $\sum\cdeg_{-\vec{u}}\mathbf{P}=\sum\vec{v}-\sum\vec{u}$.\end{lem}
\begin{proof}
In general, we have $\det\mathbf{P}\le x^{\sum\vec{v}}$, since each
increase of the $-\vec{u}$ column degree of some column of $\mathbf{P}$
also increases the order of at least one row by one. Therefore, to
show the equality holds in our special case here, we just need to
show that $\det\mathbf{P}\nless x^{\sum\vec{v}}$. If $\mathbf{A}=\left[I,0\right]$,
then it is not difficult to see that the lemma is true, since $\diag\left(\left[x^{\vec{v}},I\right]\right)$
is a $\left(\mathbf{A},\vec{v},-\vec{u}\right)$-basis.

Note that there is a $\left(\mathbf{A},\vec{v},-\vec{u}\right)$-basis
$\mathbf{Q}$ such that $\mathbf{A}\mathbf{Q}=\left[x^{\vec{v}},0\right]$.
\end{proof}
But the equality hold in the special case here.
\begin{lem}
Let $\mathbf{P}=\left[\mathbf{P}_{1},\mathbf{P}_{2}\right]$ be a
$\left(\mathbf{M}^{T},\vec{b}+1,-\vec{s}\right)$-basis as before. 
\end{lem}
The only component we need to consider 

To compute a unimodular completion of $\mathbf{F}$ with column degrees
bounded component-wise by $\vec{s}$ (or equivalently, $-\vec{s}$-row
degrees bounded by 0), let us first compute a $\vec{s}$-minimal kernel
basis $\mathbf{N}$ of $\mathbf{F}$. Let $\vec{b}$ be the $\vec{s}$-column
degrees of $\mathbf{N}$. Let $\bar{\mathbf{N}}=\colRev\left(\mathbf{N},\vec{s},\vec{b}\right)^{T}$.
We then compute a $\left(\bar{\mathbf{N}},-\vec{s},\vec{b}+1\right)$-basis
$\bar{\mathbf{P}}$, which would contain a complete kernel basis $\bar{\mathbf{F}}$
of $\bar{\mathbf{N}}$ since the row degrees of $\bar{\mathbf{N}}\bar{\mathbf{F}}$
are bounded by the corresponding entries of $\vec{b}$ and $\order(\bar{\mathbf{N}},\bar{\mathbf{P}})$
is greater than $\vec{b}$ component-wise . Note that the $-\vec{s}$-column
degrees of $\bar{\mathbf{P}}$ are bounded by $1$ since the $-\vec{s}$-row
degrees of $\bar{\mathbf{F}}^{T}$ are bounded by that of $\mathbf{F}$,
which are bounded by 0.
\begin{lem}
Given $\mathbf{F}\in\mathbb{K}\left[x\right]^{m\times n}$ with $\cdeg\mathbf{F}\le\vec{s}$
(or equivalently, $\rdeg_{-\vec{s}}\mathbf{F}\le0$). If $\mathbf{N}$
is a $\left(\mathbf{F},\infty,\vec{s}\right)$-basis with $\cdeg_{\vec{s}}\mathbf{N}=\vec{b}$, \end{lem}
\end{comment}


We now do a series of order basis computations in order to compute
a unimodular completion of $\mathbf{F}$ based on \prettyref{lem:unimodularComputationByRows}.

Let $\vec{s}_{1}=\vec{s}$. First we compute an $\left(\hat{\mathbf{M}}_{1},\vec{\sigma}_{1},-\vec{s}_{1}\right)$-basis
$\mathbf{P}_{1}=\left[\mathbf{N}_{1},\mathbf{Q}_{1}\right]$, where
$\mathbf{N}_{1}$ is a $\left(\hat{\mathbf{M}}_{1},-\vec{s}_{1}\right)$-kernel
basis%
\begin{comment}
 with $\cdeg_{-\vec{s}_{1}}\mathbf{N}_{1}\le0$
\end{comment}
. This computation can be done using \prettyref{alg:umab} with a
cost of $O^{\sim}\left(n^{\omega}s\right)$, where $s=\xi/n$. 

Let $\tilde{\mathbf{N}}_{1}=\mathbf{N}_{1}$. Let $\vec{s}_{2}=-\cdeg_{-\vec{s}}\mathbf{N}_{1}$
and $\vec{t}_{2}=-\cdeg_{-\vec{s}}\mathbf{Q}_{1}$. %
\begin{comment}
Note that $-\vec{s}_{1}\le-[\vec{s}_{2},\vec{t}_{2}]\le\left[0,\dots,0,1,\dots1\right]$
component-wise, since $\mathbf{P}_{1}$ has lower order than any $\left(\mathbf{M}^{T},\vec{b}+1,-\vec{s}\right)$-basis
$\mathbf{P}$ hence generates $\mathbf{P}$. Therefore, $\cdeg_{-\vec{s}}\mathbf{P}_{1}\le\cdeg_{-\vec{s}}\mathbf{P}\le\left[0,\dots,0,1,\dots1\right]$. 
\end{comment}
{} We then compute an $\left(\hat{\mathbf{M}}_{2}\tilde{\mathbf{N}}_{1},\vec{\sigma}_{2},-\vec{s}_{2}\right)$-basis
$\mathbf{P}_{2}=\left[\mathbf{N}_{2},\mathbf{Q}_{2}\right]$ with
$\vec{s}_{3}=-\cdeg_{-\vec{s}_{2}}\mathbf{N}_{2}$ and $\vec{t}_{3}=-\cdeg_{-\vec{s}_{2}}\mathbf{Q}_{2}$.
Let $\tilde{\mathbf{N}}_{2}=\tilde{\mathbf{N}}_{1}\mathbf{N}_{2}$
 %
\begin{comment}
Let $\mathbf{R}_{1}=\left[\mathbf{N}_{1}\mathbf{Q}_{2},\mathbf{Q}_{1}\right]$
and $\mathbf{R}_{1}^{r}=\colRev\left(\mathbf{R}_{1},-\vec{s},\cdeg_{-\vec{s}}\mathbf{R}_{1}\right)$.
Then from \prettyref{lem:unimodularComputationByRows} we know $\left[\mathbf{F}^{T},\mathbf{R}_{1}^{r}\right]$
is a unimodular matrix.
\end{comment}


Continue this process, at step $i$, we compute an $\left(\hat{\mathbf{M}}_{i}\tilde{\mathbf{N}}_{i-1},\vec{\sigma}_{i},-\vec{s}_{i}\right)$-basis
$\mathbf{P}_{i}=\left[\mathbf{N}_{i},\mathbf{Q}_{i}\right]$. Let
$\tilde{\mathbf{N}}_{i}=\prod_{j=1}^{i}\mathbf{N}_{i}=\tilde{\mathbf{N}}_{i-1}\mathbf{N}_{i}$.
Note that $\tilde{\mathbf{N}}_{\log k}$ is a $\left(\mathbf{M},-\vec{s}\right)$-kernel
basis. Let 
\[
\mathbf{R}=\left[\mathbf{Q}_{1},\tilde{\mathbf{N}}_{1}\mathbf{Q}_{2},\dots,\tilde{\mathbf{N}}_{\log k-2}\mathbf{Q}_{\log k-1},\tilde{\mathbf{N}}_{\log k-1}\mathbf{Q}_{\log k}\right]
\]
, and $\mathbf{R}^{r}=\colRev\left(\mathbf{R},-\vec{s},\cdeg_{-\vec{s}}\mathbf{R}\right)$,
then from \prettyref{lem:unimodularComputationByRows} we can conclude
that $\left[\mathbf{F}^{T},\mathbf{R}^{r}\right]$ is a unimodular
matrix. 

\begin{comment}
Note that rank sensitive computation for order basis is not as natural,
since for order basis we work with power series in general, whose
rank may never be truly computed. In addition, the computed basis
does not correspond to the rank. For these reasons, we do not pursue
rank sensitive computations for order basis. 
\end{comment}
\begin{algorithm}[t]
\caption{$\unimodularCompletion(\mathbf{F})$}
\label{alg:unimodularCompletion}

\begin{algorithmic}[1]
\REQUIRE{$\mathbf{F}\in\mathbb{K}\left[x\right]^{m\times n}$ with full row
rank; $\vec{s}$ is initially set to the column degrees of $\mathbf{F}$.
It keeps track of the degrees.}

\ENSURE{$\mathbf{G}\in\mathbb{K}\left[x\right]^{\left(n-m\right)\times n}$
such that $\begin{bmatrix}\mathbf{F}\\
\mathbf{G}
\end{bmatrix}$ is unimodular.}

\STATE{$\vec{s}:=\cdeg\mathbf{F}$;}

\STATE{$\mathbf{F}^{r}:=\rowRev\left(\mathbf{F},-\vec{s},0\right)$;}

\STATE{$\mathbf{M}:=\mnb(\mathbf{F}^{r},\vec{s})$; $\vec{b}:=\cdeg_{\vec{s}}\mathbf{M}$;
$k:=n-m;$}

\STATE{\textbf{$\left[\mathbf{M}_{1}^{T},\mathbf{M}_{2}^{T},\cdots,\mathbf{M}_{\log k-1}^{T},\mathbf{M}_{\log k}^{T}\right]:=\mathbf{M}$},
with $\mathbf{M}_{\log k},\mathbf{M}_{\log k-1},\cdots,\mathbf{M}_{2},\mathbf{M}_{1}$
having $\vec{s}$-row degrees in the range $\left[0,2\xi/k\right],(2\xi/k,4\xi/k],...,(\xi/4,\xi/2],(\xi/2,\xi].$\textbf{ }}

\FOR{$i$ \textbf{from $1$ to $\log k$ }} 

\forbody{\STATE{$\vec{\sigma}_{i}:=\left[\xi/2^{i-1}+1,\dots,\xi/2^{i-1}+1\right]$,
with the number of entries matches the row dimension of $\mathbf{M}_{i};$}}

\STATE{$\vec{\sigma}:=\left[\vec{\sigma}_{\log k},\vec{\sigma}_{\log k-1},\dots,\vec{\sigma}_{1}\right]$;}

\STATE{$\hat{\mathbf{M}}:=x^{\vec{\sigma}-\vec{b}-1}\mathbf{M};$}

\STATE{$\mathbf{N}_{0}:=I_{n}$; $\tilde{\mathbf{N}}_{0}:=I_{n};$}

\FOR{$i$ \textbf{from $1$ to $\log k$ }} 

\forbody{\STATE{$\vec{s}_{i}:=-\cdeg_{-\vec{s}}\mathbf{N}_{i-1};$ (note $\vec{s}_{1}=\vec{s}$)}

\STATE{$\mathbf{P}_{i}:=\umab\left(\hat{\mathbf{M}}_{i}\tilde{\mathbf{N}}_{i-1},\vec{\sigma}_{i},-\vec{s}_{i}\right)$; }

\STATE{$\left[\mathbf{N}_{i},\mathbf{Q}_{i}\right]:=\mathbf{P}_{i}$, where
$\mathbf{N}_{i}$ is a $\left(\hat{\mathbf{M}}_{i},-\vec{s}_{i}\right)$-nullspace
basis;}

\STATE{$\tilde{\mathbf{N}_{i}}:=\tilde{\mathbf{N}}_{i-1}\cdot\mathbf{N}_{i};$ }

\STATE{$\mathbf{R}:=\left[\mathbf{R},\tilde{\mathbf{N}}_{i-1}\mathbf{Q}_{i}\right]$;}}

\STATE{$\mathbf{R}^{r}:=\colRev\left(\mathbf{R},-\vec{s},\cdeg_{-\vec{s}}\mathbf{R}\right);$}

\RETURN $\left(\mathbf{R}^{r}\right)^{T}$ 
\end{algorithmic}
\end{algorithm}



We still need to check the cost of the multiplications $\hat{\mathbf{M}}_{i}\tilde{\mathbf{N}}_{i-1}$,
$\tilde{\mathbf{N}}_{i-1}\mathbf{N}_{i}$, and $\tilde{\mathbf{N}}_{i-1}\mathbf{Q}_{i}$. 
\begin{lem}
The multiplications $\hat{\mathbf{M}}_{i}\tilde{\mathbf{N}}_{i-1}$
can be done with a cost of $O^{\sim}\left(n^{\omega}s\right)$.\end{lem}
\begin{proof}
The dimension of $\hat{\mathbf{M}}_{i}$ is bounded by $2^{i-1}\times n$
and $\sum\rdeg_{\vec{s}}\hat{\mathbf{M}}_{i}\le2^{i-1}\cdot\xi/2^{i-1}=\xi$.
We also have $\cdeg_{-\vec{s}}\tilde{\mathbf{N}}_{i-1}\le0$, or equivalently,
$\rdeg\tilde{\mathbf{N}}_{i-1}\le\vec{s}$. We can now use \prettyref{thm:multiplyUnbalancedMatrices}
to multiply $\tilde{\mathbf{N}}_{i-1}^{T}$ and $\hat{\mathbf{M}}_{i}^{T}$
with a cost of $O^{\sim}\left(n^{\omega}s\right)$.\end{proof}
\begin{lem}
The multiplication $\tilde{\mathbf{N}}_{i-1}\mathbf{N}_{i}$ can be
done with a cost of $O^{\sim}\left(n^{\omega}s\right)$.\end{lem}
\begin{proof}
We know $\cdeg_{-\vec{s}}\tilde{\mathbf{N}}_{i-1}=-\vec{s}_{i}$,
and $\cdeg_{-\vec{s}_{i}}\mathbf{N}_{i}=-\vec{s}_{i+1}\le0.$ In other
words, $\rdeg\mathbf{N}_{i}\le\vec{s}_{i}$, and $\rdeg_{\vec{s}_{i}}\tilde{\mathbf{N}}_{i-1}\le\vec{s}$,
hence we can again use \prettyref{thm:multiplyUnbalancedMatrices}
to multiply $\mathbf{N}_{i}^{T}$ and $\tilde{\mathbf{N}}_{i-1}^{T}$
with a cost of $O^{\sim}\left(n^{\omega}s\right)$.\end{proof}
\begin{lem}
The multiplication $\tilde{\mathbf{N}}_{i-1}\mathbf{Q}_{i}$ can be
done with a cost of $O^{\sim}\left(n^{\omega}s\right)$.\end{lem}
\begin{proof}
We know $\cdeg_{-\vec{s}_{i}}\mathbf{Q}_{i}\le\max\cdeg_{\vec{s}}\mathbf{P}=1$,
or equivalently, $\rdeg\mathbf{Q}_{i}\le\vec{s}_{i}+1$. But we also
know that this $\mathbf{Q}_{i}$ from the order basis computation
has a factor $xI$. Therefore, $\rdeg\left(\mathbf{Q}_{i}/x\right)\le\vec{s}_{i}$.
In addition, $\rdeg_{\vec{s}_{i}}\tilde{\mathbf{N}}_{i-1}\le\vec{s}$
as before. So we can again use \prettyref{thm:multiplyUnbalancedMatrices}
to multiply $\mathbf{Q}_{i}^{T}$ and $\tilde{\mathbf{N}}_{i-1}^{T}$
with a cost of $O^{\sim}\left(n^{\omega}s\right)$.\end{proof}
\begin{thm}
A unimodular completion of $\mathbf{F}$ can be computed with a cost
of $O^{\sim}\left(n^{\omega}s\right)$ field operations.\end{thm}

