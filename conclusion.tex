
\chapter{Conclusion\label{sec:Future-Research}}

In this thesis, we have presented efficient deterministic algorithms
for a number of polynomial matrix computation problems, including
the computation of order basis, minimal nullspace basis, matrix inverse,
column basis, unimodular completion, determinant, and Hermite normal
form. The algorithm for nullspace basis computation also immediately
allows us to solve linear systems. The algorithm for column basis
also immediately allows us to compute matrix GCD, column reduced forms
and Popov normal forms for matrices of any dimension. 

We first gave algorithms for computing a shifted order basis of an
$m\times n$ matrix of power series over a field $\mathbb{K}$ with
$m\le n$. For a given order $\sigma$ and balanced shift $\vec{s}$
the first algorithm determines an order basis with a cost of $O^{\sim}(n^{\omega-1}m\sigma)$
field operations in $\mathbb{K}$, where $\omega$ is the exponent
of matrix multiplication. Here an input shift is balanced when $\max(\vec{s})-\min(\vec{s})\in O(m\sigma/n)$.
%Here $O^{\sim}$
%is just $O$ with log factors omitted and 
%$\MM\left(n,d\right)$ denotes
%the cost of multiplying two polynomial matrices with dimension $n$
%and degree $d$. 
\begin{comment}
This extends earlier work of Storjohann which only determines a subset
of an order basis that is within a specified degree bound $\delta$
using $O^{\sim}(n^{\omega}\delta)$ field operations for $\delta\ge\lceil m\sigma/n\rceil$.
\end{comment}
{} While the first algorithm addresses the case when the column degrees
of a complete order basis are unbalanced given a balanced input shift,
it is not efficient in the case when an unbalanced shift results in
the row degrees also becoming unbalanced. %The column degrees of a complete basis may be unbalanced, which is
%a major issue we address in the first algorithm. When the input shift
%is unbalanced, the row degrees of the basis can also be unbalanced
%in addition to the unbalanced column degrees. For this, we present
We have presented a second algorithm which balances the high degree
rows and computes an order basis also using $O^{\sim}(n^{\omega}\lceil m\sigma/n\rceil)$
field operations in the case that the shift is unbalanced but satisfies
the condition $\sum_{i=1}^{n}(\max(\vec{s})-\vec{s}_{i})\le m\sigma$.%
\begin{comment}
Every problem with any unbalanced shift can be in fact reduced to
a problem with a shift that satisfying this condition if the degrees
of a resulting order basis is known. 
\end{comment}
{} %
\begin{comment}
Many unbalanced shift problems can be in fact converted to problems
satisfying this condition. 
\end{comment}
{} This condition essentially allows us to locate those high degree
rows that need to be balanced. %
\begin{comment}
In more general unbalanced shift cases, this algorithm may not work
well directly since we do not know in advance which are the high degree
rows need to be balanced. But it may work efficiently if we have an
effective way of estimating the resulting row degrees. 
\end{comment}
{} %
\begin{comment}
This extends the earlier work by the authors from ISSAC'09.
\end{comment}


We then presented an algorithm for the computation of a minimal nullspace
basis of an $m\times n$ input matrix of univariate polynomials over
a field $\mathbb{K}$ with $m\le n$. This algorithm computes a minimal
nullspace basis of a degree $d$ input matrix with a cost of $O^{\sim}\left(n^{\omega-1}md\right)$
field operations in $\mathbb{K}$. %
\begin{comment}
Here the soft-$O$ notation is Big-$O$ with log factors removed while
$\omega$ is the exponent of matrix multiplication.
\end{comment}
{} The same algorithm also works in the more general situation on computing
a shifted minimal nullspace basis, with a given degree shift $\vec{s}\in\mathbb{Z}^{n}$
whose entries bound the corresponding column degrees of the input
matrix. In this case if $\rho$ is the sum of the $m$ largest entries
of $\vec{s}$, then a $\vec{s}$-minimal right nullspace basis can
be computed with a cost of $O^{\sim}(n^{\omega}\rho/m)$ field operations. 

Order basis computation and nullspace basis computation were then
applied to the remaining problems. An algorithm for computing the
inverse of an matrix in $\mathbb{K}\left[x\right]^{n\times n}$ was
then given with a cost of $O^{\sim}\left(n^{2}\xi\right)$ field operations,
where $\xi$ is the sum of column or row degrees of the input matrix.
The inverse represented alternatively by a product of $\left\lceil \log n\right\rceil $
matrices costs only $O^{\sim}\left(n^{\omega-1}\xi\right)$ to compute.
We then discussed the computation of a column basis of an input matrix
in $\mathbb{K}\left[x\right]^{m\times n}$ with a cost of $O^{\sim}\left(m^{\omega-1}\xi\right)$,
where $\xi$ is the sum of column degrees of the input matrix. Next,
an algorithm was presented for computing an unimodular completion
of an input matrix in $\mathbb{K}\left[x\right]^{m\times n}$, $m<n$
with a cost of $O^{\sim}\left(n^{\omega-1}\xi\right)$, where $\xi$
is the sum of column degrees of the input matrix. Then an algorithm
for computing the determinant of an input matrix in $\mathbb{K}\left[x\right]^{n\times n}$
with a cost of $O^{\sim}\left(n^{\omega-1}\xi\right)$ was given,
where $\xi$ is the sum of column or row degrees of the input matrix.
Then we looked at an algorithm for computing the Hermite normal form
of a degree $d$ input matrix in $\mathbb{K}\left[x\right]^{n\times n}$
with a cost of $O^{\sim}\left(n^{\omega}d\right)$. Finally, we provided
algorithms for rank-sensitive computations of the rank and rank profile
of an input matrix in $\mathbb{K}\left[x\right]^{m\times n}$ with
a cost of $O^{\sim}\left(mr^{\omega-2}\xi\right)$, and then applied
the rank profile algorithm to rank-sensitive computation of minimal
kernel basis to obtain a cost of $O^{\sim}(nmr^{\omega-2}d+n^{\omega-1}rd)$.

For all these problems, our algorithms are not only essentially optimal,
but in most cases they also solve more general versions of these problems
than previous algorithms. More specifically, the computational costs
are stated with the  sum of the column degrees of the input matrix,
which is more precise and tighter than using just degrees. In addition,
the shifted minimal bases computed by our algorithms are more general
than the standard minimal bases.
