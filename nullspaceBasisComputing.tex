
\section{Minimal Kernel Basis Computation}

\label{sec:Nullspace-Basis-Computation}

In this section, we describe a new, efficient algorithm for computing
a shifted minimal kernel basis. The algorithm uses two computation
processes recursively. The first process, described in \prettyref{sub:continueComputingNullspaceBasisByColumns},
uses an order basis computation to compute a subset of kernel basis
elements of lower degree, and results in a new problem of lower column
dimension. The second process, described in \prettyref{sub:continueComputingNullspaceBasisByRows},
reduces the row dimension of the problem by computing a kernel basis
of a submatrix formed by a subset of the rows of the input matrix.

We assume that the row dimension $m$ is bounded by the column dimension
$n$ in this chapter. But this assumption is later removed in \prettyref{sub:removeDimensionAssumption}
with results from \prettyref{chap:rank}.

We require that the shift $\vec{s}$ bounds the column degrees of
$\mathbf{F}$ component-wise. For example, we can set each entry of
$\vec{s}$ to be the corresponding column degree of $\mathbf{F}$,
or we can simply set each entry of $\vec{s}$ to be the maximum column
degree of $\mathbf{F}$. %As it may become evident later in this section, this condition on
%the shift is very useful, as it provides a simple bound on the column
%degrees of $\mathbf{F} \cdot \mathbf{A}$ for any polynomial matrix $\mathbf{A}$. 
 This is a very useful condition as it helps us to keep track of and
bound the shifted degrees throughout the kernel basis computation,
as we will see in \prettyref{sub:boundsBasedOnShift}. In the remainder
of this thesis, when we say a list $\vec{s}=\left[s_{1},\dots,s_{n}\right]$
bounds another list $\vec{t}=\left[t_{1},\dots,t_{n}\right]$ or if
we write $\vec{s}\ge\vec{t}$, we mean $\vec{s}$ bound $\vec{t}$
component-wise, that is, each $s_{i}\ge t_{i}$ for each $i$ in $\left\{ 1,\dots,n\right\} $. 

For simplicity, we will also assume without loss of generality that
the columns of $\mathbf{F}$ and the corresponding entries of $\vec{s}=\left[s_{1},\dots,s_{n}\right]$
are arranged so that the entries of $\vec{s}$ are in increasing order.

Let $\rho=\sum_{n-m+1}^{n}s_{i}$ be the sum of $m$ largest entries
of $\vec{s}$, and $s=\rho/m$ be their average. The algorithm we
present in this section computes a $\vec{s}$-minimal null space basis
$\mathbf{N}$ with a cost of $O^{\sim}(n^{\omega}s)$ field operations.
For uniform shift $\vec{s}=\left[s,\dots,s\right]$, we improve this
later to $O^{\sim}\left(n^{\omega-1}ms\right)=O^{\sim}\left(n^{\omega-1}\rho\right)$.


\subsection{\label{sub:boundsBasedOnShift}Bounds based on the shift}

A key requirement for efficient computation is making sure that the
intermediate computations do not blow up in size. We will see that
this requirement is satisfied by the existence of a bound, $\xi=\sum\vec{s}=\sum_{i=1}^{n}{s}_{i},$
\begin{comment}
the sum of all entries of the initial input shift $\vec{s}$, 
\end{comment}
{} on the sum of all entries of the input shift of all subproblems throughout
the computation.%
\begin{comment}
, as we will see later in \prettyref{lem:boundOfSumOfShiftedDegreesOfOrderBasis},
\prettyref{thm:boundOfSumOfShiftedDegreesOfKernelBasis}, and \prettyref{lem:boundOnShiftedDegrees}
that this quantity bounds the sum of all entries of the input shift
of all subproblems throughout the computation,. 
\end{comment}
{} 



The following lemma shows that any $(\mathbf{F},\sigma,\vec{s})$-basis
contains a partial $\vec{s}$-minimal kernel basis of $\mathbf{F}$,
and as a result, any $(\mathbf{F},\sigma,\vec{s})$-basis with high
enough $\sigma$ contains a $\vec{s}$-minimal kernel basis of $\mathbf{F}$. 
\begin{lem}
\label{lem:orderBasisContainsNullspaceBasis}Let $\mathbf{P}=\left[\mathbf{P}_{1},\mathbf{P}_{2}\right]$
be any $\left(\mathbf{F},\sigma,\vec{s}\right)$-basis and $\mathbf{N}=\left[\mathbf{N}_{1},\mathbf{N}_{2}\right]$
be any $\vec{s}$-minimal kernel basis of $\mathbf{F}$, where $\mathbf{P}_{1}$
and $\mathbf{N}_{1}$ contain all columns from $\mathbf{P}$ and $\mathbf{N}$,
respectively, whose $\vec{s}$-column degrees are less than $\sigma$.
Then $\left[\mathbf{P}_{1},\mathbf{N}_{2}\right]$ is a $\vec{s}$-minimal
kernel basis of $\mathbf{F}$, and $\left[\mathbf{N}_{1},\mathbf{P}_{2}\right]$
is a $\left(\mathbf{F},\sigma,\vec{s}\right)$-basis.\end{lem}
\begin{proof}
From \prettyref{lem:boundOnDegreesOfFA}, any column $\mathbf{p}$
of $\mathbf{P}_{1}$ satisfies $\deg\mathbf{F}\mathbf{p}\le\deg_{\vec{s}}\mathbf{p}<\sigma$.
Combining this with the fact that $\mathbf{F}\mathbf{p}\equiv0\mod x^{\sigma}$
we get $\mathbf{F}\mathbf{p}=0$. Thus $\mathbf{P}_{1}$ is generated
by $\mathbf{N}_{1}$, that is, $\mathbf{P}_{1}=\mathbf{N}_{1}\mathbf{U}$
for some polynomial matrix $\mathbf{U}$. On the other hand, $\mathbf{N}_{1}$
has order $\left(\mathbf{F},\sigma\right)$ and %
\begin{comment}
is therefore generated by $\mathbf{P}_{1}$, i.e. 
\end{comment}
{} therefore satisfies $\mathbf{N}_{1}=\mathbf{P}_{1}\mathbf{V}$ for
some polynomial matrix $\mathbf{V}$. We now have $\mathbf{P}_{1}=\mathbf{P}_{1}\mathbf{V}\mathbf{U}$
and $\mathbf{N}_{1}=\mathbf{N}_{1}\mathbf{U}\mathbf{V}$, implying
both $\mathbf{U}$ and $\mathbf{V}$ are unimodular. The result then
follows from the unimodular equivalence of $\mathbf{P}_{1}$ and $\mathbf{N}_{1}$
and the fact that they are $\vec{s}$-column reduced. %reducedness of .
\end{proof}
We can now provide a simple bound on the $\vec{s}$-minimal kernel
basis of $\mathbf{F}$. 
\begin{thm}
\label{thm:boundOfSumOfShiftedDegreesOfKernelBasis}Suppose $\mathbf{F}\in\mathbb{K}\left[x\right]^{m\times n}$
and $\vec{s}\in\mathbb{Z}_{\ge0}^{n}$ is a shift with entries bounding
the corresponding column degrees of $\mathbf{F}$. Then the sum of
the $\vec{s}$-column degrees of any $\vec{s}$-minimal kernel basis
of $\mathbf{F}$ is bounded by $\xi=\sum\vec{s}$.\end{thm}
\begin{proof}
Let $\mathbf{P}$ be a $(\mathbf{F},\sigma,\vec{s})$-basis with high
enough order $\sigma$ so that $\mathbf{P}=\left[\mathbf{N},\bar{\mathbf{N}}\right]$
contains a complete kernel basis, $\mathbf{N}$, of $\mathbf{F}$.
By \prettyref{lem:orderBasisContainsNullspaceBasis} we just need
$\sigma$ to be greater than the $\vec{s}$-column degree %
\begin{comment}
$\delta$ 
\end{comment}
of a $\vec{s}$-minimal kernel basis of $\mathbf{F}$. %
\begin{comment}
To see this, let $k$ be the column dimension of any kernel basis
of $\mathbf{F}$, and $\mathbf{P}'$ be the submatrix of $\mathbf{P}$
consists of the columns with the $k$ lowest $\vec{s}$-column degrees.
Note that the minimality of $\mathbf{P}$ implies that the $\vec{s}$-column
degrees of $\mathbf{P}'$ are no more that of any $\vec{s}$-minimal
kernel basis of $\mathbf{F}$. For any column $\mathbf{p}$ of $\mathbf{P}'$
we also have $\deg\mathbf{F}\mathbf{p}\le\deg_{\vec{s}}\mathbf{p}$
by \prettyref{lem:boundOnDegreesOfFA}. Hence we have $\deg\mathbf{F}\mathbf{p}\le\deg_{\vec{s}}\mathbf{p}\le\delta<\sigma$.
Combining this with the fact that $\mathbf{F}\mathbf{p}\equiv0\mod x^{\sigma}$
we get $\mathbf{F}\mathbf{p}=0$, which means $\mathbf{P}'$, whose
$k$ columns are linearly independent, is also a kernel basis of $\mathbf{F}$.
In fact, although not needed for the proof of this theorem, it is
not difficult to show from here that $\mathbf{P}'$ is in fact a $\vec{s}$-minimal
kernel basis because its $\vec{s}$-column degrees are no more than
that of a $\vec{s}$-minimal kernel basis of $\mathbf{F}$. 
\end{comment}
Let $r$ be the column dimension of $\bar{\mathbf{N}}$. Note that
this is the same as the rank of $\mathbf{F}$. By \prettyref{lem:boundOfSumOfShiftedDegreesOfOrderBasis}
the sum of the $\vec{s}$-column degrees of $\mathbf{P}$ is at most
$\xi+r\sigma$. By \prettyref{lem:boundOnDegreesOfFA} the sum of
the $\vec{s}$-column degrees of $\bar{\mathbf{N}}$ is greater than
or equal to the sum of the column degrees of $\mathbf{F}\cdot\bar{\mathbf{N}}$,
which is at least $r\sigma$, since every column of $\mathbf{F}\bar{\mathbf{N}}$
is nonzero and has order $\sigma$. So the sum of the $\vec{s}$-column
degrees of $\mathbf{N}$ is bounded by $\xi+r\sigma-r\sigma=\xi$. 
\end{proof}
\prettyref{thm:boundOfSumOfShiftedDegreesOfKernelBasis} specializes
to the following well-known results in the case of uniform shift:
\begin{cor}
Given $\mathbf{F}\in\mathbb{K}\left[x\right]^{m\times n}$ with degree
$d$ . The sum of the column degree of its minimal kernel basis is
bounded by $md$.\end{cor}
\begin{proof}
From \prettyref{thm:boundOfSumOfShiftedDegreesOfKernelBasis} and
the definition of shifted column degrees we have 
\[
nd\ge\sum\cdeg_{\left[d,\dots d\right]}\mathbf{N}\ge(n-m)d+\sum\cdeg\mathbf{N},
\]
 which gives 
\[
\sum\cdeg\mathbf{N}\le nd-(n-m)d=md.
\]

\end{proof}
\begin{comment}
Our intermediate computations should not blow up in size but should
also not be too expensive. Our algorithm will reduce a single null
space computation to a set of null space computations of smaller size.
These smaller problems are constructed by the computations of residuals
which in all cases involves the multiplication of two polynomial matrices
having unbalanced degrees. The following lemma, whose proof we defer
until a later section, says that this can be done efficiently.
\begin{lem}
\label{lem:multiplyUnbalancedMatrices-1} Let $\vec{s}$ be a shift
ordered in terms of increasing values and $\xi$, a bound on the sum
of the entries of $\vec{s}$. Let $\mathbf{A}\in\mathbb{K}\left[x\right]^{m\times n}$,
with column degrees bounded by $\vec{s}$ and $\mathbf{B}\in\mathbb{K}\left[x\right]^{n\times k}$
with $k\in O\left(m\right)$ and the sum $\theta$ of its $\vec{s}$-column
degrees satisfying $\theta\in O\left(\xi\right)$. Then we can multiply
$\mathbf{A}$ and $\mathbf{B}$ with a cost of $O^{\sim}(nm^{\omega-2}\xi)$. \end{lem}
\end{comment}


%For simplicity, we also assume without loss of generality that the
%columns of $\mathbf{F}$ and the corresponding entries of $\vec{s}=\left[s_{1},\dots,s_{n}\right]$
%are arranged so that the entries of $\vec{s}$ are in increasing order.
%%\begin{comment}
%A convenient bound throughout the computation is provided by $\xi=\sum\vec{s}$
%be.
%%\end{comment}
%{} 
%Let $\rho=\sum_{n-m+1}^{n}s_{i}$ be the sum of $m$ largest entries
%of $\vec{s}$, and $s=\rho/m$ be their average. The algorithm we
%present in this section computes a $\vec{s}$-minimal null space basis
%$\mathbf{N}$ with a cost of $O^{\sim}(n^{\omega}s)$ field operations.
%For uniform shift $\vec{s}=\left[s,\dots,s\right]$, we improve this
%later to $O^{\sim}\left(n^{\omega}\left\lceil ms/n\right\rceil \right)=O^{\sim}\left(n^{\omega}\left\lceil \rho/n\right\rceil \right)=O^{\sim}\left(n^{\omega-1}\left\lceil m\xi/n\right\rceil \right)$.



\subsection{\label{sub:continueComputingNullspaceBasisByColumns}Reducing the
column dimension via order basis computation}

In this subsection we look at how an order basis computation can be
used to reduce the column dimension of our problem. While order basis
computations were also used in \citep{storjohann-villard:2005} to
reduce the column dimensions of their problems, here order basis computations
are used in a more comprehensive way. In particular, \prettyref{thm:continueComputingNullspaceBasisByColumns}
given later in this section, allows us to maintain the minimality
of the bases with the use of the shifted degrees and the residuals.

We begin by computing a $\left(\mathbf{F},3s,\vec{s}\right)$-basis
$\mathbf{P}$, which can be done with a cost of $O^{\sim}\left(n^{\omega}s\right)$
using the algorithm from \citet{Giorgi2003}. Note that if $\vec{s}$
is balanced, then we can compute this with a cost of $O^{\sim}\left(n^{\omega-1}\rho\right)$
using the algorithm from \prettyref{chap:OrderBasis}. We will show
that at most %$3m/2$ 
$\frac{3m}{2}$ columns of \textbf{$\mathbf{P}$} are not elements
of a kernel basis of $\mathbf{F}$. 
\begin{rem}
Note that it is not essential to choose $3s$ for the order. The order
can be set to $\ell s$ for any constant $\ell>1$. A smaller $\ell$
means less work to compute a $(\mathbf{F},\ell s,\vec{s})$-basis,
but also results in fewer kernel basis elements computed and leaves
more work for computing the remaining basis elements. On the other
hand, a larger $\ell$ means more work is needed for order basis computation,
but leaves less remaining work. It may be possible to better balance
these computations with a better choice of $\ell$. However, as we
will see later, the resulting complexity given in this paper would
remain the same for any $\ell>1$ as long as we use the big $O$ notation
and do not care about the constant factors in the cost. \end{rem}
\begin{thm}
\label{thm:dimensionOfPartialNullspaceBasisBasedOnOrder} Let $\mathbf{P}=[\mathbf{P}_{1},\mathbf{P}_{2}]$
be a $(\mathbf{F},\sigma,\vec{s})$-basis with $\sigma>s$ and $\mathbf{P}_{1}$
containing all columns $\mathbf{n}$ of $\mathbf{P}$ satisfying $\mathbf{F}\mathbf{n}=0$.
Then for $\ell=\sigma/s$ the column dimension $\kappa$ of $\mathbf{P}_{2}$
is bounded by $\frac{\ell m}{(\ell-1)}.$ \end{thm}
\begin{proof}
Any column $\mathbf{p}$ of $\mathbf{P}_{2}$ has order $\sigma$
but also satisfies $\mathbf{F}\mathbf{p}\ne0$. Thus the degree of
$\mathbf{F}\mathbf{p}$ must be at least $\sigma$ and, by \prettyref{lem:boundOnDegreesOfFA},
$\mathbf{p}$ must have $\vec{s}$-column degree at least $\sigma$.
It follows that the sum of the $\vec{s}$-column degrees of the columns
of $\mathbf{P}_{2}$ must satisfy $\sum\deg_{\vec{s}}\mathbf{P}_{2}\ge\kappa\sigma.$
From \prettyref{lem:boundOfSumOfShiftedDegreesOfOrderBasis} we know
that the sum of the $\vec{s}$-column degrees of the columns of $\mathbf{P}$
satisfies $\sum\deg_{\vec{s}}\mathbf{P}\le\sum\vec{s}+m\sigma,$ and
hence the sum of $\vec{s}$-column degrees of the columns of $\mathbf{P}_{1}$
must satisfy 
\[
\sum\deg_{\vec{s}}\mathbf{P}_{1}=\sum\deg_{\vec{s}}\mathbf{P}-\sum\deg_{\vec{s}}\mathbf{P}_{2}\le\sum\vec{s}+m\sigma-\kappa\sigma.
\]
 On the other hand, the lowest possible value of $\sum\deg_{\vec{s}}\mathbf{P}_{1}$
is $\sum_{i=1}^{n-\kappa}s_{i}$, the sum of the $n-\kappa$ smallest
entries of $\vec{s}$ (which occurs when $\mathbf{P}_{1}=\left[\mathbf{I},0\right]^{T}$).%
\begin{comment}
based on the assumption that the entries of $\vec{s}=\left[s_{1},\dots,s_{n}\right]$
are arranged in increasing order. 
\end{comment}
{} It follows that 
\[
\sum\vec{s}+m\sigma-\kappa\sigma\ge\sum_{i=1}^{n-\kappa}s_{i},
\]
 or, after rearrangement, 
\[
m\sigma\ge\kappa\sigma-\left(\sum\vec{s}-\sum_{i=1}^{n-\kappa}s_{i}\right).
\]
 Combining this with the fact that for $\kappa\ge m$ the average
of the $\kappa$ largest entries of $\vec{s}$ is no more than the
average of the $m$ largest entries of $\vec{s}$, that is, 
\[
\left(\sum\vec{s}-\sum_{i=1}^{n-\kappa}s_{i}\right)/\kappa\le s,\mbox{ or }\sum\vec{s}-\sum_{i=1}^{n-\kappa}s_{i}\le\kappa s,
\]
 we get $m\sigma\ge\kappa\sigma-\kappa s,$ which gives $\kappa\le m\sigma/(\sigma-s)$
for $\sigma>s$. Substituting in $\sigma=\ell s$, we get $\kappa\le\frac{\ell m}{(\ell-1)}$
as required. 
\end{proof}
Let $\left[\mathbf{P}_{1},\mathbf{P}_{2}\right]=\mathbf{P}$ with
$\mathbf{P}_{1}$ consisting of the kernel basis elements computed.
Then the residual $\mathbf{F}\mathbf{P}=\left[\mathbf{0},\mathbf{F}\mathbf{P}_{2}\right]$
can be used to compute the remaining kernel basis elements. Before
showing this can be correctly done, let us first make sure that the
matrix multiplication $\mathbf{F}\mathbf{P}_{2}$ can be done efficiently,
which may not be obvious since $\mathbf{F}$, $\mathbf{P}_{2}$, and
their product $\mathbf{F}\mathbf{P}_{2}$ can all have degrees up
to $\Theta(\xi)$. But we do have the sum of the column degrees of
$\mathbf{F}$, that of $\mathbf{F}\mathbf{P}_{2}$, and the sum of
the $\vec{s}$-column degrees of $\mathbf{P}_{2}$ all bounded by
$O(\xi)$, which means their total size are not too big but their
column degrees can be quite unbalanced. We will encounter this type
of multiplication again multiple times, for computing residuals and
combining results. In fact, almost all of the matrices in our kernel
basis computation can have such unbalanced degrees. To efficiently
multiply these matrices, we provide the following theorem, whose proof
we defer until the end of this section. 
\begin{thm}
\label{thm:multiplyUnbalancedMatrices} Let $\mathbf{A}\in\mathbb{K}\left[x\right]^{m\times n}$,
$\vec{s}$ a shift with entries bounding the column degrees of $\mathbf{A}$
and $\xi$, a bound on the sum of the entries of $\vec{s}$. Let $\mathbf{B}\in\mathbb{K}\left[x\right]^{n\times k}$
with $k\in O\left(m\right)$ and the sum $\theta$ of its $\vec{s}$-column
degrees satisfying $\theta\in O\left(\xi\right)$. Then we can multiply
$\mathbf{A}$ and $\mathbf{B}$ with a cost of $O^{\sim}(nm^{\omega-2}\xi)$.
\begin{comment}
\begin{proof}
For simplicity we assume $m$ is a power of 2, something which can
be achieved by appending zero rows to $\mathbf{F}$. Divide the matrix
$\mathbf{B}$ into $\log m$ column blocks according to the $\vec{s}$-column
degrees of its columns. Let 
\[
\mathbf{B}=\left[\begin{array}{cccc}
\mathbf{B}^{\left(\log m\right)} & \mathbf{B}^{\left(\log m-1\right)} & \cdots & \mathbf{B}^{\left(1\right)}\end{array}\right],
\]
 with $\mathbf{B}^{\left(\log m\right)}$, $\mathbf{B}^{\left(\log m-1\right)},$
$\mathbf{B}^{\left(\log m-2\right)}$, ... , $\mathbf{B}^{\left(2\right)}$,
$\mathbf{B}^{\left(1\right)}$ having $\vec{s}$-column degrees in
the range $\left[0,2\xi/m\right]$, $(2\xi/m,4\xi/m]$, $(4\xi/m,8\xi/m]$,
...,$(\xi/4,\xi/2]$, $(\xi/2,\theta]$, respectively. We will multiply
$\mathbf{A}$ with each $\mathbf{B}^{\left(i\right)}$ separately.

We also divide the matrix $\mathbf{A}$ into $\log m$ column blocks
and each matrix $\mathbf{B}^{\left(i\right)}$ into $\log m$ row
blocks according to the size of the corresponding entries in $\vec{s}$.
Set 
\begin{eqnarray}
~~~~~~~~~~~~~~~~~~~~~~~~~~\vec{s} & = & \left[\begin{array}{cccc}
\vec{s}_{\log m} & \vec{s}_{\log m-1} & \cdots & \vec{s}_{1}\end{array}\right]\nonumber \\
\mathbf{A} & = & \left[\begin{array}{cccc}
\mathbf{A}_{\log m} & \mathbf{A}_{\log m-1} & \cdots & \mathbf{A}_{1}\end{array}\right]\nonumber \\
\mathbf{B} & = & \left[\begin{array}{cccc}
\mathbf{B}^{\left(\log m\right)} & \mathbf{B}^{\left(\log m-1\right)} & \cdots & \mathbf{B}^{\left(1\right)}\end{array}\right]\nonumber \\
 & = & \left[\begin{array}{cccc}
\mathbf{B}_{\log m}^{\left(\log m\right)} & \mathbf{B}_{\log m}^{\left(\log m-1\right)} & \cdots & \mathbf{B}_{\log m}^{\left(1\right)}\\
\vdots &  &  & \vdots\\
\mathbf{B}_{1}^{\left(\log m\right)} & \mathbf{B}_{1}^{\left(\log m-1\right)} & \cdots & \mathbf{B}_{1}^{\left(1\right)}
\end{array}\right]
\end{eqnarray}
 with $\vec{s}_{\log m},\vec{s}_{\log m-1},\dots,\vec{s}_{1}$ having
entries in the range $\left[0,2\xi/m\right]$, $(2\xi/m,4\xi/m]$,
$(4\xi/m,8\xi/m]$, ..., $(\xi/2,\xi]$ respectively. Also the column
dimension of $\mathbf{A}_{j}$ and the row dimension of $\mathbf{B}_{j}^{\left(i\right)}$
match that of $\vec{s}_{j}$ for $j$ from $1$ to $\log m$.

Notice that $\mathbf{B}_{(j)}^{(i)}$ for $i>j$ must be zero. Otherwise,
as $\vec{s}_{j}>\xi/2^{j}\ge\xi/2^{i-1}$, the $\vec{s}$-column degree
of $\mathbf{B}^{(i)}$ would exceed $\xi/2^{i-1}$, a contradiction
since by definition the $\vec{s}$-column degree of $\mathbf{B}^{(i)}$
is bounded by $\xi/2^{i-1}$ when $i>1$. So $\mathbf{B}$ in fact
has a block triangular shape 
\[
\mathbf{B}=\left[\begin{array}{cccc}
\mathbf{B}_{\log m}^{\left(\log m\right)} & \mathbf{B}_{\log m}^{\left(\log m-1\right)} & \cdots & \mathbf{B}_{\log m}^{\left(1\right)}\\
 & \mathbf{B}_{\log m-1}^{\left(\log m-1\right)} &  & \vdots\\
 &  & \ddots\\
 &  &  & \mathbf{B}_{1}^{\left(1\right)}
\end{array}\right]
\]
 (while remembering that the blocks have varying sizes).

First consider the multiplication 
\[
\mathbf{A}\mathbf{B}^{\left(1\right)}=\left[\begin{array}{cccc}
\mathbf{A}_{\log m} & \mathbf{A}_{\log m-1} & \cdots & \mathbf{A}_{1}\end{array}\right]\left[\begin{array}{l}
\mathbf{B}_{\log m}^{\left(1\right)}\\
\mathbf{B}_{\log m-1}^{\left(1\right)}\\
\vdots\\
\mathbf{B}_{1}^{\left(1\right)}
\end{array}\right].
\]
 Note that there are $O\left(1\right)$ columns in $\mathbf{B}^{(1)}$
since $\theta\in O\left(\xi\right)$. We do this in $\log m$ steps.
At step $j$ for $j$ from $1$ to $\log m$ we multiply $\mathbf{A}_{j}$
and $\mathbf{B}_{j}^{(1)}$. The column dimension of $\mathbf{A}_{j}$,
which is the same as the row dimension of $\mathbf{B}_{j}^{(1)}$,
is less than $2^{j}$. The degree of $\mathbf{B}_{j}^{(1)}$ is $O\left(\xi\right)$.
To use fast multiplication, we expand $\mathbf{B}_{j}^{(1)}$ to a
matrix $\bar{\mathbf{B}}_{j}^{(1)}$ with degree less than $\delta\in\Theta(\xi/2^{j})$
and column dimension $q\in O(2^{j})$ as follows. Write 
\[
\mathbf{B}_{j}^{(1)}=\mathbf{B}_{j,0}^{(1)}+\mathbf{B}_{j,1}^{(1)}x^{\delta}+\dots+\mathbf{B}_{j,q-1}^{(1)}x^{\delta(q-1)}=\sum_{k=0}^{q-1}\mathbf{B}_{j,k}^{(1)}x^{\delta k}
\]
 with each $\mathbf{B}_{j,k}^{(1)}$ having degree less than $\delta.$
Set $\bar{\mathbf{B}}_{j}^{(1)}=\left[\mathbf{B}_{j,0}^{(1)},\mathbf{B}_{j,1}^{(1)},\dots,\mathbf{B}_{j,q-1}^{(1)}\right]$.
We can then multiply $\mathbf{A}_{j}$, which has dimension $m\times O(2^{j})$
for $j<\log m$, and $\bar{\mathbf{B}}_{j}^{(1)}$, which has dimension
$O(2^{j})\times O(2^{j})$ for $j<\log m$, with a cost of 
\[
O^{\sim}\left((m/2^{j})\left(2^{j}\right)^{\omega}\xi/2^{j}\right)=O^{\sim}\left(\left(2^{j}\right)^{\omega-2}m\xi\right)\subset O^{\sim}\left(m^{\omega-1}\xi\right)\subset O^{\sim}(nm^{\omega-2}\xi).
\]
 For $j=\log m$, $\mathbf{A}_{j}$ has dimension $m\times O\left(n\right)$,
$\bar{\mathbf{B}}_{j}^{\left(1\right)}$ has dimension $O\left(n\right)\times O(m)$,
and their degrees are $O\left(\xi/m\right)$. Hence they can be multiplied
with a cost of $O^{\sim}\left((n/m)m^{\omega}(\xi/m)\right)=O^{\sim}\left(nm^{\omega-2}\xi\right)$.
Adding up the columns of $\mathbf{A}_{j}\bar{\mathbf{B}}_{j}^{(1)}$
gives $\mathbf{A}_{j}\mathbf{B}_{j}^{(1)}$ and costs $O(m\xi)$.
Therefore, multiplying $\mathbf{A}$ and $\mathbf{B}^{(1)}$ over
$\log(m)$ steps costs $O^{\sim}\left(nm^{\omega-2}\xi\right)$.

Next we multiply $\mathbf{A}$ with $\mathbf{B}^{(2)}$. We proceed
in the same way as before, but notice that $\mathbf{A}_{1}\mathbf{B}_{1}^{(2)}$
is no longer needed since $\mathbf{B}_{1}^{(2)}=0$. Multiplying $\mathbf{A}$
and $\mathbf{B}^{(2)}$ also costs $O^{\sim}\left(nm^{\omega-2}\xi\right)$.

Continue doing this, it costs $O^{\sim}\left(nm^{\omega-2}\xi\right)$.
to multiply $\mathbf{A}$ with the columns $\mathbf{B}^{(i)}$ for
$i$ from $1$ to $\log m$. As before, remember that $\mathbf{B}_{(j)}^{(i)}=0$
for $j>i$. The overall cost for $i$ from 1 to $\log m$ is $O^{\sim}\left(nm^{\omega-2}\xi\right)$
to multiply $\mathbf{A}$ and $\mathbf{B}$. \end{proof}
\end{comment}

\end{thm}
With \prettyref{thm:multiplyUnbalancedMatrices}, we can now do the
multiplication $\mathbf{F}\mathbf{P}_{2}$ efficiently. 
\begin{cor}
\label{cor:multiplyingFP2}The multiplication of $\mathbf{F}$ and
$\mathbf{P}_{2}$ can be done with a cost of $O^{\sim}\left(nm^{\omega-2}\xi\right)$.\end{cor}
\begin{proof}
Since $\mathbf{P}=[\mathbf{P}_{1},\mathbf{P}_{2}]$ is a $(\mathbf{F},3s,\vec{s})$-basis,
we have from \prettyref{lem:boundOfSumOfShiftedDegreesOfOrderBasis}
that the sum of the $\vec{s}$-column degrees of $\mathbf{P}_{2}$
satisfies $\sum\deg_{\vec{s}}\mathbf{P}_{2}\le3sm+\xi\leq4\xi$. Hence
\prettyref{thm:multiplyUnbalancedMatrices} applies. 
\end{proof}
It remains to show that the residual $\mathbf{F}\mathbf{P}_{2}$ can
be used to compute the remaining kernel basis elements. 
\begin{thm}
\label{thm:continueComputingNullspaceBasisByColumns}Let $\mathbf{P}=\left[\mathbf{P}_{1},\mathbf{P}_{2}\right]$
be a $\left(\mathbf{F},\sigma,\vec{s}\right)$-basis such that $\mathbf{P}_{1}$
consists of all the kernel basis elements of $\mathbf{F}$ in $\mathbf{P}$.
Let $\vec{b}=[\vec{b}_{1},\vec{b}_{2}]$ be the $\vec{s}$-column
degrees of $\mathbf{P}$, where $\vec{b}_{1},\vec{b}_{2}$ are the
$\vec{s}$-column degrees of $\mathbf{P}_{1},$ $\mathbf{P}_{2}$
respectively. Let $\mathbf{Q}$ be a $\vec{b}_{2}$-minimal kernel
basis of $\mathbf{F}\mathbf{P}_{2}$ with $\vec{b}_{2}$-column degrees
$\vec{b}_{2}'$. Then $\left[\mathbf{P}_{1},\mathbf{P}_{2}\mathbf{Q}\right]$
is a $\vec{s}$-minimal kernel basis of $\mathbf{F}$ with $\vec{s}$-column
degrees $[\vec{b}_{1},\vec{b}_{2}']$.\end{thm}
\begin{proof}
Let $\mathbf{Q}'=\diag(\left[I,\mathbf{Q}\right])$, where the dimension
of the identity matrix $I$ matches that of $\mathbf{P}_{1}.$ Then
$\mathbf{Q}'$ is a $\vec{b}$-minimal kernel basis of $\mathbf{F}\mathbf{P}$
since %
\begin{comment}
$\mathbf{F}\mathbf{P}=\left[0,\mathbf{F}\mathbf{P}_{2}\right]$ and 
\end{comment}
{} $\mathbf{F}\mathbf{P}\mathbf{Q}'=\left[\mathbf{F}\mathbf{P}_{1},\mathbf{F}\mathbf{P}_{2}\mathbf{Q}\right]=0$.
It follows that $\mathbf{P}\mathbf{Q}'=\left[\mathbf{P}_{1},\mathbf{P}_{2}\mathbf{Q}\right]$
is a kernel basis of $\mathbf{F}$. We now show that $\mathbf{P}\mathbf{Q}'$
is $\vec{s}$-column reduced and has $\vec{s}$-column degrees $[\vec{b}_{1},\vec{b}_{2}']$,
or equivalently, $x^{\vec{s}}\mathbf{P}\mathbf{Q}'$ is column reduced
and has column degrees $[\vec{b}_{1},\vec{b}_{2}']$. Notice that
$x^{\vec{s}}\mathbf{P}$ has column degrees $[\vec{b}_{1},\vec{b}_{2}]$
and a full rank leading column coefficient matrix $P$. Hence $x^{\vec{s}}\mathbf{P}x^{-[\vec{b}_{1},\vec{b}_{2}]}$
has column degrees $\left[0,\dots0\right]$. (If one is concerned
about the entries not being polynomials, one can simply multiply the
matrix by $x^{\xi}$ to shift the degrees up.) Similarly, $x^{\vec{b}_{2}}\mathbf{Q}x^{-\vec{b}_{2}'}$
has column degrees $[0,\dots,0]$, and so $x^{[\vec{b}_{1},\vec{b}_{2}]}\mathbf{Q}'x^{-[\vec{b}_{1},\vec{b}_{2}']}$
also has column degrees $[0,\dots,0]$ and a full rank leading column
coefficient matrix $Q'$. Putting these together, we see that $x^{\vec{s}}\mathbf{P}x^{-[\vec{b}_{1},\vec{b}_{2}]}x^{[\vec{b}_{1},\vec{b}_{2}]}\mathbf{Q}'x^{-[\vec{b}_{1},\vec{b}_{2}']}=x^{\vec{s}}\mathbf{P}\mathbf{Q}'x^{-[\vec{b}_{1},\vec{b}_{2}']}$
has column degrees $[0,\dots,0]$ and a full rank leading column coefficient
matrix $PQ'$. It follows that $x^{\vec{s}}\mathbf{P}\mathbf{Q}'$
has column degrees $[\vec{b}_{1},\vec{b}_{2}']$, or equivalently,
the $\vec{s}$-column degrees of $\mathbf{PQ}'$ is $[\vec{b}_{1},\vec{b}_{2}']$.

It remains to show that any $\mathbf{n}$ satisfying $\mathbf{F}\mathbf{n}=0$
must be a linear combination of the columns of $\mathbf{PQ}'$. Since
$\mathbf{n}\in\left\langle \left(\mathbf{F},\sigma\right)\right\rangle $,
it is generated by the $\left(\mathbf{F},\sigma\right)$-basis $\mathbf{P}$,
that is, $\mathbf{n}=\mathbf{P}\mathbf{a}$ with $\mathbf{a}=\mathbf{P}^{-1}\mathbf{n}\in\mathbb{K}\left[x\right]^{n}$.
Also, $\mathbf{F}\mathbf{n}=0$ implies $\mathbf{F}\mathbf{P}\mathbf{a}=0$,
hence $\mathbf{a}=\mathbf{Q}'\mathbf{b}$ for some vector $\mathbf{b}$
as $\mathbf{Q}'$ is a kernel basis of $\mathbf{F}\mathbf{P}$. We
now have $\mathbf{n}=\mathbf{P}\mathbf{Q}'\mathbf{b}$ as required.\end{proof}
\begin{example}
\label{exm:continueComputingNullspaceBasisByColumns}Let us look at
an example of computing kernel basis using \prettyref{thm:continueComputingNullspaceBasisByColumns}.
Let $\mathbf{F}$ be given by 
\[
\left[\begin{array}{cccc}
x+x^{2}+x^{3} & 1+x & 0 & 1+x\\
1+x^{2}+x^{3} & x+x^{2}+x^{3} & x+x^{2} & x^{3}
\end{array}\right]\in\mathbb{Z}_{2}\left[x\right]^{2\times4}.
\]
 Let $\sigma=3$, $\vec{s}=\left[3,3,3,3\right]$. We first compute
a $\left(\mathbf{F},\sigma,\vec{s}\right)$-basis 
\[
\mathbf{P}=\left[\begin{array}{cccc}
~0~ & ~0~ & x^{2} & x\\
1 & 0 & 0 & x^{2}\\
1 & x^{2} & x+x^{2} & 1+x\\
1 & 0 & 0 & 0
\end{array}\right],
\]
 with the $\vec{s}$-column degrees $\vec{b}=\left[3,5,5,5\right]$
and the residual 
\[
\mathbf{F}\mathbf{P}=\left[\begin{array}{cccc}
~0~ & 0 & x^{3}+x^{4}+x^{5} & x^{4}\\
0 & x^{3}+x^{4} & x^{5} & x^{3}+x^{5}
\end{array}\right].
\]
 Thus $\mathbf{P}_{1}=[0,1,1,1]^{T}$, %\[
%\mathbf{P}_{1}=\begin{bmatrix}0\\
%1\\
%1\\
%1
%\end{bmatrix}
%\]
with $\vec{s}$-column degree $3$, is the only kernel basis element
computed. Let $\mathbf{P}_{2}$ contain the remaining columns of $\mathbf{P}$
and $\vec{b}_{2}=\left[5,5,5\right]$ be its $\vec{s}$-column degrees.
Next we compute a $\vec{b}_{2}$-minimal kernel basis of $\mathbf{F}\mathbf{P}_{2}$
\[
\mathbf{Q}=[1+x+x^{4},\ x+x^{2},\ 1+x^{3}]^{T}
\]
 which has $\vec{b}_{2}$-column degree 9. Then 
\[
\left[\mathbf{P}_{1},\mathbf{P}_{2}\mathbf{Q}\right]=\left[\begin{array}{cc}
0 & x+x^{3}\\
1 & x^{2}+x^{5}\\
1 & 1+x+x^{6}\\
1 & 0
\end{array}\right]
\]
 is a complete $\vec{s}$-minimal kernel basis of $\mathbf{F}$ with
$\vec{s}$-column degrees $\left[3,9\right]$. 
\end{example}
\prettyref{thm:continueComputingNullspaceBasisByColumns} shows that
the remaining $\vec{s}$-minimal kernel basis elements $\mathbf{P}_{2}\mathbf{Q}$
can be correctly computed from the residual $\mathbf{F}\mathbf{P}_{2}$.
\begin{comment}
We still need to show it can be done efficiently. 
\end{comment}
{} Before discussing the computation of a $\vec{b}_{2}$-minimal kernel
basis $\mathbf{Q}$ of $\mathbf{F}\mathbf{P}_{2}$, let us first note
that the multiplication $\mathbf{P}_{2}\mathbf{Q}$ can be done efficiently,
which again follows from \prettyref{thm:multiplyUnbalancedMatrices}. 
\begin{lem}
\label{lem:multiplyingP2Q}The multiplication of $\mathbf{P}_{2}$
and $\mathbf{Q}$ can be done with a cost of $O^{\sim}\left(nm^{\omega-2}\xi\right)$.\end{lem}
\begin{proof}
Note that the dimension of $\mathbf{P}_{2}$ is $n\times O(m)$ from
\prettyref{thm:dimensionOfPartialNullspaceBasisBasedOnOrder} and
the dimension of $\mathbf{Q}$ is $O\left(m\right)\times O\left(m\right)$.
The column degrees of $\mathbf{P}_{2}$ are bounded by the $\vec{s}$-column
degrees $\vec{b}_{2}$ of $\mathbf{P}_{2}$ since $\vec{s}$ is non-negative.
Also recall that $\sum\vec{b}_{2}\le4\xi$ from the proof of \prettyref{cor:multiplyingFP2}.
By \prettyref{lem:boundOnDegreesOfFA} the column degrees of $\mathbf{F}\mathbf{P}_{2}$
are bounded by the $\vec{s}$-column degrees $\vec{b}_{2}$ of $\mathbf{P}_{2}$.
By \prettyref{thm:boundOfSumOfShiftedDegreesOfKernelBasis}, the sum
of the $\vec{b}_{2}$-column degrees of $\mathbf{Q}$ is %
\begin{comment}
bounded by the sum of the column degrees of $\mathbf{F}\mathbf{P}_{2}$,
which is 
\end{comment}
{} also bounded by $\sum\vec{b}_{2}\le4\xi$. Now if we separate $\mathbf{P}_{2}$
to $n/m$ blocks rows each with no more than $m$ rows, \prettyref{thm:multiplyUnbalancedMatrices}
can be used to multiply each block row with $\mathbf{Q}$. Each multiplication
involves matrices of dimension $O\left(m\right)\times O\left(m\right)$.
In addition, both the sum of the column degrees of $\mathbf{P}_{2}$
and the sum of the $\vec{b}_{2}$-column degrees of $\mathbf{Q}$
are bounded by $4\xi$. So each multiplication costs $O^{\sim}(m^{\omega-1}\xi)$.
Hence doing this for all $n/m$ block rows costs $O^{\sim}\left(nm^{\omega-2}\xi\right)$. 
\end{proof}

\subsection{Reducing the degrees}

Our next task is computing a $\vec{b}_{2}$-minimal kernel basis of
the residual $\mathbf{F}\mathbf{P}_{2}$. It is useful to note that
the lower degree terms of $\mathbf{F}\mathbf{P}_{2}$ are zero since
it has order $\sigma$. Hence we can use $\mathbf{G}=\mathbf{F}\mathbf{P}_{2}/x^{\sigma}$
instead to compute the remaining basis elements. In the following,
we show that just like the original input matrix $\mathbf{F}$, this
new input matrix $\mathbf{G}$ has column degrees bounded by the corresponding
entries of $\vec{s}$.%
\begin{comment}
However we still need to verify that the size of $\mathbf{G}$ is
small enough to allow for the efficient computation of the remaining
basis elements. 
\end{comment}
\begin{comment}
\prettyref{lem:boundOnShiftedDegrees} and \prettyref{cor:boundOnFPAfterLowerTermsRemoved}
shows that $\vec{b}-\sigma$, which bound the column degrees of $\mathbf{F}\mathbf{P}/x^{\sigma}$
by \prettyref{lem:boundOnDegreesOfFA}, are bounded by the corresponding
entries of $\vec{s}$. 
\end{comment}

\begin{lem}
\label{lem:boundOnShiftedDegrees}%Let $\vec{s}$, $\mathbf{F}$ be as before. 
If an $(\mathbf{F},\sigma,\vec{s})$-basis has columns arranged in
increasing $\vec{s}$-column degrees with $\vec{s}$-column degrees
$\vec{b}$, then the entries of $\vec{b}-[\sigma,\dots,\sigma]=\left[b_{1}-\sigma,\dots,b_{n}-\sigma\right]$
are bounded component-wise by $\vec{s}$. \end{lem}
\begin{proof}
A $(\mathbf{F},0,\vec{s})$-basis of order $0$ has $\vec{s}$-column
degrees given by $\vec{s}$. For each order increase, any column of
the basis has its $\vec{s}$-column degree increases by at most one,
which occurs when its order is increased by multiplying the column
by $x$. Hence at order $\sigma$, the $\vec{s}$-column degree increase
for each column is at most $\sigma$. \end{proof}
\begin{cor}
\label{cor:boundOnFPAfterLowerTermsRemoved}The column degrees of
$\mathbf{F}\mathbf{P}/x^{\sigma}$ are \\
bounded component-wise by $\vec{s}$.\end{cor}
\begin{proof}
From \prettyref{lem:boundOnDegreesOfFA}, the column degrees of $\mathbf{F}\mathbf{P}$
are bounded component-wise by $\vec{b}$, the $\vec{s}$-column degrees
of $\mathbf{P}$. Hence the column degrees of $\mathbf{F}\mathbf{P}/x^{\sigma}$
are bounded component-wise by $\vec{b}-[\sigma,\dots,\sigma]$. The
result then follows from \prettyref{lem:boundOnShiftedDegrees}. 
\end{proof}
From \prettyref{cor:boundOnFPAfterLowerTermsRemoved}, the column
degrees of $\mathbf{F}\mathbf{P}_{2}/x^{\sigma}$ are bounded by the
entries of the corresponding subset $\vec{t}$ of $\vec{b}-[\sigma,\dots,\sigma]$,
which is in turn bounded by the entries of the corresponding subset
of $\vec{s}$. 
\begin{example}
\label{exm:reducingDegree}From \prettyref{exm:continueComputingNullspaceBasisByColumns},
note that instead of using the residual 
\[
\mathbf{F}\mathbf{P}_{2}=\left[\begin{array}{ccc}
0 & x^{3}+x^{4}+x^{5} & x^{4}\\
x^{3}+x^{4} & x^{5} & x^{3}+x^{5}
\end{array}\right]
\]
 to compute a $[5,5,5]$-minimal kernel basis of $\mathbf{F}$,%
\begin{comment}
$\left(\mathbf{F}\mathbf{P}_{2},\tau,[5,5,5]\right)$-basis,
\end{comment}
{} we can instead use 
\[
\mathbf{G}=\mathbf{F}\mathbf{P}_{2}/x^{3}=\left[\begin{array}{ccc}
0 & 1+x+x^{2} & x\\
1+x & x^{2} & 1+x^{2}
\end{array}\right]
\]
 to compute a $[2,2,2]$-minimal kernel basis of $\mathbf{G}$%
\begin{comment}
$\left(\mathbf{G},\tau,[2,2,2]\right)$-basis
\end{comment}
. The column degrees of $\mathbf{G}$ are bounded by the new shift
$\left[2,2,2\right]$, which is in turn bounded by the corresponding
entries $\left[3,3,3\right]$ of $\vec{s}$. 
\end{example}
At this point, using \prettyref{thm:continueComputingNullspaceBasisByColumns}
and \prettyref{cor:boundOnFPAfterLowerTermsRemoved}, the problem
is reduced to computing a $\vec{t}$-minimal kernel basis of $\mathbf{G}=\mathbf{F}\mathbf{P}_{2}/x^{3s}$,
which still has row dimension $m$. But its column dimension is now
bounded by $3m/2$. Also notice that as in the original problem, the
column degrees of the new input matrix $\mathbf{G}$ are bounded by
the corresponding entries of the new shift $\vec{t}$. In addition,
as the new shift $\vec{t}$ is bounded component-wise by a subset
of the old shift $\vec{s}$, the new problem is no more difficult
than the original problem.


\subsection{\label{sub:continueComputingNullspaceBasisByRows}Reducing the row
dimension }

We now turn to the new problem of computing a $\vec{t}$-minimal kernel
basis of $\mathbf{G}$. Let 
\[
\mathbf{G}=\left[\begin{array}{c}
\mathbf{G}_{1}\\
\mathbf{G}_{2}
\end{array}\right]
\]
 %[\mathbf{G}_{1},\mathbf{G}_{2}]^{T} \]
 with $\mathbf{G}_{1}$ having $\left\lfloor m/2\right\rfloor $ rows
and $\mathbf{G}_{2}$ having $\left\lceil m/2\right\rceil $ rows.
If we compute a $\vec{t}$-minimal kernel basis $\mathbf{N}_{1}$
of $\mathbf{G}_{1}$, where $\mathbf{N}_{1}$ has $\vec{t}$-column
degrees $\vec{u}$, then compute a $\vec{u}$-minimal kernel basis
$\mathbf{N}_{2}$ of $\mathbf{G}_{2}\mathbf{N}_{1}$, then the next
theorem %\prettyref{thm:continueComputingkernelBasisByRows} 
shows that $\mathbf{N}_{1}\mathbf{N}_{2}$ is a $\vec{t}$-minimal
kernel basis of $\mathbf{G}$. 
\begin{thm}
\label{thm:continueComputingNullspaceBasisByRows}Let %an input matrix
$\mathbf{G}=\left[\mathbf{G}_{1}^{T},\mathbf{G}_{2}^{T}\right]^{T}\in\mathbb{K}\left[x\right]^{m\times n}$
and $\vec{t}$ a shift vector. If $\mathbf{N}_{1}$ is a $\vec{t}$-minimal
kernel basis of $\mathbf{G}_{1}$ with $\vec{t}$-column degrees $\vec{u}$,
and $\mathbf{N}_{2}$ is a $\vec{u}$-minimal kernel basis of $\mathbf{G}_{2}\mathbf{N}_{1}$
with $\vec{u}$-column degrees $\vec{v}$, then $\mathbf{N}_{1}\mathbf{N}_{2}$
is a $\vec{t}$-minimal kernel basis of $\mathbf{G}$ with $\vec{t}$-column
degrees $\vec{v}$.\end{thm}
\begin{proof}
The proof is very similar to the proof of \prettyref{thm:continueComputingNullspaceBasisByColumns}.
It is clear that $\mathbf{G}\mathbf{N}_{1}\mathbf{N}_{2}=0$ hence
$\mathbf{N}_{1}\mathbf{N}_{2}$ is a kernel basis of $\mathbf{G}$.
We now show that $\mathbf{N}_{1}\mathbf{N}_{2}$ is $\vec{t}$-column
reduced and has $\vec{t}$-column degrees $\vec{v}$, or equivalently,
$x^{\vec{t}}\mathbf{N}_{1}\mathbf{N}_{2}$ is column reduced. Notice
that $x^{\vec{t}}\mathbf{N}_{1}$ has column degrees $\vec{u}$ and
a full rank leading column coefficient matrix $N_{1}$. Hence $x^{\vec{t}}\mathbf{N}_{1}x^{-\vec{u}}$
has column degrees $\left[0,\dots,0\right]$. Again, if one is concerned
about the entries not being polynomials, one can simply multiply the
matrix by $x^{\xi}$ to shift the degrees up. Similarly, $x^{\vec{u}}\mathbf{N}_{2}x^{\vec{v}}$
has column degrees $\left[0,\dots,0\right]$ and a full rank leading
column coefficient matrix $N_{2}$. Putting them together, $x^{\vec{t}}\mathbf{N}_{1}x^{-\vec{u}}x^{\vec{u}}\mathbf{N}_{2}x^{-\vec{v}}=x^{\vec{t}}\mathbf{N}_{1}\mathbf{N}_{2}x^{-\vec{v}}$
has column degrees $[0,\dots,0]$ and a full rank leading column coefficient
matrix $N_{1}N_{2}$. It follows that $x^{\vec{t}}\mathbf{N}_{1}\mathbf{N}_{2}$
has column degrees $\vec{v}$, or equivalently, the $\vec{t}$-column
degrees of $\mathbf{N}_{1}\mathbf{N}_{2}$ is $\vec{v}$.

It remains to show that any $\mathbf{n}$ satisfying $\mathbf{G}\mathbf{n}=0$
must be a linear combination of the columns of $\mathbf{N}_{1}\mathbf{N}_{2}$.
First notice that $\mathbf{n}=\mathbf{N}_{1}\mathbf{a}$ for some
polynomial vector $\mathbf{a}$ since $\mathbf{N}_{1}$ is a kernel
basis of $\mathbf{G}_{1}$. Also, $\mathbf{G}\mathbf{n}=0$ implies
that $\mathbf{G}_{2}\mathbf{N}_{1}\mathbf{a}=0$, hence $\mathbf{a}=\mathbf{N}_{2}\mathbf{b}$
for some vector $\mathbf{b}$ as $\mathbf{N}_{2}$ is a kernel basis
of $\mathbf{G}_{2}\mathbf{N}_{1}$. We now have $\mathbf{n}=\mathbf{N}_{1}\mathbf{N}_{2}\mathbf{b}$
as required.\end{proof}
\begin{example}
Let us compute a $\vec{t}$-minimal kernel basis of 
\[
\mathbf{G}=\left[\begin{array}{ccc}
0 & 1+x+x^{2} & x\\
1+x & x^{2} & 1+x^{2}
\end{array}\right]
\]
 from \prettyref{exm:reducingDegree}, where $\vec{t}=\left[2,2,2\right]$.
Then 
\begin{align*}
\mathbf{G}_{1} & =\left[\begin{array}{ccc}
0 & 1+x+x^{2} & x\end{array}\right]\mbox{and }\mathbf{G}_{2}=\left[\begin{array}{ccc}
1+x & x^{2} & 1+x^{2}\end{array}\right].
\end{align*}


We first compute a $\vec{t}$-minimal kernel basis $\mathbf{N}_{1}$
of $\mathbf{G}_{1}$: 
\[
\mathbf{N}_{1}=\left[\begin{array}{cc}
1 & 0\\
0 & x\\
0 & 1+x+x^{2}
\end{array}\right]
\]
 with its $\vec{t}$-column degrees $\vec{u}=\left[2,4\right]$. Next,
we compute a $\vec{u}$-minimal kernel basis $\mathbf{N}_{2}$ of
$\mathbf{G}_{2}\mathbf{N}_{1}=\left[\begin{array}{cc}
1+x & 1+x+x^{4}\end{array}\right]$: 
\[
\mathbf{N}_{2}=[1+x+x^{4},\ 1+x]^{T}.
\]
 Then 
\[
\mathbf{N}_{1}\mathbf{N}_{2}=[1+x+x^{4},\ x+x^{2},\ 1+x^{3}]^{T}
\]
 is a $\vec{t}$-minimal kernel basis of $\mathbf{G}$. 
\end{example}
While \prettyref{thm:continueComputingNullspaceBasisByColumns} allows
us to compute kernel bases by columns, which then reduces the column
dimensions, \prettyref{thm:continueComputingNullspaceBasisByRows}
shows that that the kernel bases can also be computed by rows, which
then reduces the row dimensions. Again, we need to check that these
computations can be done efficiently. In the following, %
\begin{comment}
\prettyref{lem:sizeOfG2N1} shows that the size of $\mathbf{G}_{2}\mathbf{N}_{1}$
is within a required bound. Then 
\end{comment}
{} \prettyref{lem:mutiplyingG2N1} and \prettyref{lem:multiplyingN1N2}
show that the multiplication $\mathbf{G}_{2}\mathbf{N}_{1}$ and the
multiplication $\mathbf{N}_{1}\mathbf{N}_{2}$ can be done efficiently,
which are again consequences of \prettyref{thm:multiplyUnbalancedMatrices}.%
\begin{comment}
\begin{lem}
\label{lem:sizeOfG2N1}The sum of the column degrees of $\mathbf{G}_{2}\mathbf{N}_{1}$
is bounded by $\xi$.\end{lem}
\begin{proof}
This follows from \prettyref{thm:boundOfSumOfShiftedDegreesOfKernelBasis}
and \prettyref{lem:boundOnDegreesOfFA}.\end{proof}
\end{comment}

\begin{lem}
\label{lem:mutiplyingG2N1}The multiplication of $\mathbf{G}_{2}$
and $\mathbf{N}_{1}$ can be done with a cost of $O^{\sim}(m^{\omega-1}\xi)$.\end{lem}
\begin{proof}
\prettyref{thm:multiplyUnbalancedMatrices} applies directly here.\end{proof}
\begin{lem}
\label{lem:multiplyingN1N2}The multiplication of $\mathbf{N}_{1}$
and $\mathbf{N}_{2}$ can be done with a cost of $O^{\sim}(m^{\omega-1}\xi)$.\end{lem}
\begin{proof}
\prettyref{thm:multiplyUnbalancedMatrices} applies because the sum
of the column degrees of $\mathbf{N}_{1}$ is bounded by the sum of
the $\vec{t}$-column degrees of $\mathbf{N}_{1}$, which is $\sum\vec{u}\le\xi$,
and by \prettyref{thm:boundOfSumOfShiftedDegreesOfKernelBasis} the
sum of $\vec{u}$-column degrees of $\mathbf{N}_{2}$ is also bounded
by $\xi$. 
\end{proof}

\subsection{Recursive computation}

The computation of $\mathbf{N}_{1}$ and $\mathbf{N}_{2}$ is identical
to the original problem, only the dimension has decreased. For computing
$\mathbf{N}_{1}$, the dimension of the input matrix $\mathbf{G}_{1}$
is bounded by $\left\lfloor m/2\right\rfloor \times\left(3m/2\right)$.
For computing $\mathbf{N}_{2}$ , the dimension of input matrix $\mathbf{G}_{2}\mathbf{N}_{1}$
is bounded by $\left\lceil m/2\right\rceil \times(3m/2)$. The column
degrees of $\mathbf{G}_{1}$ are bounded by the entries of $\vec{t}$,
with $\sum\vec{t}\le\xi$. Similarly, the column degrees of $\mathbf{G}_{2}\mathbf{N}_{1}$
are bounded by the entries of $\vec{u}$, with $\sum\vec{u}\le\xi$.
Hence, the same computation process can be repeated on these two smaller
problems. This gives a recursive algorithm, shown in \prettyref{alg:minimalNullspaceBasisWithRankProfile}.

\begin{algorithm}[t]
\caption{$\mnb(\mathbf{F},\vec{s})$}
\label{alg:minimalNullspaceBasis}

\begin{algorithmic}[1]
\REQUIRE{$\mathbf{F}\in\mathbb{K}\left[x\right]^{m\times n}$, $\vec{s}=[s_{1},\dots,s_{n}]\in\mathbb{Z}^{n}$
with entries arranged in non-decreasing order and bounding the corresponding
column degrees of $\mathbf{F}$. }

\ENSURE{A $\vec{s}$-minimal kernel basis of $\mathbf{F}$.}

\STATE{$\xi:=\sum_{i=1}^{n}s_{i}$; $\rho:=\sum_{i=n-m+1}^{n}s_{i};$ $s:=\rho/m$; }

\STATE{$\left[\mathbf{P},\vec{b}\right]:=\mab\left(\mathbf{F},3s,\vec{s}\right)$,
a $\left(\mathbf{F},3s,\vec{s}\right)$-basis with the columns of
$\mathbf{P}$ and the entries of is $\vec{s}$-column degrees $\vec{b}$
arranged so that the entries of $\vec{b}$ are in non-decreasing order;}\label{line:orderBasis}

\STATE{$\left[\mathbf{P}_{1},\mathbf{P}_{2}\right]:=\mathbf{P}$ where $\mathbf{P}_{1}$
consists of all columns $\mathbf{p}$ of $\mathbf{P}$ satisfying
$\mathbf{F}\mathbf{p}=0$;}

\IF{$m=1$}

\ifbody{\RETURN $\mathbf{P}_{1}$ 

\ELSE{}

\STATE{ $\vec{t}:=\deg_{\vec{s}}\mathbf{P}_{2}-\left[3s,3s,\dots,3s\right];$}

\STATE{$\mathbf{G}:=\mathbf{F}\mathbf{P}_{2}/x^{3s}$;}\label{line:multiplyFP2}

\STATE{$\left[\mathbf{G}_{1}^{T},\mathbf{G}_{2}^{T}\right]^{T}:=\mathbf{G}$,
with $\mathbf{G}_{1}$ having $\left\lfloor m/2\right\rfloor $ rows
and $\mathbf{G}_{2}$ having $\left\lceil m/2\right\rceil $ rows;}

\STATE{$\mathbf{N}_{1}:=\mnb\left(\mathbf{G}_{1},\vec{t}\right);$ }

\STATE{$\mathbf{N}_{2}:=\mnb\left(\mathbf{G}_{2}\mathbf{N}_{1},\cdeg_{\vec{t}}\mathbf{N}_{1}\right);$}\label{line:multiplyG2N1}

\STATE{$\mathbf{Q}:=\mathbf{N}_{1}\mathbf{N}_{2}$;}\label{line:multiplyN1N2}

\RETURN $\left[\mathbf{P}_{1},\mathbf{P}_{2}\mathbf{Q}\right]$\label{line:multiplyP2Q}}
\end{algorithmic}
\end{algorithm}



Before analyzing the computational complexity of \prettyref{alg:minimalNullspaceBasisWithRankProfile}
in the following section, we provide a proof of \prettyref{thm:multiplyUnbalancedMatrices},
which is needed to efficiently multiply matrices with unbalanced degrees
in the algorithm.


\subsection{Proof of \prettyref{thm:multiplyUnbalancedMatrices}}

%The recursive procedure described in the next section reduces a single null space
%computation to a set of null space computations of smaller size. These smaller problems
%are constructed by the computations of residuals which in all cases involves the multiplication
%of two polynomial matrices having unbalanced degrees. In this section, we present a method for 
%multiplying polynomial matrices that occur in the algorithm given in the following section.  These matrices have unbalanced degrees so it would be inefficient to directly apply the 
%standard matrix multiplication.


%\begin{lem}
%\label{lem:multiplyUnbalancedMatrices} Let $\vec{s}$ be a shift
%ordered in terms of increasing values and $\xi$, a bound on the sum
%of the entries of $\vec{s}$. Let $\mathbf{A}\in\mathbb{K}\left[x\right]^{m\times n}$,
%with column degrees bounded by $\vec{s}$ and $\mathbf{B}\in\mathbb{K}\left[x\right]^{n\times k}$
%with $k\in O\left(m\right)$ and the sum $\theta$ of its $\vec{s}$-column
%degrees satisfying $\theta\in O\left(\xi\right)$. Then we can multiply
%$\mathbf{A}$ and $\mathbf{B}$ with a cost of $O^{\sim}(nm^{\omega-2}\xi)$. \end{lem}


In this subsection we give a proof of \prettyref{thm:multiplyUnbalancedMatrices}.
\begin{proof}
Recall that $\vec{s}$ is a shift with entries ordered in terms of
increasing values and $\xi$ is a bound on the sum of the entries
of $\vec{s}$. We wish to determine the cost of multiplying the two
polynomials matrices $\mathbf{A}\in\mathbb{K}\left[x\right]^{m\times n}$
and $\mathbf{B}\in\mathbb{K}\left[x\right]^{n\times k}$ where $\mathbf{A}$
has column degrees bounded by $\vec{s}$ and where $k\in O\left(m\right)$
and the sum $\theta$ of its $\vec{s}$-column degrees satisfies $\theta\in O\left(\xi\right)$.
The goal is to show that these polynomial matrices can be multiplied
%Then we can multiply %$\mathbf{A}$ and $\mathbf{B}$ 
with a cost of $O^{\sim}(nm^{\omega-2}\xi)$.

For simplicity we assume $m$ is a power of $2$, something which
can be achieved by appending zero rows to $\mathbf{F}$. We divide
the matrix $\mathbf{B}$ into $\log m$ column blocks according to
the $\vec{s}$-column degrees of its columns. Let 
\[
\mathbf{B}=\left[\begin{array}{ccccc}
\mathbf{B}^{\left(\log m\right)} & \mathbf{B}^{\left(\log m-1\right)} & \cdots & \mathbf{B}^{\left(2\right)} & \mathbf{B}^{\left(1\right)}\end{array}\right],
\]
 with $\mathbf{B}^{\left(\log m\right)}$, $\mathbf{B}^{\left(\log m-1\right)},$
$\mathbf{B}^{\left(\log m-2\right)}$, ... , $\mathbf{B}^{\left(2\right)}$,
$\mathbf{B}^{\left(1\right)}$ having $\vec{s}$-column degrees in
the range $\left[0,2\xi/m\right]$, $(2\xi/m,4\xi/m]$, $(4\xi/m,8\xi/m]$,
...,$(\xi/4,\xi/2]$, $(\xi/2,\theta]$, respectively. We will multiply
$\mathbf{A}$ with each $\mathbf{B}^{\left(i\right)}$ separately.

We also divide the matrix $\mathbf{A}$ into $\log m$ column blocks
and each matrix $\mathbf{B}^{\left(i\right)}$ into $\log m$ row
blocks according to the size of the corresponding entries in $\vec{s}$.
Set 
\begin{eqnarray*}
~~~~\vec{s} & = & \left[\begin{array}{cccc}
\vec{s}_{\log m} & \vec{s}_{\log m-1} & \cdots & \vec{s}_{1}\end{array}\right]\\
\mathbf{A} & = & \left[\begin{array}{cccc}
\mathbf{A}_{\log m} & \mathbf{A}_{\log m-1} & \cdots & \mathbf{A}_{1}\end{array}\right]\\
\mathbf{B} & = & \left[\begin{array}{cccc}
\mathbf{B}^{\left(\log m\right)} & \mathbf{B}^{\left(\log m-1\right)} & \cdots & \mathbf{B}^{\left(1\right)}\end{array}\right]\\
 & = & \left[\begin{array}{cccc}
\mathbf{B}_{\log m}^{\left(\log m\right)} & \mathbf{B}_{\log m}^{\left(\log m-1\right)} & \cdots & \mathbf{B}_{\log m}^{\left(1\right)}\\
\vdots &  &  & \vdots\\
\mathbf{B}_{1}^{\left(\log m\right)} & \mathbf{B}_{1}^{\left(\log m-1\right)} & \cdots & \mathbf{B}_{1}^{\left(1\right)}
\end{array}\right]
\end{eqnarray*}
 %
\begin{comment}
\wei{Shouldn't they be reversed? We are not assign to $\vec{s},\mathbf{A},\mathbf{B},$
but to $\vec{s}_{i},\mathbf{A}_{i},\mathbf{B}_{i}$. Same issue with
other assignments as well} \george{I think it is more readable this
way.} 
\end{comment}
with $\vec{s}_{\log m},\vec{s}_{\log m-1},\dots,\vec{s}_{1}$ having
entries in the range $\left[0,2\xi/m\right]$, $(2\xi/m,4\xi/m]$,
$(4\xi/m,8\xi/m]$, ..., $(\xi/2,\xi]$ respectively. Also the column
dimension of $\mathbf{A}_{j}$ and the row dimension of $\mathbf{B}_{j}^{\left(i\right)}$
match that of $\vec{s}_{j}$ for $j$ from $1$ to $\log m$.

Notice that $\mathbf{B}_{(j)}^{(i)}$ for $i>j$ must be zero. Otherwise,
as $\vec{s}_{j}>\xi/2^{j}\ge\xi/2^{i-1}$, the $\vec{s}$-column degree
of $\mathbf{B}^{(i)}$ would exceed $\xi/2^{i-1}$, a contradiction
since by definition the $\vec{s}$-column degree of $\mathbf{B}^{(i)}$
is bounded by $\xi/2^{i-1}$ when $i>1$. So $\mathbf{B}$ in fact
has a block triangular shape 
\[
\mathbf{B}=\left[\begin{array}{cccc}
\mathbf{B}_{\log m}^{\left(\log m\right)} & \mathbf{B}_{\log m}^{\left(\log m-1\right)} & \cdots & \mathbf{B}_{\log m}^{\left(1\right)}\\
 & \mathbf{B}_{\log m-1}^{\left(\log m-1\right)} &  & \vdots\\
 &  & \ddots\\
 &  &  & \mathbf{B}_{1}^{\left(1\right)}
\end{array}\right]
\]
 (while remembering that the blocks have varying sizes).

First consider the multiplication 
\[
\mathbf{A}\mathbf{B}^{\left(1\right)}=\left[\begin{array}{ccc}
\mathbf{A}_{\log m} & \cdots & \mathbf{A}_{1}\end{array}\right]\left[\begin{array}{l}
\mathbf{B}_{\log m}^{\left(1\right)}\\
\vdots\\
\mathbf{B}_{1}^{\left(1\right)}
\end{array}\right].
\]
 Note that there are $O\left(1\right)$ columns in $\mathbf{B}^{(1)}$
since $\theta\in O\left(\xi\right)$. We do this in $\log m$ steps.
At step $j$ for $j$ from $1$ to $\log m$ we multiply $\mathbf{A}_{j}$
and $\mathbf{B}_{j}^{(1)}$. The column dimension of $\mathbf{A}_{j}$,
which is the same as the row dimension of $\mathbf{B}_{j}^{(1)}$,
is less than $2^{j}$. The degree of $\mathbf{B}_{j}^{(1)}$ is $O\left(\xi\right)$.
To use fast multiplication, we expand $\mathbf{B}_{j}^{(1)}$ to a
matrix $\bar{\mathbf{B}}_{j}^{(1)}$ with degree less than $\delta\in\Theta(\xi/2^{j})$
and column dimension $q\in O(2^{j})$ as follows. Write 
\[
\mathbf{B}_{j}^{(1)}=\mathbf{B}_{j,0}^{(1)}+\mathbf{B}_{j,1}^{(1)}x^{\delta}+\dots+\mathbf{B}_{j,q-1}^{(1)}x^{\delta(q-1)}=\sum_{k=0}^{q-1}\mathbf{B}_{j,k}^{(1)}x^{\delta k}
\]
 with each $\mathbf{B}_{j,k}^{(1)}$ having degree less than $\delta.$
Set 
\[
\bar{\mathbf{B}}_{j}^{(1)}=\left[\mathbf{B}_{j,0}^{(1)},\mathbf{B}_{j,1}^{(1)},\dots,\mathbf{B}_{j,q-1}^{(1)}\right]
\]
. We can then multiply $\mathbf{A}_{j}$, which has dimension $m\times O(2^{j})$
for $j<\log m$, and $\bar{\mathbf{B}}_{j}^{(1)}$, which has dimension
$O(2^{j})\times O(2^{j})$ for $j<\log m$, with a cost of 
\begin{align*}
O^{\sim}\left((m/2^{j})\left(2^{j}\right)^{\omega}\xi/2^{j}\right) & =O^{\sim}\left(\left(2^{j}\right)^{\omega-2}m\xi\right).\\
 & \subset O^{\sim}\left(m^{\omega-1}\xi\right)\subset O^{\sim}(nm^{\omega-2}\xi)
\end{align*}
 For $j=\log m$, $\mathbf{A}_{j}$ has dimension $m\times O\left(n\right)$,
$\bar{\mathbf{B}}_{j}^{\left(1\right)}$ has dimension $O\left(n\right)\times O(m)$,
and their degrees are $O\left(\xi/m\right)$. Hence they can be multiplied
with a cost of $O^{\sim}\left((n/m)m^{\omega}(\xi/m)\right)=O^{\sim}\left(nm^{\omega-2}\xi\right)$.
Adding up the columns of $\mathbf{A}_{j}\bar{\mathbf{B}}_{j}^{(1)}$
gives $\mathbf{A}_{j}\mathbf{B}_{j}^{(1)}$ and costs $O(m\xi)$.
Therefore, multiplying $\mathbf{A}$ and $\mathbf{B}^{(1)}$ over
$\log(m)$ steps costs $O^{\sim}\left(nm^{\omega-2}\xi\right)$.

Next we multiply $\mathbf{A}$ with $\mathbf{B}^{(2)}$. We proceed
in the same way as before, but notice that $\mathbf{A}_{1}\mathbf{B}_{1}^{(2)}$
is no longer needed since $\mathbf{B}_{1}^{(2)}=0$. Multiplying $\mathbf{A}$
and $\mathbf{B}^{(2)}$ also costs $O^{\sim}\left(nm^{\omega-2}\xi\right)$.

Continuing to doing this, gives a costs of $O^{\sim}\left(nm^{\omega-2}\xi\right)$
to multiply $\mathbf{A}$ with the columns $\mathbf{B}^{(i)}$ for
$i$ from $1$ to $\log m$. As before, we recall that $\mathbf{B}_{(j)}^{(i)}=0$
for $j>i$. The overall cost for $i$ from 1 to $\log m$ is therefore
$O^{\sim}\left(nm^{\omega-2}\xi\right)$ to multiply $\mathbf{A}$
and $\mathbf{B}$. \end{proof}




\section{Computational Complexity}

\label{sec:complexityNullspaceBasis}

\begin{comment}
The result of the previous section is a recursive algorithm, shown
in \prettyref{alg:minimalNullspaceBasisWithRankProfile} for the computation
of a minimal kernel. It remains to determine the complexity.

%\input{algorithmkernelBasis.tex}



\subsection{Cost Analysis of Algorithm $\mnb$}
\end{comment}


For the cost analysis we first consider the case where the column
dimension $n$ is not much bigger than the row dimension $m$. 
\begin{thm}
\label{thm:costLowColDimension}If $n\in O\left(m\right)$, then the
cost of \prettyref{alg:minimalNullspaceBasisWithRankProfile} is $O^{\sim}\left(m^{\omega-1}\xi\right)=O^{\sim}\left(m^{\omega-1}\rho\right)$
field operations.%
\begin{comment}
to compute a $\vec{s}$-minimal kernel basis of $\mathbf{F}$. 
\end{comment}
\end{thm}
\begin{proof}
We may assume $m$ is a power of $2$, which can be achieved by appending
zero rows to $\mathbf{F}$. Note that $\rho\in\Theta\left(\xi\right)$
when $n\in O\left(m\right)$. Then the order basis computation at
\prettyref{line:orderBasis} costs $O^{\sim}\left(n^{\omega}s\right)=O^{\sim}\left(m^{\omega-1}\rho\right)$.
The multiplications at \prettyref{line:multiplyFP2} and \prettyref{line:multiplyP2Q}
cost $O^{\sim}\left(nm^{\omega-2}\xi\right)=O^{\sim}\left(m^{\omega-1}\xi\right)$.
The remaining operations including multiplications at \prettyref{line:multiplyG2N1}
and \prettyref{line:multiplyN1N2} cost $O^{\sim}\left(m^{\omega-1}\xi\right)$.
Let $g(m,\xi)$ be the computational cost of the original problem.
Then we have the recurrence relation 
\[
g(m,\xi)\in O^{\sim}(m^{\omega-1}\xi)+g(m/2,\xi)+g(m/2,\xi),
\]
 with the base case $g(1,\xi)\in O^{\sim}\left(\xi\right)$, the cost
of just an order basis computation at $m=1.$ This gives $g(m,\xi)\in O^{\sim}(m^{\omega-1}\xi)$
field operations as the cost of the algorithm. 
\end{proof}
We now consider the general case where the column dimension $n$ can
be much bigger than the row dimension $m$. 
\begin{thm}
\label{thm:costGeneral}\prettyref{alg:minimalNullspaceBasisWithRankProfile}
costs $O^{\sim}\left(n^{\omega}s\right)$ field operations in general.\end{thm}
\begin{proof}
The order basis computation at \prettyref{line:orderBasis} costs
$O^{\sim}\left(n^{\omega}s\right)$ in general, which dominates the
cost of other operations. The problem is then reduced to one where
we have column dimension $O\left(m\right)$, which is handled by \prettyref{thm:costLowColDimension}
with a cost of $O^{\sim}\left(m^{\omega-1}\xi\right)\in O^{\sim}\left(n^{\omega}s\right)$. 
\end{proof}
When we have the important special case where the shift $\vec{s}=\left[s,\dots,s\right]$
is uniform then \prettyref{alg:minimalNullspaceBasisWithRankProfile}
has a lower cost. Indeed we notice that the order basis computation
at \prettyref{line:orderBasis} costs $O^{\sim}\left(n^{\omega-1}ms\right)$
using the algorithm from \prettyref{chap:OrderBasis}. In addition,
the multiplication of $\mathbf{F}$ and $\mathbf{P}_{2}$ at \prettyref{line:multiplyFP2}
and the multiplication of $\mathbf{P}_{2}$ and $\mathbf{Q}$ at \prettyref{line:multiplyP2Q}
both cost $O^{\sim}\left(nm^{\omega-1}s\right)$ as shown in \prettyref{lem:multiplyFP2WithUniformShift}
and \prettyref{lem:multiplyP2QWithUniformShift}. 
\begin{lem}
\label{lem:multiplyFP2WithUniformShift}If the degree of $\mathbf{F}$
is bounded by $s$, then the multiplication of $\mathbf{F}$ and $\mathbf{P}_{2}$
at \prettyref{line:multiplyFP2} costs $O^{\sim}\left(nm^{\omega-1}s\right)$.\end{lem}
\begin{proof}
Since $\mathbf{P}_{2}$ is a part of a $\left(\mathbf{F},3s,\vec{s}\right)$-basis,
its degree is bounded by $3s$. It has dimension $n\times O\left(m\right)$
from \prettyref{thm:dimensionOfPartialNullspaceBasisBasedOnOrder}.
Multiplying $\mathbf{F}$ and $\mathbf{P}_{2}$ therefore costs $(n/m)O^{\sim}\left(m^{\omega}s\right)=O^{\sim}\left(nm^{\omega-1}s\right)$.\end{proof}
\begin{lem}
\label{lem:multiplyP2QWithUniformShift}If $\mathbf{F}$ has degree
$s$, then the multiplication of $\mathbf{P}_{2}$ and $\mathbf{Q}$
at \prettyref{line:multiplyP2Q} costs $O^{\sim}\left(nm^{\omega-1}s\right)$.\end{lem}
\begin{proof}
First note that the dimension of $\mathbf{Q}$ is $O\left(m\right)\times O\left(m\right)$
since it is a $\vec{t}$-minimal kernel basis of $\mathbf{G}=\mathbf{F}\mathbf{P}_{2}/x^{3s}$,
which has dimension $m\times O\left(m\right)$. In addition, by \prettyref{thm:boundOfSumOfShiftedDegreesOfKernelBasis},
the sum of the $\vec{t}$-column degrees of $\mathbf{Q}$ is bounded
by $\sum\vec{t}$, which is bounded by $O\left(ms\right)$ since $\vec{t}$
has $O\left(m\right)$ entries all bounded by $s$.

Now \prettyref{thm:multiplyUnbalancedMatrices} and its proof still
work. The current situation is even simpler as we do not need to subdivide
the columns of $\mathbf{P}_{2}$, which has degree bounded by $3s$
and dimension $n\times O\left(m\right)$. We just need to separate
the columns of $\mathbf{Q}$ to $O\left(\log m\right)$ groups with
degree ranges $\left[0,2s\right],$ $(2s,4s],$ $(4s,8s],$ $\dots$,
and multiply $\mathbf{P}_{2}$ with each group in the same way as
in \prettyref{thm:multiplyUnbalancedMatrices}, with each of these
$O\left(\log m\right)$ multiplications costs $(n/m)O^{\sim}\left(m^{\omega}s\right)=O^{\sim}\left(nm^{\omega-1}s\right)$.\end{proof}
\begin{thm}
\label{thm:costOfMinimalNullspaceBasisWithUniformShift}If $\vec{s}=\left[s,\dots,s\right]$
is uniform, then \prettyref{alg:minimalNullspaceBasisWithRankProfile}
costs $O^{\sim}\left(n^{\omega-1}ms\right)$. \end{thm}
\begin{proof}
After the initial order basis computation, which costs $O^{\sim}\left(n^{\omega-1}ms\right),$
and the multiplication of $\mathbf{F}$ and $\mathbf{P}_{2}$, which
costs $O^{\sim}\left(nm^{\omega-1}s\right)$ from \prettyref{lem:multiplyFP2WithUniformShift},
the column dimension is reduced to $O\left(m\right)$, allowing \prettyref{thm:costLowColDimension}
to apply for computing a $\vec{t}$-minimal kernel basis of $\mathbf{F}\mathbf{P}_{2}/x^{3s}$.
Hence the remaining work costs $O^{\sim}\left(m^{\omega}s\right)$.
The overall cost is therefore dominated by the cost $O^{\sim}\left(n^{\omega-1}ms\right)$
of the initial order basis computation.\end{proof}
\begin{cor}
\label{cor:costOfMinimalNullspaceBasis}If the input matrix $\mathbf{F}$
has degree $d$, then a minimal kernel basis of $\mathbf{F}$ can
be computed with a cost of $O^{\sim}\left(n^{\omega-1}md\right)$. \end{cor}
\begin{proof}
We can just set the shift $\vec{s}$ to $\left[d,\dots,d\right]$
and apply \prettyref{thm:costOfMinimalNullspaceBasisWithUniformShift}.
\end{proof}

