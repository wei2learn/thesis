
\chapter{Preliminaries}

\label{chap:Background}

In this chapter, we provide some of the background needed in order
to understand the basic concepts needed for order basis computation
and nullspace basis computation. %
\begin{comment}
, which provide lower bounds for the computational cost 
\end{comment}
{} %
\begin{comment}
The challenges of balancing input and handling unbalanced output are
discussed along with the techniques which we plan to use to overcome
the difficulties. We review the construction by \citet{Storjohann:2006}
which transforms the inputs to those having dimensions and degree
balance better suited for fast computation and discuss an idea from
\citet{storjohann-villard:2005} for handling the case where the output
degree is unbalanced.
\end{comment}



\section{\label{sec:Notation}Notation}

Since we are interested in computing bases with minimal degrees, it
is useful to have convenient notations for comparing two lists of
degrees.   In addition, our matrices often represent sets of column
vectors, so the arrangement of these columns are not important. To
compare two lists of column degrees from two matrices, we first sort
each list in increasing order, and then do the comparison.
\begin{description}
\item [{Comparing~Unordered~Lists}] For two lists $\vec{a}\in\mathbb{Z}^{n}$
and $\vec{b}\in\mathbb{Z}^{n}$, let $\bar{a}=\left[\bar{a}_{1},\dots,\bar{a}_{n}\right]$
and $\bar{b}=\left[\bar{b}_{1},\dots,\bar{b}_{n}\right]$ be the lists
consists of the entries of $\vec{a}$ and $\vec{b}$ but sorted in
increasing order. 
\[
\begin{cases}
\vec{a}\ge\vec{b} & \mbox{if }\bar{a}_{i}\ge\bar{b}_{i}\mbox{ for all }i\in\left[1,\dots n\right]\\
\vec{a}\le\vec{b} & \mbox{if }\bar{a}_{i}\le\bar{b}_{i}\mbox{ for all }i\in\left[1,\dots n\right]\\
\vec{a}>\vec{b} & \mbox{if }\vec{a}\ge\vec{b}\mbox{ and }\bar{a}_{j}>\bar{b}_{j}\mbox{ for at least one }j\in\left[1,\dots n\right]\\
\vec{a}<\vec{b} & \mbox{if }\vec{a}\le\vec{b}\mbox{ and }\bar{a}_{j}<\bar{b}_{j}\mbox{ for at least one }j\in\left[1,\dots n\right].
\end{cases}
\]

\item [{Summation~Notation}] For a list $\vec{a}=\left[a_{1},\dots,a_{n}\right]\in\mathbb{Z}^{n}$,
we write $\sum\vec{a}$ without index to denote the summation of all
entries in $\vec{a}$. 
\item [{}]~
\item [{Uniformly~Shift~a~List}] For a list $\vec{a}=\left[a_{1},\dots,a_{n}\right]\in\mathbb{Z}^{n}$
and $c\in\mathbb{Z}$, we write $\vec{a}+c$ to denote $\vec{a}+\left[c,\dots,c\right]=\left[a_{1}+c,\dots,a_{n}+c\right]$,
and similarly for $-$.
\item [{Compare~a~List~with~a~Integer}] For a list $\vec{a}=\left[a_{1},\dots,a_{n}\right]\in\mathbb{Z}^{n}$
and $c\in\mathbb{Z}$, we write $\vec{a}<c$ to denote $\vec{a}<\left[c,\dots,c\right]$,
and similarly for $>,\le,\ge,=$.\end{description}
\begin{example}
Let $\mathbf{A}=\left[1,x^{2}\right]$, and $\mathbf{B}=\left[x,1\right]$,
we can write $\cdeg\mathbf{A}>\cdeg\mathbf{B}$, since the sorted
lists of column degrees of $\mathbf{A}$ and $\mathbf{B}$ are $\left[0,2\right]$
and $\left[0,1\right]$ respectively.
\end{example}

\section{Model of Computation}

The computational cost in this thesis is analyzed by bounding the
number of arithmetic operations (additions, subtractions, multiplications,
and divisions) in the coefficient field $\mathbb{K}$ on an algebraic
random access machine. We reduce all the problems to polynomial matrix
multiplications. We use $\MM(n,d)$ to denote the cost of multiplying
two polynomial matrices with dimension $n$ and degree $d$, and $\M(n)$
to denote the cost of multiplying two polynomials with degree $d$.
We assume that $\M(st)\in O\left(\M(s)\M(t)\right)$ and $\M(n)\in O(n^{\omega-1})$,
where the multiplication exponent $\omega$ is assumed to satisfy
$2<\omega\le3$. We take $\MM(n,d)\in O\left(n^{\omega}\M(d)\right)\subset O^{\sim}\left(n^{\omega}d\right)$.
We refer to the book by \citet{vonzurgathen} for more details and
reference about the cost of polynomial multiplication and matrix multiplication.


\section{Computational Cost in Terms of Average Degrees}

In this thesis, we often state the computational costs in terms of
the average column degrees, in contrast to the matrix degrees that
are also the maximum column degrees typically used in the literature.
For example, the cost of a problem involving an input matrix with
dimension $n\times n$ and column degrees $\vec{s}=\left[s_{1},\dots,s_{n}\right]$
is usually $O^{\sim}\left(n^{\omega}s\right)$, where $s=\sum\vec{s}/n$
is the average column degree of the input matrix. This is similar
to the cost of multiplying two matrices with dimension $n$ and degree
$s$.

It may be tempting to also write the cost as $O^{\sim}\left(n^{\omega-1}\xi\right)$
with $\xi=\sum\vec{s}$ being the sum of the column degrees. However,
note that if $\xi$ is asymptotically much smaller than $n$, such
as $\xi\in O\left(\log\left(n\right)\right)$, this cost becomes $O^{\sim}\left(n^{\omega-1}\right)$,
which is incorrect since this is smaller than the size of the input
matrix. This issue is caused by the subtleties involved with using
degrees to measure the size of input matrices. In particular, a degree
0 polynomial does not have size 0. It still has one coefficient, or
one field element to store. The zero polynomial is even more problematic.
Its size is zero but its degree is often defined as $-\infty$, certainly
not zero.

One simple way to address this issue is to use the actual number of
coefficients instead of the degree to measure the size of a polynomial.
The number of coefficients of a polynomial $p$ is 
\[
\begin{cases}
0 & \mbox{if }p=0\\
1+\deg p & \mbox{if }\deg p\ge0.
\end{cases}
\]
This can be applied to polynomial vectors and polynomial matrices
as well. For example, a polynomial column vector with column degree
2 then has 3 column coefficients. Then we can use $\xi$ in the cost
$O^{\sim}\left(n^{\omega-1}\xi\right)$ to represent the sum of the
number of column coefficients for all columns of the input matrix,
which must be at least the number of nonzero columns. If we assume
the zero columns are removed form the input matrix, then $\xi$ must
be at least $n$ , so $n^{\omega-1}\xi$ must be at least $n^{\omega}$,
and a cost of $O^{\sim}\left(n^{\omega-1}\right)$ is not possible.

Perhaps a even easier way is to just use the average column degrees
$s=\sum\vec{s}/n$ in the cost. The cost $O^{\sim}(n^{\omega}s)$
is then similar to the cost of multiplying two matrices of dimension
$n$ and degree $s$. The use of $s$ in the asymptotic Big-$O$ notation
assumes $s$ tends to infinity, hence the problematic cases of zero
polynomials and degree zero polynomials no longer exist. For simplicity,
this is the approach we take in this thesis, that is, for most problems
we describe the costs in terms of the average column degrees instead
of the sum of the column degrees. But keep in mind that the first
approach can always be used to give a more refined cost, as $\sum\vec{s}/n$
in the first approach is not assumed to tend to infinity, but can
be an arbitrary rational number not less than one.

For order basis computations considered in this thesis, it is more
appropriate to measure the sizes using orders instead of column degrees.
The size of a problem with dimension $m\times n$ and order $\sigma$
is $O\left(mn\sigma\right)$. If $m\le n$, then the typical cost
of our algorithms is $O^{\sim}\left(n^{\omega}a\right)$, where $a=m\sigma/n$
can be viewed as the average size of the rows if we treat the original
input matrix as a square matrix by appending $n-m$ zero rows. If
the cost is stated as $O^{\sim}\left(n^{\omega-1}m\sigma\right)$
instead of $O^{\sim}\left(n^{\omega}a\right)$, then we have a similar
situation as before. When $m\sigma\in o\left(n\right)$ such as $m\sigma\in O\left(\log\left(n\right)\right)$,
the cost $O^{\sim}\left(n^{\omega-1}m\sigma\right)$ again becomes
$O^{\sim}\left(n^{\omega-1}\right)$, which is again less than the
cost of multiplying two matrices of dimension $n$ and degree 0. However,
this time it does not violate the upper bound based on the input size,
which is at most $nm\sigma\in o(n^{2})$. In fact, for order basis
computation with balanced shift, we do provide a way, given in \prettyref{sec:removeCeilingFunction}
of \prettyref{chap:OrderBasis}., to handle the case of $m\sigma\in o\left(n\right)$
with a cost of $O^{\sim}\left(n^{\omega-1}m\sigma\right)$.


\section{Shifted Degrees}

In this section, we look at some properties of shifted degrees, which
may help in understanding their usefulness in efficient computations
of polynomial matrix problems.


\begin{lem}
\label{lem:columnDegreesRowDegreesSymmetry}A matrix $\mathbf{A}\in\mathbb{K}\left[x\right]^{m\times n}$
has $\vec{u}$-column degrees bounded by $\vec{v}$ if and only if
its $-\vec{v}$-row degrees are bounded by $-\vec{u}$. %
\begin{comment}
In addition, for a matrix $\mathbf{A}$ with $\vec{u}$-column degrees
$\vec{v}$, it has a full-rank leading $\vec{u}$-column coefficient
matrix if and only if it has a full-rank leading $-\vec{v}$-row coefficient
matrix.
\end{comment}
\end{lem}
\begin{proof}
The lemma follows from the fact $x^{\vec{u}}\mathbf{A}x^{-\vec{v}}$
has degree no more than 0 (Note that the negative degrees are defined
here by setting $\deg x^{-d}=-d$ for $d\in\mathbb{Z}_{>0}$. If one
wishes to avoid negative degrees, one can simply shift the degrees
up by multiplying the matrix by $x^{a}$ for some large $a$). Note
the symmetry between the shifted row degrees and the shifted column
degrees.\end{proof}
\begin{example}
Let $\mathbf{A}=\begin{bmatrix}1,x^{2}\end{bmatrix}$, $\vec{u}=\left[1\right]$
and $\vec{v}=\left[2,4\right]$. Then $\cdeg_{\vec{u}}\mathbf{A}=\left[1,3\right]\le\vec{v}$,
and $\rdeg_{-\vec{v}}\mathbf{A}=\left[-2\right]\le-\vec{u}$. If $\vec{u}=\left[1\right]$
and $\vec{v}=[1,1]$, then $\cdeg_{\vec{u}}\mathbf{A}=\left[1,3\right]\nleq\vec{v}$,
and $\rdeg_{-\vec{v}}\mathbf{A}=\left[1\right]\nleq-\vec{u}$.\end{example}
\begin{lem}
\label{lem:productDegreeBound}If the $\vec{u}$-column degrees of
$\mathbf{A}\in\mathbb{K}\left[x\right]^{m\times n}$ are bounded by
the corresponding entries of an integer list $\vec{v}\in\mathbb{Z}^{n}$,
(or equivalently, the $-\vec{v}$-row degrees of $\mathbf{A}$ are
bounded by $-\vec{u}$) and the $\vec{v}$-column degrees of $\mathbf{B}\in\mathbb{K}\left[x\right]^{n\times k}$
are bounded by $\vec{w}\in\mathbb{Z}^{k}$, then the $\vec{u}$-column
degrees of $\mathbf{A}\mathbf{B}$ are bounded by $\vec{w}$. \end{lem}
\begin{proof}
Note that $x^{\vec{u}}\mathbf{A}x^{-\vec{v}}$ and $x^{\vec{v}}\mathbf{B}^{-\vec{w}}$
have degrees bounded by 0. Therefore 
\[
x^{\vec{u}}\mathbf{A}x^{-\vec{v}}x^{\vec{v}}\mathbf{B}^{-\vec{w}}=x^{\vec{u}}\mathbf{A}\mathbf{B}^{-\vec{w}}
\]
 also has degree bounded by 0, or equivalently, $\cdeg_{\vec{u}}\mathbf{A}\mathbf{B}\le\vec{w}$. \end{proof}
\begin{cor}
\label{lem:boundOnDegreesOfFA}Let $\vec{v}$ be a shift whose entries
bound the corresponding column degrees of $\mathbf{A}$. Then for
any polynomial matrix $\mathbf{B}\in\mathbb{K}\left[x\right]^{n\times k}$,
the column degrees of $\mathbf{A}\mathbf{B}$ are bounded %component-wise 
by the corresponding $\vec{v}$-column degrees of $\mathbf{B}$.\end{cor}
\begin{proof}
Just set the shift $\vec{u}$ to 0 in \prettyref{lem:productDegreeBound}
\end{proof}
\begin{comment}
\begin{lem}
\label{lem:productDegreeBoundWithRowDegrees}If the $\vec{u}$-row
degrees of $\mathbf{A}\in\mathbb{K}\left[x\right]^{m\times n}$ are
bounded by the corresponding entries of an integer list $\vec{v}\in\mathbb{Z}^{n}$,
and the $-\vec{u}$-column degrees of $\mathbf{B}\in\mathbb{K}\left[x\right]^{n\times k}$
are bounded by $\vec{w}\in\mathbb{Z}^{k}$, then the $-\vec{v}$-column
degrees of $\mathbf{A}\mathbf{B}$ are bounded by $\vec{w}$. \end{lem}
\begin{proof}
Note that $x^{-\vec{v}}\mathbf{A}x^{\vec{u}}$ and $x^{-\vec{u}}\mathbf{B}^{-\vec{w}}$
have degrees bounded by 0. Therefore, $x^{-\vec{v}}\mathbf{A}x^{\vec{u}}x^{-\vec{u}}\mathbf{B}^{-\vec{w}}=x^{-\vec{v}}\mathbf{A}\mathbf{B}^{-\vec{w}}$
also has degree bounded by 0, or equivalently, $\cdeg_{-\vec{v}}\mathbf{A}\mathbf{B}\le\vec{w}$.
Alternatively, this also follows from the fact $\cdeg_{-\vec{v}}\mathbf{A}\le-\vec{u}$
from \prettyref{lem:productDegreeBound}.\end{proof}
\end{comment}



\section{Unimodular Matrices and Unimodular Transformations}

A square matrix in $\mathbb{K}\left[x\right]^{n\times n}$ is said
to be \emph{unimodular} if its determinant is in $\mathbb{K}\backslash\left\{ 0\right\} $.
Equivalently, a matrix in $\mathbb{K}\left[x\right]^{n\times n}$
is unimodular if and only if it has an inverse in $\mathbb{K}\left[x\right]^{n\times n}$.
Therefore, a unimodular transformation can be undone by the multiplication
with the inverse transformation matrix in $\mathbb{K}\left[x\right]^{n\times n}$.
This allows us to talk about unimodular equivalence of the polynomial
matrices. Two matrices $\mathbf{A},\mathbf{B}\in\mathbb{K}\left[x\right]^{m\times n}$
are said to be right unimodularly equivalent if $\mathbf{A}=\mathbf{B}\mathbf{U}$
for some unimodular matrix $\mathbf{U}$. This also means that the
columns of $\mathbf{A}$ and $\mathbf{B}$ generates the same set
of vectors. That is, if a vector $\mathbf{q}=\mathbf{A}\mathbf{p}$
for some $\mathbf{p}\in\mathbb{K}\left[x\right]^{n\times1}$, then
$\mathbf{q}=\mathbf{B}\mathbf{U}\mathbf{p}$.


\section{Column Basis}

The column module of a nonzero matrix $\mathbf{A}\in\mathbb{K}\left[x\right]^{m\times n}$
is the $\mathbb{K}\left[x\right]$-module generated by the columns
of $\mathbf{A}$, that is, this module contains all the column vectors
that are $\mathbb{K}\left[x\right]$-linear combinations of the columns
of $\mathbf{A}$. A column basis of $\mathbf{A}$ is just a basis
for this module. Any column basis of $\mathbf{A}$ can be represented
as a matrix $\mathbf{T}$, whose columns are the basis elements. The
matrix $\mathbf{T}$ has full-rank since basis elements must be linearly
independent. Therefore, a column basis $\mathbf{T}$ can always be
obtained from the unimodular matrix that transforms $\mathbf{A}$
to $\mathbf{A}\mathbf{U}=\left[0,\mathbf{T}\right]$ with a full rank
$\mathbf{T}$. A row basis is defined in a similar way as column basis.
\begin{example}
If $\mathbf{A}=\begin{bmatrix}1 & 1+x\\
x & x+x^{2}
\end{bmatrix}$, then $\mathbf{T}=\begin{bmatrix}1\\
x
\end{bmatrix}$ is a column basis of $\mathbf{A}$, as $\mathbf{A}=\mathbf{T}\left[1,1+x\right]$.
Note also that the right factor $\left[1,1+x\right]$ here also turns
out to be a row basis of $\mathbf{A}$.
\end{example}
Any two bases for the matrix are unimodularly equivalent.
\begin{lem}
\label{lem:basisEquivalence}If the matrices $\mathbf{T}_{1}$ and
$\mathbf{T}_{2}$ are both column bases of $\mathbf{A}$, then $\mathbf{T}_{1}$
and $\mathbf{T}_{2}$ are right unimodularly equivalent.\end{lem}
\begin{proof}
Any column of $\mathbf{T}_{1}$ or $\mathbf{T}_{2}$ is generated
by $\mathbf{T}_{1}$ and also by $\mathbf{T}_{2}$. In other words,
$\mathbf{T}_{1}=\mathbf{T}_{2}\mathbf{U}$ and $\mathbf{T}_{2}=\mathbf{T}_{1}\mathbf{V}$
for polynomial matrices $\mathbf{U}$ and $\mathbf{V}$. Hence $\mathbf{T}_{1}=\mathbf{T}_{1}\mathbf{V}\mathbf{U}$
and $\mathbf{T}_{2}=\mathbf{T}_{2}\mathbf{U}\mathbf{V}$, implying
$\mathbf{U}\mathbf{V}=\mathbf{V}\mathbf{U}=I$, which requires both
$\mathbf{U}$ and $\mathbf{V}$ to be unimodular.
\end{proof}
\noindent A column basis is not unique and indeed any column basis
right multiplied by a unimodular polynomial matrix gives another column
basis. As a result, a column basis can have arbitrarily high degree


\section{\label{sec:minimality}Minimality and Column Reducedness}

For many polynomial matrix computation problems, we would like the
output matrix to be not only easy to describe, but also convenient
to use. This usually means the column degrees or the more general
shifted column degrees are small comparing to other matrices that
are right unimodularly equivalent. For example a matrix $\mathbf{A}=\left[x,x^{2},x^{2}\right]$
with column degrees $\left[1,2,2\right]$ can be unimodularly transformed
to a matrix $\mathbf{B}=\left[x,x,x^{2}\right]$ with column degrees
$\left[1,1,2\right]$, which is more desirable with lower degrees. 

To unimodularly transform a matrix to one with lower column degrees,
we can look at its\emph{ leading column coefficient matrix}, which
is defined as follows. 
\begin{defn}
Given a matrix $\mathbf{A}=\left[\mathbf{a}_{1},\dots,\mathbf{a}_{n}\right]\in\mathbb{K}\left[x\right]^{m\times n}$,
the leading column coefficient matrix $A$ of $\mathbf{A}$ is
\begin{align*}
A & =\lcoeff\left(\mathbf{A}\right)\\
 & =\left[\lcoeff\left(\mathbf{a}_{1}\right),\dots,\lcoeff\left(\mathbf{a}_{k}\right)\right]\\
 & =\left[\coeff\left(\mathbf{a}_{1},\cdeg\left(\mathbf{a}_{1}\right)\right),\dots,\coeff\left(\mathbf{a}_{k},\cdeg\left(\mathbf{a}_{k}\right)\right)\right].
\end{align*}

\end{defn}
Then, the matrix $\mathbf{A}$ can be unimodularly reduced to another
matrix with lower column degrees if $\lcoeff\left(\mathbf{A}\right)$
is not full-rank. %
\begin{comment}
In the following, we assume with out loss of generality that matrices
have columns arranged in increasing column degrees. 
\end{comment}

\begin{lem}
\label{lem:columnOperationToReduceDegree}Given a matrix $\mathbf{A}\in\mathbb{K}\left[x\right]^{m\times n}$
with no zero columns. If $\lcoeff\left(\mathbf{A}\right)$ is not
full-rank, then there is a unimodular matrix $\mathbf{U}$ such that
$\cdeg\left(\mathbf{A}\mathbf{U}\right)<\cdeg\mathbf{A}$.\end{lem}
\begin{proof}
We may assume the columns of $\mathbf{A}$ are arranged in increasing
column degrees. Let the column degrees of $\mathbf{A}$ be $\left[d_{1},\dots,d_{n}\right]$.
Let $A=\lcoeff\left(\mathbf{A}\right)$. Suppose the $i$th column
$A_{i}$ of $A$ is a linear combination of the first $i-1$ columns.
That is, $A_{i}=A'a$, where $A'$ is the submatrix of $A$ consists
of the first $i-1$ columns of $A$, and $a=\left[a_{1},\dots,a_{i-1}\right]^{T}\in\mathbb{K}^{(i-1)\times1}$.
Let 
\[
\mathbf{U}=\begin{bmatrix}1 &  &  &  & -a_{1}x^{d_{i}-d_{1}}\\
 & 1 &  &  & -a_{2}x^{d_{i}-d_{2}}\\
 &  & \ddots &  & \vdots\\
 &  &  & 1 & -a_{i-1}x^{d_{i}-d_{i-1}}\\
 &  &  &  & 1\\
 &  &  &  &  & \ddots\\
 &  &  &  &  &  & 1
\end{bmatrix}.
\]
Then $\mathbf{B}=\mathbf{A}\mathbf{U}$ has column degrees $\left[d_{1},\dots,d_{i-1},\bar{d}{}_{i},d_{i+1},\dots,d_{n}\right],$
with $\bar{d}_{i}<d_{i}$.
\end{proof}
As one would expect, $\lcoeff\left(\mathbf{A}\right)$ can only be
full rank if $\mathbf{A}$ is full rank.
\begin{lem}
\label{lem:fullRankLcoeffImpliesFullRank}If a matrix $\mathbf{A}\in\mathbb{K}\left[x\right]^{m\times n}$
is not full rank, then $\lcoeff\left(\mathbf{A}\right)$ is also not
full rank.\end{lem}
\begin{proof}
If $\mathbf{A}$ is not full rank, then it satisfies $\mathbf{A}\mathbf{p}=0$
for some nonzero $\mathbf{p}\in\mathbb{K}\left[x\right]^{n\times1}$.
Let $\vec{a}=\cdeg\mathbf{A}$. Then $\mathbf{A}\mathbf{p}=\left(\mathbf{A}x^{-\vec{a}}\right)\cdot\left(x^{\vec{a}}\mathbf{p}\right)=0$,
which requires the leading coefficient of $\left(\mathbf{A}x^{-\vec{a}}\right)\cdot\left(x^{\vec{a}}\mathbf{p}\right)$
to be zero, that is, $\lcoeff\left(\mathbf{A}x^{-\vec{a}}\right)\lcoeff\left(x^{\vec{a}}\mathbf{p}\right)=0$,
implying $\lcoeff\left(\mathbf{A}x^{-\vec{a}}\right)=\lcoeff\left(\mathbf{A}\right)$
is not full rank.
\end{proof}
One consequence of \prettyref{lem:columnOperationToReduceDegree}
and \prettyref{lem:fullRankLcoeffImpliesFullRank} is that a column
basis can always be obtained by applying the unimodular transformation
from \prettyref{lem:columnOperationToReduceDegree}.
\begin{cor}
\label{cor:unimodularlyReduceToColumnBasis}Any matrix $\mathbf{A}\in\mathbb{K}\left[x\right]^{m\times n}$
can be unimodularly transformed to $\left[0,\mathbf{T}\right]\in\mathbb{K}\left[x\right]^{m\times n}$
with a full rank matrix $\mathbf{T}\in\mathbb{K}\left[x\right]^{m\times r}$,
that is, $\mathbf{A}\mathbf{U}=\left[0,\mathbf{T}\right]$ for some
unimodular matrix $\mathbf{U}$. Any such matrix $\mathbf{T}$ is
a column basis of $\mathbf{A}$.\end{cor}
\begin{proof}
We can just repeatedly apply the unimodular transformation from \prettyref{lem:columnOperationToReduceDegree}.
By \prettyref{lem:fullRankLcoeffImpliesFullRank}, this can eventually
gives a matrix $\left[0,\mathbf{T}\right]$ with a full rank $\mathbf{T}$,
implying that $\mathbf{T}$ is a column basis of $\mathbf{F}$.
\end{proof}
The leading column coefficient matrix can also help to determine the
degree of the determinant of the matrix.
\begin{lem}
\label{lem:fullRankLeadingCoefficientAndDegreeOfDetermant}The degree
of the determinant of a matrix $\mathbf{A}\in\mathbb{K}\left[x\right]^{n\times n}$
is bounded by $\sum\cdeg\mathbf{A}$. If $\lcoeff\left(\mathbf{A}\right)$
is nonsingular, then $\deg\det\mathbf{A}=\sum\cdeg\mathbf{A}$. More
generally, for any shift $\vec{s}\in\mathbb{Z}^{n}$, $\deg\det\mathbf{A}\le\sum\cdeg_{\vec{s}}\mathbf{A}-\sum\vec{s}$.
If $\lcoeff\left(x^{\vec{s}}\mathbf{A}\right)$ is nonsingular, then
$\deg\det\mathbf{A}=\sum\cdeg_{\vec{s}}\mathbf{A}-\sum\vec{s}$.\end{lem}
\begin{proof}
The determinant can be defined as a sum of products, with each product
involving exactly one entry from each column. So the largest possible
degree of each product is $\sum\cdeg\mathbf{A}$. For the second statement,
note that the coefficient of $\det\mathbf{A}$ corresponds to the
largest possible degree $\sum\cdeg\mathbf{A}$ is $\det\lcoeff\left(\mathbf{A}\right)$.
The more general results with shift can be shown by considering the
the determinant of $x^{\vec{s}}\mathbf{A}$.
\end{proof}
It is still not always possible to order two lists of integer degrees
using the comparison described in ``'\prettyref{sec:Notation}, as
in the case of matrices $\left[x,x^{2},x^{2}\right]$ and $\left[x,x,x^{3}\right]$
with column degrees $\left[1,2,2\right]$ and $\left[1,1,3\right]$.
Although the lists of column degrees of the set of unimodular equivalent
matrices are not well-ordered, there always exists some matrices with
the \emph{minimal} column degrees, as shown below below by \prettyref{lem:matrixGCDlowerDegrees}
and \prettyref{cor:minimalUnimodularEquivalentMatrix}.
\begin{lem}
\label{lem:matrixGCDlowerDegrees}Given any two unimodularly equivalent
matrices $\mathbf{A}\in\mathbb{K}\left[x\right]^{m\times n}$ and
$\mathbf{B}\in\mathbb{K}\left[x\right]^{m\times n}$. The matrix $\left[\mathbf{A},\mathbf{B}\right]$
can be unimodularly reduced to $\left[0,\mathbf{C}\right]$ with a
matrix $\mathbf{C}\in\mathbb{K}\left[x\right]^{m\times n}$ that is
unimodularly equivalent to both $\mathbf{A}$ and $\mathbf{B}$, and
satisfies $\cdeg\mathbf{C}\le\cdeg\mathbf{A}$ and $\cdeg\mathbf{C}\le\cdeg\mathbf{B}$.
More generally, with a shift $\vec{s},$ the matrix $\left[\mathbf{A},\mathbf{B}\right]$
can be unimodularly reduced to $\left[0,\mathbf{D}\right]$ with a
matrix $\mathbf{D}$ that is unimodularly equivalent to both $\mathbf{A}$
and $\mathbf{B}$, and satisfies $\cdeg_{\vec{s}}\mathbf{D}\le\cdeg_{\vec{s}}\mathbf{A}$
and $\cdeg_{\vec{s}}\mathbf{D}\le\cdeg_{\vec{s}}\mathbf{B}$. \end{lem}
\begin{proof}
If $r$ is the rank of $\mathbf{A}$ and $\mathbf{B}$, we can repeatedly
apply \prettyref{lem:fullRankLcoeffImpliesFullRank} and \prettyref{lem:columnOperationToReduceDegree}
to compute a matrix $\mathbf{C}$ that has $r$ nonzero linearly independent
columns with column degrees bounded by the column degrees of the $r$
linear independent columns of $\left[\mathbf{A},\mathbf{B}\right]$
with the smallest column degrees. For the more general result with
shift, we can again multiply $x^{\vec{s}}$ to the matrices.\end{proof}
\begin{cor}
\label{cor:minimalUnimodularEquivalentMatrix}Given a matrix $\mathbf{F}\in\mathbb{K}\left[x\right]^{m\times n}$
and a shift $\vec{s}\in\mathbb{Z}^{n}$. There exists a matrix $\mathbf{G}$
that is right unimodularly equivalent to $\mathbf{F}$ and $\cdeg\mathbf{G}\le\cdeg(\mathbf{F}\mathbf{U})$
for any unimodular $\mathbf{U}$. More generally, with a shift $\vec{s}$,
there exists a matrix $\mathbf{H}$ that is right unimodularly equivalent
to $\mathbf{F}$ and $\cdeg_{\vec{s}}\mathbf{G}\le\cdeg_{\vec{s}}(\mathbf{F}\mathbf{U})$
for any unimodular $\mathbf{U}$\end{cor}
\begin{proof}
Just repeatedly apply \prettyref{lem:matrixGCDlowerDegrees} to the
nonzero columns of the matrices that are right unimodularly equivalent
to $\mathbf{F}$. 
\end{proof}
The existence of matrices with minimal column degrees allows us to
define \emph{column reduced.}
\begin{defn}
A matrix $\mathbf{A}\in\mathbb{K}\left[x\right]^{m\times n}$ is said
to be \emph{column reduced} if $\cdeg\mathbf{A}\le\cdeg\mathbf{A}\mathbf{U}$
for any unimodular matrix $\mathbf{U}$. More generally, for a shift
$\vec{s}\in\mathbb{Z}^{n}$, a matrix $\mathbf{A}\in\mathbb{K}\left[x\right]^{m\times n}$
is said to be $\vec{s}$\emph{-column reduced} if $\cdeg_{\vec{s}}\mathbf{A}\le\cdeg_{\vec{s}}\mathbf{A}\mathbf{U}$
for any unimodular matrix $\mathbf{U}$.


\end{defn}
The following results show that the leading column coefficient matrix
provides a useful test for being column reduced.
\begin{lem}
\label{lem:predictableDegree0}If $\mathbf{A}\in\mathbb{K}\left[x\right]^{m\times n}$
has a full rank leading column coefficient matrix $\lcoeff\left(\mathbf{A}\right)$
and $\cdeg\mathbf{A}=\vec{v}$. Then a matrix $\mathbf{B}\in\mathbb{K}\left[x\right]^{n\times k}$
has $\vec{v}$-column degrees $\cdeg_{\vec{v}}\mathbf{B}=\vec{w}$
if and only if $\cdeg\left(\mathbf{A}\mathbf{B}\right)=\vec{w}$. \end{lem}
\begin{proof}
The leading coefficient matrix $B$ of $x^{\vec{v}}\mathbf{B}^{-\vec{w}}$
has no zero column if and only if the leading column coefficient matrix
$AB$ of $\mathbf{A}x^{-\vec{v}}x^{\vec{v}}\mathbf{B}^{-\vec{w}}=\mathbf{A}\mathbf{B}^{-\vec{w}}$
has no zero column, in other words, $x^{\vec{v}}\mathbf{B}^{-\vec{w}}$
has column degrees $\left[0,\dots,0\right]$ if and only if $\mathbf{A}\mathbf{B}^{-\vec{w}}$
has column degrees $\left[0,\dots,0\right]$.\end{proof}
\begin{cor}
\label{cor:fullColumnRankLeadingCoefficientMatrixImpliesColumnReduced}If
$\mathbf{A}\in\mathbb{K}\left[x\right]^{m\times n}$ has a full rank
leading column coefficient matrix, then $\mathbf{A}$ is column reduced. \end{cor}
\begin{proof}
Let $\vec{v}=\cdeg\mathbf{A}$. Any unimodular matrix $\mathbf{U}\in\mathbb{K}\left[x\right]^{n\times n}$
must satisfy $\cdeg_{\vec{v}}\mathbf{U}\ge\vec{v}$. Then by \prettyref{lem:predictableDegree0}
\[
\cdeg\left(\mathbf{A}\mathbf{U}\right)=\cdeg_{\vec{v}}\mathbf{U}\ge\vec{v}=\cdeg\mathbf{A}.
\]
\end{proof}
\begin{cor}
\label{cor:columnReducedLeadingCoefficient}A matrix $\mathbf{A}\in\mathbb{K}\left[x\right]^{m\times n}$
with no zero columns is $\vec{u}$-column reduced if and only if $\lcoeff\left(x^{\vec{u}}\mathbf{A}\right)$
has full column rank.\end{cor}
\begin{proof}
This follows from \prettyref{lem:columnOperationToReduceDegree} and
\prettyref{cor:fullColumnRankLeadingCoefficientMatrixImpliesColumnReduced}.


\end{proof}
In \prettyref{lem:productDegreeBound}, when the matrix $\mathbf{A}$
is $\vec{u}$-column reduced, the bound becomes an equality, which
then gives the following lemma. This can be viewed as a stronger version
of the predictable-degree property \citep{kailath:1980}.
\begin{lem}
\label{lem:predictableDegree}Let $\mathbf{A}\in\mathbb{K}\left[x\right]^{m\times n}$
be a $\vec{u}$-column reduced matrix with no zero columns and with
$\cdeg_{\vec{u}}\mathbf{A}=\vec{v}$. Then a matrix $\mathbf{B}\in\mathbb{K}\left[x\right]^{n\times k}$
has $\vec{v}$-column degrees $\cdeg_{\vec{v}}\mathbf{B}=\vec{w}$
if and only if $\cdeg_{\vec{u}}\left(\mathbf{A}\mathbf{B}\right)=\vec{w}$. \end{lem}
\begin{proof}
From \prettyref{cor:columnReducedLeadingCoefficient}, $\vec{u}$-column
reduced means the leading column coefficient matrix $A$ of $x^{\vec{u}}\mathbf{A}x^{-\vec{v}}$
has linearly independent columns. Now the result follows by using
\prettyref{lem:predictableDegree0} on $x^{\vec{u}}\mathbf{A}$.
\end{proof}
We also have the following similar result on column reducedness.
\begin{lem}
\label{lem:predictableColumnReducedness}If $\mathbf{A}$ is a full-rank
$\vec{u}$-column reduced matrix with $\cdeg_{\vec{u}}\mathbf{A}=\vec{v}$,
then $\mathbf{B}$ is $\vec{v}$-column reduced if and only if $\mathbf{A}\mathbf{B}$
is column reduced.\end{lem}
\begin{proof}
This again follows by looking at the full rank leading column coefficient
matrices.
\end{proof}

\section{Order Basis}

We now look at order basis in more detail.

Let $\mathbb{K}$ be a field, $\mathbf{F}\in\mathbb{K}\left[\left[x\right]\right]^{m\times n}$
a matrix of power series and $\vec{\sigma}=\left[\sigma_{1},\dots,\sigma_{m}\right]$
a vector of non-negative integers. 
\begin{defn}
We say a column vector of polynomials $\mathbf{p}\in\mathbb{K}\left[x\right]^{n\times1}$
has \emph{order} $\left(\mathbf{F},\vec{\sigma}\right)$ (or \emph{order}
$\vec{\sigma}$ with respect to $\mathbf{F}$) if $\mathbf{F}\cdot\mathbf{p}\equiv\mathbf{0}\mod x^{\vec{\sigma}}$,
that is, 
\[
\mathbf{F}\cdot\mathbf{p}=x^{\vec{\sigma}}\mathbf{r}=\begin{bmatrix}x^{\sigma_{1}}\\
 & \ddots\\
 &  & x^{\sigma_{m}}
\end{bmatrix}\mathbf{r}
\]
 for some $\mathbf{r}\in\mathbb{K}\left[\left[x\right]\right]^{m\times1}$.
If $\vec{\sigma}=\left[\sigma,\dots,\sigma\right]$ has entries uniformly
equal to $\sigma$, then we say that $\mathbf{p}$ has order $\left(\mathbf{F},\sigma\right).$
\begin{comment}
The vector of power series $\mathbf{r}$ is called the order $\left(\mathbf{F},\sigma\right)$-residual
of \textbf{$\mathbf{p}$}. 
\end{comment}
{} The set of all order $\left(\mathbf{F},\vec{\sigma}\right)$ vectors
is a free $\mathbb{K}\left[x\right]$-module denoted by $\left\langle \left(\mathbf{F},\vec{\sigma}\right)\right\rangle $. 
\end{defn}
An order basis for $\mathbf{F}$ and $\vec{\sigma}$ is simply a basis
for the $\mathbb{K}\left[x\right]$-module $\left\langle \left(\mathbf{F},\vec{\sigma}\right)\right\rangle $.
In this thesis we compute those order bases having minimal or shifted
minimal degrees (also referred to as a reduced order basis in \citep{BL1997}).



An order basis \citep{BeLa94,BL1997} $\mathbf{P}$ of $\mathbf{F}$
with order $\vec{\sigma}$ and shift $\vec{s}$, or simply an $\left(\mathbf{F},\vec{\sigma},\vec{s}\right)$-basis,
is a basis for the module $\left\langle \left(\mathbf{F},\vec{\sigma}\right)\right\rangle $
\begin{comment}
\[
\left\langle \left(\mathbf{F},\vec{\sigma}\right)\right\rangle =\{\mathbf{p}\in\mathbb{K}\left[x\right]^{n\times1}\|\mathbf{F}\cdot\mathbf{p}=x^{\vec{\sigma}}\mathbf{r},\mathbf{r}\in\mathbb{K}[[x]]^{m\times1}\}
\]
\end{comment}
{} having minimal $\vec{s}$-column degrees. If $\vec{\sigma}=\left[\sigma,\dots,\sigma\right]$
is uniform then we simply write $\left(\mathbf{F},\sigma,\vec{s}\right)$-basis.
The precise definition of an $\left(\mathbf{F},\vec{\sigma},\vec{s}\right)$-basis
is as follows.
\begin{defn}
\label{def:orderBasis}A polynomial matrix $\mathbf{P}$ is an order
basis of $\mathbf{F}$ of order $\vec{\sigma}$ and shift $\vec{s}$,
denoted by $\left(\mathbf{F},\vec{\sigma},\vec{s}\right)$-basis,
if the following properties hold: 
\begin{enumerate}
\item $\mathbf{P}$ is a nonsingular matrix of dimension $n$ and is $\vec{s}$-column
reduced. 
\item $\mathbf{P}$ has order $\left(\mathbf{F},\vec{\sigma}\right)$ (or
equivalently, each column of $\mathbf{P}$ is in $\left\langle (\mathbf{F},\vec{\sigma})\right\rangle $). 
\item Any $\mathbf{q}\in\left\langle \left(\mathbf{F},\vec{\sigma}\right)\right\rangle $
can be expressed as a linear combination of the columns of $\mathbf{P}$,
given by $\mathbf{P}^{-1}\mathbf{q}$. 
\end{enumerate}
\end{defn}
\begin{comment}
Note that the module $\left\langle \left(\mathbf{F},\vec{\sigma}\right)\right\rangle $
does not depend on the shift $\vec{s}$. 
\end{comment}




It follows from \prettyref{def:orderBasis} and \prettyref{lem:basisEquivalence}
that any pair of $\left(\mathbf{F},\vec{\sigma},\vec{s}\right)$-bases
$\mathbf{P}$ and $\mathbf{Q}$ are column bases of each other and
are unimodularly equivalent.

From \citep{BL1997} we have the following lemma. 
\begin{lem}
\label{lem:orderBasisEquivalence}The following are equivalent for
a polynomial matrix \textbf{$\mathbf{P}$}: 
\begin{enumerate}
\item $\mathbf{P}$ is a $\left(\mathbf{F},\vec{\sigma},\vec{s}\right)$-basis. 
\item $\mathbf{P}$ is comprised of a set of $n$ minimal $\vec{s}$-column
degree polynomial vectors that are linearly independent and each having
order $\left(\mathbf{F},\vec{\sigma}\right)$. 
\item $\mathbf{P}$ does not contain a zero column, has order $\left(\mathbf{F},\vec{\sigma}\right)$,
is $\vec{s}$-column reduced, and any $\mathbf{q}\in\left\langle \left(\mathbf{F},\vec{\sigma}\right)\right\rangle $
can be expressed as a linear combination of the columns of $\mathbf{P}$.

\end{enumerate}
\end{lem}
In some cases an entire order basis is unnecessary and instead one
looks for a minimal basis that generates only the elements of $\left\langle \left(\mathbf{F},\vec{\sigma}\right)\right\rangle $
with $\vec{s}$-column degrees bounded by a given $\delta$. Such
a minimal basis is a partial $\left(\mathbf{F},\vec{\sigma},\vec{s}\right)$-basis
comprised of elements of a $\left(\mathbf{F},\vec{\sigma},\vec{s}\right)$-basis
with $\vec{s}$-column degrees bounded by $\delta$. This is called
a \emph{minbasis} in \citet{Storjohann:2006}. 
\begin{defn}
\label{def:genset} Let $\left\langle \left(\mathbf{F},\vec{\sigma},\vec{s}\right)\right\rangle _{\delta}\subset\left\langle \left(\mathbf{F},\vec{\sigma}\right)\right\rangle $
denote the set of order $\left(\mathbf{F},\vec{\sigma}\right)$ polynomial
vectors with $\vec{s}$-column degree bounded by $\delta$. A $\left(\mathbf{F},\vec{\sigma},\vec{s}\right)_{\delta}$-basis
is a polynomial matrix $\mathbf{P}$ not containing a zero column
and satisfying: \end{defn}
\begin{enumerate}
\item $\mathbf{P}$ has order $\left(\mathbf{F},\vec{\sigma}\right).$ 

\begin{enumerate}
\item Any element of $\left\langle \left(\mathbf{F},\vec{\sigma},\vec{s}\right)\right\rangle _{\delta}$
can be expressed as a linear combination of the columns of $\mathbf{P}$. 
\item $\mathbf{P}$ is $\vec{s}$-column reduced. 
\end{enumerate}
\end{enumerate}
\begin{comment}
As before, the linear combination here is in fact unique. 
\end{comment}
A $\left(\mathbf{F},\vec{\sigma},\vec{s}\right)_{\delta}$-basis is,
in general, not square unless $\delta$ is large enough to contain
all $n$ basis elements in which case it is a complete $\left(\mathbf{F},\vec{\sigma},\vec{s}\right)$-basis.


\section{Kernel Basis}

Recall that the kernel of $\mathbf{F}\in\mathbb{K}\left[x\right]^{m\times n}$
is the $\mathbb{F}\left[x\right]$-module 
\[
\left\{ \mathbf{p}\in\mathbb{K}\left[x\right]^{n}~|~\mathbf{F}\mathbf{p}=0\right\} .
\]
 A kernel basis of $\mathbf{F}$ is just a basis of this module. Kernel
bases are closely related to order bases, as can be seen from the
following definitions. 
\begin{defn}
\label{def:kernelBasis}Given $\mathbf{F}\in\mathbb{K}\left[x\right]^{m\times n}$,
a polynomial matrix $\mathbf{N}\in\mathbb{K}\left[x\right]^{n\times*}$
is a (right) kernel basis of $\mathbf{F}$ if the following properties
hold: 
\begin{enumerate}
\item $\mathbf{N}$ is full-rank. 
\item $\mathbf{N}$ satisfies $\mathbf{F}\cdot\mathbf{N}=0$. 
\item Any $\mathbf{q}\in\mathbb{K}\left[x\right]^{n}$ satisfying $\mathbf{F}\mathbf{q}=0$
can be expressed as a linear combination of the columns of $\mathbf{N}$,
that is, there exists some polynomial vector $\mathbf{p}$ such that
$\mathbf{q}=\mathbf{N}\mathbf{p}$. 
\end{enumerate}
\end{defn}
Again, it follows from \prettyref{def:kernelBasis} and \prettyref{lem:basisEquivalence}
that any pair of kernel bases $\mathbf{N}$ and $\mathbf{M}$ of $\mathbf{F}$
are column bases of each other and are unimodularly equivalent.

An $\vec{s}$-minimal kernel basis of $\mathbf{F}$ is just a kernel
basis that is $\vec{s}$-column reduced.
\begin{defn}
Given $\mathbf{F}\in\mathbb{K}\left[x\right]^{m\times n}$, a polynomial
matrix $\mathbf{N}\in\mathbb{K}\left[x\right]^{n\times*}$ is a $\vec{s}$-minimal
(right) kernel basis of $\mathbf{F}$ if\textbf{ $\mathbf{N}$} is
a kernel basis of $\mathbf{F}$ and $\mathbf{N}$ is $\vec{s}$-column
reduced. We also call a $\vec{s}$-minimal (right) kernel basis of
$\mathbf{F}$ a $\left(\mathbf{F},\vec{s}\right)$-kernel basis in
this thesis.
\end{defn}




\begin{comment}
Note that the module $\left\langle \left(\mathbf{F},\vec{\sigma}\right)\right\rangle $
does not depend on the shift $\vec{s}$. 
\end{comment}


\begin{comment}
Minimal kernel bases can be directly computed via order basis computation.
Indeed if the order $\sigma$ of a $\left(\mathbf{F},\sigma,\vec{s}\right)$-basis
$\mathbf{P}$ is high enough, then $\mathbf{P}$ contains a $\vec{s}$-minimal
kernel basis $\mathbf{N}$. However, this approach may require the
order $\sigma$ to be quite high. For example, if $\mathbf{F}$ has
degree $d$ and $\vec{s}$ is uniform, then its minimal kernel bases
can have degree up to $md$. In that case the order $\sigma$ would
need to be set to $d+md$ in the order basis computation in order
to fully compute a minimal kernel basis. The fastest method of computing
such a $\left(\mathbf{F},d+md\right)$-basis would cost $O^{\sim}\left(n^{\omega}\left\lceil m^{2}d/n\right\rceil \right)$
using the algorithm from \citep{za2009}. We can see from this cost
that there is room for improvement when $m$ is large. For example,
in the worst case when $m\in\Theta\left(n\right)$, this cost would
be $O^{\sim}\left(n^{\omega+1}d\right)$. This points to a root cause
for the inefficiency in this approach. Namely, when $m$ is large,
the computed kernel basis, which can have a column dimension of $n-m$,
is a much smaller subset of the order basis computed. Hence considerable
effort is put in the computation of order basis elements that are
not part of a nullspace basis. A key to reducing the cost is therefore
to reduce such computation of unneeded order basis elements, which
is achieved in our algorithm by only using order basis computation
to compute partial nullspace bases of low degrees.
\end{comment}

