
\section{The Case $m\sigma\in o(n)$}

\label{sec:removeCeilingFunction}

We have been assuming $m\sigma\in\Omega(n)$ so far. With this assumption,
we can transform the original problem with dimension $m\times n$
and degree $\sigma$ to one with dimension $\Theta(n)\times\Theta(n)$
and degree $\Theta(d)=\Theta(m\sigma/n)$, allowing order basis computation
to be efficient with a final cost of $O^{\sim}(n^{\omega}d)$. However,
if $m\sigma\in o(n)$, the average degree $d=m\sigma/n\in o(1)$ but
$1$ is the lowest possible degree and $m\sigma$ is the maximum possible
row dimension of our transformed problems. In other words, we cannot
obtain a nearly square transformed problem for our algorithms to behave
efficiently. In this case, our algorithms still require $O^{\sim}(n^{\omega}$$\left\lceil d\right\rceil )=O^{\sim}(n^{\omega})$
field operations. We now look how this cost can be improved to $O^{\sim}(n^{\omega}d)=O^{\sim}(n^{\omega-1}m\sigma)$.


\subsection{Balanced Case}

First note that in this case, using \prettyref{def:balancedShift},
a balanced shift $\vec{s}$ is also uniform, since $\max\left(\vec{s}\right)-\min\left(\vec{s}\right)\in O\left(m\sigma/n\right)\subseteq o(1)$,
which makes $\max\left(\vec{s}\right)-\min\left(\vec{s}\right)=0$.
So let us just consider the uniform shift case.

We first compute all degree 0 basis elements, which then helps to
eliminate the columns of the input that are never going to be needed
as pivots. The remaining columns can then be used as the input to
compute the remaining basis elements efficiently. The degree 0 elements
of a $\left(\mathbf{F},\sigma\right)$-basis%
\begin{comment}
nullspace basis elements of $\mathbf{F}$
\end{comment}
{} correspond to a nullspace basis of a linearized matrix 
\[
\bar{F}=\left[\begin{array}{c}
F_{0}\\
F_{1}\\
F_{2}\\
\vdots\\
F_{\sigma-1}
\end{array}\right]\in\mathbb{K}^{(m\sigma)\times n}
\]
of $\mathbf{F}=F_{0}+F_{1}x+F_{2}x^{2}+\cdots+F_{\sigma-1}x^{\sigma-1}$.
\begin{lem}
The elements of a nullspace basis of $\bar{F}$ over $\mathbb{K}$
are also the degree 0 elements of a $\left(\mathbf{F},\sigma\right)$-basis%
\begin{comment}
a minimal nullspace basis of $\mathbf{F}$
\end{comment}
.\end{lem}
\begin{proof}
The columns of $\bar{F}$ and the columns of $\mathbf{F}$ are equivalent
representations of the same elements of the same $\mathbb{K}$-module,
which is also a vector space over $\mathbb{K}$.
\end{proof}
To compute these basis elements, we can use the Gauss Jordan transform
algorithm from \citet{storjohann:phd2000} on $\bar{F}$ with a cost
of $O\left(nm\sigma\bar{r}^{\omega-2}\right)$, where $\bar{r}\le m\sigma$
is the rank of $\bar{F}$. The algorithm finds a permutation matrix
$P$ and a unimodular matrix $U$ in $\mathbb{K}^{n\times n}$ such
that $\bar{F}PU$ is in the reduced column echelon form of $\bar{F}$.
Note that $P$ permutes the columns of $\bar{F}$ so that the first
$\bar{r}$ columns of $\bar{F}$ are linearly independent. Let $\left[U_{1},U_{0}\right]:=U$
with $U_{0}$ correspond to the zero columns of $\bar{F}PU$. Then
the matrix consists of the bottom $n-\bar{r}$ rows of $U_{0}$ is
the identity matrix, and only the first $\bar{r}$ rows of $U_{1}$
are nonzero. Because of this simpler structure after permutation,
let us compute a $\left(\mathbf{F}P,\sigma\right)$-basis $\mathbf{P}$
instead, which also gives us a $\left(\mathbf{F},\sigma\right)$-basis
$P\mathbf{P}$. Notice that $U_{0}$ consists of all the degree 0
elements of a $\left(\mathbf{F}P,\sigma\right)$-basis. We can then
use $\mathbf{F}PU_{1}$ as the input matrix to compute the remaining
basis elements. But to further simplify our future computation, let
us replace $U_{1}$ with $V=\left[I,0\right]^{T}$ of the same dimension,
where the identity matrix $I$ replaces the first nonzero $\bar{r}$
rows in $U_{1}$. In essence, $PV$ picks $\bar{r}$ columns from
$\mathbf{F}$ for computing the remaining basis elements. Since $U_{0}$
has at least $n-m\sigma$ columns, there are at most $m\sigma$ columns
in $U_{1}$, and hence at most $m\sigma$ columns in $V$ and in $\mathbf{F}PV$. 
\begin{lem}
\begin{comment}
Let If $\bar{F}P\left[U_{2},U_{0}\right]$ is in the reduced column
echelon form of $\bar{F}$ with $U_{0}$ being a nullspace basis of
$\bar{F}P$, and i
\end{comment}
If we compute a $\left(\mathbf{F}PV,\sigma\right)$-basis $\mathbf{Q}$,
then $\left[V\mathbf{Q},U_{0}\right]$ is a $\left(\mathbf{F}P,\sigma\right)$-basis.\end{lem}
\begin{proof}
Note that the matrix $\left[V,U_{0}\right]$, which has the structure
\[
\begin{bmatrix}I & *\\
0 & I
\end{bmatrix}
\]
 with $*$ representing the first $r$ rows of $U_{0}$, is a $\left(\mathbf{F}P,0\right)$-basis
since it is unimodular and column reduced. From \ref{thm:combineOrderBases},
we can use the residual $\mathbf{F}P[V,U_{0}]=\left[\mathbf{F}PV,0\right]$
to compute a $\left(\left[\mathbf{F}PV,0\right],\sigma\right)$-basis
$\bar{\mathbf{Q}}$, then $[V,U_{0}]\bar{\mathbf{Q}}$ is a $\left(\mathbf{F}P,\sigma\right)$-basis.
Also note that if $\mathbf{Q}$ is a $\left(\mathbf{F}PV,\sigma\right)$-basis,
then 
\[
\bar{\mathbf{Q}}=\begin{bmatrix}\mathbf{Q}\\
 & I
\end{bmatrix}
\]
 is a $\left(\left[\mathbf{F}PV,0\right],\sigma\right)$-basis, and
$[V,U_{0}]\bar{\mathbf{Q}}=\left[V\mathbf{Q},U_{0}\right]$ is a $\left(\mathbf{F}P,\sigma\right)$-basis.
\end{proof}
Our new problem of computing a $\left(\mathbf{F}PV,\sigma\right)$-basis
now satisfies the condition of having column dimension bounded by
$m\sigma$.%
\begin{comment}
\begin{lem}
The column dimensions of $V$ and $\mathbf{F}PV$ are bounded by $m\sigma$.\end{lem}
\begin{proof}
The rank of $\bar{F}$ is bounded by $m\sigma$. This means its nullspace
basis $U_{0}$ has at least $n-m\sigma$ columns. Therefore $V$ has
at most $m\sigma$ columns.\end{proof}
\end{comment}
{} We can therefore compute a $\left(\mathbf{F}PV,\sigma\right)$-basis
using \prettyref{alg:mab} with a cost of $O^{\sim}\left(\left(m\sigma\right)^{\omega}\right)=O^{\sim}\left(n^{\omega-1}m\sigma\right)$.

The last thing to check is making sure that the multiplications for
computing the residual $\mathbf{F}PV$, and for combining the results
$V\mathbf{Q}$, and for obtaining the final result $P\left[V\mathbf{Q},U_{0}\right]$
can all be done efficiently, which is not difficult since $P$ is
a permutation matrix, and $V$ consists of an identity matrix and
zeros. Therefore, the $(\mathbf{F},\sigma)$-basis $P\left[V\mathbf{Q},U_{0}\right]$
can be computed with a cost of $O^{\sim}\left(n^{\omega-1}m\sigma\right)$.
This allows us to remove the assumption of $m\sigma\in\Omega(n)$,
and to extend \prettyref{thm:balancedCost} to the situation of $m\sigma\in o(n)$.
\begin{thm}
\label{thm:orderBasisCostCeilingRemoved}When $m\sigma\in o(n)$ and
the shift $\vec{s}$ is balanced, a $\left(\mathbf{F},\sigma,\vec{s}\right)$-basis
can be computed with a cost of $O\left(n^{\omega}\bar{\M}(d)\log\sigma)\right)=O\left(n^{\omega}d\log d\log\log d\log\sigma)\right)\subset O^{\sim}\left(n^{\omega}d\right)$
field operations, where $d=m\sigma/n$. %
\begin{comment}
Should easily work for the second unbalanced case as well. But need
to write it out.
\end{comment}
\end{thm}

