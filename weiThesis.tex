%% LyX 2.0.5.1 created this file.  For more info, see http://www.lyx.org/.
%% Do not edit unless you really know what you are doing.
\documentclass[12pt,english,letterpaper,titlepage,oneside,final]{report}
\usepackage[T1]{fontenc}
\usepackage[latin9]{inputenc}
\setcounter{secnumdepth}{3}
\setcounter{tocdepth}{3}
\usepackage{color}
\usepackage{verbatim}
\usepackage{prettyref}
\usepackage{refstyle}
\usepackage{float}
\usepackage{amsthm}
\usepackage{amsmath}
\usepackage{amssymb}
\usepackage{setspace}
\usepackage[authoryear]{natbib}
\doublespacing

\makeatletter

%%%%%%%%%%%%%%%%%%%%%%%%%%%%%% LyX specific LaTeX commands.

\let\pr@chap=\pr@cha
\AtBeginDocument{\providecommand\secref[1]{\ref{sec:#1}}}
\AtBeginDocument{\providecommand\lemref[1]{\ref{lem:#1}}}
\AtBeginDocument{\providecommand\chapref[1]{\ref{chap:#1}}}
\floatstyle{ruled}
\newfloat{algorithm}{tbp}{loa}[chapter]
\providecommand{\algorithmname}{Algorithm}
\floatname{algorithm}{\protect\algorithmname}
\RS@ifundefined{subref}
  {\def\RSsubtxt{section~}\newref{sub}{name = \RSsubtxt}}
  {}
\RS@ifundefined{thmref}
  {\def\RSthmtxt{theorem~}\newref{thm}{name = \RSthmtxt}}
  {}
\RS@ifundefined{lemref}
  {\def\RSlemtxt{lemma~}\newref{lem}{name = \RSlemtxt}}
  {}


%%%%%%%%%%%%%%%%%%%%%%%%%%%%%% Textclass specific LaTeX commands.
 \usepackage{algorithmic}
\floatstyle{ruled}
\newfloat{algorithm}{tbp}{loa}
\floatname{algorithm}{Algorithm}
\usepackage{algorithmic}
\newcommand{\forbody}[1]{ #1 \ENDFOR }
\newcommand{\ifbody}[1]{ #1  \ENDIF}
\newcommand{\whilebody}[1]{ #1  \ENDWHILE}
\renewcommand{\algorithmicprint}{\textbf{draw}}
\renewcommand{\algorithmicrequire}{\textbf{Input:}}
\renewcommand{\algorithmicensure}{\textbf{Output:}}

\theoremstyle{plain}
\ifx\thechapter\undefined
\newtheorem{thm}{\protect\theoremname}
\else
\newtheorem{thm}{\protect\theoremname}[chapter]
\fi
  \theoremstyle{definition}
  \newtheorem{example}[thm]{\protect\examplename}
  \theoremstyle{plain}
  \newtheorem{lem}[thm]{\protect\lemmaname}
  \theoremstyle{plain}
  \newtheorem{cor}[thm]{\protect\corollaryname}
  \theoremstyle{definition}
  \newtheorem{defn}[thm]{\protect\definitionname}
 \usepackage{algolyx}
 \usepackage{algolyx}
  \theoremstyle{remark}
  \newtheorem{rem}[thm]{\protect\remarkname}

%%%%%%%%%%%%%%%%%%%%%%%%%%%%%% User specified LaTeX commands.
% UW Thesis Template for LaTeX -- Updated Mar 27, 2008 by Stephen Carr
% FOR ASSISTANCE, please send mail to rt-IST-CSmathsci@ist.uwaterloo.ca

% Effective October 2006, the University of Waterloo 
% requires electronic thesis submission
% (http://www.grad.uwaterloo.ca/students/current/thesis_regulations.asp).
% However, many faculties/departments also require one or more printed
% copies. This template attempts to satisfy both types of output. 
% It is based on the standard "report" document class. The "book" document 
% class can be substituted if you have a very large, multi-part thesis.
% To the best of our knowledge, it satisfies the current UW
% requirements (http://www.grad.uwaterloo.ca/Thesis_Regs/thesistofc.asp).
% However, it is your responsibility to assure that you have met all 
% requirements of the University and your particular department.
% Many thanks to Colin Alie for assistance in preparing this updated template.

% -----------------------------------------------------------------------

% By default, output is produced that is geared toward generating a PDF 
% version optimized for viewing on an electronic display, including 
% hyperlinks within the PDF.
 
% To process a thesis called "mythesis.tex" based on this template, run:

% pdflatex mythesis	-- first pass of the pdflatex processor
% bibtex mythesis	-- generates bibliography from .bib data file(s) 
% pdflatex mythesis	-- fixes cross-references, bibliographic references, etc
% pdflatex mythesis	-- fixes cross-references, bibliographic references, etc

% N.B. The "pdftex" program allows graphics in the following formats to be
% included with the "\includegraphics" command: PNG, PDF, JPEG, TIFF
% Tip 1: Generate your figures and photos in the size you want them to appear
% in your thesis, rather than scaling them with \includegraphics options.
% Tip 2: Any drawings you do should be in scalable vector graphic formats:
% SVG, PNG, WMF, EPS and then converted to PNG or PDF, so they are scalable in
% the final PDF as well.
% Tip 3: Photographs should be cropped and compressed so as not to be too large.

% To create PDF output that is optimized for double-sided printing: 
%
% 1) comment-out the \documentclass statement in the preamble below, and
% un-comment the second \documentclass line.
%
% 2) change the value assigned below to the boolean variable
% "ElectronicVersion" from "true" to "false".

% --------------------- Start of Document Preamble -----------------------

% Specify the document class, default style attributes, and page dimensions
% For hyperlinked PDF, suitable for viewing on a computer, use this:

 
% For un-hyperlinked PDF, suitable for double-sided printing, use this:
%\documentclass[letterpaper,12pt,titlepage,openright,twoside,final]{report}

% Some LaTeX commands I define for my own nomenclature.
% If you have to, it's better to change nomenclature once here than in a 
% million places throughout your thesis!
\newcommand{\package}[1]{\textbf{#1}} % package names in bold text
\newcommand{\cmmd}[1]{\textbackslash\texttt{#1}} % command name in tt font 
\newcommand{\href}[1]{#1} % does nothing, but defines the command so the
    % print-optimized version will ignore \href tags (redefined by hyperref pkg).
%\newcommand{\texorpdfstring}[2]{#1} % does nothing, but defines the command
% Anything defined here may be redefined by packages added below...

% This package allows if-then-else control structures.
\usepackage{ifthen}

\newboolean{ElectronicVersion}
\setboolean{ElectronicVersion}{true} % CHANGE THIS AS REQUIRED

% allows easy specifying of the page layout
\usepackage{geometry}

\usepackage{nomencl}

\usepackage{amstext}
 % Lots of math symbols and environments
 % For including graphics N.B. pdftex graphics driver 
\usepackage{geometry}
 % allows easy specifying of the page layout

% Hyperlinks make it very easy to navigate an electronic document.
% The following statement places "bookmarks" in the PDF document
% linking Table of Contents, List of Figures, List of Tables, and
% List of References entries to their appearance in the body of the
% text.  In addition, this is where one would specify the thesis title
% and author as it should appear in the properties of the PDF
% document.
% Use the "hyperref" package only for the electronic version:
% N.B. HYPERREF MUST BE THE LAST PACKAGE LOADED; ADD ADDITIONAL PKGS ABOVE
\ifthenelse{\boolean{ElectronicVersion}}{
    \usepackage[letterpaper=true,pdftex,bookmarks,pagebackref,
plainpages=false,         pdfpagelabels=true,         pdftitle=UW\ LaTeX\ Thesis\ Template,         pdfauthor=Pat\ Neugraad	         ]{hyperref}
}{}

% UW thesis requirements specify a minimum of 1" (72pt) margin at the
% top, bottom, and outside page edges and a 1.125" (81pt) gutter
% margin.  More space is needed for the gutter margin so that the text
% remains legible after the printed material is bound (not an issue
% for an electronic document).  The following statement specifies a
% page layout which is suitable for both electronic and print versions
% of the thesis (a 20pt shift to the outside of the page is
% automatically imposed when the print version is being generated
% providing the wider gutter margin).
\geometry{
  verbose,
  dvips,
  width=422.695pt, marginparsep=0pt, marginparwidth=0pt,
  top=72.27pt, headheight=12pt, headsep=36pt, footskip=30pt, bottom=72.27pt
}

% The following statement specifies the amount of space between
% paragraphs.  Other reasonable specifications are \bigskipamount and \smallskipamount.


% The following statement controls the line spacing.  The default
% spacing corresponds to good typographic conventions and only slight
% changes (e.g., perhaps "1.2"), if any, should be made.
\renewcommand{\baselinestretch}{1}     

% By default, each chapter will start on a recto (right-hand side)
% page.  In some cases, this will require that the verso page be
% blank and, while it should be counted, a page number should not be
% printed.  The following statements ensure a page number is not
% printed on an otherwise blank verso page.
\let\origdoublepage\cleardoublepage
\newcommand{\clearemptydoublepage}{%
  \clearpage{\pagestyle{empty}\origdoublepage}}
\let\cleardoublepage\clearemptydoublepage


%%%%%%%%%%%%%%
\renewcommand{\algorithmicrequire}{\textbf{Input:}}
\renewcommand{\algorithmicensure}{\textbf{Output:}}

\def\diag{\mbox{diag}}
\def\cdeg{\qopname\relax n{cdeg}}
\def\rdeg{\qopname\relax n{rdeg}}
%\keycomment{\{\{}{\}\}}
\def\MM{\qopname\relax n{MM}}
\def\M{\qopname\relax n{M}}
\def\ord{\qopname\relax n{ord}}

\def\StorjohannTransform{\qopname\relax n{StorjohannTransform}}
\def\TransformUnbalanced{\qopname\relax n{TransformUnbalanced}}
\def\rowDimension{\qopname\relax n{rowDimension}}
\def\columnDimension{\qopname\relax n{columnDimension}}
\DeclareMathOperator{\re}{rem}
\DeclareMathOperator{\coeff}{coeff}
\DeclareMathOperator{\lcoeff}{lcoeff}
\DeclareMathOperator{\inv}{inverse}
\DeclareMathOperator{\rev}{rev}
\DeclareMathOperator{\colRev}{colRev}
\DeclareMathOperator{\rowRev}{rowRev}
\DeclareMathOperator{\unimodularCompletion}{unimodularCompletion}
\DeclareMathOperator{\hermiteDiagonal}{hermiteDiagonal}
\DeclareMathOperator{\hermiteDiagonalWithScale}{hermiteDiagonalWithScale}
\DeclareMathOperator{\mnbr}{minimaKernelBasisReversed}
\DeclareMathOperator{\mnbrp}{minimalKernelBasisWithRankProfile}
\DeclareMathOperator{\colBasis}{colBasis}
\DeclareMathOperator{\rankProfile}{rankProfile}
\def\mab{\qopname\relax n{orderBasis}}
\def\mnbrs{\qopname\relax n{minimalNullspacerBasisRankSensitive}}
\def\mmab{\qopname\relax n{fastOrderBasis}}
\def\umab{\qopname\relax n{unbalancedFastOrderBasis}}
\def\mnb{\qopname\relax n{minimalKernelBasis ~ }}
\newcommand{\bb}{\\}
%\newrefformat{chap}{chapr \ref{#1}}
\newrefformat{lem}{Lemma~\ref{#1}}
\newrefformat{thm}{Theorem~\ref{#1}}
\newrefformat{cha}{Chapter~\ref{#1}}
\newrefformat{chap}{Chapter~\ref{#1}}
\newrefformat{sec}{Section~\ref{#1}}
\newrefformat{rem}{Remark~\ref{#1}}
\newrefformat{fac}{Fact~\ref{#1}}
\newrefformat{sub}{Subsection~\ref{#1}}
\newrefformat{cor}{Corollary~\ref{#1}}
\newrefformat{cond}{Condition~\ref{#1}}
\newrefformat{con}{Conjecture~\ref{#1}}
\newrefformat{def}{Definition~\ref{#1}}
\newrefformat{pro}{Proposition~\ref{#1}}
\newrefformat{alg}{Algorithm~\ref{#1}}
\newrefformat{exm}{Example~\ref{#1}}
\newrefformat{line}{line~\ref{#1}}



%======================================================================
%   L O G I C A L    D O C U M E N T -- the content of your thesis
%======================================================================

\makeatother

\usepackage{babel}
  \providecommand{\corollaryname}{Corollary}
  \providecommand{\definitionname}{Definition}
  \providecommand{\examplename}{Example}
  \providecommand{\lemmaname}{Lemma}
  \providecommand{\remarkname}{Remark}
\providecommand{\theoremname}{Theorem}

\begin{document}
% T I T L E   P A G E
% -------------------
% The title page is counted as page `i' but we need to suppress the
% page number.  We also don't want any headers or footers.
\pagestyle{empty}
\pagenumbering{roman}

% The contents of the title page are specified in the "titlepage"
% environment.
\begin{titlepage}
        \begin{center}
        \vspace*{1.0cm}

        \Huge
   {\bf Fast Order Basis and Kernel Basis Computation and Related Problems}

        \vspace*{1.0cm}

        \normalsize
        by \\

        \vspace*{1.0cm}

        \Large
        Wei Zhou \\

        \vspace*{3.0cm}

        \normalsize
        A thesis \\
        presented to the University of Waterloo \\ 
        in fulfillment of the \\
        thesis requirement for the degree of \\
  Doctor of Philosophy \\
        in \\
        Computer Science \\

        \vspace*{2.0cm}

        Waterloo, Ontario, Canada, 2012 \\

        \vspace*{1.0cm}

        \copyright\ Wei Zhou 2012 \\
        \end{center}
\end{titlepage}


% The rest of the front pages should contain no headers and be
% numbered using roman numerals starting with `ii'
\pagestyle{plain} \setcounter{page}{2}

\noindent % D E C L A R A T I O N   P A G E
% -------------------------------
\ifthenelse{\boolean{ElectronicVersion}}{ % The following is the sample Delaration Page as provided by the GSO
 % December 13th, 2006.  It is designed for an electronic thesis.


\noindent I hereby declare that I am the sole author of this thesis.
This is a true copy of the thesis, including any required final revisions,
as accepted by my examiners.\bigskip{}


\noindent I understand that my thesis may be made electronically available
to the public.}{ % The following text was what was required back when the GSO acceped
 % printed versions and you may want to continue to use it for your
 % printed version.


\noindent I hereby declare that I am the sole author of this thesis.

\noindent \smallskip{}


\noindent I authorize the University of Waterloo to lend this thesis
to other institutions or individuals for the purpose of scholarly
research.

\bigskip{}


\noindent I further authorize the University of Waterloo to reproduce
this thesis by photocopying or by other means, in total or in part,
at the request of other institutions or individuals for the purpose
of scholarly research. } \newpage{}

\begin{center}
\textbf{Abstract}
\par\end{center}

\noindent In this thesis, we present efficient deterministic algorithms
for polynomial matrix computation problems, including the computation
of order basis, minimal kernel basis, matrix inverse, column basis,
unimodular completion, determinant, Hermite normal form, rank and
rank profile for matrices of univariate polynomials over a field.
The algorithm for kernel basis computation also immediately provides
an efficient deterministic algorithm for solving linear systems. The
algorithm for column basis also gives efficient deterministic algorithms
for computing matrix GCDs, column reduced forms, and Popov normal
forms for matrices of any dimension and any rank. 

We reduce all these problems to polynomial matrix multiplications.
The computational costs of our algorithms are then similar to the
costs of multiplying matrices, whose dimensions match the input matrix
dimensions in the original problems, and whose degrees equal the average
column degrees of the original input matrices in most cases. The use
of the average column degrees instead of the commonly used matrix
degrees, or equivalently the maximum column degrees, makes our computational
costs more precise and tighter. In addition, the shifted minimal bases
computed by our algorithms are more general than the standard minimal
bases.\newpage{}

\begin{center}
\textbf{Acknowledgments}
\par\end{center}

Thanks to my supervisor, George Labahn, for supporting me and providing
me with the opportunity and the freedom to explore in this fascinating
world of polynomial matrix computation. Also thanks to my other committee
members, Mark Giesbrecht, Cameron Stewart, Arne Storjohann, and Gilles
Villard for reading this thesis. Thanks especially to Arne, whose
accomplishments and magical abilities to come up with new ideas have
always amazed me and influenced my work. Thanks to Gilles for flying
all the way to Canada from France to attend my defense and for all
the encouraging words. Thanks to Mark for the helpful comments on
this thesis.

Thanks to all my friends in Waterloo, especially my longtime roommate
and good friend Zhirong Li, my resourceful academic brother Reinhold
Burger, and my good friends Wei Li and Jun Chen, for all the fun conversations
and memorable activities. 

I am grateful for having a wonderful family that made everything in
my life possible.



\noindent \newpage{}

\begin{center}
\textbf{Dedication}
\par\end{center}

\noindent To my parents and my sister, for always believing in me

\noindent To my daughter, for bringing new energy and joy

\noindent To my wife, for patience and support\newpage{}

\tableofcontents{}\newpage{}

\listof{algorithm}{List of Algorithms}
\addcontentsline{toc}{chapter}{List of Algorithms}

\newpage{}\pagenumbering{arabic}

\doublespacing


\chapter{Introduction}

In this thesis, we present efficient algorithms for a number of problems
involving matrices of univariate power series or polynomials over
a field. The first problem we consider is the computation of order
bases, which can be viewed as the most fundamental among all the problems
considered in this thesis, since order basis computation is used by
the algorithms for all other problems. The second problem, minimal
kernel basis computation, provides another essential tool used by
the algorithms for the remaining problems, including the computation
of matrix inverse, determinant, column basis, unimodular completion,
and Hermite normal form. The algorithm for kernel basis computation
also immediately allows us to solve linear systems. The algorithm
for column basis also immediately allows us to compute matrix GCDs,
column reduced forms and Popov normal forms for matrices of any dimension. 



Let us first look at order bases and kernel bases in more detail.

Let $\mathbf{F}\in\mathbb{K}\left[\left[x\right]\right]^{m\times n}$
be a matrix of power series over a field $\mathbb{K}$. Given a nonnegative
integer $\sigma$, we say a vector $\mathbf{p}\in\mathbb{K}\left[x\right]^{n\times1}$
of polynomials gives an \emph{order} $\sigma$ approximation of $\mathbf{F}$,
or $\mathbf{p}$ has order $\left(\mathbf{F},\sigma\right)$, if 
\[
\mathbf{F}\cdot\mathbf{p}\equiv\mathbf{0}\mod x^{\sigma},
\]
 that is, the first $\sigma$ terms of $\mathbf{F}\cdot\mathbf{p}$
are zero. Historically such problems date back to their use in Hermite's
proof of the transcendence of $e$ in 1873. In 1893 Pad�, a student
of Hermite, formalized the concepts introduced by Hermite and defined
what is now known as Hermite-Pad� approximants (where $m=1$), Pad�
approximants (where $m=1,n=2$) and simultaneous Pad� approximants
(where $\mathbf{F}$ has a special structure). Such rational approximations
also specified degree constraints on the polynomials $\mathbf{p}$
and had their order conditions related to these degree constraints.
Additional order problems include vector and matrix versions of rational
approximation, partial realizations of matrix sequences and vector
rational reconstruction just to name a few (cf. the references in
\citet{BL1997}). As an example, the factorization of differential
operators algorithm of \citet{vanHoeij} makes use of vector Hermite-Pad�
approximation to reconstruct differential factorizations over rational
functions from factorizations of differential operators over power
series domains.

The set of all such order $\left(\mathbf{F},\sigma\right)$ approximations
forms a module over $\mathbb{K}\left[x\right]$. An {\em order basis}
- or minimal approximant basis or $\sigma$-basis - is a basis of
this module having a type of minimal degree property (called reduced
order basis in \citep{BL1997}). The minimal degree property parameterizes
solutions to an order problem by the degrees of the columns of the
order basis. In the case of rational approximation, order bases can
be viewed as a natural generalization of the Pad� table of a power
series \citep{gravesmorris} since they are able to describe {\em
all} solutions to such problems given particular degree bounds. They
can even be used to show the well known block structure of the Pad�
and related Rational Interpolation tables \citep{BL1997}. Order bases
are used in such diverse applications as the inversion of structured
matrices \citep{La92}, normal forms of matrix polynomials \citep{BLV:1999,BLV:jsc06},
and other important problems in matrix polynomial arithmetic including
matrix inversion, determinant and kernel computation \citep{Giorgi2003,storjohann-villard:2005}.
\begin{comment}
In our case we also allow the minimal degree property to include a
shift $\vec{s}$. Such a shift is important, for example, for matrix
normal form problems \citep{BLV:1999,BLV:jsc06}.
\end{comment}


Kernel bases are closely related to order bases.

For a matrix of polynomials $\mathbf{F}\in\mathbb{K}\left[x\right]^{m\times n}$
with rank $r$. The set

\[
\left\{ \mathbf{p}\in\mathbb{K}\left[x\right]^{n}~|~\mathbf{F}\mathbf{p}=0\right\} ,
\]
 is the (right) kernel of $\mathbf{F}$, which is also a $\mathbb{K}[x]$-module.
It can be generated by a basis -- a kernel basis of $\mathbf{F}$,
that can be represented as a matrix in $\mathbb{K}\left[x\right]^{n\times\left(n-r\right)}$,
with the columns being the basis elements.

Kernel bases of polynomial matrices appear in a large number of applications,
being first used as an algebraic formalism in the area of control
theory \citep{Kucera:1979}. For example, in linear system theory
if a system is represented by a transfer function given in terms of
a left coprime matrix fraction decomposition $\mathbf{T}=\mathbf{D}_{\ell}^{-1}\mathbf{N}_{\ell}$,
with $\mathbf{D}_{\ell}$ and $\mathbf{N}_{\ell}$ polynomial matrices,
then one often wants to find a right coprime matrix fraction representation
$\mathbf{T}=\mathbf{N}_{r}\mathbf{D}_{r}^{-1}$ with $\mathbf{D}_{r}$
and $\mathbf{N}_{r}$ polynomial matrices of appropriate dimensions
\citep{kailath:1980}. This is equivalent to the kernel basis computation
\begin{equation}
[\mathbf{D}_{\ell}~~-\mathbf{N}_{\ell}]\left[\begin{array}{c}
\mathbf{N}_{r}\\
\mathbf{D}_{r}
\end{array}\right]=0.\label{rightfactor}
\end{equation}
 Solving and determining fundamental properties of the basic matrix
equation $\mathbf{A}\mathbf{Z}=\mathbf{B}$ where $\mathbf{A}$ and
$\mathbf{B}$ have polynomial elements can be determined by finding
a complete description (that is, a basis) of the kernel of $[\mathbf{A},-\mathbf{B}]$.
Other examples of the use of kernels and their bases include fault
diagnostics \citep{frisk:phd} and column reduction of matrix polynomials
\citep{BVP:1988,neven:1993}.

% a bit more specific description of our problem.


In most applications one is interested in finding a {\em minimal
kernel basis} of $\mathbf{F}$ in $\mathbb{K}\left[x\right]^{n}$
\citep{forney:1975}. A kernel basis $\mathbf{N}$ of $\mathbf{F}$
is said to be minimal if it has the minimal possible column degrees
among all right kernel bases. This is also often referred to as a
{\em minimal polynomial basis}. Examples where minimality are needed
include the right coprime matrix factorization problem and the problem
of column reducing a polynomial matrix. As an example, finding a basis
for the kernel corresponding to the right matrix fraction problem
(\ref{rightfactor}) finds a matrix fraction while a minimal kernel
basis finds such a fraction in reduced form having a minimal column
degree denominator (needed for example in minimal partial realization
problems). 


\section{Shifted Degrees and Their Sums}

The standard way to measure the size of a matrix is to use its dimension
and its degree. A major complication in many polynomial matrix computation
problems is that the degrees of the intermediate results or the output
can be much larger than the input. This seems to prevent these problems
to be computed efficiently, since the size of the intermediate results
and the size of the output provides lower bounds on the computational
cost of any algorithm.  But  it is possible that the degree just
may not be the best choice to be used in these computations. In this
thesis, instead of the standard degrees, we use the more general shifted
degrees to guide the computations, and use the sum of the shifted
degrees to measure the size of polynomial matrices. We will see that
the shifted degree is in fact a more natural choice, as it guides
the computation so that the sizes of the output and the intermediate
results are indeed bounded by the size as the input for all these
problems.  Closely examination of the shifted degrees reveals new
structures of the problems in this thesis, leading to better understanding
of the problems, and allowing the development of simple and efficient
algorithms. 

\begin{comment}
Although the degree is simple and seems natural, 

In such cases, one may wonder why the extra space is needed if we
are just transforming the original input and not creating extra information
to store. One possibilities is that the output contains some redundant
information that makes it easier to use in someway. Another possibility
is that the measurement is flawed, which turns out to be the case
for all the problems we consider in this thesis. By using a better
criterion to measure the sizes, we will see that the sizes of the
output and the intermediate results are indeed bounded by the size
as the input for all these problems. The importance of the criterion
for measuring the sizes cannot be understated, as it provides clean,
elegant structures to the previous messy problems and guides us efficiently
to the output. 

In this thesis, we use the sum of the shifted degrees to measure the
size of polynomial matrices. We introduce the shifted column degrees
in this section. 
\end{comment}


For a column vector $\mathbf{p}=\left[p_{1},\dots,p_{n}\right]^{T}$
of univariate polynomials over a field $\mathbb{K}$, its column degree
is just the maximum of the degrees of the entries of $\mathbf{p}$,
that is, 
\[
\cdeg\mathbf{p}=\max_{1\le i\le n}\deg p_{i}.
\]
The \emph{shifted column degree} generalizes this standard column
degree by taking the maximum after shifting the degrees by a given
integer vector that is known as a \emph{shift}. More specifically,
the shifted column degree of $\mathbf{p}$ with respect to a shift
$\vec{s}=\left[s_{1},\dots,s_{n}\right]\in\mathbb{Z}^{n}$, or the
\emph{$\vec{s}$-column degree} of $\mathbf{p}$ is

\[
\cdeg_{\vec{s}}\mathbf{p}=\max_{1\le i\le n}[\deg p^{\left(i\right)}+s_{i}]=\deg(x^{\vec{s}}\cdot\mathbf{p}),
\]
 where 
\[
x^{\vec{s}}=\diag\left(\left[x^{s_{1}},x^{s_{2}},\dots,x^{s_{n}}\right]\right)=\begin{bmatrix}x^{s_{1}}\\
 & x^{s_{2}}\\
 &  & \ddots\\
 &  &  & x^{s_{1}}
\end{bmatrix}.
\]
For a matrix $\mathbf{P}$, we use $\cdeg\mathbf{P}$ and $\cdeg_{\vec{s}}\mathbf{P}$
to denote respectively the list of its column degrees and the list
of its shifted $\vec{s}$-column degrees. When $\vec{s}=\left[0,\dots,0\right]$,
the shifted column degree specializes to the standard column degree.
The shifted row degree of a row vector \textbf{$\mathbf{q}$ }is defined
in the same way.
\[
\rdeg_{\vec{s}}\mathbf{q}=\max_{1\le i\le n}[\deg p^{\left(i\right)}+s_{i}]=\deg(\mathbf{q}\cdot x^{\vec{s}}).
\]


The shifted degrees have been used previously in polynomial matrix
computations and to generalize matrix normal forms \citep{BLV:jsc06}.
The shifted column degree is equivalent to the notion of \emph{defect}
commonly used in the literature. Our definition of $\vec{s}$-column
degree is a special case of the $\mathbf{H}$-degree from \citep{BL1997},
where in this case $\mathbf{H}=x^{\vec{s}}$.%
\begin{comment}
The shifted column degree of a column polynomial vector $\mathbf{p}$
with shift $\vec{s}=\left[s_{1},\dots,s_{n}\right]\in\mathbb{Z}^{n}$
is given by 
\[
\deg_{\vec{s}}\mathbf{p}=\max_{1\le i\le n}[\deg p^{\left(i\right)}+s_{i}]=\deg(x^{\vec{s}}\cdot\mathbf{p}).
\]
 We call this the \emph{$\vec{s}$-column degree}, or simply the \emph{$\vec{s}$-degree}
of $\mathbf{p}$. A shifted column degree defined this way is equivalent
to the notion of \emph{defect} commonly used in the literature. Our
definition of $\vec{s}$-degree is a special case of the $\mathbf{H}$-degree
from \citep{BL1997}, where in this case $\mathbf{H}=x^{\vec{s}}$.
As in the uniform shift case, we say a matrix is \emph{$\vec{s}$-column
reduced} or \emph{$\vec{s}$-reduced} if its $\vec{s}$-degrees cannot
be decreased by unimodular column operations. More precisely, if $\mathbf{P}$
is a $\vec{s}$-column reduced and $[d_{1},\dots,d_{n}]$ are the
$\vec{s}$-degrees of columns of $\mathbf{P}$ sorted in nondecreasing
order, then $[d_{1},\dots,d_{n}]$ is lexicographically minimal among
all matrices right equivalent to $\mathbf{P}$. Note that a matrix
$\mathbf{P}$ is $\vec{s}$-column reduced if and only if $x^{\vec{s}}\cdot\mathbf{P}$
is column reduced \citep{BLV:1999,BLV:jsc06}.
\end{comment}
{} 


\section{Order Basis Computation}

The first problem considered in this thesis is the efficient computation
of order basis. Algorithms for fast computation of order basis include
that of \citet{BeLa94} which converts the matrix problem into a vector
problem of higher order (which they called the Power Hermite-Pad�
problem). Their divide and conquer algorithm has complexity of $O^{\sim}(n^{2}m\sigma+nm^{2}\sigma)$
field operations. As usual, the soft-$O$ notation $O^{\sim}$ is
simply Big-$O$ with polylogarithmic factors $(\log(nm\sigma))^{O(1)}$
omitted. By working more directly on the input $m\times n$ input
matrix, \citet{Giorgi2003} give a divide and conquer method with
cost $O^{\sim}\left(n^{\omega}\sigma\right)$ arithmetic operations.
Their method is very efficient if $m$ is close to the size of $n$
but can be improved if $m$ is small.

In a novel construction, \citet{Storjohann:2006} effectively reverses
the approach of Beckermann and Labahn. Namely, rather than convert
a high dimension matrix order problem into a lower dimension vector
problem of higher order, Storjohann converts a low dimension problem
to a high dimension problem with lower order. For example, computing
an order basis for a $1\times n$ vector input $\mathbf{f}$ and order
$\sigma$ can be converted to a problem of order basis computation
with an $O\left(n\right)\times O\left(n\right)$ input matrix and
an order $O\left(\left\lceil \sigma/n\right\rceil \right)$. Combining
this conversion with the method of Giorgi et al. can then be used
effectively for problems with small row dimensions to achieve a cost
of $O^{\sim}\left(n^{\omega}\left\lceil m\sigma/n\right\rceil \right)$.

However, while order bases of the original problem can have degree
up to $\sigma$, the nature of Storjohann's conversion limits the
degree of an order basis of the converted problem to $O\left(\left\lceil m\sigma/n\right\rceil \right)$
in order to be computationally efficient. In other words, this approach
does not in general compute a complete order basis. Rather, in order
to achieve efficiency, it only computes a partial order basis containing
basis elements with degrees within $O\left(\left\lceil m\sigma/n\right\rceil \right)$,
referred to by Storjohann as a {\em minbasis}. Fast methods for
computing a minbasis are particularly useful for certain problems,
for example, in the case of inversion of structured block matrices
where one needs only precisely the minbasis \citep{La92}. However,
in other applications, such as those arising in polynomial matrix
arithmetic, one needs a complete basis which specifies all solutions
of a given order, not just those within a particular degree bound
(cf. \citet{BL1997}).

In \chapref{OrderBasis} we present algorithms which compute an entire
order basis with a cost of $O^{\sim}(n^{\omega-1}m\sigma)$ field
operations when $n\in O\left(m\sigma\right)$. The algorithms differ
depending on the nature of the degree shift required for the reduced
order basis. In the first case we use a transformation that can be
considered as an extension of Storjohann's transformation. This new
transformation provides a way to extend the results from one transformed
problem to another transformed problem of a higher degree. This enables
us to use an idea from the null space basis algorithm found in \citep{storjohann-villard:2005}
in order to achieve efficient computation. At each iteration, basis
elements within a specified degree bound are computed via a Storjohann
transformed problem. Then the partial result is used to simplify the
next Storjohann transformed problem of a higher degree, allowing basis
elements within a higher degree bound to be computed efficiently.
This is repeated until all basis elements are computed.

In order to compute an order basis efficiently, the first algorithm
requires that the degree shifts are balanced. In the case where the
shift is not balanced, the row degrees of the basis can also become
unbalanced in addition to the unbalanced column degrees. We give a
second algorithm that balances the high degree rows and uses $O^{\sim}(n^{\omega}\lceil m\sigma/n\rceil)$
field operations when the shift $\vec{s}$ is unbalanced but satisfies
the condition $\sum_{i=1}^{n}(\max(\vec{s})-\vec{s}_{i})\le m\sigma$.
This condition essentially allows us to locate the high degree unbalanced
rows that need to be balanced. %
\begin{comment}
In fact, every problem with any unbalanced shift can be reformulated
to a problem with a shift satisfying this condition if the degrees
of the resulting order basis is known. 
\end{comment}
{} The algorithm converts a problem of unbalanced shift to one with
balanced shift, based on a second idea from \citep{Storjohann:2006}.
Then the first algorithm is used to efficiently compute the elements
of an order basis whose shifted degrees exceed a specified parameter.
The problem is then reduced to one where we remove the computed elements.
This results in a new problem with smaller dimension and higher degree.
The same process is repeated again on this new problem in order to
compute the elements with the next highest shifted degrees.

At the end of \chapref{OrderBasis}, we discuss how the assumption
of $n\in O\left(m\sigma\right)$ can be removed while an order basis
can still be computed with a cost of $O^{\sim}(n^{\omega-1}m\sigma)$
field operations when the shifts are balanced.

Some results on order basis computation have appeared in \citep{za2009,ZL2012}.

\begin{comment}
The remaining paper is structured as follows. Basic definitions and
properties of order bases are given in the next section. \secref{transform}
provides an extension to Storjohann's transformation to allow higher
degree basis elements to be computed. Based on this new transformation,
\secref{Order-Basis-Computation} establishes a link between two Storjohann
transformed problems of different degrees, from which an recursive
method and then an iterative algorithm are derived. The time complexity
is analyzed in the next section. After this, \chapref{Unbalanced-Shift}
describes an algorithm which handles problems with a type of unbalanced
shift. This is followed by a conclusion along with a description for
topics for future research. 
\end{comment}



\section{Kernel Basis Computation}

We are interested in fast computation of minimal kernel bases and
shifted minimal kernel bases in exact environments. Historically computation
of a minimal kernel basis has made use of either matrix pencil or
resultant methods (often called a linearized approaches) or use of
elimination methods for matrix polynomials. Matrix pencil methods
convert a kernel basis computation problem to one of larger matrix
size but having polynomial degree one. In this case a minimal kernel
basis is determined from the computation of the Kronecker canonical
form, with efficient algorithms given by \citet{beelen:1988,varga:1994,Oara:1997}.
The cost of these algorithms is $O(m^{2}nd^{3})$. Resultant methods
convert the kernel basis computation of the matrix polynomial $\mathbf{F}$
into a block Toeplitz kernel problem with much higher dimension with
the resulting complexity again being high. In \citep{storjohann-villard:2005}
the authors give a randomized Las Vegas algorithm for computing a
set of $n-r$ linearly independent elements in the kernel of $\mathbf{F}$
with a cost of $O^{\sim}\left(nmr^{\omega-1}\right)$ where $O^{\sim}$
is the same as Big-$O$ but without log factors and where $\omega$
is the power of fast matrix multiplication. These linearly independent
elements do not in general form a basis for the kernel, as they generally
do not generate all the elements in the kernel. A set of any such
$n-r$ linearly independent elements only form a basis for the $\mathbb{F}\left(x\right)$-vector
space $\left\{ \mathbf{p}\in\mathbb{K}\left(x\right)^{n}~|~\mathbf{F}\mathbf{p}=0\right\} $
when the ring $\mathbb{K}\left[x\right]$ is extended to the field
$\mathbb{K}\left(x\right)$.



In \chapref{NullspaceBasis} we present a deterministic algorithm
for computing a minimal kernel basis with a cost of $O^{\sim}\left(n^{\omega-1}md\right)$
field operations in $\mathbb{K}$. The same algorithm can also compute
a $\vec{s}$-minimal kernel basis of $\mathbf{F}$ with a cost of
$O^{\sim}(n^{\omega}\rho/m)$ if the entries of $\vec{s}$ bound the
corresponding column degrees of $\mathbf{F}$, where $\rho$ is the
sum of the $m$ largest entries of $\vec{s}$. 

% a bit more information about our method


A key component of the algorithm is the computation of order basis%
\begin{comment}
 (also known as minimal approximant basis or $\sigma$-basis) \citep{BeLa94}.
Order bases are efficiently computed using the algorithms from \citet{Giorgi2003}
and \citet{za2009}
\end{comment}
. We use order basis computation to compute a partial kernel basis,
which also reduces the column dimension of the problem. The problem
can then be separated to two subproblems of smaller row dimensions,
which can then be handled in the same way as the original problem.

Some results on kernel basis computation have appeared in \citep{za2012}.

\begin{comment}
% the rest


The remainder of this paper is structured as follows. Basic definitions
and properties of order bases and kernel bases are given in the next
section. The details of our kernel basis computation and a formal
statement of the algorithm can be found in \secref{Nullspace-Basis-Computation}.
A complexity analysis of the algorithm is provided in the following
section. The paper ends with a conclusion and topics for future research. 
\end{comment}



\section{Overview}

The remainder of this thesis is structured as follows. Basic definitions
and properties are given in the next chapter. The details of our order
basis computation can be found in \chapref{OrderBasis}. Kernel basis
computation is described in \chapref{NullspaceBasis}.  \chapref{Matrix-inverse}
discusses the algorithm for computing matrix inverse. \chapref{Matrix-GCD}
discusses the computation of column bases. Unimodular completion is
then discussed in \chapref{Unimodular-Completion}. Then we look at
determinant computation in \chapref{determinant}, Hermite normal
form computation in \chapref{hermite}, and rank profile and rank-sensitive
computations in \chapref{rank}. %
\begin{comment}
The paper ends with a conclusion and topics for future research. 
\end{comment}






\chapter{Preliminaries}

\label{chap:Background}

In this chapter, we provide some of the background needed in order
to understand the basic concepts needed for order basis computation
and nullspace basis computation. %
\begin{comment}
, which provide lower bounds for the computational cost 
\end{comment}
{} %
\begin{comment}
The challenges of balancing input and handling unbalanced output are
discussed along with the techniques which we plan to use to overcome
the difficulties. We review the construction by \citet{Storjohann:2006}
which transforms the inputs to those having dimensions and degree
balance better suited for fast computation and discuss an idea from
\citet{storjohann-villard:2005} for handling the case where the output
degree is unbalanced.
\end{comment}



\section{Notations}

Since we are interested in computing bases with minimal degrees, it
is useful to have convenient notations for comparing two lists of
degrees.   In addition, our matrices often represent sets of column
vectors, so the arrangement of these columns are not important. To
compare two lists of column degrees from two matrices, we first sort
each list in increasing order, and then do the comparison.
\begin{description}
\item [{Comparing~Unordered~Lists}] For two lists $\vec{a}\in\mathbb{Z}^{n}$
and $\vec{b}\in\mathbb{Z}^{n}$, let $\bar{a}=\left[\bar{a}_{1},\dots,\bar{a}_{n}\right]\in\mathbb{Z}^{n}$
and $\bar{b}=\left[\bar{b}_{1},\dots,\bar{b}_{n}\right]\in\mathbb{Z}^{n}$
be the lists consists of the entries of $\vec{a}$ and $\vec{b}$
but sorted in increasing order. 
\[
\begin{cases}
\vec{a}\ge\vec{b} & \mbox{if }\bar{a}_{i}\ge\bar{b}_{i}\mbox{ for all }i\in\left[1,\dots n\right]\\
\vec{a}\le\vec{b} & \mbox{if }\bar{a}_{i}\le\bar{b}_{i}\mbox{ for all }i\in\left[1,\dots n\right]\\
\vec{a}>\vec{b} & \mbox{if }\vec{a}\ge\vec{b}\mbox{ and }\bar{a}_{j}>\bar{b}_{j}\mbox{ for at least one }j\in\left[1,\dots n\right]\\
\vec{a}<\vec{b} & \mbox{if }\vec{a}\le\vec{b}\mbox{ and }\bar{a}_{j}<\bar{b}_{j}\mbox{ for at least one }j\in\left[1,\dots n\right].
\end{cases}
\]

\item [{Summation~Notation}] For a list $\vec{a}=\left[a_{1},\dots,a_{n}\right]\in\mathbb{Z}^{n}$,
we write $\sum\vec{a}$ without index to denote the summation of all
entries in $\vec{a}$. 
\item [{}]~
\item [{Uniformly~Shift~a~List}] For a list $\vec{a}=\left[a_{1},\dots,a_{n}\right]\in\mathbb{Z}^{n}$
and $c\in\mathbb{Z}$, we write $\vec{a}+c$ to denote $\vec{a}+\left[c,\dots,c\right]=\left[a_{1}+c,\dots,a_{n}+c\right]$,
and similarly for $-$.
\item [{Compare~a~List~with~a~Integer}] For a list $\vec{a}=\left[a_{1},\dots,a_{n}\right]\in\mathbb{Z}^{n}$
and $c\in\mathbb{Z}$, we write $\vec{a}<c$ to denote $\vec{a}<\left[c,\dots,c\right]$,
and similarly for $>,\le,\ge,=$.
\end{description}

\section{Model of Computation}

The computational cost in this paper is analyzed by bounding the number
of arithmetic operations (additions, subtractions, multiplications,
and divisions) in the coefficient field $\mathbb{K}$ on an algebraic
random access machine. We use $\MM(n,d)$ to denote the cost of multiplying
two polynomial matrices with dimension $n$ and degree $d$, and $\M(n)$
to denote the cost of multiplying two polynomials with degree $d$.
We define a cost function $\bar{\M}(d)=d\log d\log\log d$, then $\bar{\M}(ab)\in O\left(\bar{\M}(a)\bar{\M}(b)\right)$
and $\bar{\M}(n)\in O(n^{\omega-1})$. We take $\MM(n,d)\in O\left(n^{\omega}\M(d)\right)\subset O(n^{\omega}\bar{\M}(d))$,
where the multiplication exponent $\omega$ is assumed to satisfy
$2<\omega\le3$. We refer to the book by \citet{vonzurgathen} for
more details and reference about the cost of polynomial multiplication
and matrix multiplication.


\section{Shifted Degrees}

In this section, we look at some properties of shifted degrees, which
may help in understanding their usefulness in efficient computations
of polynomial matrix problems.


\begin{lem}
\label{lem:columnDegreesRowDegreesSymmetry}A matrix $\mathbf{A}\in\mathbb{K}\left[x\right]^{m\times n}$
has $\vec{u}$-column degrees bounded by $\vec{v}$ if and only if
its $-\vec{v}$-row degrees are bounded by $-\vec{u}$. %
\begin{comment}
In addition, for a matrix $\mathbf{A}$ with $\vec{u}$-column degrees
$\vec{v}$, it has a full-rank leading $\vec{u}$-column coefficient
matrix if and only if it has a full-rank leading $-\vec{v}$-row coefficient
matrix.
\end{comment}
\end{lem}
\begin{proof}
The lemma follows from the fact $x^{\vec{u}}\mathbf{A}x^{-\vec{v}}$
has degree bounded by 0 (Note that the negative degrees are define
here by setting$\deg x^{-d}=-d$ for $d\in\mathbb{Z}_{>0}$. If one
wishes to avoid this negative degrees, one can simply shift the degrees
up by multiplying the matrix by $x^{a}$ for some large $a$). Note
the symmetry between the shifted row degrees and the shifted column
degrees.\end{proof}
\begin{lem}
\label{lem:productDegreeBound}If the $\vec{u}$-column degrees of
$\mathbf{A}\in\mathbb{K}\left[x\right]^{m\times n}$ are bounded by
the corresponding entries of an integer list $\vec{v}\in\mathbb{Z}^{n}$,
(or equivalently, the $-\vec{v}$-row degrees of $\mathbf{A}$ are
bounded by $-\vec{u}$) and the $\vec{v}$-column degrees of $\mathbf{B}\in\mathbb{K}\left[x\right]^{n\times k}$
are bounded by $\vec{w}\in\mathbb{Z}^{k}$, then the $\vec{u}$-column
degrees of $\mathbf{A}\mathbf{B}$ are bounded by $\vec{w}$. \end{lem}
\begin{proof}
Note that $x^{\vec{u}}\mathbf{A}x^{-\vec{v}}$ and $x^{\vec{v}}\mathbf{B}^{-\vec{w}}$
have degrees bounded by 0. Therefore 
\[
x^{\vec{u}}\mathbf{A}x^{-\vec{v}}x^{\vec{v}}\mathbf{B}^{-\vec{w}}=x^{\vec{u}}\mathbf{A}\mathbf{B}^{-\vec{w}}
\]
 also has degree bounded by 0, or equivalently, $\cdeg_{\vec{u}}\mathbf{A}\mathbf{B}\le\vec{w}$. \end{proof}
\begin{cor}
\label{lem:boundOnDegreesOfFA}Let $\vec{v}$ be a shift whose entries
bound the corresponding column degrees of $\mathbf{A}$. Then for
any polynomial matrix $\mathbf{B}\in\mathbb{K}\left[x\right]^{n\times k}$,
the column degrees of $\mathbf{A}\mathbf{B}$ are bounded %component-wise 
by the corresponding $\vec{s}$-column degrees of $\mathbf{B}$.\end{cor}
\begin{proof}
Just set the shift $\vec{u}$ to 0 in \prettyref{lem:productDegreeBound}
\end{proof}
\begin{comment}
\begin{lem}
\label{lem:productDegreeBoundWithRowDegrees}If the $\vec{u}$-row
degrees of $\mathbf{A}\in\mathbb{K}\left[x\right]^{m\times n}$ are
bounded by the corresponding entries of an integer list $\vec{v}\in\mathbb{Z}^{n}$,
and the $-\vec{u}$-column degrees of $\mathbf{B}\in\mathbb{K}\left[x\right]^{n\times k}$
are bounded by $\vec{w}\in\mathbb{Z}^{k}$, then the $-\vec{v}$-column
degrees of $\mathbf{A}\mathbf{B}$ are bounded by $\vec{w}$. \end{lem}
\begin{proof}
Note that $x^{-\vec{v}}\mathbf{A}x^{\vec{u}}$ and $x^{-\vec{u}}\mathbf{B}^{-\vec{w}}$
have degrees bounded by 0. Therefore, $x^{-\vec{v}}\mathbf{A}x^{\vec{u}}x^{-\vec{u}}\mathbf{B}^{-\vec{w}}=x^{-\vec{v}}\mathbf{A}\mathbf{B}^{-\vec{w}}$
also has degree bounded by 0, or equivalently, $\cdeg_{-\vec{v}}\mathbf{A}\mathbf{B}\le\vec{w}$.
Alternatively, this also follows from the fact $\cdeg_{-\vec{v}}\mathbf{A}\le-\vec{u}$
from \prettyref{lem:productDegreeBound}.\end{proof}
\end{comment}



\section{Unimodular Matrices and Unimodular Transformations}

A square matrix in $\mathbb{K}\left[x\right]^{n\times n}$ is said
to be \emph{unimodular} if its determinant is in $\mathbb{K}$. A
matrix in $\mathbb{K}\left[x\right]^{n\times n}$ is unimodular if
and only if it has an inverse in $\mathbb{K}\left[x\right]^{n\times n}$.
Therefore, a unimodular transformation can be undone by the multiplication
with the inverse transformation matrix. This allows us to talk about
unimodular equivalence of the polynomial matrices. Two matrices $\mathbf{A},\mathbf{B}\in\mathbb{K}\left[x\right]^{m\times n}$
are said to be right unimodularly equivalent if $\mathbf{A}=\mathbf{B}\mathbf{U}$
for some unimodular matrix $\mathbf{U}$. This also means that the
columns of $\mathbf{A}$ and $\mathbf{B}$ generates the same set
of vectors. That is, if a vector $\mathbf{q}=\mathbf{A}\mathbf{p}$
for some $\mathbf{p}\in\mathbb{K}\left[x\right]^{n\times1}$, then
$\mathbf{q}=\mathbf{B}\mathbf{U}\mathbf{p}$.


\section{Column Basis}

The column module of a matrix $\mathbf{A}\in\mathbb{K}\left[x\right]^{m\times n}$
is the $\mathbb{K}\left[x\right]$-module generated by the columns
of $\mathbf{A}$, that is, this module contains all the column vectors
that are $\mathbb{K}\left[x\right]$-linear combinations of the columns
of $\mathbf{A}$. A column basis of $\mathbf{A}$ is just a basis
for this module. Any column basis of $\mathbf{A}$ can be represented
as a matrix $\mathbf{T}$, whose columns are the basis elements. The
matrix $\mathbf{T}$ has full-rank since basis elements must be linearly
independent. In addition, any two bases for the same module are unimodularly
equivalent.
\begin{lem}
\label{lem:basisEquivalence}If the matrices $\mathbf{T}_{1}$ and
$\mathbf{T}_{2}$ are both column bases of $\mathbf{A}$, then $\mathbf{T}_{1}$
and $\mathbf{T}_{2}$ are right unimodularly equivalent.\end{lem}
\begin{proof}
Any column of $\mathbf{T}_{1}$ or $\mathbf{T}_{2}$ is generated
by $\mathbf{T}_{1}$ and also by $\mathbf{T}_{2}$. In other words,
$\mathbf{T}_{1}=\mathbf{T}_{2}\mathbf{U}$ and $\mathbf{T}_{2}=\mathbf{T}_{1}\mathbf{V}$
for polynomial matrices $\mathbf{U}$ and $\mathbf{V}$. Hence $\mathbf{T}_{1}=\mathbf{T}_{1}\mathbf{V}\mathbf{U}$
and $\mathbf{T}_{2}=\mathbf{T}_{2}\mathbf{U}\mathbf{V}$, implying
$\mathbf{U}\mathbf{V}=\mathbf{V}\mathbf{U}=I$, which requires both
$\mathbf{U}$ and $\mathbf{V}$ to be unimodular.
\end{proof}

\section{\label{sec:minimality}Minimality and Column Reducedness}

For many polynomial matrix computation problems, we would like the
output matrix to be not only easy to describe, but also convenient
to use. This usually means the column degrees or the more general
shifted column degrees are small comparing to other matrices that
are right unimodularly equivalent. For example a matrix $\mathbf{A}=\left[x,x^{2},x^{2}\right]$
with column degrees $\left[1,2,2\right]$ can be unimodularly reduced
to a matrix $\mathbf{B}=\left[x,x,x^{2}\right]$ with column degrees
$\left[1,1,2\right]$, which is more desirable with lower degrees. 

To unimodularly reduce a matrix to one with lower column degrees,
we can look at its\emph{ leading column coefficient matrix}, which
is defined as follows. 
\begin{defn}
Given a matrix $\mathbf{A}\in\mathbb{K}\left[x\right]^{m\times n}$,
the leading column coefficient matrix $A$ of $\mathbf{A}$ is
\begin{align*}
A & =\lcoeff\left(\mathbf{A}\right)\\
 & =\left[\lcoeff\left(\mathbf{a}_{1}\right),\dots,\lcoeff\left(\mathbf{a}_{k}\right)\right]\\
 & =\left[\coeff\left(\mathbf{a}_{1},\deg\left(\mathbf{a}_{1}\right)\right),\dots,\coeff\left(\mathbf{a}_{k},\deg\left(\mathbf{a}_{k}\right)\right)\right].
\end{align*}

\end{defn}
Then, the matrix $\mathbf{A}$ can be unimodularly reduced to another
matrix with lower column degrees if $\lcoeff\left(\mathbf{A}\right)$
is not full-rank. %
\begin{comment}
In the following, we assume with out loss of generality that matrices
have columns arranged in increasing column degrees. 
\end{comment}

\begin{lem}
\label{lem:columnOperationToReduceDegree}Given a matrix $\mathbf{A}\in\mathbb{K}\left[x\right]^{m\times n}$
with no zero columns. If $\lcoeff\left(\mathbf{A}\right)$ is not
full-rank, then there is a unimodular matrix $\mathbf{U}$ such that
$\cdeg\left(\mathbf{A}\mathbf{U}\right)<\cdeg\mathbf{A}$.\end{lem}
\begin{proof}
We may assume the columns of $\mathbf{A}$ are arranged in increasing
column degrees. Let the column degrees of $\mathbf{A}$ be $\left[d_{1},\dots,d_{n}\right]$.
Let $A=\lcoeff\left(\mathbf{A}\right)$. Suppose the $i$th column
$A_{i}$ of $A$ is a linear combination of the first $i-1$ columns.
That is, $A_{i}=A'a$, where $A'$ is the submatrix of $A$ consists
of the first $i-1$ columns of $A$, and $a=\left[a_{1},\dots,a_{i-1}\right]^{T}\in\mathbb{K}^{(i-1)\times1}$.
Let 
\[
\mathbf{U}=\begin{bmatrix}1 &  &  &  & -a_{1}x^{d_{i}-d_{1}}\\
 & 1 &  &  & -a_{2}x^{d_{i}-d_{2}}\\
 &  & \ddots &  & \vdots\\
 &  &  & 1 & -a_{i-1}x^{d_{i}-d_{i-1}}\\
 &  &  &  & 1\\
 &  &  &  &  & \ddots\\
 &  &  &  &  &  & 1
\end{bmatrix}.
\]
Then $\mathbf{B}=\mathbf{A}\mathbf{U}$ has column degrees $\left[d_{1},\dots,d_{i-1},\bar{d}{}_{i},d_{i+1},\dots,d_{n}\right],$
with $\bar{d}_{i}<d_{i}$.
\end{proof}
The leading column coefficient matrix can also help to determine the
degree of the matrix.
\begin{lem}
\label{lem:fullRankLeadingCoefficientAndDegreeOfDetermant}The degree
of the determinant of a matrix $\mathbf{A}\in\mathbb{K}\left[x\right]^{n\times n}$
is bounded by $\sum\cdeg\mathbf{A}$. If $\lcoeff\left(\mathbf{A}\right)$
is nonsingular, then $\deg\det\mathbf{A}=\sum\cdeg\mathbf{A}$. More
generally, for any shift $\vec{s}\in\mathbb{Z}^{n}$, $\deg\det\mathbf{A}\le\sum\cdeg_{\vec{s}}\mathbf{A}-\sum\vec{s}$.
If $\lcoeff\left(x^{\vec{s}}\mathbf{A}\right)$ is nonsingular, then
$\deg\det\mathbf{A}=\sum\cdeg_{\vec{s}}\mathbf{A}-\sum\vec{s}$.\end{lem}
\begin{proof}
The determinant is the sum of products, with each product involving
exactly one entry from each column. So the largest possible degree
of each product is $\sum\cdeg\mathbf{A}$. For the second statement,
note that the coefficient of $\det\mathbf{A}$ corresponds to the
largest possible degree $\sum\cdeg\mathbf{A}$ is $\det\lcoeff\left(\mathbf{A}\right)$.
The more general results with shift can be shown by considering the
the determinant of $x^{\vec{s}}\mathbf{A}$.
\end{proof}
It is still not always possible to order two lists of integer degrees,
as in the case of matrices $\left[x,x^{2},x^{2}\right]$ and $\left[x,x,x^{3}\right]$
with column degrees $\left[1,2,2\right]$ and $\left[1,1,3\right]$.
Fortunately, although the lists of column degrees of the set of unimodular
equivalent matrices are not well-ordered, there always exists some
matrices with the \emph{minimal} column degrees, as shown below by
\prettyref{lem:matrixGCDlowerDegrees} and \prettyref{cor:minimalUnimodularEquivalentMatrix}.
\begin{lem}
\label{lem:matrixGCDlowerDegrees}Given any two right unimodularly
equivalent matrices $\mathbf{A}$ and $\mathbf{B}$, the matrix $\left[\mathbf{A},\mathbf{B}\right]$
can be unimodularly reduced to $\left[0,\mathbf{C}\right]$ with a
matrix $\mathbf{C}$ that is unimodularly equivalent to both $\mathbf{A}$
and $\mathbf{B}$, and satisfies $\cdeg\mathbf{C}\le\cdeg\mathbf{A}$
and $\cdeg\mathbf{C}\le\cdeg\mathbf{B}$. More generally, with a shift
$\vec{s},$ the matrix $\left[\mathbf{A},\mathbf{B}\right]$ can be
unimodularly reduced to $\left[0,\mathbf{D}\right]$ with a matrix
$\mathbf{D}$ that is unimodularly equivalent to both $\mathbf{A}$
and $\mathbf{B}$, and satisfies $\cdeg_{\vec{s}}\mathbf{D}\le\cdeg_{\vec{s}}\mathbf{A}$
and $\cdeg_{\vec{s}}\mathbf{D}\le\cdeg_{\vec{s}}\mathbf{B}$. \end{lem}
\begin{proof}
If $r$ is the rank of $\mathbf{A}$ and $\mathbf{B}$, the column
degrees of $\mathbf{C}$ are bounded by the column degrees of the
$r$ linear independent columns of $\left[\mathbf{A},\mathbf{B}\right]$
with the smallest column degrees, as higher degree columns can be
reduced using \prettyref{lem:columnOperationToReduceDegree}. For
the more general result with shift, we can again multiply $x^{\vec{s}}$
to the matrices.\end{proof}
\begin{cor}
\label{cor:minimalUnimodularEquivalentMatrix}Given a matrix $\mathbf{F}\in\mathbb{K}\left[x\right]^{m\times n}$
and a shift $\vec{s}\in\mathbb{Z}^{n}$. There exists a matrix $\mathbf{G}$
that is right unimodularly equivalent to $\mathbf{F}$ and $\cdeg\mathbf{G}\le\cdeg(\mathbf{F}\mathbf{U})$
for any unimodular $\mathbf{U}$. More generally, with a shift $\vec{s}$,
there exists a matrix $\mathbf{H}$ that is right unimodularly equivalent
to $\mathbf{F}$ and $\cdeg_{\vec{s}}\mathbf{G}\le\cdeg_{\vec{s}}(\mathbf{F}\mathbf{U})$
for any unimodular $\mathbf{U}$\end{cor}
\begin{proof}
Just repeatedly apply \prettyref{lem:matrixGCDlowerDegrees} to matrices
that are right unimodularly equivalent to $\mathbf{F}$. 
\end{proof}
The existence of matrices with minimal column degrees allows us to
define \emph{column reducedness.}
\begin{defn}
A matrix $\mathbf{A}\in\mathbb{K}\left[x\right]^{m\times n}$ is said
to be \emph{column reduced} if $\cdeg\mathbf{A}\le\cdeg\mathbf{A}\mathbf{U}$
for any unimodular matrix $\mathbf{U}$. More generally, for a shift
$\vec{s}\in\mathbb{Z}^{n}$, a matrix $\mathbf{A}\in\mathbb{K}\left[x\right]^{m\times n}$
is said to be $\vec{s}$\emph{-column reduced} if $\cdeg_{\vec{s}}\mathbf{A}\le\cdeg_{\vec{s}}\mathbf{A}\mathbf{U}$
for any unimodular matrix $\mathbf{U}$.

Note that the nonzero columns of a column reduced matrix form a column
basis. So computing a column reduced form is at least as hard as computing
a column basis.

The leading column coefficient matrix provides a useful test for being
column reduced.\end{defn}
\begin{lem}
\label{lem:columnReducedLeadingCoefficient}A matrix $\mathbf{A}\in\mathbb{K}\left[x\right]^{m\times n}$
with no zero columns is column reduced if and only if \textup{$\lcoeff\left(\mathbf{A}\right)$
has full column rank.}\end{lem}
\begin{proof}
We just need to show that the full-rank $\lcoeff\left(\mathbf{A}\right)$
implies a column reduced $\mathbf{A}$, since the other direction
is covered in \prettyref{lem:columnOperationToReduceDegree}. Suppose
$\lcoeff\left(\mathbf{A}\right)$ is full-rank but not column reduced.
Let $\mathbf{B}$ be a unimodularly equivalent column reduced matrix,
which exists from \prettyref{cor:minimalUnimodularEquivalentMatrix},
then $\cdeg\mathbf{A}>\cdeg\mathbf{B}$. Let $\mathbf{U}$ be the
unimodular matrix satisfying $\mathbf{A}\mathbf{U}=\mathbf{B}$. Let
$\bar{\mathbf{A}}$ be a square matrix with $n$ rows chosen from
$\mathbf{A}$ such that $\cdeg\bar{\mathbf{A}}=\cdeg\mathbf{A}$.
Then $\bar{\mathbf{B}}=\bar{\mathbf{A}}\mathbf{U}$ is a matrix consisting
of rows from $\mathbf{B}$ with the same indices. It follows that
$\deg\det\bar{\mathbf{A}}=\deg\det\bar{\mathbf{B}}$. But $\cdeg\bar{\mathbf{B}}\le\cdeg\mathbf{B}<\cdeg\mathbf{A}=\cdeg\bar{\mathbf{A}}$,
while from \prettyref{lem:fullRankLeadingCoefficientAndDegreeOfDetermant}
$\deg\det\bar{\mathbf{B}}\le\sum\cdeg\bar{\mathbf{B}}$ and $\deg\det\bar{\mathbf{A}}=\sum\cdeg\bar{\mathbf{A}}$,
which gives $\deg\det\bar{\mathbf{B}}<\deg\det\bar{\mathbf{A}}$,
a contradiction.
\end{proof}
In \prettyref{lem:productDegreeBound}, when the matrix $\mathbf{A}$
is $\vec{u}$-column reduced, the bound becomes an equality, which
then gives the following lemma. This can be viewed as a stronger version
of the predictable-degree property \citep{kailath:1980}.
\begin{lem}
\label{lem:predictableDegree}If $\mathbf{A}$ is a full-rank $\vec{u}$-column
reduced matrix with $\cdeg_{\vec{u}}\mathbf{A}=\vec{v}$, then $\cdeg_{\vec{v}}\mathbf{B}=\vec{w}$
if and only if $\cdeg_{\vec{u}}\left(\mathbf{A}\mathbf{B}\right)=\vec{w}$.\end{lem}
\begin{proof}
$\vec{u}$-column reduced means the leading column coefficient matrix
$A$ of $x^{\vec{u}}\mathbf{A}x^{-\vec{v}}$ has linearly independent
columns. Now the leading coefficient matrix $B$ of $x^{\vec{v}}\mathbf{B}^{-\vec{w}}$
has no zero column if and only if the leading column coefficient matrix
$AB$ of $x^{\vec{u}}\mathbf{A}x^{-\vec{v}}x^{\vec{v}}\mathbf{B}^{-\vec{w}}=x^{\vec{u}}\mathbf{A}\mathbf{B}^{-\vec{w}}$
has no zero column, in other words, $x^{\vec{v}}\mathbf{B}^{-\vec{w}}$
has column degrees $\left[0,\dots,0\right]$ if and only if $x^{\vec{u}}\mathbf{A}\mathbf{B}^{-\vec{w}}$
has column degrees $\left[0,\dots,0\right]$.
\end{proof}

\section{Order Basis}

We now look at order basis in more detail.

Let $\mathbb{K}$ be a field, $\mathbf{F}\in\mathbb{K}\left[\left[x\right]\right]^{m\times n}$
a matrix of power series and $\vec{\sigma}=\left[\sigma_{1},\dots,\sigma_{m}\right]$
a vector of non-negative integers. 
\begin{defn}
We say a vector of polynomials $\mathbf{p}\in\mathbb{K}\left[x\right]^{n\times1}$
has \emph{order} $\left(\mathbf{F},\vec{\sigma}\right)$ (or \emph{order}
$\vec{\sigma}$ with respect to $\mathbf{F}$) if $\mathbf{F}\cdot\mathbf{p}\equiv\mathbf{0}\mod x^{\vec{\sigma}}$,
that is, 
\[
\mathbf{F}\cdot\mathbf{p}=x^{\vec{\sigma}}\mathbf{r}=\begin{bmatrix}x^{\sigma_{1}}\\
 & \ddots\\
 &  & x^{\sigma_{m}}
\end{bmatrix}\mathbf{r}
\]
 for some $\mathbf{r}\in\mathbb{K}\left[\left[x\right]\right]^{m\times1}$.
If $\vec{\sigma}=\left[\sigma,\dots,\sigma\right]$ is uniform, then
we say that $\mathbf{p}$ has order $\left(\mathbf{F},\sigma\right).$
\begin{comment}
The vector of power series $\mathbf{r}$ is called the order $\left(\mathbf{F},\sigma\right)$-residual
of \textbf{$\mathbf{p}$}. 
\end{comment}
{} The set of all order $\left(\mathbf{F},\vec{\sigma}\right)$ vectors
is a free $\mathbb{K}\left[x\right]$-module denoted by $\left\langle \left(\mathbf{F},\vec{\sigma}\right)\right\rangle $. 
\end{defn}
An order basis for $\mathbf{F}$ and $\vec{\sigma}$ is simply a basis
for the $\mathbb{K}\left[x\right]$-module $\left\langle \left(\mathbf{F},\vec{\sigma}\right)\right\rangle $.
In this thesis we compute those order bases having minimal or shifted
minimal degrees (also referred to as a reduced order basis in \citep{BL1997}).



An \emph{order basis} \citep{BeLa94,BL1997} $\mathbf{P}$ of $\mathbf{F}$
with order $\vec{\sigma}$ and shift $\vec{s}$, or simply an $\left(\mathbf{F},\vec{\sigma},\vec{s}\right)$-basis,
is a basis for the module $\left\langle \left(\mathbf{F},\vec{\sigma}\right)\right\rangle $
\begin{comment}
\[
\left\langle \left(\mathbf{F},\vec{\sigma}\right)\right\rangle =\{\mathbf{p}\in\mathbb{K}\left[x\right]^{n\times1}\|\mathbf{F}\cdot\mathbf{p}=x^{\vec{\sigma}}\mathbf{r},\mathbf{r}\in\mathbb{K}[[x]]^{m\times1}\}
\]
\end{comment}
{} having minimal $\vec{s}$-column degrees. If $\vec{\sigma}=\left[\sigma,\dots,\sigma\right]$
are constant vectors then we simply write $\left(\mathbf{F},\sigma,\vec{s}\right)$-basis.
The precise definition of an $\left(\mathbf{F},\vec{\sigma},\vec{s}\right)$-basis
is as follows.
\begin{defn}
\label{def:orderBasis}A polynomial matrix $\mathbf{P}$ is an order
basis of $\mathbf{F}$ of order $\vec{\sigma}$ and shift $\vec{s}$,
denoted by $\left(\mathbf{F},\vec{\sigma},\vec{s}\right)$-basis,
if the following properties hold: 
\begin{enumerate}
\item $\mathbf{P}$ is a nonsingular matrix of dimension $n$ and is $\vec{s}$-column
reduced. 
\item $\mathbf{P}$ has order $\left(\mathbf{F},\vec{\sigma}\right)$ (or
equivalently, each column of $\mathbf{P}$ is in $\left\langle (\mathbf{F},\vec{\sigma})\right\rangle $). 
\item Any $\mathbf{q}\in\left\langle \left(\mathbf{F},\vec{\sigma}\right)\right\rangle $
can be expressed as a linear combination of the columns of $\mathbf{P}$,
given by $\mathbf{P}^{-1}\mathbf{q}$. 
\end{enumerate}
\end{defn}
\begin{comment}
Note that the module $\left\langle \left(\mathbf{F},\vec{\sigma}\right)\right\rangle $
does not depend on the shift $\vec{s}$. 
\end{comment}




It follows from \prettyref{def:orderBasis} and \prettyref{lem:basisEquivalence}
that any pair of $\left(\mathbf{F},\vec{\sigma},\vec{s}\right)$-bases
$\mathbf{P}$ and $\mathbf{Q}$ are column bases of each other and
are unimodularly equivalent.

From \citep{BL1997} we have the following lemma. 
\begin{lem}
\label{lem:orderBasisEquivalence}The following are equivalent for
a polynomial matrix \textbf{$\mathbf{P}$}: 
\begin{enumerate}
\item $\mathbf{P}$ is a $\left(\mathbf{F},\vec{\sigma},\vec{s}\right)$-basis. 
\item $\mathbf{P}$ is comprised of a set of $n$ minimal $\vec{s}$-degree
polynomial vectors that are linearly independent and each having order
$\left(\mathbf{F},\vec{\sigma}\right)$. 
\item $\mathbf{P}$ does not contain a zero column, has order $\left(\mathbf{F},\vec{\sigma}\right)$,
is $\vec{s}$-column reduced, and any $\mathbf{q}\in\left\langle \left(\mathbf{F},\vec{\sigma}\right)\right\rangle $
can be expressed as a linear combination of the columns of $\mathbf{P}$.

\end{enumerate}
\end{lem}
In some cases an entire order basis is unnecessary and instead one
looks for a minimal basis that generates only the elements of $\left\langle \left(\mathbf{F},\vec{\sigma}\right)\right\rangle $
with $\vec{s}$-degrees bounded by a given $\delta$. Such a minimal
basis is a partial $\left(\mathbf{F},\vec{\sigma},\vec{s}\right)$-basis
comprised of elements of a $\left(\mathbf{F},\vec{\sigma},\vec{s}\right)$-basis
with $\vec{s}$-degrees bounded by $\delta$. This is called a \emph{minbasis}
in \citet{Storjohann:2006}. 
\begin{defn}
\label{def:genset} Let $\left\langle \left(\mathbf{F},\vec{\sigma},\vec{s}\right)\right\rangle _{\delta}\subset\left\langle \left(\mathbf{F},\vec{\sigma}\right)\right\rangle $
denote the set of order $\left(\mathbf{F},\vec{\sigma}\right)$ polynomial
vectors with $\vec{s}$-degree bounded by $\delta$. A $\left(\mathbf{F},\vec{\sigma},\vec{s}\right)_{\delta}$-basis
is a polynomial matrix $\mathbf{P}$ not containing a zero column
and satisfying: \end{defn}
\begin{enumerate}
\item $\mathbf{P}$ has order $\left(\mathbf{F},\vec{\sigma}\right).$ 

\begin{enumerate}
\item Any element of $\left\langle \left(\mathbf{F},\vec{\sigma},\vec{s}\right)\right\rangle _{\delta}$
can be expressed as a linear combination of the columns of $\mathbf{P}$. 
\item $\mathbf{P}$ is $\vec{s}$-column reduced. 
\end{enumerate}
\end{enumerate}
\begin{comment}
As before, the linear combination here is in fact unique. 
\end{comment}
A $\left(\mathbf{F},\vec{\sigma},\vec{s}\right)_{\delta}$-basis is,
in general, not square unless $\delta$ is large enough to contain
all $n$ basis elements in which case it is a complete $\left(\mathbf{F},\vec{\sigma},\vec{s}\right)$-basis.


\section{Kernel Basis}

Recall that the kernel of $\mathbf{F}\in\mathbb{K}\left[x\right]^{m\times n}$
is the $\mathbb{F}\left[x\right]$-module 
\[
\left\{ \mathbf{p}\in\mathbb{K}\left[x\right]^{n}~|~\mathbf{F}\mathbf{p}=0\right\} .
\]
 A kernel basis of $\mathbf{F}$ is just a basis of this module. Kernel
bases are closely related to order bases, as can be seen from the
following definitions. 
\begin{defn}
\label{def:kernelBasis}Given $\mathbf{F}\in\mathbb{K}\left[x\right]^{m\times n}$,
a polynomial matrix $\mathbf{N}\in\mathbb{K}\left[x\right]^{n\times*}$
is a (right) kernel basis of $\mathbf{F}$ if the following properties
hold: 
\begin{enumerate}
\item $\mathbf{N}$ is full-rank. 
\item $\mathbf{N}$ satisfies $\mathbf{F}\cdot\mathbf{N}=0$. 
\item Any $\mathbf{q}\in\mathbb{K}\left[x\right]^{n}$ satisfying $\mathbf{F}\mathbf{q}=0$
can be expressed as a linear combination of the columns of $\mathbf{N}$,
that is, there exists some polynomial vector $\mathbf{p}$ such that
$\mathbf{q}=\mathbf{N}\mathbf{p}$. 
\end{enumerate}
\end{defn}
Again, it follows from \prettyref{def:kernelBasis} and \prettyref{lem:basisEquivalence}
that any pair of kernel bases $\mathbf{N}$ and $\mathbf{M}$ of $\mathbf{F}$
are column bases of each other and are unimodularly equivalent.

An $\vec{s}$-minimal kernel basis of $\mathbf{F}$ is just a kernel
basis that is $\vec{s}$-column reduced.
\begin{defn}
Given $\mathbf{F}\in\mathbb{K}\left[x\right]^{m\times n}$, a polynomial
matrix $\mathbf{N}\in\mathbb{K}\left[x\right]^{n\times*}$ is a $\vec{s}$-minimal
(right) kernel basis of $\mathbf{F}$ if\textbf{ $\mathbf{N}$} is
a kernel basis of $\mathbf{F}$ and $\mathbf{N}$ is $\vec{s}$-column
reduced. We also call a $\vec{s}$-minimal (right) kernel basis of
$\mathbf{F}$ a $\left(\mathbf{F},\vec{s}\right)$-kernel basis in
this thesis.
\end{defn}




\begin{comment}
Note that the module $\left\langle \left(\mathbf{F},\vec{\sigma}\right)\right\rangle $
does not depend on the shift $\vec{s}$. 
\end{comment}


\begin{comment}
Minimal kernel bases can be directly computed via order basis computation.
Indeed if the order $\sigma$ of a $\left(\mathbf{F},\sigma,\vec{s}\right)$-basis
$\mathbf{P}$ is high enough, then $\mathbf{P}$ contains a $\vec{s}$-minimal
kernel basis $\mathbf{N}$. However, this approach may require the
order $\sigma$ to be quite high. For example, if $\mathbf{F}$ has
degree $d$ and $\vec{s}$ is uniform, then its minimal kernel bases
can have degree up to $md$. In that case the order $\sigma$ would
need to be set to $d+md$ in the order basis computation in order
to fully compute a minimal kernel basis. The fastest method of computing
such a $\left(\mathbf{F},d+md\right)$-basis would cost $O^{\sim}\left(n^{\omega}\left\lceil m^{2}d/n\right\rceil \right)$
using the algorithm from \citep{za2009}. We can see from this cost
that there is room for improvement when $m$ is large. For example,
in the worst case when $m\in\Theta\left(n\right)$, this cost would
be $O^{\sim}\left(n^{\omega+1}d\right)$. This points to a root cause
for the inefficiency in this approach. Namely, when $m$ is large,
the computed kernel basis, which can have a column dimension of $n-m$,
is a much smaller subset of the order basis computed. Hence considerable
effort is put in the computation of order basis elements that are
not part of a nullspace basis. A key to reducing the cost is therefore
to reduce such computation of unneeded order basis elements, which
is achieved in our algorithm by only using order basis computation
to compute partial nullspace bases of low degrees.
\end{comment}




\chapter{Order Basis with Balanced Shifts\label{chap:OrderBasis}}

In this chapter and the next chapter we give algorithms for computing
a shifted order basis of an $m\times n$ matrix of power series over
a field $\mathbb{K}$ with $m\le n$. For a given order $\sigma$
and balanced shift $\vec{s}$ the algorithm in this chapter determines
an order basis with a cost of $O^{\sim}(n^{\omega}a)$ field operations
in $\mathbb{K}$, where $\omega$ is the exponent of matrix multiplication
and $a=m\sigma/n$. Here, an input shift is balanced when $\max(\vec{s})-\min(\vec{s})\in O(a)$.
%Here $O^{\sim}$
%is just $O$ with log factors omitted and 
%$\MM\left(n,d\right)$ denotes
%the cost of multiplying two polynomial matrices with dimension $n$
%and degree $d$. 
This extends earlier work of Storjohann which only determines a subset
of an order basis that is within a specified degree bound $\delta$
using $O^{\sim}(n^{\omega}\delta)$ field operations for $\delta\ge\lceil a\rceil$.
In the end of this chapter, we show how a more refined cost of $O^{\sim}(n^{\omega-1}m\sigma)$
instead of $O^{\sim}\left(n^{\omega}a\right)$ field operations can
be achieved when the shifts are balanced.



In this chapter, we assume, without any loss of generality, that $n/m$
and $\sigma$ are powers of two. This can be achieved by padding zero
rows to the input matrix and multiplying it by some power of $x$.

We first give a brief description of Storjohann's transformation for
computing a partial order basis. %
\begin{comment}
We then provide another transformation that allows us to extend the
result from a lower order Storjohann's transformation to the result
from a higher order transformation. This lead to an algorithm that
correctly computes a complete order basis. We then show that this
algorithm is efficient on problems with balanced shifts. Finally,
we present the second algorithm that works efficiently when the shift
is unbalanced but satisfies the condition 

a special case of unbalanced shift.
\end{comment}



\section{Balancing Input with Storjohann's Transformation}

\label{sub:storjohannTransformation}

For computing a $\left(\mathbf{F},\sigma,\vec{s}\right)$-basis with
input matrix $\mathbf{F}\in\mathbb{K}\left[\left[x\right]\right]^{m\times n}$,
shift $\vec{s}$ and order $\sigma$ one can view $\mathbf{F}$ as
a polynomial matrix with degree $\sigma-1$, as higher order terms
are not needed in the computation. As such the total input size of
an order basis problem is $mn\sigma$ coefficients. One can apply
the method of \citet{Giorgi2003} directly, which gives a cost of
\begin{align}
\sum_{i=0}^{\log\sigma}2^{i}\MM(n,2^{-i}\sigma)= & \sum_{i=0}^{\log\sigma}2^{-i}\sigma\MM(n,2^{i})\nonumber \\
\subset & O\left(\sum_{i=0}^{\log\sigma}\sigma n^{\omega}\M\left(2^{i}\right)2^{-i}\right)\nonumber \\
\subset & O\left(n^{\omega}\sum_{i=0}^{\log\sigma}\M\left(\sigma\right)\right)\label{eq:polynomialMultiplicationBound}\\
= & O(n^{\omega}\M(\sigma)\log\sigma).\nonumber 
\end{align}
Equation (\ref{eq:polynomialMultiplicationBound}) follows from the
fact $2^{i}\le\sigma$ implies $\M\left(2^{i}\right)2^{-i}\le\M\left(\sigma\right)/\sigma$.
This cost is close to the cost of multiplying two matrices with dimension
$n$ and degree $\sigma$. Note that this cost is independent of the
degree shift. This is very efficient if $m\in\Theta\left(n\right)$.
However, for small $m$, say $m=1$ as in Hermite Pad� approximation,
the total input size is only $n\sigma$ coefficients. Matrix multiplication
cannot be used effectively on a such vector input.

\citet{Storjohann:2006} provides a novel way to transform an order
basis problem with small row dimension to a problem with higher row
dimension and possibly lower degree to take advantage of \citet{Giorgi2003}'s
algorithm. We provide a quick overview of a slightly modified version
of Storjohann's method. Our small modification allows a nonuniform
degree shift for the input and provides a slightly simpler degree
shift, degree, and order for the transformed problem. The proof of
its correctness is provided in \secref{transform}. In order to compute
a $\left(\mathbf{F},\sigma,\vec{s}\right)$-basis, assuming without
loss of generality that $\min\left(\vec{s}\right)=0$, we first write
\[
\mathbf{F}=\mathbf{F}_{0}+\mathbf{F}_{1}x^{\delta}+\mathbf{F}_{2}x^{2\delta}+\cdots+\mathbf{F}_{l}x^{l\delta},
\]
 with $\deg\mathbf{F}_{i}<\delta$%
\begin{comment}
I used $\deg\mathbf{F}_{i}\le\delta-1$ before, but the reviewer suggested
changing to $\deg\mathbf{F}_{i}<\delta$ and said it's slightly easier
to read. A reason to use $\le$ is to make it consistent with the
definition of minbasis. For example, $\left(\mathbf{F},\sigma,\vec{s}\right)_{\delta-1}$-basis
indicates that the degree bound is $\delta-1$. I still prefer to
use $\le$. 
\end{comment}
{} for a positive integer $\delta$, and where we assume (again without
loss of generality) that $\sigma=\left(l+1\right)\delta$. Set 
\[
{\bar{\mathbf{F}}}=\left[\begin{array}{c|cccc}
\mathbf{F}_{0}+\mathbf{F}_{1}x^{\delta} & \mathbf{0}_{m} & \mathbf{0}_{m} & \cdots & \mathbf{0}_{m}\\
\hline \mathbf{F}_{1}+\mathbf{F}_{2}x^{\delta} & \mathbf{I}_{m} & \mathbf{0}_{m}\\
\mathbf{F}_{2}+\mathbf{F}_{3}x^{\delta} & \mathbf{0}_{m} & \mathbf{I}_{m}\\
\vdots &  &  & \ddots\\
\mathbf{F}_{l-1}+\mathbf{F}_{l}x^{\delta} &  &  &  & \mathbf{I}_{m}
\end{array}\right]_{ml\times(n+m(l-1))}.
\]
 On the left side of $\bar{\mathbf{F}}$, each block $\mathbf{F}_{i}+\mathbf{F}_{i+1}x^{\delta}$
has dimension $m\times n$. On the right side, there are $l\times(l-1)$
blocks of $\mathbf{0}_{m}$'s or $\mathbf{I}_{m}$'s each having dimension
$m\times m$. The overall dimension of $\bar{\mathbf{F}}$ is $ml\times(n+m(l-1))$.
Set $\vec{s'}=\left[\vec{s},0,\dots,0\right]$ ($\vec{s}$ followed
by $m\left(l-1\right)$ $0$'s). A $({\bar{\mathbf{F}}},2\delta,\vec{s'})$-basis
can then be computed by the method of Giorgi et al. with a cost of
$O^{\sim}\left(n^{\omega}\delta\right)$ for $\delta\ge\left\lceil a\right\rceil $,
where $a=m\sigma/n$. This transformation of Storjohann can be viewed
as a partial linearization of the original problem, where $\bar{\mathbf{F}}$
is analogous to the coefficient matrix of $\mathbf{F}$. Note that
$\bar{\mathbf{F}}$ has $l$ block rows each containing $m$ rows.
We continue to use each block row to represent $m$ rows for the remainder
of this chapter.

Clearly an $(\bar{\mathbf{F}},2\delta,\vec{s'})$-basis $\bar{\mathbf{P}}$
of the transformed problem is not a $\left(\mathbf{F},\sigma,\vec{s}\right)$-basis
of the original problem, as $\bar{\mathbf{P}}$ has a higher dimension
and lower degree. However, the first $n$ rows of the $(\bar{\mathbf{F}},2\delta,\vec{s'})_{\delta-1}$-basis
contained in $\bar{\mathbf{P}}$ is a $\left(\mathbf{F},\sigma,\vec{s}\right)_{\delta-1}$-basis.

Note that there is no need to set the degree parameter $\delta$ to
less than $\left\lceil a\right\rceil $, as this produces fewer basis
elements without a better cost. The lowest cost is achieved when $\bar{\mathbf{F}}$
is close to square so matrix multiplication can be used most effectively.
This requires the number of block rows $l$ of $\bar{\mathbf{F}}$
to be close to $n/m$, which requires $\delta=\Theta\left(a\right)$.
Recall that $mn\sigma$ is the total size of the original $m\times n$
input matrix $\mathbf{F}$, hence $a=mn\sigma/n^{2}=m\sigma/n$ is
the average size of each entry of $\mathbf{F}$ if the $m$ rows of
$\mathbf{F}$ are spread out over $n$ rows. Choosing $\delta=\Theta\left(a\right)$,
the cost of computing a $({\bar{\mathbf{F}}},2\delta,\vec{s'})$-basis
is then $O^{\sim}\left(n^{\omega}a\right)$. In the first part of
this chapter, we use the average size $a=m\sigma/n$ in the asymptotic
cost notation. Therefore, $a$ is assumed to be tending to infinity,
which means $m\sigma>n$. Together with the assumption that $\sigma$
and $n/m$ are both powers of two, $m\sigma/n$ is then always a positive
integer in this paper.

%Let us now look at a concrete example that illustrate Storjohann's method. 

\begin{example}
\label{exm:StorjohannTransformation}Let $\mathbb{K}=\mathbb{Z}_{2}$,
$\sigma=8$, $\delta=2$ and 
\[
\mathbf{F}=[x+x^{2}+x^{3}+x^{4}+x^{5}+x^{6},~1+x+x^{5}+x^{6}+x^{7},~1+x^{2}+x^{4}+x^{5}+x^{6}+x^{7},~1+x+x^{3}+x^{7}]
\]
 a vector of size $1\times4$. Then 
\[
\bar{\mathbf{F}}=\left[{\begin{array}{cccc|cc}
x+x^{2}+x^{3} & 1+x & 1+x^{2} & 1+x+x^{2} & ~~0~ & ~0~\\
\hline 1+x+x^{2}+x^{3} & x^{3} & 1+x^{2}+x^{3} & x & ~~1~ & ~0~\\
1+x+x^{2} & x+x^{2}+x^{3} & 1+x+x^{2}+x^{3} & x^{3} & ~~0~ & ~1~
\end{array}}\right]_{3\times6}
\]
 and a $\left(\bar{\mathbf{F}},4,\vec{0}\right)$-basis is given by
\[
\bar{\mathbf{P}}=\left[{\begin{array}{cc|cccc}
~1~ & ~x~ & 1 & x^{2}+x^{3} & 0 & x+x^{2}+x^{3}\\
0 & 1 & 0 & x^{2} & x^{2}+x^{3} & 0\\
1 & 1+x & x+x^{2} & x^{2} & x^{2} & x^{2}\\
1 & 0 & 0 & 0 & 0 & 0\\
\hline 0 & 1 & 1 & 0 & x^{2} & x+x^{2}+x^{3}\\
0 & 1 & 1+x^{2} & 0 & x^{2} & x+x^{2}
\end{array}}\right].
\]
 The first two columns of $\bar{\mathbf{P}}$ have degree less than
$2$, hence its top left $4\times2$ submatrix is a $\left(\mathbf{F},8,\vec{0}\right)_{1}$-basis.
This is a low degree part of the\textbf{ $(\mathbf{F},8,\vec{0})$}-basis
\[
\mathbf{P}=\begin{bmatrix}1 & x & 1 & x^{2}\\
0 & 1 & x^{2}+x^{3} & 0\\
1 & 1+x & x & x^{3}+x^{4}\\
1 & 0 & 0 & 0
\end{bmatrix}.
\]
 Note that if $\delta$ is set to $\sigma/2=4$, then the transformed
problem is the same as the original problem. 
\end{example}

\section{\label{sub:Unbalanced-Output}Unbalanced Output }

Storjohann's transformation can be used to efficiently compute a $\left(\mathbf{F},\sigma,\vec{s}\right)_{\delta-1}$-basis
if the degree parameter $\delta$ is close to the average degree $d=m\sigma/n$.
However, if $\delta$ is large, say $\delta=\Theta\left(\sigma\right)$,
or if we want to compute a complete $\left(\mathbf{F},\sigma,\vec{s}\right)$-basis,
then the current analysis for the computation still gives the cost
estimate of  $O^{\sim}\left(n^{\omega}\sigma\right)$.

The underlying difficulty with computing a complete order basis is
that the basis can have degree up to $\sigma$. As the output of this
problem has dimension $n\times n$ and degree up to $\Theta\left(\sigma\right)$,
this may seem to suggest $O^{\sim}\left(n^{\omega}\sigma\right)$
is about the best that can be done. However, the total size of the
output, that is, the total number of coefficients of all $n^{2}$
polynomial entries can still be bounded by $O\left(mn\sigma\right)$,
the same as the size of the input. This gives some hope for a more
efficient method. 
\begin{lem}
\label{lem:boundOfSumOfShiftedDegreesOfOrderBasis}Let $\vec{t}$
be the $\vec{s}$-column degrees of a $\left(\mathbf{F},\sigma,\vec{s}\right)$-basis.
Then $\sum\vec{t}~\le~m\sigma+\sum\vec{s}$%
\begin{comment}
 and $\max_{i}\left(\vec{t}_{i}-\vec{s}_{i}\right)\le\sigma$
\end{comment}
\textup{}%
\begin{comment}
need to permute the columns to put the pivots on the diagonal.
\end{comment}
. In addition, the total size of any $\left(\mathbf{F},\sigma,\vec{s}\right)$-basis
in $\vec{s}$-Popov form is bounded by $nm\sigma$. \end{lem}
\begin{proof}
The sum of the $\vec{s}$-column degrees is $\sum\vec{s}$ at order
0, since the identity matrix is a $\left(\mathbf{F},0,\vec{s}\right)$-basis.
This sum increases by at most $r$ for each order increase, as can
be seen from the iterative computation of order bases in \citep{BeLa94,Giorgi2003}.
The second statement follows from the fact that the row degrees and
the $\vec{s}$-column degrees of any $\vec{s}$-Popov form are represented
by the pivot entries.. 
\end{proof}
\begin{comment}
As a result, the average degree of the entries of the output matrix
can be also bounded by $d=m\sigma/n$. 
\end{comment}


Let us now look at the average column degree of the output. In the
first part of our discussion on order basis computation, we assume,
without loss of generality, that $\min\left(\vec{s}\right)=0$ so
$\deg\mathbf{q}\le\deg_{\vec{s}}\mathbf{q}$ for any $\mathbf{q}\in\mathbb{K}\left[x\right]^{n}$.
The situation is simpler if the shift $\vec{s}$ is uniform since
then $\sum\vec{t}\le m\sigma$ by \lemref{boundOfSumOfShiftedDegreesOfOrderBasis}
and the average column degree is therefore bounded by $a=m\sigma/n$.
In the first part of this thesis, we consider a slightly more general
case, when the shift $\vec{s}$ is \emph{balanced}, which is defined
as follows. 
\begin{defn}
\label{def:balancedShift}A shift $\vec{s}$ is balanced if $\max\vec{s}-\min\vec{s}\in O(a)$
or if $\max\vec{s}-\min\vec{s}\in O(m\sigma/n)$. 
\end{defn}
Note that we only need to use the second definition using $\max\vec{s}-\min\vec{s}\in O(m\sigma/n)$
when we discuss the more refined cost in \secref{removeCeilingFunction}. 

By assuming $\min\vec{s}=0$, $\vec{s}$ is balanced if $\max\vec{s}\in O(a)$.
In this case, \lemref{boundOfSumOfShiftedDegreesOfOrderBasis} implies
$\sum\left(\vec{t}\right)\le m\sigma+\sum\left(\vec{s}\right)\in O\left(m\sigma+na\right)=O\left(m\sigma\right)$.
Hence the average column degree of the output basis remains $O\left(a\right)$.

%From the iterative algorithms for computing order basis, computing
%a order basis to order $\sigma$ requires up to $\sigma$ iterations,
%each iteration increases the sum of column degrees of the order basis
%by at most $m$. Therefore, the sum of column degrees of an order
%$\sigma$ order basis is at most $m\sigma$.


\begin{comment}
In fact, if $\mathbf{F}\left(0\right)$ is full rank, the sum of column
degrees of an order $\sigma$ order basis is exactly $m\sigma$, as
we need exactly $\sigma$ iterations, each increases the sum of the
column degrees by exactly $m$. 
\end{comment}


The fact that a $\left(\mathbf{F},\sigma,\vec{s}\right)$-basis can
have degree up to $\sigma$ while its average column degree is $O\left(a\right)$
implies that an order basis can have quite unbalanced column degrees,
especially if $m$ is small. A similar problem with unbalanced output
is encountered in null space basis computation. \citet{storjohann-villard:2005}
deal with this in the following way.

Let $d$ be the average column degree of the output. Set the degree
parameter $\delta$ to twice that of $d$. This allows one to compute
at least half the columns of a basis (since the number of columns
with degree at least $\delta$ must be at most a half of the total
number of columns). One can then simplify the problem, so that the
computed basis elements are completely removed from the problem. This
reduces the dimension of the problem by at least a factor of $2$.
One then doubles the degree bound $\delta$ in order to have at least
$3/4$ of the basis elements computed. Repeating this, at iteration
$i$, at most $1/2^{i}$ of the basis elements are remaining. Therefore,
no more than $\log n$ iterations are needed to compute all basis
elements.


\section{Extending Storjohann's Transformation}

\label{sec:transform}

\begin{comment}
notations: 
\begin{itemize}
\item Matrices

\begin{itemize}
\item Polynomial matrices: bold capital $\mathbf{A,B,F,G,P,R}$

\begin{itemize}
\item $\mathbf{F}$: input matrix,

\begin{itemize}
\item $\bar{\mathbf{F}}$: Storjohann transformed matrix, 
\item $\check{\mathbf{F}}$: the transformation that extends $\bar{\mathbf{F}}$. 
\end{itemize}
\item $\mathbf{G}$: input matrix for the second subproblem. 
\end{itemize}
\item scalar matrices: non-bold Capital, used to distinguish coefficient
matrix 
\item Identity matrix: $\mathbf{I}$ this is to be consistent with polynomial
matrices, as identity is considered as an important element of polynomial
matrices. 
\end{itemize}
\item order

\begin{itemize}
\item $\sigma,\omega$ 
\end{itemize}
\item vectors

\begin{itemize}
\item polynomial vectors: $\mathbf{p,q,r,s,t}$ 
\item $\mathbf{p}$: element in basis $\mathbf{P}$ 
\item $\mathbf{q}:$ some test element in $\left\langle \left(\mathbf{F},\sigma\right)\right\rangle $ 
\item $\bar{\mathbf{t}}$: test element in $\left\langle \left(\check{\mathbf{F}},\vec{\omega}\right)\right\rangle $ 
\end{itemize}
\item shifts:

\begin{itemize}
\item $\vec{e}$: uniform 0 shift used for transformation. 
\item $\vec{s}$: original input shift 
\item $\vec{t}$: output shifted degrees of the current basis, with result
from previous subproblems included. i.e., $\left[\vec{s},0,\dots,0\right]$-degrees
of the current basis. 
\item $\vec{b},\vec{a}$: input shift and output shifted degrees for each
reduced subproblem that calls OrderBasis. In the iterative computational
process, input shift $\vec{b}^{\left(i-1\right)}$ corresponds to
input shift $\vec{a}^{\left(i\right)}$. The high degree entries of
$\vec{a}^{\left(i\right)}$ at iteration $i$ is then used as the
input shift $\vec{b}^{\left(i\right)}$ for iteration $i$. 
\end{itemize}
\item dimension:

\begin{itemize}
\item $m,n$: row, column dimension 
\item $k$: number of columns with $\left[\vec{s},0,\dots,0\right]$-degrees
less than degree bound 
\item $l$: number of block rows 
\end{itemize}
\item degrees, order:

\begin{itemize}
\item $\sigma$: order 
\item $\delta$: degree bound 
\item $d$: average degree (=$m\sigma/n$) \end{itemize}
\end{itemize}
\end{comment}


In this section, we introduce a transformation that can be viewed
as an extension of Storjohann's transformation which allows for computation
of a full, rather than partial, order basis. More generally (as discussed
in the next section) this transformation provides a link between two
Storjohann transformed problems constructed using different degree
parameters. For easier understanding, we first focus on a particular
case of this transformation in \prettyref{sub:particularCase} and
then generalize this in \prettyref{sub:generalTransform}.


\subsection{\label{sub:particularCase}A Particular Case}

Consider the problem of computing a $\left(\mathbf{F},\sigma,\vec{s}\right)$-basis.
We assume $\sigma=4\delta$ %
\begin{comment}
(The results below hold for any positive integer great than one, but
4 is used for simplicity. ) 
\end{comment}
for a positive integer $\delta$ and write the input matrix polynomial
as $\mathbf{F}=\mathbf{F}_{0}+\mathbf{F}_{1}x^{\delta}+\mathbf{F}_{2}x^{2\delta}+\mathbf{F}_{3}x^{3\delta}$
with $\deg\mathbf{F}_{i}<\delta$.%
\begin{comment}
again, changed from $\deg\mathbf{F}_{i}\le\delta-1$ to $\deg\mathbf{F}_{i}<\delta$
even though I prefer $\le$ 
\end{comment}
{} In the following, we show that computing a $\left(\mathbf{F},\sigma,\vec{s}\right)$-basis
can be done by computing a $(\mathbf{F}',\vec{\omega},\vec{s'})$-basis
where 
\begin{equation}
\mathbf{F}'=\left[\begin{array}{cc}
\mathbf{F} & \mathbf{0}\\
\mathbf{F}'_{21} & \mathbf{F}'_{22}
\end{array}\right]=\left[\begin{array}{c|cr}
\mathbf{F}_{0}+\mathbf{F}_{1}x^{\delta}+\mathbf{F}_{2}x^{2\delta}+\mathbf{F}_{3}x^{3\delta} & \mathbf{0} & \mathbf{0}\\
\hline \mathbf{F}_{1}+\mathbf{F}_{2}x^{\delta} & \mathbf{I}_{m} & \mathbf{0}\\
\mathbf{F}_{2}+\mathbf{F}_{3}x^{\delta} & \mathbf{0} & \mathbf{I}_{m}
\end{array}\right]\label{eq:extendedSTransformTop}
\end{equation}
 with order $\vec{\omega}=\left[4\delta,\dots,4\delta,2\delta,\dots,2\delta\right]$
(with $m$ $4\delta$'s and $2m$ $2\delta$'s) and degree shift $\vec{s'}=\left[\vec{s},e,\dots,e\right]$
(with $2m$ $e$'s), where $e$ is an integer less than or equal to
$1$. %
\begin{comment}
actually true for $e\le\min\vec{s}+1$. keeping this $e$ helps to
show that this hold true in more general situations, while Storjohann
only used $e=1$. it can be negative as well. 
\end{comment}
{} We set $e$ to $0$ in this paper for simplicity%
\footnote{Storjohann used $e=1$ in \citep{Storjohann:2006}. All results in
this section still hold for any other $e\le1.$%
}. %
\begin{comment}
In fact, it is quite easy to construct a $(\check{\mathbf{F}},\vec{\omega},\vec{s'})$-basis
from a $\left(\mathbf{F},\sigma,\vec{s}\right)$-basis, as we show
later in \prettyref{lem:FtoAbasis}. However, it requires more work
to extract a $\left(\mathbf{F},\sigma,\vec{s}\right)$-basis from
a $(\check{\mathbf{F}},\vec{\omega},\vec{s'})$-basis, which is addressed
eventually in \prettyref{cor:extractingFbasisFromGbasis}. Note that
although constructing a $(\check{\mathbf{F}},\vec{\omega},\vec{s'})$-basis
from a $\left(\mathbf{F},\sigma,\vec{s}\right)$-basis in \prettyref{lem:FtoAbasis}
is the reverse of what we want, this well-formed $(\check{\mathbf{F}},\vec{\omega},\vec{s'})$-basis
restricts the elements of $\langle(\check{\mathbf{F}},\vec{\omega},\vec{s'})\rangle$
to a simple form shown in \prettyref{cor:FtauBasisForm}, which helps
to establish a close correspondence between a $(\check{\mathbf{F}},\vec{\omega},\vec{s'})$-basis
and a $\left(\mathbf{F},\sigma,\vec{s}\right)$-basis in \prettyref{lem:2delta-1Basis},
\prettyref{lem:2deltaBasis}, and \prettyref{thm:mainTheorem}. 
\end{comment}


We first look at the correspondence bettween the elements of $\left\langle \left(\mathbf{F},\sigma,\vec{s}\right)\right\rangle _{\tau}$
and the elements of $\langle(\mathbf{F}',\vec{\omega},\vec{s'})\rangle_{\tau}$
in \prettyref{lem:qToBqOrder} to \prettyref{lem:bqToqOrder}. The
correspondence between $\left(\mathbf{F},\sigma,\vec{s}\right)$-bases
and $(\mathbf{F}',\vec{\omega},\vec{s'})$-bases is then considered
in \prettyref{cor:FtauBasisForm} to \prettyref{thm:mainTheorem}.

Let 
\[
\mathbf{B}=\left[\begin{array}{c}
\mathbf{I}_{n}\\
x^{-\delta}\mathbf{F}_{0}\\
x^{-2\delta}\left(\mathbf{F}_{0}+\mathbf{F}_{1}x^{\delta}\right)
\end{array}\right].
\]

\begin{lem}
\label{lem:qToBqOrder}If $\mathbf{q}\in\left\langle \left(\mathbf{F},\sigma\right)\right\rangle $,
then $\mathbf{B}\mathbf{q}\in\langle(\mathbf{F}'\vec{,\omega})\rangle$.\end{lem}
\begin{proof}
The lemma follows from 
\[
\mathbf{F}'\mathbf{B}\mathbf{q}=\left[\begin{array}{r}
\mathbf{F}_{0}+\mathbf{F}_{1}x^{\delta}+\mathbf{F}_{2}x^{2\delta}+\mathbf{F}_{3}x^{3\delta}\\
\mathbf{F}_{0}x^{-\delta}+\mathbf{F}_{1}+\mathbf{F}_{2}x^{\delta}\\
\mathbf{F}_{0}x^{-2\delta}+\mathbf{F}_{1}x^{-\delta}+\mathbf{F}_{2}+\mathbf{F}_{3}x^{\delta}
\end{array}\right]\mathbf{q}\equiv\mathbf{0}\mod x^{\vec{\omega}}.
\]
 Note that the bottom rows of $\mathbf{B}$ may not be polynomials.
However, $\mathbf{B}\mathbf{q}$ is a polynomial vector since $\mathbf{q}\in\left\langle \left(\mathbf{F},\sigma\right)\right\rangle $
implies $\mathbf{q}\in\left\langle \left(\mathbf{F}_{0},\delta\right)\right\rangle $
and $\mathbf{q}\in\left\langle \left(\mathbf{F}_{0}+\mathbf{F}_{1}x^{\delta},2\delta\right)\right\rangle $. 
\end{proof}
\noindent The following lemma shows that the condition $e\le1$ forces
$\deg_{\vec{s'}}\mathbf{B}\mathbf{q}$ to be determined by $\mathbf{q}$. 
\begin{lem}
\label{lem:qToBqDegree}If $\mathbf{q}\in\left\langle \left(\mathbf{F},\sigma,\vec{s}\right)\right\rangle _{\tau}$
for any degree bound $\tau\in\mathbb{Z}$, then $\deg_{\vec{s'}}\mathbf{B}\mathbf{q}=\deg_{\vec{s}}\mathbf{q}$.\end{lem}
\begin{proof}
By assumption $s_{i}\ge0$, so $\deg\mathbf{q}\le\deg_{\vec{s}}\mathbf{q}.$
Now consider the degree of the bottom $2m$ entries, $\mathbf{q}_{2},\mathbf{q}_{3}$,
of 
\[
\begin{bmatrix}\mathbf{q}\\
\mathbf{q}_{2}\\
\mathbf{q}_{3}
\end{bmatrix}=\mathbf{B}\mathbf{q}=\left[\begin{array}{r}
\mathbf{q}\\
x^{-\delta}\mathbf{F}_{0}\cdot\mathbf{q}\\
x^{-2\delta}\left(\mathbf{F}_{0}+\mathbf{F}_{1}x^{\delta}\right)\cdot\mathbf{q}
\end{array}\right].
\]
 Our goal is to show $\deg_{\vec{e}}\left[\mathbf{q}_{2}^{T},\mathbf{q}_{3}^{T}\right]^{T}\le\deg_{\vec{s}}\mathbf{q}$.
Note that 
\[
\deg\mathbf{q}_{2}=\deg\left(\mathbf{F}_{0}\mathbf{q}/x^{\delta}\right)\le\deg\mathbf{q}+\delta-1-\delta\le\deg_{\vec{s}}\mathbf{q}-1,
\]
 and similarly $\deg\mathbf{q}_{3}\le\deg_{\vec{s}}\mathbf{q}-1$.
Therefore 
\[
\deg_{\vec{e}}\begin{bmatrix}\mathbf{q}_{2}\\
\mathbf{q}_{3}
\end{bmatrix}=\deg\begin{bmatrix}\mathbf{q}_{2}\\
\mathbf{q}_{3}
\end{bmatrix}+e\le\deg_{\vec{s}}\mathbf{q}-1+e\le\deg_{\vec{s}}\mathbf{q}.
\]
 \end{proof}
\begin{cor}
\label{cor:sToBs}If $\mathbf{q}\in\left\langle \left(\mathbf{F},\sigma,\vec{s}\right)\right\rangle _{\tau}$
for any degree bound $\tau\in\mathbb{Z}$ , then\textup{ }$\mathbf{B}\mathbf{q}\in\langle(\mathbf{F}',\vec{\omega},\vec{s'})\rangle_{\tau}$\textup{.} 
\end{cor}
%\begin{pf}
%This follows from \prettyref{lem:qToBqOrder} and \prettyref{lem:qToBqDegree}.\end{pf}

\begin{cor}
\label{cor:linearCombinationOfFirstnRows}Let $\bar{\mathbf{S}}_{\tau}$
be a $(\mathbf{F}',\vec{\omega},\vec{s'})_{\tau}$-basis and $\mathbf{S}_{\tau}$
be the top $n$ rows of $\bar{\mathbf{S}}_{\tau}$ for any bound $\tau\in\mathbb{Z}$.
Then any $\mathbf{q}\in\left\langle \left(\mathbf{F},\sigma,\vec{s}\right)\right\rangle _{\tau}$
is a linear combination of the columns of $\mathbf{S}_{\tau}$.\end{cor}
\begin{proof}
By \prettyref{cor:sToBs}, $\mathbf{B}\mathbf{q}\in\langle(\mathbf{F}',\vec{\omega},\vec{s'})\rangle_{\tau}$,
and so is a linear combination of columns of $\bar{\mathbf{S}}_{\tau}$.
That is, there exists a polynomial vector $\mathbf{u}$ such that
$\mathbf{B}\mathbf{q}=\bar{\mathbf{S}}_{\tau}\mathbf{u}$. This remains
true if we restrict the equation to the top $n$ rows, that is, $\mathbf{q}=\left[\mathbf{I}_{n},\mathbf{0}\right]\mathbf{B}\mathbf{q}=\left[\mathbf{I}_{n},\mathbf{0}\right]\bar{\mathbf{S}}_{\tau}\mathbf{u}=\mathbf{S}_{\tau}\mathbf{u}$.\end{proof}
\begin{lem}
\label{lem:bqToqOrder}Let $\bar{\mathbf{q}}\in\langle(\mathbf{F}',\vec{\omega},\vec{s'})\rangle_{\tau}$
for any degree bound $\tau\in\mathbb{Z}$, and $\mathbf{q}_{1}$ the
first $n$ entries of $\mathbf{\bar{q}}$. Then $\mathbf{q}_{1}\in\left\langle \left(\mathbf{F},\sigma,\vec{s}\right)\right\rangle _{\tau}$.\end{lem}
\begin{proof}
The top rows of 
\[
\mathbf{F}'\mathbf{q}=\left[\begin{array}{cc}
\mathbf{F} & \mathbf{0}\\
\mathbf{F}'_{21} & \mathbf{F}'_{22}
\end{array}\right]\left[\begin{array}{c}
\mathbf{q}_{1}\\
\mathbf{q}_{2}
\end{array}\right]=\begin{bmatrix}\mathbf{F}\mathbf{q}_{1}\\
\mathbf{F}'_{21}\mathbf{q}_{1}+\mathbf{F}'_{22}\mathbf{q}_{2}
\end{bmatrix}\equiv\mathbf{0}\mod x^{\vec{\omega}}
\]
 give $\mathbf{F}\mathbf{q}_{1}\equiv\mathbf{0}\mod x^{\sigma}$. 
\end{proof}
The next lemma shows a $(\mathbf{F}',\vec{\omega},\vec{s'})$-basis
can be constructed from a $\left(\mathbf{F},\sigma,\vec{s}\right)$-basis.
This well-formed $(\mathbf{F}',\vec{\omega},\vec{s'})$-basis restricts
the elements of $\langle(\mathbf{F}',\vec{\omega},\vec{s'})\rangle$
to a simple form shown in \prettyref{cor:FtauBasisForm}. This in
turn helps to establish a close correspondence between a $(\mathbf{F}',\vec{\omega},\vec{s'})$-basis
and a $\left(\mathbf{F},\sigma,\vec{s}\right)$-basis in \prettyref{lem:2delta-1Basis},
\prettyref{lem:2deltaBasis}, and \prettyref{thm:mainTheorem}. 
\begin{lem}
\label{lem:FtoAbasis}If $\mathbf{P}$ is a $\left(\mathbf{F},\sigma,\vec{s}\right)$-basis,
then 
\begin{eqnarray*}
\bar{\mathbf{T}} & = & \left[\mathbf{B}\mathbf{P}~\begin{array}{|c}
\mathbf{0}_{n\times2m}\\
x^{2\delta}\mathbf{I}_{2m}
\end{array}\right]=\left[\begin{array}{r|cc}
\mathbf{P} & \mathbf{0}_{n\times m} & \mathbf{0}_{n\times m}\\
\hline x^{-\delta}\mathbf{F}_{0}\cdot\mathbf{P} & x^{2\delta}\mathbf{I}_{m} & \mathbf{0}_{m}\\
x^{-2\delta}\left(\mathbf{F}_{0}+\mathbf{F}_{1}x^{\delta}\right)\cdot\mathbf{P} & \mathbf{0}_{m} & x^{2\delta}\mathbf{I}_{m}
\end{array}\right]
\end{eqnarray*}
 is a $(\mathbf{F}',\vec{\omega},\vec{s'})$-basis.\end{lem}
\begin{proof}
By \prettyref{lem:qToBqOrder}, $\bar{\mathbf{T}}$ has order $(\mathbf{F}',\vec{\omega})$
and is $\vec{s'}$-column reduced since $\mathbf{P}$ dominates the
$\vec{s'}$-degrees of $\bar{\mathbf{T}}$ on the left side by \prettyref{lem:qToBqDegree}.
It remains to show that any $\bar{\mathbf{q}}\in\langle(\mathbf{F}',\vec{\omega},\vec{s'})\rangle$
is a linear combination of the columns of $\mathbf{\bar{\mathbf{T}}}$.

Let $\mathbf{q}$ be the top $n$ entries of $\bar{\mathbf{q}}$.
Then by \prettyref{lem:bqToqOrder}, $\mathbf{q}\in\left\langle \left(\mathbf{F},\sigma,\vec{s}\right)\right\rangle $,
hence is a linear combination of the columns of $\mathbf{P}$, that
is $\mathbf{q}=\mathbf{P}\mathbf{u}$ with $\mathbf{u}=\mathbf{P}^{-1}\mathbf{q}\in\mathbb{K}\left[x\right]^{n\times1}$.
Subtracting the contribution of $\mathbf{P}$ from $\bar{\mathbf{q}}$,
we get 
\[
\mathbf{q}'=\bar{\mathbf{q}}-\mathbf{B}\mathbf{P}\mathbf{u}=\bar{\mathbf{q}}-\mathbf{B}\mathbf{q}=\left[\begin{array}{c}
\mathbf{0}\\
\mathbf{v}
\end{array}\right],
\]
 which is still in $\langle(\mathbf{F}',\vec{\omega},\vec{s'})\rangle$,
that is, 
\[
\mathbf{F}'\mathbf{q}'=\begin{bmatrix}\mathbf{0}\\
\mathbf{I}_{2m}\mathbf{v}
\end{bmatrix}\equiv\mathbf{0}\mod x^{\vec{\omega}}.
\]
 This forces $\mathbf{v}$ to be a linear combination of the columns
of $x^{2\delta}\mathbf{I}_{2m}$, the bottom right submatrix of $\bar{\mathbf{T}}$.
Now $\bar{\mathbf{q}}=\bar{\mathbf{T}}\left[\mathbf{u}^{T},\mathbf{v}^{T}\right]^{T}$
as required.\end{proof}
\begin{cor}
\label{cor:FtauBasisForm}Let $\tau\in\mathbb{Z}$ be any degree bound
and $\mathbf{P}_{\tau}\in\mathbb{K}\left[x\right]^{n\times t}$ be
a $\left(\mathbf{F},\sigma,\vec{s}\right)_{\tau}$-basis. If $\bar{\mathbf{q}}\in\langle(\mathbf{F}',\vec{\omega},\vec{s'})\rangle_{\tau}$
and $\mathbf{q}$ is the top $n$ entries of $\bar{\mathbf{q}}$,
then $\bar{\mathbf{q}}$ must have the form 
\[
\bar{\mathbf{q}}=\mathbf{B}\mathbf{P}_{\tau}\mathbf{u}+x^{2\delta}\begin{bmatrix}\mathbf{0}\\
\mathbf{v}
\end{bmatrix}=\mathbf{B}\mathbf{q}+x^{2\delta}\begin{bmatrix}\mathbf{0}\\
\mathbf{v}
\end{bmatrix}
\]
 for some polynomial vector $\mathbf{u}\in\mathbb{K}\left[x\right]^{t\times1}$
and $\mathbf{v}\in\mathbb{K}\left[x\right]^{2m\times1}$. In particular,
if $\deg_{\vec{s'}}\bar{\mathbf{q}}<2\delta$, then $\bar{\mathbf{q}}=\mathbf{B}\mathbf{P}_{\tau}\mathbf{u}=\mathbf{B}\mathbf{q}$. \end{cor}
\begin{proof}
This follows directly from \prettyref{lem:FtoAbasis} with $\vec{s'}$-degrees
restricted to $\tau$.\end{proof}
\begin{lem}
\label{lem:2delta-1Basis}If $\bar{\mathbf{S}}^{\left(1\right)}$
is a $(\check{\mathbf{F}},\vec{\omega},\vec{s'})_{2\delta-1}$-basis,
then the matrix $\mathbf{S}^{\left(1\right)}$ consisting of its first
$n$ rows is a $\left(\mathbf{F},\sigma,\vec{s}\right)_{2\delta-1}$-basis.\end{lem}
\begin{proof}
By \prettyref{lem:bqToqOrder}, $\mathbf{S}^{\left(1\right)}$ has
order $\left(\mathbf{F},\sigma\right)$. By \prettyref{cor:linearCombinationOfFirstnRows},
any $\mathbf{q}\in\left\langle \left(\mathbf{F},\sigma,\vec{s}\right)\right\rangle _{2\delta-1}$
is a linear combination of $\mathbf{S}^{\left(1\right)}$. It remains
to show that $\mathbf{S}^{\left(1\right)}$ is $\vec{s}$-column reduced.

By \prettyref{cor:FtauBasisForm}, $\bar{\mathbf{S}}^{\left(1\right)}=\mathbf{B}\mathbf{S}^{\left(1\right)}$,
and by \prettyref{lem:bqToqOrder}, the columns of $\mathbf{S}^{\left(1\right)}$
are in $\left\langle \left(\mathbf{F},\sigma,\vec{s}\right)\right\rangle _{2\delta-1}$.
Thus, by \prettyref{lem:qToBqDegree}, $\mathbf{S}^{\left(1\right)}$
determines the $\vec{s'}$-column degrees of $\mathbf{S}^{\left(1\right)}$.
Therefore, $\bar{\mathbf{S}}^{\left(1\right)}$ being $\vec{s'}$-column
reduced implies that $\mathbf{S}^{\left(1\right)}$ is $\vec{s}$-column
reduced.\end{proof}
\begin{lem}
\label{lem:2deltaBasis} Let $\bar{\mathbf{S}}^{\left(12\right)}=[\bar{\mathbf{S}}^{\left(1\right)},\bar{\mathbf{S}}^{\left(2\right)}]$
be a $(\mathbf{F}',\vec{\omega},\vec{s'})_{2\delta}$-basis, with
$\deg_{\vec{s'}}\bar{\mathbf{S}}^{\left(1\right)}\le2\delta-1$ and
$\deg_{\vec{s'}}\bar{\mathbf{S}}^{\left(2\right)}=2\delta$, and $\mathbf{S}^{\left(12\right)},\mathbf{S}^{\left(1\right)},\mathbf{S}^{\left(2\right)}$
the first $n$ rows of $\bar{\mathbf{S}}^{\left(12\right)},\bar{\mathbf{S}}^{\left(1\right)},\bar{\mathbf{S}}^{\left(2\right)}$,
respectively. Let $I$ be the column rank profile (the lexicographically
smallest sequence of column indices that indicates a full column rank
submatrix) of $\mathbf{S}^{\left(12\right)}$. %which contains all columns of $\mathbf{S}^{\left(1\right)}$ by \prettyref{lem:2delta-1Basis}.
Then the submatrix\textbf{ $\mathbf{S}_{I}^{\left(12\right)}$ }comprised
of the columns of $\mathbf{S}^{\left(12\right)}$ indexed by $I$
is a $\left(\mathbf{F},\sigma,\vec{s}\right)_{2\delta}$-basis. \end{lem}
\begin{proof}
Consider doing $\vec{s}$-column reduction on $\mathbf{S}^{\left(12\right)}$.
From \prettyref{lem:2delta-1Basis}, we know that $\mathbf{S}^{\left(1\right)}$
is a $\left(\mathbf{F},\sigma,\vec{s}\right)_{2\delta-1}$-basis.
Therefore, only $\mathbf{S}^{\left(2\right)}$ may be $\vec{s}$-reduced.
If a column $\mathbf{c}$ of $\mathbf{S}^{\left(2\right)}$ can be
further $\vec{s}$-reduced, then it becomes an element of $\left\langle \left(\mathbf{F},\sigma,\vec{s}\right)\right\rangle _{2\delta-1}$,
which is generated by $\mathbf{S}^{\left(1\right)}$. Thus $\mathbf{c}$
must be reduced to zero by $\mathbf{S}^{\left(1\right)}$. The only
nonzero columns of $\mathbf{S}^{\left(12\right)}$ remaining after
$\vec{s}$-column reduction are therefore the columns that cannot
be $\vec{s}$-reduced. Hence $\mathbf{S}^{\left(12\right)}$ $\vec{s}$-reduces
to \textbf{$\mathbf{S}_{I}^{\left(12\right)}$}. In addition, \textbf{$\mathbf{S}_{I}^{\left(12\right)}$}
has order $\left(\mathbf{F},\sigma\right)$ as $\mathbf{S}^{\left(12\right)}$
has order $\left(\mathbf{F},\sigma\right)$ by \prettyref{lem:bqToqOrder}.
From \prettyref{cor:linearCombinationOfFirstnRows} any $\mathbf{q}\in\left\langle \left(\mathbf{F},\sigma,\vec{s}\right)\right\rangle _{2\delta}$
is a linear combination of $\mathbf{S}^{\left(12\right)}$ and hence
is also a linear combination of $\mathbf{S}_{I}^{\left(12\right)}$. 
\end{proof}
To extract $\mathbf{S}_{I}^{\left(12\right)}$ from $\mathbf{S}^{\left(12\right)}$,
note that doing $\vec{s}$-column reduction on $\mathbf{S}^{\left(12\right)}$
is equivalent to the more familiar problem of doing column reduction
on $x^{\vec{s}}\mathbf{S}^{\left(12\right)}$. As $\mathbf{S}^{\left(12\right)}$
$\vec{s}$-column reduces to \textbf{$\mathbf{S}_{I}^{\left(12\right)}$},
this corresponds to determining the column rank profile of the\emph{
leading column coefficient matrix }of\emph{ }\textbf{$x^{\vec{s}}\mathbf{S}^{\left(12\right)}$}%
\begin{comment}
\emph{ }$S^{\left(12\right)}=\lcoeff(x^{\vec{s}}\cdot\mathbf{S}^{\left(12\right)})$ 
\end{comment}
\emph{. }Recall that the leading column coefficient matrix of a matrix
$\mathbf{A}=\left[\mathbf{a}_{1},\dots,\mathbf{a}_{k}\right]$ used
for column reduction is\emph{ 
\begin{eqnarray*}
\lcoeff\left(\mathbf{A}\right) & = & \left[\lcoeff\left(\mathbf{a}_{1}\right),\dots,\lcoeff\left(\mathbf{a}_{k}\right)\right]\\
 & = & \left[\coeff\left(\mathbf{a}_{1},\deg\left(\mathbf{a}_{1}\right)\right),\dots,\coeff\left(\mathbf{a}_{k},\deg\left(\mathbf{a}_{k}\right)\right)\right].
\end{eqnarray*}
 }The column rank profile of $\lcoeff(x^{\vec{s}}\mathbf{S}^{\left(12\right)})$
can be determined by (the transposed version of) LSP factorization
\citep{IbarraMH82}, which factorizes $\lcoeff(x^{\vec{s}}\mathbf{S}^{\left(12\right)})=PSU$
as the product of a permutation matrix $P$, a matrix $S$ with its
nonzero columns forming a lower triangular submatrix, and an upper
triangular matrix $U$ with $1$'s on the diagonal. The indices, $I$,
of the nonzero columns of $S$ then give $\mathbf{S}_{I}^{\left(12\right)}$
in $\mathbf{S}^{\left(12\right)}$. 
\begin{thm}
\label{thm:mainTheorem}Let $\bar{\mathbf{S}}=[\bar{\mathbf{S}}^{\left(12\right)},\bar{\mathbf{S}}^{\left(3\right)}]$
be a $(\mathbf{F}',\vec{\omega},\vec{s'})$-basis, with $\deg_{\vec{s'}}\bar{\mathbf{S}}^{\left(12\right)}\le2\delta$
and $\deg_{\vec{s'}}\bar{\mathbf{S}}^{\left(3\right)}\ge2\delta+1$,
and $\mathbf{S},\mathbf{S}^{\left(12\right)},\mathbf{S}^{\left(3\right)}$
the first $n$ rows of $\bar{\mathbf{S}},\bar{\mathbf{S}}^{\left(12\right)},\bar{\mathbf{S}}^{\left(3\right)}$,
respectively. If $I$ is the column rank profile of $\mathbf{S}^{\left(12\right)}$,
then the submatrix \textbf{ $[\mathbf{S}_{I}^{\left(12\right)},\mathbf{S}^{\left(3\right)}]$
}of $\mathbf{S}$ is a $\left(\mathbf{F},\sigma,\vec{s}\right)$-basis. \end{thm}
\begin{proof}
By \prettyref{lem:bqToqOrder}, $\mathbf{S}$ has order $\left(\mathbf{F},\sigma\right)$,
and so $[\mathbf{S}_{I}^{\left(12\right)},\mathbf{S}^{\left(3\right)}]$
also has order $\left(\mathbf{F},\sigma\right)$. By \prettyref{cor:linearCombinationOfFirstnRows},
any $\mathbf{q}\in\left\langle \left(\mathbf{F},\sigma,\vec{s}\right)\right\rangle $
is a linear combination of the columns of $\mathbf{S}$, and so $\mathbf{q}$
is also a linear combination of the columns of $[\mathbf{S}_{I}^{\left(12\right)},\mathbf{S}^{\left(3\right)}]$.
It only remains to show that $[\mathbf{S}_{I}^{\left(12\right)},\mathbf{S}^{\left(3\right)}]$
is $\vec{s}$-column reduced.

Let $\mathbf{P}$ be a $\left(\mathbf{F},\sigma,\vec{s}\right)$-basis
and $\mathbf{\bar{\mathbf{T}}}$ be the $(\mathbf{F}',\vec{\omega},\vec{s'})$-basis
constructed from $\mathbf{P}$ as in \prettyref{lem:FtoAbasis}. Let
$\bar{\mathbf{T}}^{\left(3\right)}$ be the columns of $\bar{\mathbf{T}}$
with $\vec{s'}$-degrees greater than $2\delta,$ and $\mathbf{P}^{\left(3\right)}$
be the columns of $\mathbf{P}$ with $\vec{s}$-degrees greater than
$2\delta.$ Assume without loss of generality that $\mathbf{S},$
$\mathbf{P}$, and $\bar{\mathbf{T}}$ have their columns sorted according
to their $\vec{s}$-degrees and $\vec{s'}$-degrees, respectively.
Then $\deg_{\vec{s}}\mathbf{S}^{\left(3\right)}\le\deg_{\vec{s'}}\bar{\mathbf{S}}^{\left(3\right)}=\deg_{\vec{s'}}\bar{\mathbf{T}}^{\left(3\right)}=\deg_{\vec{s}}\mathbf{P}^{\left(3\right)}$.
Combining this with the $\vec{s}$-minimality of $\mathbf{S}_{I}^{\left(12\right)}$
from \prettyref{lem:2deltaBasis}, it follows that $\deg_{\vec{s}}[\mathbf{S}_{I}^{\left(12\right)},\mathbf{S}^{\left(3\right)}]\le\deg_{\vec{s}}\mathbf{P}$.
This combined with the fact that $[\mathbf{S}_{I}^{\left(12\right)},\mathbf{S}^{\left(3\right)}]$
still generates $\left\langle \left(\mathbf{F},\sigma,\vec{s}\right)\right\rangle $
implies that $\deg_{\vec{s}}[\mathbf{S}_{I}^{\left(12\right)},\mathbf{S}^{\left(3\right)}]=\deg_{\vec{s}}\mathbf{P}$.
Therefore, $[\mathbf{S}_{I}^{\left(12\right)},\mathbf{S}^{\left(3\right)}]$
is a $\left(\mathbf{F},\sigma,\vec{s}\right)$-basis. \end{proof}
\begin{cor}
\label{cor:extractingFbasisFromGbasis}Let $\bar{\mathbf{S}}$ be
a $(\mathbf{F}',\vec{\omega},\vec{s'})$-basis with its columns sorted
in an increasing order of their $\vec{s'}$ degrees, and $\mathbf{S}$
the first $n$ rows of $\bar{\mathbf{S}}$. If $J$ is the column
rank profile of $\lcoeff(x^{\vec{s}}\mathbf{S})$, then the submatrix
$\mathbf{S}_{J}$ of $\mathbf{S}$ indexed by $J$ is a $\left(\mathbf{F},\sigma,\vec{s}\right)$-basis.\end{cor}
\begin{proof}
This follows directly from \prettyref{thm:mainTheorem}. 
\end{proof}
This rank profile $J$ can be determined by LSP factorization on $\lcoeff(x^{\vec{s}}\cdot\mathbf{S}^{\left(12\right)})$.
%as discussed before.

\begin{example}
\label{exm:auxiliaryTransformation}For the problem in \prettyref{exm:StorjohannTransformation},
$\check{\mathbf{F}}$ is given by 
\begin{align*}
 & \left[{\begin{array}{cccccr}
x+x^{2}+x^{3}+x^{4}+x^{5}+x^{6} & \ 1+x+x^{5}+x^{6}+x^{7}\  & 1+x^{2}+x^{6}+x^{7} & \ 1+x+x^{3}+x^{7}\  & 0\  & 0\\
1+x+x^{2}+x^{3} & x^{3} & 1+x^{2}+x^{3} & x & 1\  & 0\\
1+x+x^{2} & x+x^{2}+x^{3} & 1+x+x^{2}+x^{3} & x^{3} & 0\  & 1
\end{array}}\right],
\end{align*}
 and a $\left(\mathbf{F}',\left[8,4,4\right],\vec{0}\right)$-basis
is given as 
\[
\left[\begin{array}{cccc|cc}
~1~ & x & 1 & x^{2} & x^{2}+x^{4} & 1+x^{2}+x^{3}+x^{4}\\
0 & 1 & x^{2}+x^{3} & 0 & x^{3} & 0\\
1 & 1+x & x & x^{3}+x^{4} & 0 & x+x^{2}+x^{3}\\
1 & 0 & 0 & 0 & 0 & 0\\
\hline 0 & 1 & 1+x^{2} & x^{2} & x^{2}+x^{3} & 1+x^{2}+x^{3}+x^{4}\\
0 & 1 & 1 & x^{2}+x^{4} & x^{2}+x^{3} & 1+x^{3}
\end{array}\right].
\]
 Column reduction on the top 4 rows gives the top left $4\times4$
submatrix, which is a \textbf{$(\mathbf{F},8,\vec{0})$}-basis. 
\end{example}
The following two lemmas verify Storjohann's result in the case of
degree parameter $\delta=\sigma/4$. More specifically, we show that
the matrix of the top $n$ rows of a $(\bar{\mathbf{F}},2\delta,\vec{s'})_{\delta-1}$-basis
is a $\left(\mathbf{F},\sigma,\vec{s}\right)_{\delta-1}$-basis, with
the transformed input matrix 
\begin{equation}
\bar{\mathbf{F}}=\left[\begin{array}{l|cc}
\mathbf{F}_{0}+\mathbf{F}_{1}x^{\delta} & \mathbf{0} & \mathbf{0}\\
\hline \mathbf{F}_{1}+\mathbf{F}_{2}x^{\delta} & \mathbf{I}_{m} & \mathbf{0}\\
\mathbf{F}_{2}+\mathbf{F}_{3}x^{\delta} & \mathbf{0} & ~\mathbf{I}_{m}
\end{array}\right]\equiv\mathbf{F}'\mod x^{2\delta}.\label{eq:storjohannTransformation4parts}
\end{equation}

\begin{lem}
\label{lem:A_delta-1Form}If $\bar{\mathbf{q}}\in\langle(\bar{\mathbf{F}},2\delta,\vec{s'})\rangle_{\delta-1}$
and $\mathbf{q}$ denotes the first $n$ entries of $\bar{\mathbf{q}}$,
then $\bar{\mathbf{q}}$ must have the form 
\[
\bar{\mathbf{q}}=\mathbf{B}\mathbf{q}=\left[\begin{array}{r}
\mathbf{q}\\
x^{-\delta}\mathbf{F}_{0}\cdot\mathbf{q}\\
x^{-2\delta}\left(\mathbf{F}_{0}+\mathbf{F}_{1}x^{\delta}\right)\cdot\mathbf{q}
\end{array}\right]
\]
 and $\mathbf{q}\in\left\langle \left(\mathbf{F},\sigma,\vec{s}\right)\right\rangle _{\delta-1}$.\end{lem}
\begin{proof}
Let $\mathbf{q},\mathbf{q}_{2},\mathbf{q}_{3}$ consist of the top
$n$ entries, middle $m$ entries, and bottom $m$ entries, respectively,
of $\mathbf{\bar{\mathbf{q}}}$ so that 
\begin{align}
\bar{\mathbf{F}}\bar{\mathbf{q}} & \equiv\left[\begin{array}{r}
\mathbf{F}_{0}\mathbf{q}+x^{\delta}\mathbf{F}_{1}\mathbf{q}\\
\mathbf{q}_{2}+\mathbf{F}_{1}\mathbf{q}+x^{\delta}\mathbf{F}_{2}\mathbf{q}\\
\mathbf{q}_{3}+\mathbf{F}_{2}\mathbf{q}+x^{\delta}\mathbf{F}_{3}\mathbf{q}
\end{array}\right]\equiv\mathbf{0}\mod x^{2\delta}.\label{eq:Aq}
\end{align}
 From the first and second block rows, we get $\mathbf{F}_{0}\mathbf{q}+x^{\delta}\mathbf{F}_{1}\mathbf{q}\equiv\mathbf{0}\mod x^{2\delta}$
and $\mathbf{q}_{2}+\mathbf{F}_{1}\mathbf{q}\equiv\mathbf{0}\mod x^{\delta}$,
which implies 
\begin{equation}
\mathbf{F}_{0}\mathbf{q}\equiv x^{\delta}\mathbf{q}_{2}\mod x^{2\delta}.\label{eq:q1q2}
\end{equation}
 Similarly, from the second and third rows, we get $\mathbf{q}_{2}+\mathbf{F}_{1}\mathbf{q}+x^{\delta}\mathbf{F}_{2}\mathbf{q}\equiv\mathbf{0}\mod x^{2\delta}$
and $\mathbf{q}_{3}+\mathbf{F}_{2}\mathbf{q}\equiv\mathbf{0}\mod x^{\delta}$,
which implies $\mathbf{q}_{2}+\mathbf{F}_{1}\mathbf{q}\equiv x^{\delta}\mathbf{q}_{3}\mod x^{2\delta}$.

Since $\deg\mathbf{q}\le\deg_{\vec{s}}\mathbf{q}=\delta-1$, we have
$\deg\mathbf{F}_{0}\mathbf{q}\le2\delta-2$, hence from \prettyref{eq:q1q2}
$\deg\mathbf{q}_{2}\le\delta-2$ and $\mathbf{q}_{2}x^{\delta}=\mathbf{F}_{0}\mathbf{q}$.
Similarly, $\deg\mathbf{q}_{3}\le\delta-2$ and $\mathbf{q}_{3}x^{2\delta}=\mathbf{q}_{2}x^{\delta}+\mathbf{F}_{1}\mathbf{q}x^{\delta}=\mathbf{F}_{0}\mathbf{q}+\mathbf{F}_{1}\mathbf{q}x^{\delta}$.
Substituting this to $\mathbf{F}\mathbf{q}=(\mathbf{F}_{0}\mathbf{q}+\mathbf{F}_{1}\mathbf{q}x^{\delta})+(\mathbf{F}_{2}\mathbf{q}x^{2\delta}+\mathbf{F}_{3}\mathbf{q}x^{3\delta})$,
we get $\mathbf{F}\mathbf{q}=\mathbf{q}_{3}x^{2\delta}+(\mathbf{F}_{2}\mathbf{q}x^{2\delta}+\mathbf{F}_{3}\mathbf{q}x^{3\delta})\equiv\mathbf{0}\mod x^{4\delta}$
using the bottom block row of \prettyref{eq:Aq}.\end{proof}
\begin{lem}
\label{lem:delta-1Basis} If $\bar{\mathbf{S}}_{\delta-1}$ is a $(\bar{\mathbf{F}},2\delta,\vec{s'})_{\delta-1}$-basis,
then the matrix of its first $n$ rows, $\mathbf{S}_{\delta-1}$,
is a $\left(\mathbf{F},\sigma,\vec{s}\right)_{\delta-1}$-basis.\end{lem}
\begin{proof}
By \prettyref{lem:A_delta-1Form}, $\mathbf{S}_{\delta-1}$ has order
$\left(\mathbf{F},\sigma\right)$. Following Lemmas \ref{lem:qToBqOrder}
and \ref{lem:qToBqDegree} and Corollaries \ref{cor:sToBs} and \ref{cor:linearCombinationOfFirstnRows}
(replacing $\vec{\omega}$ by $2\delta$), we conclude that any $\mathbf{q}\in\left\langle \left(\mathbf{F},\sigma,\vec{s}\right)\right\rangle _{\delta-1}$
is a linear combination of the columns of $\mathbf{S}_{\delta-1}$.
In addition, since $\bar{\mathbf{S}}_{\delta-1}=\mathbf{B}\mathbf{S}_{\delta-1}$
by \prettyref{lem:A_delta-1Form}, and the columns of $\mathbf{S}_{\delta-1}$
are in $\left\langle \left(\mathbf{F},\sigma,\vec{s}\right)\right\rangle _{\delta-1}$,
it follows from \prettyref{lem:qToBqDegree} that $\mathbf{S}_{\delta-1}$
determines the $\vec{s'}$-column degrees of $\bar{\mathbf{S}}_{\delta-1}$.
Hence $\bar{\mathbf{S}}_{\delta-1}$ $\vec{s'}$-column reduced implies
that $\mathbf{S}_{\delta-1}$ is $\vec{s}$-column reduced. 
\end{proof}

\subsection{\label{sub:generalTransform}More General Results}

Let us now consider an immediate extension of the results in the previous
subsection. Suppose that instead of a $\left(\mathbf{F},\sigma,\vec{s}\right)$-basis
we now want to compute a $(\bar{\mathbf{F}}^{\left(i\right)},2\delta^{\left(i\right)},\vec{s}^{\left(i\right)})$-basis
with a Storjohann transformed input matrix 
\[
\bar{\mathbf{F}}^{\left(i\right)}=\left[\begin{array}{c|cccc}
\mathbf{F}_{0}+\mathbf{F}_{1}x^{\delta^{\left(i\right)}} & \mathbf{0}_{m} & \cdots & \mathbf{\cdots} & \mathbf{0}_{m}\\
\hline \mathbf{F}_{1}+\mathbf{F}_{2}x^{\delta^{\left(i\right)}} & \mathbf{I}_{m}\\
\mathbf{F}_{2}+\mathbf{F}_{3}x^{\delta^{\left(i\right)}} &  & \mathbf{I}_{m}\\
\vdots &  &  & \ddots\\
\mathbf{F}_{l^{\left(i\right)}-1}+\mathbf{F}_{l^{\left(i\right)}}x^{\delta^{\left(i\right)}} &  &  &  & \mathbf{I}_{m}
\end{array}\right]_{ml^{(i)}\times(n+m(l^{(i)}-1))}
\]
 made with degree parameter $\delta^{\left(i\right)}=2^{i}d$ for
some integer $i$ between $2$%
\begin{comment}
the base case is $i=1$ and problem is not to be subdivided 
\end{comment}
{} and $\log\left(\sigma/d\right)-1$, and a shift $\vec{s}^{\left(i\right)}=[\vec{s},0,\dots,0]$
(with $m(l^{\left(i\right)}-1)$ 0's), where $l^{\left(i\right)}=\sigma/\delta^{\left(i\right)}-1$
is the number of block rows%
\footnote{Recall that $d=m\sigma/n$ is the average degree of the input matrix
$\mathbf{F}$ if we treat $\mathbf{F}$ as a square $n\times n$ matrix.
Also, $i$ starts at $2$ because $i=1$ is our base case in the computation
of an order basis, which may become more clear in the next section.
The base case can be computed efficiently using the method of Giorgi
et al. \citeyearpar{Giorgi2003} directly and does not require the
transformation discussed in this section.%
}. To apply a transformation analogous to \prettyref{eq:extendedSTransformTop},
we write each $\mathbf{F}_{j}=\mathbf{F}_{j0}+\mathbf{F}_{j1}\delta^{\left(i-1\right)}$
and set 
\begin{equation}
\mathbf{F}'^{\left(i\right)}=\left[\begin{array}{l|c}
\mathbf{F}_{00}+\mathbf{F}_{01}x^{\delta^{\left(i-1\right)}}+\mathbf{F}_{10}x^{2\delta^{\left(i-1\right)}}+\mathbf{F}_{11}x^{3\delta^{\left(i-1\right)}} & ~~\mathbf{0}~~\\
\hline \mathbf{F}_{01}+\mathbf{F}_{10}x^{\delta^{\left(i-1\right)}}\\
\mathbf{F}_{10}+\mathbf{F}_{11}x^{\delta^{\left(i-1\right)}}+\mathbf{F}_{20}x^{2\delta^{\left(i-1\right)}}+\mathbf{F}_{21}x^{3\delta^{\left(i-1\right)}}\\
\mathbf{F}_{11}+\mathbf{F}_{20}x^{\delta^{\left(i-1\right)}}\\
\vdots & ~~\mathbf{I}~~\\
\mathbf{F}_{\left(l^{\left(i\right)}-1\right)0}+\mathbf{F}_{\left(l^{\left(i\right)}-1\right)1}x^{\delta^{\left(i-1\right)}}+\mathbf{F}_{l^{\left(i\right)}0}x^{2\delta^{\left(i-1\right)}}+\mathbf{F}_{l^{\left(i\right)}1}x^{3\delta^{\left(i-1\right)}}\\
\mathbf{F}_{\left(l^{\left(i\right)}-1\right)1}+\mathbf{F}_{l^{\left(i\right)}0}x^{\delta^{\left(i-1\right)}}\\
\mathbf{F}_{l^{\left(i\right)}0}+\mathbf{F}_{l^{\left(i\right)}1}x^{\delta^{\left(i-1\right)}}
\end{array}\right],\label{eq:extendedStorjohannTransform}
\end{equation}
 %
\begin{comment}
This is not ideal, but no better idea. 
\end{comment}
{} and $\vec{\omega}^{\left(i\right)}=\left[\left[[2\delta^{\left(i\right)}]^{m},[\delta^{\left(i\right)}]^{m}\right]^{l^{\left(i\right)}},[\delta^{\left(i\right)}]^{m}\right]$,
where $\left[\circ\right]^{k}$ represents $\circ$ repeated $k$
times%
\begin{comment}
Not sure if using this notation is a good thing to do, but it saves
space and makes presentation easier 
\end{comment}
. The order entries $2\delta^{\left(i\right)}$, $\delta^{\left(i\right)}$
in $\vec{\omega}^{\left(i\right)}$ correspond to the degree $2\delta^{\left(i\right)}-1$,
degree $\delta^{\left(i\right)}-1$ rows in $\mathbf{F}'^{\left(i\right)}$
respectively. Let 
\[
\mathbf{E}^{\left(i\right)}=\left[\begin{array}{c||cc|cc|cc|cc||cc}
\mathbf{I}_{n} &  &  &  &  &  &  &  &  & \mathbf{0}_{n\times m} & \mathbf{0}_{n\times m}\\
\hline\hline  & \mathbf{0}_{m} & \mathbf{I}_{m} &  &  &  &  &  &  & \ \\
\hline  &  &  & \mathbf{0}_{m} & \mathbf{I}_{m} &  &  &  &  & \ \\
\hline  &  &  &  &  & \ddots & \ddots &  &  & \ \\
\hline  &  &  &  &  &  &  & \mathbf{0}_{m} & \mathbf{I}_{m}
\end{array}\right]
\]
 with $l^{\left(i\right)}-1$ blocks of $\left[\mathbf{0}_{m},\mathbf{I}_{m}\right]$
and hence an overall dimension of $(n+m(l^{\left(i\right)}-1))\times(n+m(l^{\left(i-1\right)}-1))$.
Thus $\mathbf{E}^{\left(i\right)}\mathbf{M}$ picks out from $\mathbf{M}$
the first $n$ rows and the even block rows from the remaining rows
except the last block row for a matrix $\mathbf{M}$ with $n+m(l^{\left(i-1\right)}-1)$
rows. In particular, if $i=\log\left(n/m\right)-1$, then $(\mathbf{F}'^{\left(i\right)},\vec{\omega}^{\left(i\right)},\vec{s}^{\left(i-1\right)})=(\mathbf{F}',\vec{\omega},\vec{s'})$,
which for $d=m\sigma/n$ gives the problem considered earlier in \prettyref{sub:particularCase},
and $\mathbf{E}^{\left(i\right)}=\left[\mathbf{I}_{n},\mathbf{0}_{n\times m},\mathbf{0}_{n\times m}\right]$
is used to select the top $n$ rows of a $(\mathbf{F}',\vec{\omega},\vec{s'})$-basis
for a $\left(\mathbf{F},\sigma,\vec{s}\right)$-basis to be extracted.

We can now state the analog of \prettyref{cor:extractingFbasisFromGbasis}: 
\begin{thm}
\label{thm:extractingOrderBasis}Let $\mathbf{S}'^{\left(i\right)}$
be a $(\mathbf{F}'^{\left(i\right)},\vec{\omega}^{\left(i\right)},\vec{s}^{\left(i-1\right)})$-basis
with its columns sorted in an increasing order of their $\vec{s}^{\left(i-1\right)}$
degrees. Let $\hat{\mathbf{S}}^{\left(i\right)}=\mathbf{E}^{\left(i\right)}\mathbf{S}'^{\left(i\right)}$.
Let $J$ be the column rank profile of $\lcoeff(x^{\vec{s}^{\left(i\right)}}\hat{\mathbf{S}}^{\left(i\right)})$.
Then $\hat{\mathbf{S}}_{J}^{\left(i\right)}$ is a $(\bar{\mathbf{F}}^{\left(i\right)},2\delta^{\left(i\right)},\vec{s}^{\left(i\right)})$-basis.\end{thm}
\begin{proof}
One can follow the same arguments used before from \prettyref{lem:qToBqOrder}
to \prettyref{cor:extractingFbasisFromGbasis}. Alternatively, this
can be derived from \prettyref{cor:extractingFbasisFromGbasis} by
noticing the redundant block rows that can be disregarded after applying
transformation \prettyref{eq:extendedSTransformTop} directly to the
input matrix $\bar{\mathbf{F}}^{\left(i\right)}$. 
\end{proof}
\prettyref{lem:delta-1Basis} can also be extended in the same way
to capture Storjohann's transformation with more general degree parameters: 
\begin{lem}
\label{lem:linkStorjohanTransform}If $\bar{\mathbf{P}}_{1}^{\left(i-1\right)}$
is a $(\bar{\mathbf{F}}^{\left(i-1\right)},2\delta^{\left(i-1\right)},\vec{s}^{\left(i-1\right)})_{\delta^{\left(i-1\right)}-1}$-basis,
then $\mathbf{E}^{\left(i\right)}\bar{\mathbf{P}}_{1}^{\left(i-1\right)}$
is a $(\bar{\mathbf{F}}^{\left(i\right)},2\delta^{\left(i\right)},\vec{s}^{\left(i\right)})_{\delta^{\left(i-1\right)}-1}$-basis
and the matrix of the top $n$ rows of $\bar{\mathbf{P}}_{1}^{\left(i-1\right)}$
is a $(\mathbf{F},\sigma,\vec{s})_{\delta^{\left(i-1\right)}-1}$-basis.\end{lem}
\begin{proof}
Again, this can be justified as done in \prettyref{lem:delta-1Basis}.
Alternatively, one can apply Storjohann's transformation with degree
parameter $\delta^{\left(i-1\right)}$ to $\bar{\mathbf{F}}^{\left(i\right)}$
as in \prettyref{eq:storjohannTransformation4parts}. The lemma then
follows from \prettyref{lem:delta-1Basis} after noticing the redundant
block rows that can be disregarded. 
\end{proof}
Notice that if $i=\log\left(n/m\right)-1,$ then \prettyref{thm:extractingOrderBasis}
and \prettyref{lem:linkStorjohanTransform} specialize to \prettyref{cor:extractingFbasisFromGbasis}
and \prettyref{lem:delta-1Basis}. 



\section{Computation of Order Bases}

\label{sec:Order-Basis-Computation}

In this section, we establish a link between two different Storjohann
transformed problems by dividing the transformed problem from the
previous section into two subproblems and then simplifying the second
subproblem. This leads to a recursive method for computing order bases.
We also present an equivalent, iterative method for computing order
bases. The iterative approach is usually more efficient in practice,
as it uses just $O(1)$ iterations in the generic case.


\subsection{\label{sub:Dividing-to-Subproblems}Dividing into Subproblems }

In \prettyref{sec:transform} we have shown that the problem of computing
a $\left(\mathbf{F},\sigma,\vec{s}\right)$-basis can be converted
to the problem of computing a $(\mathbf{F}',\vec{\omega},\vec{s'})$-basis
and, more generally, that the computation of a $(\bar{\mathbf{F}}^{\left(i\right)},2\delta^{\left(i\right)},\vec{s}^{\left(i\right)})$-basis,
a Storjohann transformed problem with degree parameter $\delta^{\left(i\right)}$,
can be converted to the problem of computing a $(\mathbf{F}'^{\left(i\right)},\vec{\omega}^{\left(i\right)},\vec{s}^{\left(i-1\right)})$-basis.
We now consider dividing the new converted problem into two subproblems.

The first subproblem is to compute a $(\mathbf{F}'^{\left(i\right)},2\delta^{\left(i-1\right)},\vec{s}^{\left(i-1\right)})$-basis
or equivalently a $(\bar{\mathbf{F}}^{\left(i-1\right)},2\delta^{\left(i-1\right)},\vec{s}^{\left(i-1\right)})$-basis
$\bar{\mathbf{P}}^{\left(i-1\right)}$, a Storjohann transformed problem
with degree parameter $\delta^{\left(i-1\right)}$. The second subproblem
is computing a $(\mathbf{F}'^{\left(i\right)}\bar{\mathbf{P}}^{\left(i-1\right)},\vec{\omega}^{\left(i\right)},\vec{t}^{\left(i-1\right)})$-basis
$\bar{\mathbf{Q}}^{\left(i\right)}$ using the residual $\mathbf{F}'^{\left(i\right)}\bar{\mathbf{P}}^{\left(i-1\right)}$
from the first subproblem along with a degree shift $\vec{t}^{\left(i-1\right)}=\deg_{\vec{s}^{\left(i-1\right)}}\bar{\mathbf{P}}^{\left(i-1\right)}$.
From Theorem 5.1 in \citep{BL1997} we then know that the product
$\bar{\mathbf{P}}^{\left(i-1\right)}\bar{\mathbf{Q}}^{\left(i\right)}$
is a $(\mathbf{F}'^{\left(i\right)},\vec{\omega}^{\left(i\right)},\vec{s}^{\left(i-1\right)})$-basis
and $\deg_{\vec{s}^{\left(i-1\right)}}\bar{\mathbf{P}}^{\left(i-1\right)}\bar{\mathbf{Q}}^{\left(i\right)}=\deg_{\vec{t}^{\left(i-1\right)}}\bar{\mathbf{Q}}^{\left(i\right)}$.
For completeness, we state a version of this theorem specialized for
our needs below and provide a simpler proof.
\begin{thm}
\label{thm:combineOrderBases}For an input matrix $\mathbf{F}\in\mathbb{K}\left[x\right]^{m\times n}$,
an order vector $\vec{\sigma}$, and a shift vector $\vec{s}$, if
$\mathbf{P}$ is a $\left(\mathbf{F},\vec{\sigma},\vec{s}\right)$-basis
with $\vec{s}$-column degrees $\vec{t}$, and $\mathbf{Q}$ is a
$(\mathbf{F}\mathbf{P},\vec{\tau},\vec{t})$ -basis with $\vec{t}$-column
degrees $\vec{u}$, where $\vec{\tau}\ge\vec{\sigma}$ component-wise,
then $\mathbf{P}\mathbf{Q}$ is a $(\mathbf{F},\vec{\tau},\vec{s})$-basis
with $\vec{s}$-column degrees $\vec{u}$.\end{thm}
\begin{proof}
It is clear that $\mathbf{P}\mathbf{Q}$ has order $(\mathbf{F},\vec{\tau})$.
We now show that $\mathbf{P}\mathbf{Q}$ is $\vec{s}$-column reduced
and has $\vec{s}$-column degrees $\vec{u}$, or equivalently, $x^{\vec{s}}\mathbf{P}\mathbf{Q}$
is column reduced and has column degrees $\vec{u}$. Notice that $x^{\vec{s}}\mathbf{P}$
has column degrees $\vec{t}$ and a full rank leading column coefficient
matrix $P$. Hence $x^{\vec{s}}\mathbf{P}x^{-\vec{t}}$ has column
degrees $\left[0,\dots0\right]$. Similarly, $x^{\vec{t}}\mathbf{Q}x^{-\vec{u}}$
has column degrees $[0,\dots,0]$ and a full rank leading column coefficient
matrix $Q$. Therefore, $x^{\vec{s}}\mathbf{P}x^{-\vec{t}}x^{\vec{t}}\mathbf{Q}x^{-\vec{u}}=x^{\vec{s}}\mathbf{P}\mathbf{Q}x^{-\vec{u}}$
has column degrees $[0,\dots,0]$ and a full rank leading column coefficient
matrix $PQ$, implying that $x^{\vec{s}}\mathbf{P}\mathbf{Q}$ is
column reduced and that $x^{\vec{s}}\mathbf{P}\mathbf{Q}$ has column
degrees $\vec{u}$, or equivalently, the $\vec{s}$-column degrees
of $\mathbf{PQ}$ is $\vec{u}$.

It remains to show that any $\mathbf{t}\in\left\langle \left(\mathbf{F},\vec{\tau}\right)\right\rangle $
is generated by the columns of $\mathbf{PQ}$. Since $\mathbf{t}\in\left\langle \left(\mathbf{F},\vec{\sigma}\right)\right\rangle $,
it is generated by the $\left(\mathbf{F},\vec{\sigma}\right)$-basis
$\mathbf{P}$, that is, $\mathbf{t}=\mathbf{P}\mathbf{a}$ for $\mathbf{a}=\mathbf{P}^{-1}\mathbf{t}\in\mathbb{K}\left[x\right]^{n}$.
Also, $\mathbf{t}\in\left\langle \left(\mathbf{F},\vec{\tau}\right)\right\rangle $
implies that $\mathbf{a}\in\left\langle \left(\mathbf{FP},\vec{\tau}\right)\right\rangle $
since $\mathbf{F}\mathbf{P}\mathbf{a}=\mathbf{F}\mathbf{t}\equiv0\mod x^{\vec{\tau}}$.
It follows that $\mathbf{a}=\mathbf{Q}\mathbf{b}$ for $\mathbf{b}=\mathbf{Q}^{-1}\mathbf{a}\in\mathbb{K}\left[x\right]^{n}$.
Therefore, $\mathbf{a}=\mathbf{P}^{-1}\mathbf{t}=\mathbf{Q}\mathbf{b}$,
which gives $\mathbf{t}=\mathbf{P}\mathbf{Q}\mathbf{b}$.\end{proof}
\begin{example}
\label{exm:subproblems} Let us continue with \prettyref{exm:StorjohannTransformation}
and \prettyref{exm:auxiliaryTransformation} in order to compute a
$\left(\mathbf{F},8,\vec{0}\right)$-basis (or equivalently a $(\bar{\mathbf{F}}^{\left(2\right)},8,\vec{0})$-basis).
This can be determined by computing a $(\mathbf{F}'^{\left(2\right)},[8,4,4],\vec{0})$-basis
as shown in \prettyref{exm:auxiliaryTransformation} where we have
$\mathbf{F}'^{\left(2\right)}=\mathbf{F}'$. Computing a $(\mathbf{F}'^{\left(2\right)},[8,4,4],\vec{0})$-basis
can be divided into two subproblems. The first subproblem is computing
a $(\bar{\mathbf{F}}^{\left(1\right)},4,\vec{0})$-basis $\bar{\mathbf{P}}^{\left(1\right)}$,
the Storjohann partial linearized problem in \prettyref{exm:StorjohannTransformation}.
The residual 
\[
\mathbf{F}'^{(2)}\bar{\mathbf{P}}^{(1)}=\left[{\begin{array}{rcccccccccc}
0 & \  & x^{8} & \  & x^{6}+x^{9} & \  & x^{4}+x^{6}+x^{9} & \  & x^{6}+x^{8}+x^{9}+x^{10} & \  & x^{5}+x^{8}\\
0 &  & 0 &  & x^{5} &  & x^{4}+x^{6} &  & x^{4}+x^{6} &  & x^{5}+x^{6}\\
0 &  & x^{4} &  & x^{5} &  & x^{5} &  & x^{4}+x^{5}+x^{6} &  & x^{4}
\end{array}}\right]
\]
 is then used as the input matrix for the second subproblem. The shift
for the second subproblem $\vec{t}^{(1)}=[0,1,2,3,3,3]$ is the list
of column degrees of $\bar{\mathbf{P}}^{(1)}$ and so the second subproblem
is to compute a $(\mathbf{F}'^{(2)}\bar{\mathbf{P}}^{(1)},\left[8,4,4\right],[0,1,2,3,3,3])$-basis,
which is 
\begin{equation}
\bar{\mathbf{Q}}^{(2)}=\left[{\begin{array}{cccccc}
~~1~ & ~~0~ & ~0~ & ~0~ & ~0~ & ~0~\\
0 & 1 & 0 & 0 & 0 & 0\\
0 & 0 & 1 & ~x^{2}~ & ~x~ & ~1~\\
0 & 0 & 0 & 0 & x & 0\\
0 & 0 & 1 & 0 & 0 & 0\\
0 & 0 & 0 & 0 & 1 & ~x~
\end{array}}\right].\label{eq:Qbar2}
\end{equation}
 Then $\bar{\mathbf{P}}^{(1)}\bar{\mathbf{Q}}^{(2)}$ gives the $(\mathbf{F}'^{\left(2\right)},[8,4,4],\vec{0})$-basis
shown in \prettyref{exm:auxiliaryTransformation}. 
\end{example}
We now show that the dimension of the second subproblem can be significantly
reduced. First, the row dimension can be reduced by over a half. Let
$\hat{\mathbf{P}}^{\left(i-1\right)}=\mathbf{E}^{\left(i\right)}\bar{\mathbf{P}}^{\left(i-1\right)}$. 
\begin{lem}
\label{lem:simplifySecondSubproblem}A $(\bar{\mathbf{F}}^{\left(i\right)}\hat{\mathbf{P}}^{\left(i-1\right)},2\delta^{\left(i\right)},\vec{t}^{\left(i-1\right)})$-basis
is a $(\mathbf{F}'^{\left(i\right)}\bar{\mathbf{P}}^{\left(i-1\right)},\vec{\omega}^{\left(i\right)},\vec{t}^{\left(i-1\right)})$-basis.\end{lem}
\begin{proof}
This follows because $\bar{\mathbf{F}}^{\left(i\right)}\hat{\mathbf{P}}^{\left(i-1\right)}$
is a submatrix of $\mathbf{F}'^{\left(i\right)}\bar{\mathbf{P}}^{\left(i-1\right)}$
after removing rows which already have the correct order $2\delta^{\left(i-1\right)}$. 
\end{proof}
The column dimension of the second subproblem can be reduced by disregarding
the $(\bar{\mathbf{F}}^{\left(i\right)},2\delta^{\left(i\right)},\vec{s}^{\left(i\right)})_{\delta^{\left(i-1\right)}-1}$-basis
which has already been computed. More specifically, after sorting
the columns of $\bar{\mathbf{P}}^{\left(i-1\right)}$ in an increasing
order of their $\vec{s}^{\left(i-1\right)}$-degrees, let $[\bar{\mathbf{P}}_{1}^{\left(i-1\right)},\bar{\mathbf{P}}_{2}^{\left(i-1\right)}]=\bar{\mathbf{P}}^{\left(i-1\right)}$
be such that $\deg_{\vec{s}^{\left(i-1\right)}}\bar{\mathbf{P}}_{1}^{\left(i-1\right)}\le\delta^{\left(i-1\right)}-1$
and $\deg_{\vec{s}^{\left(i-1\right)}}\bar{\mathbf{P}}_{2}^{\left(i-1\right)}\ge\delta^{\left(i-1\right)}$.
Then $\hat{\mathbf{P}}_{1}^{\left(i-1\right)}=\mathbf{E}^{\left(i\right)}\bar{\mathbf{P}}_{1}^{\left(i-1\right)}$
is a $(\bar{\mathbf{F}}^{\left(i\right)},2\delta^{\left(i\right)},\vec{s}^{\left(i\right)})_{\delta^{\left(i-1\right)}-1}$-basis
by \prettyref{lem:linkStorjohanTransform}. In the second subproblem,
the remaining basis elements of a $(\bar{\mathbf{F}}^{\left(i\right)},2\delta^{\left(i\right)},\vec{s}^{\left(i\right)})$-basis
can then be computed without $\bar{\mathbf{P}}_{1}^{\left(i-1\right)}$.

Let $\hat{\mathbf{P}}_{2}^{\left(i-1\right)}=\mathbf{E}^{\left(i\right)}\bar{\mathbf{P}}_{2}^{\left(i-1\right)}$,
$\vec{b}^{\left(i-1\right)}=\deg_{\vec{s}^{\left(i-1\right)}}\bar{\mathbf{P}}_{2}^{\left(i-1\right)}$,
$\bar{\mathbf{Q}}_{2}^{\left(i\right)}$ be a $(\bar{\mathbf{F}}^{\left(i\right)}\hat{\mathbf{P}}_{2}^{\left(i-1\right)},2\delta^{\left(i\right)},\vec{b}^{\left(i-1\right)})$-basis
(or equivalently a $(\mathbf{F}'^{\left(i\right)}\bar{\mathbf{P}}_{2}^{\left(i-1\right)},\vec{\omega}^{\left(i\right)},\vec{b}^{\left(i-1\right)})$-basis),
and $k^{\left(i-1\right)}$ be the column dimension of $\bar{\mathbf{P}}_{1}^{\left(i-1\right)}$.
We then have the following result. 
\begin{lem}
\label{lem:disregardComputedBasisElements} The matrix 
\[
\bar{\mathbf{Q}}^{\left(i\right)}=\left[\begin{array}{cc}
\mathbf{I}_{k^{\left(i-1\right)}}\\
 & \bar{\mathbf{Q}}_{2}^{\left(i\right)}
\end{array}\right]
\]
 is a $(\bar{\mathbf{F}}^{\left(i\right)}\hat{\mathbf{P}}^{\left(i-1\right)},2\delta^{\left(i\right)},\vec{t}^{\left(i-1\right)})$-basis
(equivalently a $(\mathbf{F}'^{\left(i\right)}\bar{\mathbf{P}}^{\left(i-1\right)},\vec{\omega}^{\left(i\right)},\vec{t}^{\left(i-1\right)})$-basis).\end{lem}
\begin{proof}
First note that $\bar{\mathbf{Q}}^{\left(i\right)}$ has order $(\bar{\mathbf{F}}^{\left(i\right)}\hat{\mathbf{P}}^{\left(i-1\right)},2\delta^{\left(i\right)})$
as 
\[
\bar{\mathbf{F}}^{\left(i\right)}\hat{\mathbf{P}}^{\left(i-1\right)}\bar{\mathbf{Q}}^{\left(i\right)}=[\bar{\mathbf{F}}^{\left(i\right)}\hat{\mathbf{P}}_{1}^{\left(i-1\right)},\bar{\mathbf{F}}^{\left(i\right)}\hat{\mathbf{P}}_{2}^{\left(i-1\right)}\bar{\mathbf{Q}}_{2}^{\left(i\right)}]\equiv0\mod x^{2\delta^{\left(i\right)}}.
\]
 In addition, $\bar{\mathbf{Q}}^{\left(i\right)}$ has minimal $\vec{t}^{\left(i-1\right)}$
degrees as $\bar{\mathbf{Q}}_{2}^{\left(i\right)}$ is $\vec{b}$-minimal.
Hence, by \prettyref{lem:orderBasisEquivalence}, $\bar{\mathbf{Q}}^{\left(i\right)}$
is a $(\bar{\mathbf{F}}^{\left(i\right)}\cdot\hat{\mathbf{P}}^{\left(i-1\right)},2\delta^{\left(i\right)},\vec{t}^{\left(i-1\right)})$-basis. 
\end{proof}
\begin{comment}
Alternatively, one can also argue that since $\hat{\mathbf{P}}_{1}^{\left(i-1\right)}$
already has order $(\bar{\mathbf{F}}^{\left(i\right)},2\delta^{\left(i\right)})$,
it cannot contribute in any way to and cannot be affected in any way
by the computations of a $(\bar{\mathbf{F}}^{\left(i\right)}\hat{\mathbf{P}}^{\left(i-1\right)},2\delta^{\left(i\right)},\vec{b}^{\left(i-1\right)})$-basis,
hence it is sufficient to just use $\bar{\mathbf{F}}^{\left(i\right)}\hat{\mathbf{P}}_{2}^{\left(i-1\right)}$
to compute a $(\bar{\mathbf{F}}^{\left(i\right)}\cdot\hat{\mathbf{P}}_{2}^{\left(i-1\right)},2\delta^{\left(i\right)},\vec{b}^{\left(i-1\right)})$-basis. 
\end{comment}


\prettyref{lem:disregardComputedBasisElements} immediately leads
to the following. 
\begin{lem}
\label{lem:computationAtTopLevel}Let $\hat{\mathbf{S}}=[\hat{\mathbf{P}}_{1}^{\left(i-1\right)},\hat{\mathbf{P}}_{2}^{\left(i-1\right)}\bar{\mathbf{Q}}_{2}^{\left(i\right)}]$,
and let $I$ be the column rank profile of $\lcoeff(x^{\vec{s}^{\left(i\right)}}\hat{\mathbf{S}})$.
Then $\hat{\mathbf{S}}_{I}$ is a $(\bar{\mathbf{F}}^{\left(i\right)},2\delta^{\left(i\right)},\vec{s}^{\left(i\right)})$-basis.\end{lem}
\begin{proof}
From \prettyref{lem:disregardComputedBasisElements}, $\bar{\mathbf{Q}}^{\left(i\right)}$
is a $(\mathbf{F}'^{\left(i\right)}\bar{\mathbf{P}}^{\left(i-1\right)},\vec{\omega}^{\left(i\right)},\vec{t}^{\left(i-1\right)})$-basis
and hence $\bar{\mathbf{P}}^{\left(i-1\right)}\bar{\mathbf{Q}}^{\left(i\right)}$
is a $(\mathbf{F}'^{\left(i\right)},\vec{\omega}^{\left(i\right)},\vec{s}^{\left(i-1\right)})$-basis.
Since $[\hat{\mathbf{P}}_{1}^{\left(i-1\right)},\hat{\mathbf{P}}_{2}^{\left(i-1\right)}\bar{\mathbf{Q}}_{2}^{\left(i\right)}]=\mathbf{E}^{\left(i\right)}\bar{\mathbf{P}}^{\left(i-1\right)}\bar{\mathbf{Q}}^{\left(i\right)}$,
the result follows from \prettyref{thm:extractingOrderBasis}. %
\begin{comment}
$=\mathbf{E}^{\left(i\right)}[\bar{\mathbf{P}}_{1}^{\left(i-1\right)},\bar{\mathbf{P}}_{2}^{\left(i-1\right)}\bar{\mathbf{Q}}_{2}^{\left(i\right)}]$ 
\end{comment}
\end{proof}
\begin{example}
Continuing with \prettyref{exm:StorjohannTransformation}, \prettyref{exm:auxiliaryTransformation},
and \prettyref{exm:subproblems}, %
\begin{comment}
notice that after the first subproblem, the second subproblem of computing
$\bar{\mathbf{Q}}^{(2)}$ in \prettyref{exm:subproblems} is really
a smaller problem of computing the lower right $4\times4$ submatrix
$\bar{\mathbf{Q}}_{2}^{(2)}$, which is a $(\bar{\mathbf{F}}^{\left(2\right)}\hat{\mathbf{P}}_{2}^{\left(1\right)},8,\vec{b}^{\left(1\right)})$-basis
(or equivalently a $(\mathbf{F}'^{\left(2\right)}\bar{\mathbf{P}}_{2}^{\left(1\right)},[8,4,4],\vec{b}^{\left(1\right)})$-basis),
where $\bar{\mathbf{P}}_{2}^{\left(1\right)}$ is the last $4$ columns
of $\bar{\mathbf{P}}^{(1)}$, $\vec{b}^{(1)}=[2,3,3,3]$ is the list
of column degrees of $\bar{\mathbf{P}}_{2}^{(1)}$, and $\hat{\mathbf{P}}_{2}^{\left(1\right)}$
is the first $4$ rows of $\bar{\mathbf{P}}_{2}^{(1)}$. 
\end{comment}
notice that in the computation of the second subproblem, instead of
using $\mathbf{F}'^{\left(2\right)},$ $\bar{\mathbf{P}}^{\left(1\right)}$,
$\bar{\mathbf{Q}}^{(2)}$, and $\bar{\mathbf{P}}^{(1)}\bar{\mathbf{Q}}^{(2)}$,
the previous lemmas show that we can just use their submatrices, $\bar{\mathbf{F}}^{(2)}$
the top left $1\times4$ submatrix of $\mathbf{F}'^{\left(2\right)}$,
$\hat{\mathbf{P}}_{2}^{(1)}$ the top right $4\times4$ submatrix
of $\bar{\mathbf{P}}^{(1)}$, $\bar{\mathbf{Q}}_{2}^{(2)}$ the bottom
right $4\times4$ submatrix of $\bar{\mathbf{Q}}^{(2)}$, and $\hat{\mathbf{P}}_{2}^{(1)}\bar{\mathbf{Q}}_{2}^{(2)}$
the top right $4\times4$ submatrix of $\bar{\mathbf{P}}^{(1)}\bar{\mathbf{Q}}^{(2)}$of
lower dimensions. 
\end{example}
\prettyref{lem:computationAtTopLevel} gives us a way of computing
a $\left(\mathbf{F},\sigma,\vec{s}\right)$-basis. We can set $i$
to $\log\left(n/m\right)-1$ so that $(\bar{\mathbf{F}}^{\left(i\right)},2\delta^{\left(i\right)},\vec{s}^{\left(i\right)})$=$\left(\mathbf{F},\sigma,\vec{s}\right)$,
and compute a $(\bar{\mathbf{F}}^{\left(i\right)},2\delta^{\left(i\right)},\vec{s}^{\left(i\right)})$-basis.
By \prettyref{lem:computationAtTopLevel}, this can be divided into
two subproblems. The first produces $[\hat{\mathbf{P}}_{1}^{\left(i-1\right)},\hat{\mathbf{P}}_{2}^{\left(i-1\right)}]=\hat{\mathbf{P}}^{\left(i-1\right)}=\mathbf{E}^{\left(i\right)}\bar{\mathbf{P}}^{\left(i-1\right)}$
from computing a $(\bar{\mathbf{F}}^{\left(i-1\right)},2\delta^{\left(i-1\right)},\vec{s}^{\left(i-1\right)})$-basis
$\bar{\mathbf{P}}^{\left(i-1\right)}$. The second subproblem then
computes a $(\bar{\mathbf{F}}^{\left(i\right)}\hat{\mathbf{P}}_{2}^{\left(i-1\right)},2\delta^{\left(i\right)},\vec{b}^{\left(i-1\right)})$-basis
$\bar{\mathbf{Q}}_{2}^{\left(i\right)}$. Note the first subproblem
of computing a $(\bar{\mathbf{F}}^{\left(i-1\right)},2\delta^{\left(i-1\right)},\vec{s}^{\left(i-1\right)})$-basis
can again be divided into two subproblems just as before. This can
be repeated recursively until we reach the base case with degree parameter
$\delta^{\left(1\right)}=2d$. The total number of recursion levels
is therefore $\log\left(n/m\right)-1$.

Notice that the transformed matrix $\mathbf{F}'^{\left(i\right)}$
is not used explicitly in the computation, even though it is crucial
for deriving our results.


\subsection{The Iterative View}

In this subsection we present our algorithm, which uses an iterative
version of the computation discussed above.%
\begin{comment}
The recursive top-down approach of the previous subsection is useful
for giving an overall picture of the computation process. Algorithm
\prettyref{alg:mab} uses an equivalent corresponding bottom-up iterative
approach. 
\end{comment}
\begin{comment}
, allowing the complexity to be more easily analyzed. 
\end{comment}
\begin{comment}
. In practice, bottom-up iterative approaches are more efficient than
the corresponding top-down recursive approaches. For our purpose,
it is also easier to analyze the computational cost of the iterative
procedure. 
\end{comment}
{} The iterative version is usually more efficient in practice, considering
that the generic case has balanced output that can be computed with
just one iteration, whereas the recursive method has to go through
$\log(n/m)-1$ levels of recursion.

\prettyref{alg:mab} uses a subroutine $\mab$, the algorithm from
Giorgi et al. \citeyearpar{Giorgi2003}, for computing order bases
with balanced input. Specifically, $\left[\mathbf{Q},\vec{a}\right]=\mab(\mathbf{G},\sigma,\vec{b})$
computes a $(\mathbf{G},\sigma,\vec{b})$-basis and also returns its
$\vec{b}$-column degrees $\vec{a}$. The other subroutine $\StorjohannTransform$
is the transformation described in \prettyref{sub:storjohannTransformation}.

\prettyref{alg:mab} proceeds as follows. In the first iteration,
which is the base case of the recursive approach, we set the degree
parameter $\delta^{\left(1\right)}$ to be twice the average degree
$d$ and apply Storjohann's transformation to produce a new input
matrix $\bar{\mathbf{F}}^{\left(1\right)}$, which has $l^{\left(1\right)}$
block rows. Then a $(\bar{\mathbf{F}}^{\left(1\right)},2\delta^{\left(1\right)},\vec{s}^{\left(1\right)})$-basis
$\bar{\mathbf{P}}^{\left(1\right)}$ is computed. Note this is in
fact the first subproblem of computing a $(\bar{\mathbf{F}}^{\left(2\right)},2\delta^{\left(2\right)},\vec{s}^{\left(2\right)})$-basis,
which is another Storjohann transformed problem and also the problem
of the second iteration. At the second iteration, we work on a new
Storjohann transformed problem with the degree doubled and the number
of block rows $l^{\left(2\right)}=(l^{\left(1\right)}-1)/2$ reduced
by over a half. The column dimension is reduced by using the result
from the previous iteration. More specifically, we know that the basis
$\bar{\mathbf{P}}^{\left(1\right)}$ already provides a $(\bar{\mathbf{F}}^{\left(2\right)},2\delta^{\left(2\right)},\vec{s}^{\left(2\right)})_{\delta^{\left(1\right)}-1}$-basis
$\hat{\mathbf{P}}_{1}^{\left(1\right)}$, which can be disregarded
in the remaining computation. The remaining work in the second iteration
is to compute a $(\bar{\mathbf{F}}^{\left(2\right)}\hat{\mathbf{P}}_{2}^{\left(1\right)},2\delta^{\left(2\right)},\vec{b}^{\left(1\right)})$-basis
$\bar{\mathbf{Q}}^{\left(2\right)}$, where $\vec{b}^{\left(1\right)}=\deg_{\vec{s}^{\left(1\right)}}\bar{\mathbf{P}}_{2}^{\left(1\right)}$,
and then to combine it with the result from the previous iteration
to form a matrix $[\hat{\mathbf{P}}_{1}^{\left(1\right)},\hat{\mathbf{P}}_{2}^{\left(1\right)}\bar{\mathbf{Q}}^{\left(2\right)}]$
in order to extract a $(\bar{\mathbf{F}}^{\left(2\right)},2\delta^{\left(2\right)},\vec{s}^{\left(2\right)})$-basis
$\bar{\mathbf{P}}^{\left(2\right)}$.

With a $(\bar{\mathbf{F}}^{\left(2\right)},2\delta^{\left(2\right)},\vec{s}^{\left(2\right)})$-basis
computed, we can repeat the same process to use it for computing a
$(\bar{\mathbf{F}}^{\left(3\right)},2\delta^{\left(3\right)},\vec{s}^{\left(3\right)})$-basis.
Continue, using the computed $(\bar{\mathbf{F}}^{\left(i-1\right)},2\delta^{\left(i-1\right)},\vec{s}^{\left(i-1\right)})$-basis
to compute a $(\bar{\mathbf{F}}^{\left(i\right)},2\delta^{\left(i\right)},\vec{s}^{\left(i\right)})$-basis,
until all $n$ elements of a $\left(\mathbf{F},\sigma,\vec{s}\right)$-basis
have been determined.

\begin{algorithm}[t]
\caption{$\mmab\left(\mathbf{F},\sigma,\vec{s}\right)$ }


\label{alg:mab} 
\begin{algor}
\item [{{*}}] \textbf{Input:} $\mathbf{F}\in\mathbb{K}\left[x\right]^{m\times n}$,
$\sigma\in\mathbb{Z}_{\ge0}$,$\vec{s}\in\mathbb{Z}^{n}$ satisfying
$n\ge m,$ $n/m$ and $\sigma$ are powers of 2, $m\sigma\in\Omega(n)$
and $\min\left(\vec{s}\right)=0$ 
\item [{{*}}] \textbf{Output:} a $\left(\mathbf{F},\sigma,\vec{s}\right)$-basis
$\mathbf{P}\in K\left[x\right]^{n\times n}$ and $\deg_{\vec{s}}\mathbf{P}$
\item [{{*}}]~\end{algor}
\begin{algor}[1]
\item [{{*}}] \textbf{if }$2m\ge n$ \textbf{then return} $\mab\left(\mathbf{F},\sigma,\vec{s}\right);$ 
\item [{{{*}}}] $i:=1;$ $d:=m\sigma/n;$ $\delta^{\left(1\right)}:=2d;$ 
\item [{{{*}}}] $\bar{\mathbf{F}}^{\left(1\right)}:=\StorjohannTransform(\mathbf{F},\delta^{\left(1\right)});$ 
\item [{{{*}}}] $l^{\left(1\right)}:=\rowDimension(\bar{\mathbf{F}}^{\left(1\right)})/m;$ 
\item [{{{*}}}] $\vec{b}^{\left(0\right)}:=\left[\vec{s},0,\dots,0\right];$\ \ \ \ \ \ \ //
$m(l_{1}-1)$ $0$'s
\item [{{{*}}}] $[\bar{\mathbf{P}}^{\left(1\right)},\vec{a}^{\left(1\right)}]:=\mab(\bar{\mathbf{F}}^{\left(1\right)},2\delta^{\left(1\right)},\vec{b}^{\left(0\right)});$ 
\item [{{{*}}}] Sort the columns of $\bar{\mathbf{P}}^{\left(i\right)}$
and $\vec{a}^{\left(i\right)}$ by the shifted column degrees $\vec{a}^{\left(i\right)}=\deg_{\vec{b}}\bar{\mathbf{P}}^{\left(i\right)}$
in increasing order; 
\item [{{*}}] $\vec{t}^{\left(i\right)}:=\vec{a}^{\left(i\right)};$ 
\item [{{{*}}}] $k^{\left(i\right)}:=$ number of entries of $\vec{a}^{\left(i\right)}$
less than $\delta^{\left(i\right)}$;
\item [{{*}}] $[\bar{\mathbf{P}}_{1}^{\left(i\right)},\bar{\mathbf{P}}_{2}^{\left(i\right)}]:=\bar{\mathbf{P}}^{\left(i\right)}$
with $\bar{\mathbf{P}}_{1}^{\left(i\right)}\in K\left[x\right]^{n\times k^{\left(i\right)}}$;
\item [{{while}}] $\columnDimension(\bar{\mathbf{P}}_{1}^{\left(i\right)})<n$ 
\item [{{{*}}}] $i:=i+1;$ $\delta^{\left(i\right)}:=2\delta^{\left(i-1\right)};$
$l^{\left(i\right)}:=(l^{\left(i-1\right)}-1)/2;$ 
\item [{{{*}}}] $\bar{\mathbf{F}}^{\left(i\right)}:=\StorjohannTransform(\mathbf{F},\delta^{\left(i\right)});$ 
\item [{{{*}}}] $\hat{\mathbf{P}}_{2}^{\left(i-1\right)}:=\mathbf{E}^{\left(i\right)}\bar{\mathbf{P}}_{2}^{\left(i-1\right)}$;
\item [{{{*}}}] \label{line:matrixProduct1}$\mathbf{G}^{\left(i\right)}:=\bar{\mathbf{F}}^{\left(i\right)}\hat{\mathbf{P}}_{2}^{\left(i-1\right)};$ 
\item [{{{*}}}] $\vec{b}^{\left(i-1\right)}:=\vec{t}^{\left(i-1\right)}[k^{\left(i-1\right)}+1\dots n+m(l^{\left(i-1\right)}-1)];$
\\
$\mbox{ // w:=v[k..l] means that w receives a slice of v whose indices range from k to l}$
\item [{{{*}}}] \label{line:orderBasisComputation}$[\mathbf{Q}^{\left(i\right)},\vec{a}^{\left(i\right)}]:=\mab(\mathbf{G}^{\left(i\right)},2\delta^{\left(i\right)},\vec{b}^{\left(i-1\right)});$ 
\item [{{{*}}}] Sort the columns of $\mathbf{Q}^{\left(i\right)}$ and
$\vec{a}^{\left(i\right)}$ by $\vec{a}^{\left(i\right)}=\deg_{\vec{b}^{\left(i-1\right)}}\mathbf{Q}^{\left(i\right)}$
in increasing order; 
\item [{{{*}}}] \label{line:matrixProduct2}$\check{\mathbf{P}}^{\left(i\right)}:=\hat{\mathbf{P}}_{2}^{\left(i-1\right)}\mathbf{Q}^{\left(i\right)};$ 
\item [{{{*}}}] \label{line:LSP}$J:=$ the column rank profile of $\lcoeff(x^{\left[\vec{s},0,\dots,0\right]}[\mathbf{E}^{\left(i\right)}\bar{\mathbf{P}}_{1}^{\left(i-1\right)},\check{\mathbf{P}}^{\left(i\right)}])$; 
\item [{{*}}] $\bar{\mathbf{P}}^{\left(i\right)}:=[\mathbf{E}^{\left(i\right)}\bar{\mathbf{P}}_{1}^{\left(i-1\right)},\check{\mathbf{P}}^{\left(i\right)}]_{J}$,
\item [{{*}}] $\vec{t}^{\left(i\right)}:=\deg_{\left[\vec{s},0,\dots,0\right]}\bar{\mathbf{P}}^{\left(i\right)}$; 
\item [{{{*}}}] $k^{\left(i\right)}:=$ number of entries of $\vec{t}^{\left(i\right)}$
less than $\delta^{\left(i\right)}$;
\item [{{*}}] $[\bar{\mathbf{P}}_{1}^{\left(i\right)},\bar{\mathbf{P}}_{2}^{\left(i\right)}]:=\bar{\mathbf{P}}^{\left(i\right)}$
with $\bar{\mathbf{P}}_{1}^{\left(i\right)}\in K\left[x\right]^{n\times k^{\left(i\right)}}$
;
\item [{{endwhile}}] ~ 
\item [{{{*}}}] \textbf{return} the top $n$ rows of $\bar{\mathbf{P}}_{1}^{\left(i\right)}$,
$\vec{t}^{\left(i\right)}\left[1..n\right]$; \end{algor}
\end{algorithm}



\begin{comment}
\begin{thm}
Algorithm \prettyref{alg:mab} computes a $\left(\mathbf{F},\sigma,\vec{s}\right)$-basis
correctly.\end{thm}
\begin{proof}
This follows from \prettyref{lem:simplifySecondSubproblem}, \prettyref{lem:disregardComputedBasisElements},
and \prettyref{lem:computationAtTopLevel}. \end{proof}
\end{comment}
{} 



\section{Computational Complexity}

\label{sec:complexity}

In this section, we analyze the computational complexity of \prettyref{alg:mab}.
\begin{lem}
\prettyref{alg:mab} computes a $\left(\mathbf{F},\sigma,\vec{s}\right)$-basis
in no more than $\log\left(n/m\right)-1$ iterations.\end{lem}
\begin{proof}
Each iteration $i$ computes a $(\bar{\mathbf{F}}^{\left(i\right)},2\delta^{\left(i\right)},\vec{s}^{\left(i\right)})$-basis.
At iteration ${i}^{*}=\log(n/m)-1$, the degree parameter is $\sigma/2$
and $(\bar{\mathbf{F}}^{\left({i}^{*}\right)},2\delta^{\left({i}^{*}\right)},\vec{s}^{\left({i}^{*}\right)})=\left(\mathbf{F},\sigma,\vec{s}\right)$.\end{proof}
\begin{lem}
\label{lem:remainingNumberElements}If the shift $\vec{s}=\left[0,\dots,0\right]$,
then a $\left(\mathbf{F},\sigma,\vec{s}\right)_{\delta^{\left(i\right)}-1}$-basis
(or equivalently a $(\bar{\mathbf{F}}^{\left(i\right)},2\delta^{\left(i\right)},\vec{s}^{\left(i\right)})_{\delta^{\left(i\right)}-1}$-basis)
computed at iteration $i$ has at least $n-n/2^{i}$ elements, and
hence at most $n/2^{i}$ elements remain to be computed. If the shift
$\vec{s}$ is balanced, that is, $\max\vec{s}\in O(a)$ assuming $\min\vec{s}=0$,
then the number $n^{\left(i\right)}$ of remaining basis elements
at iteration $i$ is $O(n/2^{i})$.\end{lem}
\begin{proof}
The uniform case follows from the idea of \citet{storjohann-villard:2005}
on null space basis computation discussed in \prettyref{sub:Unbalanced-Output}.
For the balanced case, the average column degree is bounded by $ca=cm\sigma/n$
for some constant $c$. The first iteration $\lambda$ such that $\delta^{\left(\lambda\right)}$
reaches $ca$ is therefore a constant. That is, $\delta^{\left(\lambda\right)}=2^{\lambda}a\ge ca>\delta^{\left(\lambda-1\right)}$
and hence $\lambda=\left\lceil \log c\right\rceil $. By the same
argument as in the uniform case, the number of remaining basis elements
$n^{\left(i\right)}\le n/2^{i-\lambda}=2^{\lambda}(n/2^{i})\in O(n/2^{i})$
at iteration $i\ge\lambda$. For iterations $i<\lambda$, certainly
$n^{\left(i\right)}\le n<2^{\lambda}(n/2^{i})\in O(n/2^{i})$.\end{proof}
\begin{thm}
\label{thm:balancedCost}If the shift $\vec{s}$ is balanced with
$\min\left(\vec{s}\right)=0$, then \prettyref{alg:mab} computes
a $\left(\mathbf{F},\sigma,\vec{s}\right)$-basis with a cost of $O\left(n^{\omega}\M(a)\log\sigma)\right)\subset O^{\sim}\left(n^{\omega}a\right)$
 field operations. \end{thm}
\begin{proof}
The computational cost depends on the degree, the row dimension, and
the column dimension of the problem at each iteration. The degree
parameter $\delta^{\left(i\right)}$ is $2^{i}a$ at iteration $i$.
The number of block rows $l^{\left(i\right)}$ is $\sigma/\delta^{\left(i\right)}-1$,
which is less than $\sigma/(2^{i}a)=n/(2^{i}m)$ at iteration $i$.
The row dimension is therefore less than $n/2^{i}$ at iteration $i$.

The column dimension of interest at iteration $i$ is the column dimension
of $\hat{\mathbf{P}}_{2}^{\left(i-1\right)}$ (equivalently the column
dimension of $\bar{\mathbf{P}}_{2}^{\left(i-1\right)}$), which is
the sum of two components, $n^{\left(i-1\right)}+(l^{\left(i-1\right)}-1)m$.
The first component $n^{\left(i-1\right)}\in O(n/2^{i})$ by \prettyref{lem:remainingNumberElements}.
The second component $(l^{\left(i-1\right)}-1)m<n/2^{i-1}-m<n/2^{i-1}$
comes from the size of the identity matrix added in Storjohann's transformation.
Therefore, the overall column dimension of the problem at iteration
$i$ is $O(n/2^{i})$.

At each iteration, the four most expensive operations are the multiplications
at \prettyref{line:matrixProduct1} and \prettyref{line:matrixProduct2},
the order basis computation at \prettyref{line:orderBasisComputation},
and extracting the basis at \prettyref{line:LSP}.

The matrices $\bar{\mathbf{F}}^{\left(i\right)}$ and $\hat{\mathbf{P}}_{2}^{\left(i-1\right)}$
have degree $O(2^{i}a)$ and dimensions $O(n/2^{i})\times O\left(n\right)$
and $O\left(n\right)\times O(n/2^{i})$. The multiplication cost is
therefore $2^{i}\MM(n/2^{i},2^{i}a)$ field operations, which is bounded
by
\begin{align}
2^{i}\MM(n/2^{i},2^{i}a) & \in O\left(2^{i}\left(n/2^{i}\right)^{\omega}\M(2^{i}a)\right)\nonumber \\
 & \subset O\left(n^{\omega}\left(2^{i}\right)^{1-\omega}\M\left(2^{i}\right)\M(a)\right)\label{eq:separateMultiplication}\\
 & \subset O\left(n^{\omega}\left(2^{i}\right)^{1-\omega}\left(2^{i}\right)^{\omega-1}\M(a)\right)\label{eq:MultiplicationToExponent}\\
 & \subset O\left(n^{\omega}\M(a)\right).\nonumber 
\end{align}
Equation (\ref{eq:separateMultiplication}) follows from $\M(st)\in O\left(\M(s)\M(t)\right)$.
Equation (\ref{eq:MultiplicationToExponent}) follows from $\M(t)\in O(t^{\omega-1})$.

The matrices $\hat{\mathbf{P}}_{2}^{\left(i-1\right)}$ and $\bar{\mathbf{Q}}^{\left(i\right)}$
of the second multiplication have the same degree $O(2^{i}a)$ and
dimensions $O\left(n\right)\times O(n/2^{i})$ and $O(n/2^{i})\times O(n/2^{i})$
and can also be multiplied with a cost of $O\left(n^{\omega}\M(a)\right)$
field operations. The total cost of the multiplications over $O(\log\left(n/m\right))$
iterations is therefore $O\left(n^{\omega}\M(a)\log(n/m)\right)$.

The input matrix $\mathbf{G}^{\left(i\right)}=\bar{\mathbf{F}}^{\left(i\right)}\hat{\mathbf{P}}_{2}^{\left(i-1\right)}$
of the order basis computation problem at iteration $i$ has dimension
$O(n/2^{i})\times O(n/2^{i})$ and the order of the problem is $2\delta^{\left(i\right)}\in O(2^{i}a)$.
Thus, the cost of the order basis computation at iteration $i$ is
$O\left(\left(n/2^{i}\right)^{\omega}\M\left(2^{i}a\right)\log\left(2^{i}a\right)\right)$.
The total cost over $O(\log\left(n/m\right))$ iterations is bounded
by 
\begin{align*}
 & O\left(\sum_{i=1}^{\infty}\left(\left(n/2^{i}\right)^{\omega}\M\left(2^{i}a\right)\log\left(2^{i}a\right)\right)\right)\\
\subset & O\left(\sum_{i=1}^{\infty}\left(\left(n/2^{i}\right)^{\omega}\M\left(2^{i}\right)\log\left(2^{i}\right)\M\left(a\right)\log\left(a\right)\right)\right)\\
\subset & O\left(\sum_{i=1}^{\infty}\left(n^{\omega}\left(2^{i}\right)^{-\omega}\left(2^{i}\right)^{\omega-1}\M\left(a\right)\log\left(a\right)\right)\right)\\
\subset & O\left(n^{\omega}\M\left(a\right)\log\left(a\right)\sum_{i=1}^{\infty}\left(2^{-i}\right)\right)\\
\subset & O\left(n^{\omega}\M\left(a\right)\log\left(a\right)\right).
\end{align*}


Finally, extracting an order basis by LSP factorization costs $O\left(n^{\omega}\right)$,
which is dominated by the other costs. Combining the above gives 
\[
O\left(n^{\omega}\M\left(a\right)\log(n/m)+n^{\omega}\M\left(a\right)\log a\right)=O\left(n^{\omega}\M\left(a\right)\log\sigma)\right)
\]
 as the total cost of the algorithm. \end{proof}




\section{More Refined Cost and the Case $m\sigma\in o(n)$}

\label{sec:removeCeilingFunction}

\prettyref{thm:balancedCost} states the cost of computing a $\left(\mathbf{F},\sigma,\vec{s}\right)$-basis
as $O^{\sim}\left(n^{\omega}a\right)$, where $a=m\sigma/n$. In this
cost, $a$ is assumed to tend to infinity, which means $m\sigma>n$.
This allows us to transform the original problem with dimension $m\times n$
and degree $\sigma$ to one with dimension $\Theta(n)\times\Theta(n)$
and degree $\Theta(a)=\Theta(m\sigma/n)$, allowing order basis computation
to be efficient with a final cost of $O^{\sim}(n^{\omega}a)$. However,
if we attempt to state the cost as $O^{\sim}\left(n^{\omega-1}m\sigma\right)$,
the case of $m\sigma\in o\left(n\right)$ becomes problematic and
requires special attention. In this case, the average degree $a=m\sigma/n\in o(1)$
but $1$ is the lowest possible degree and $m\sigma$ is the maximum
possible row dimension of our transformed problems. In other words,
we cannot obtain a nearly square transformed problem for our algorithms
to behave efficiently, which means our algorithms still require $O^{\sim}(n^{\omega})$
field operations. We now look how this cost can be improved to $O^{\sim}(n^{\omega-1}m\sigma)$
in the case of $m\sigma\in o\left(n\right)$.


\subsection{Balanced Case}

First note that in this case, using \prettyref{def:balancedShift},
a balanced shift $\vec{s}$ is also uniform, since $\max\left(\vec{s}\right)-\min\left(\vec{s}\right)\in O\left(m\sigma/n\right)\subseteq o(1)$,
which makes $\max\left(\vec{s}\right)-\min\left(\vec{s}\right)=0$.
So let us just consider the uniform shift case.

We first compute all degree 0 basis elements, which then helps to
eliminate the columns of the input that are never going to be needed
as pivots. The remaining columns can then be used as the input to
compute the remaining basis elements efficiently. The degree 0 elements
of a $\left(\mathbf{F},\sigma\right)$-basis%
\begin{comment}
nullspace basis elements of $\mathbf{F}$
\end{comment}
{} correspond to a nullspace basis of a linearized matrix 
\[
\bar{F}=\left[\begin{array}{c}
F_{0}\\
F_{1}\\
F_{2}\\
\vdots\\
F_{\sigma-1}
\end{array}\right]\in\mathbb{K}^{(m\sigma)\times n}
\]
of $\mathbf{F}=F_{0}+F_{1}x+F_{2}x^{2}+\cdots+F_{\sigma-1}x^{\sigma-1}$.
\begin{lem}
The elements of a nullspace basis of $\bar{F}$ over $\mathbb{K}$
are also the degree 0 elements of a $\left(\mathbf{F},\sigma\right)$-basis%
\begin{comment}
a minimal nullspace basis of $\mathbf{F}$
\end{comment}
.\end{lem}
\begin{proof}
The columns of $\bar{F}$ and the columns of $\mathbf{F}$ are equivalent
representations of the same elements of the same $\mathbb{K}$-module,
which is also a vector space over $\mathbb{K}$.
\end{proof}
To compute these basis elements, we can use the Gauss Jordan transform
algorithm from \citet{storjohann:phd2000} on $\bar{F}$ with a cost
of $O\left(nm\sigma\bar{r}^{\omega-2}\right)$, where $\bar{r}\le m\sigma$
is the rank of $\bar{F}$. The algorithm finds a permutation matrix
$P$ and a unimodular matrix $U$ in $\mathbb{K}^{n\times n}$ such
that $\bar{F}PU$ is in the reduced column echelon form of $\bar{F}$.
Note that $P$ permutes the columns of $\bar{F}$ so that the first
$\bar{r}$ columns of $\bar{F}$ are linearly independent. Let $\left[U_{1},U_{0}\right]:=U$
with $U_{0}$ correspond to the zero columns of $\bar{F}PU$. Then
the matrix consists of the bottom $n-\bar{r}$ rows of $U_{0}$ is
the identity matrix, and only the first $\bar{r}$ rows of $U_{1}$
are nonzero. Because of this simpler structure after permutation,
let us compute a $\left(\mathbf{F}P,\sigma\right)$-basis $\mathbf{P}$
instead, which also gives us a $\left(\mathbf{F},\sigma\right)$-basis
$P\mathbf{P}$. Notice that $U_{0}$ consists of all the degree 0
elements of a $\left(\mathbf{F}P,\sigma\right)$-basis. We can then
use $\mathbf{F}PU_{1}$ as the input matrix to compute the remaining
basis elements. But to further simplify our future computation, let
us replace $U_{1}$ with $V=\left[I,0\right]^{T}$ of the same dimension,
where the identity matrix $I$ replaces the first nonzero $\bar{r}$
rows in $U_{1}$. In essence, $PV$ picks $\bar{r}$ columns from
$\mathbf{F}$ for computing the remaining basis elements. Since $U_{0}$
has at least $n-m\sigma$ columns, there are at most $m\sigma$ columns
in $U_{1}$, and hence at most $m\sigma$ columns in $V$ and in $\mathbf{F}PV$. 
\begin{lem}
\begin{comment}
Let If $\bar{F}P\left[U_{2},U_{0}\right]$ is in the reduced column
echelon form of $\bar{F}$ with $U_{0}$ being a nullspace basis of
$\bar{F}P$, and i
\end{comment}
If we compute a $\left(\mathbf{F}PV,\sigma\right)$-basis $\mathbf{Q}$,
then $\left[V\mathbf{Q},U_{0}\right]$ is a $\left(\mathbf{F}P,\sigma\right)$-basis.\end{lem}
\begin{proof}
Note that the matrix $\left[V,U_{0}\right]$, which has the structure
\[
\begin{bmatrix}I & *\\
0 & I
\end{bmatrix}
\]
 with $*$ representing the first $r$ rows of $U_{0}$, is a $\left(\mathbf{F}P,0\right)$-basis
since it is unimodular and column reduced. From \ref{thm:combineOrderBases},
we can use the residual $\mathbf{F}P[V,U_{0}]=\left[\mathbf{F}PV,0\right]$
to compute a $\left(\left[\mathbf{F}PV,0\right],\sigma\right)$-basis
$\bar{\mathbf{Q}}$, then $[V,U_{0}]\bar{\mathbf{Q}}$ is a $\left(\mathbf{F}P,\sigma\right)$-basis.
Also note that if $\mathbf{Q}$ is a $\left(\mathbf{F}PV,\sigma\right)$-basis,
then 
\[
\bar{\mathbf{Q}}=\begin{bmatrix}\mathbf{Q}\\
 & I
\end{bmatrix}
\]
 is a $\left(\left[\mathbf{F}PV,0\right],\sigma\right)$-basis, and
$[V,U_{0}]\bar{\mathbf{Q}}=\left[V\mathbf{Q},U_{0}\right]$ is a $\left(\mathbf{F}P,\sigma\right)$-basis.
\end{proof}
Our new problem of computing a $\left(\mathbf{F}PV,\sigma\right)$-basis
now satisfies the condition of having column dimension bounded by
$m\sigma$.%
\begin{comment}
\begin{lem}
The column dimensions of $V$ and $\mathbf{F}PV$ are bounded by $m\sigma$.\end{lem}
\begin{proof}
The rank of $\bar{F}$ is bounded by $m\sigma$. This means its nullspace
basis $U_{0}$ has at least $n-m\sigma$ columns. Therefore $V$ has
at most $m\sigma$ columns.\end{proof}
\end{comment}
{} We can therefore compute a $\left(\mathbf{F}PV,\sigma\right)$-basis
using \prettyref{alg:mab} with a cost of $O^{\sim}\left(\left(m\sigma\right)^{\omega}\right)\subset O^{\sim}\left(n^{\omega-1}m\sigma\right)$.

The last thing to check is making sure that the multiplications for
computing the residual $\mathbf{F}PV$, and for combining the results
$V\mathbf{Q}$, and for obtaining the final result $P\left[V\mathbf{Q},U_{0}\right]$
can all be done efficiently, which is not difficult since $P$ is
a permutation matrix, and $V$ consists of an identity matrix and
zeros. Therefore, the $(\mathbf{F},\sigma)$-basis $P\left[V\mathbf{Q},U_{0}\right]$
can be computed with a cost of $O^{\sim}\left(n^{\omega-1}m\sigma\right)$.
This allows us to refine the cost $O^{\sim}\left(n^{\omega}d\right)$
to $O^{\sim}\left(n^{\omega-1}m\sigma\right)$.
\begin{thm}
\label{thm:orderBasisCostCeilingRemoved}A $\left(\mathbf{F},\sigma,\vec{s}\right)$-basis
can be computed with a cost of 
\[
O\left(n^{\omega}\M(m\sigma/n)\log\sigma)\right)\subset O^{\sim}\left(n^{\omega-1}m\sigma\right)
\]
 field operations. %
\begin{comment}
Should easily work for the second unbalanced case as well. But need
to write it out.
\end{comment}
\end{thm}




\chapter{Unbalanced Shifts}

\label{chap:Unbalanced-Shift}

\prettyref{thm:orderBasisCostCeilingRemoved} shows that \prettyref{alg:mab}
can efficiently compute a $\left(\mathbf{F},\sigma,\vec{s}\right)$-basis
when the shift $\vec{s}$ is balanced. When $\vec{s}$ is unbalanced
(something important for example in normal form computation \citep{BLV:1999,BLV:jsc06}),
then \prettyref{alg:mab} still returns a correct answer but may be
less efficient. The possible inefficiency results because there may
not be enough partial results from the intermediate subproblems to
sufficiently reduce the column dimension of the subsequent subproblem.
This is clear from the fact that the column degrees of the output
can be much larger and no longer sum up to $O\left(m\sigma\right)$
as in the balanced shift case. The shifted $\vec{s}$-column degrees,
however, still behave well. In particular, the total $\vec{s}$-degree
increase is still bounded by $m\sigma$ as stated in \prettyref{lem:boundOfSumOfShiftedDegreesOfOrderBasis},
while the shifted degree of any column can also increase by up to
$\sigma$. Recall that \prettyref{lem:boundOfSumOfShiftedDegreesOfOrderBasis}
states that for any shift $\vec{s}$, there exists a $\left(\mathbf{F},\sigma,\vec{s}\right)$-basis
still having a total size bounded by $nm\sigma$ which gives hope
for efficient computation. In the following, we look at two special
cases of unbalanced shift. In the first case where the input shift
$\vec{s}$ satisfies $\sum_{i=1}^{n}(\vec{s}_{i}-\min(\vec{s}))\in O(m\sigma)$,
the sum of the column degrees of a $\left(\mathbf{F},\sigma,\vec{s}\right)$-basis
is still in $O(m\sigma)$, which allows us to use \prettyref{alg:mab}
to compute a $\left(\mathbf{F},\sigma,\vec{s}\right)$-basis efficiently
as in the balanced case. The second case where $\vec{s}$ satisfies
$\sum_{i=1}^{n}(\max(\vec{s})-\vec{s}_{i})\in O(m\sigma)$ is more
complicated and is the main focus of this section.


\section{First unbalanced case}

We first consider the case where the input shift $\vec{s}$ satisfies
\[
\sum_{i=1}^{n}(\vec{s}_{i}-\min(\vec{s}))\in O(m\sigma).
\]
As before, we may use the equivalent condition
\begin{equation}
\vec{s}\ge0\mbox{ and }\sum\vec{s}\in O(m\sigma),\label{con:unbalanced1}
\end{equation}
which can always be obtained from the previous condition by using
$\vec{s}-\min\vec{s}$ as the new shift. Note that translating every
entry of the shift by the same constant does not change the problem.
In this case, \prettyref{alg:mab} works efficiently as before.
\begin{lem}
If the shift $\vec{s}$ satisfies condition (\ref{con:unbalanced1}),
then a $\left(\mathbf{F},\sigma,\vec{s}\right)$-basis can be computed
with $O\left(n^{\omega}\bar{\M}(d)\log\sigma)\right)=O\left(n^{\omega}d\log d\log\log d\log\sigma)\right)\subset O^{\sim}\left(n^{\omega}d\right)$
field operations. 
\end{lem}
From \prettyref{lem:boundOfSumOfShiftedDegreesOfOrderBasis}, we know
that the sum of the $\vec{s}$-column degrees of any $\left(\mathbf{F},\mathbf{\sigma},\vec{s}\right)$-basis
is $\vec{t}=\sum\vec{s}+m\sigma\in O(m\sigma)$, and since the entries
of $\vec{s}$ are non-negative, the sum of the column degrees is less
than $\sum\vec{t}$. So the sum of the column degrees of any $\left(\mathbf{F},\mathbf{\sigma},\vec{s}\right)$-basis
is also in $O(m\sigma)$. Now the same analysis from \prettyref{sec:complexity}
applies.


\section{Second unbalanced case}

We now look at another important case of unbalanced shift -- when
the input shift $\vec{s}$ satisfies the condition:
\[
\sum_{i=1}^{n}(\max(\vec{s})-\vec{s}_{i})\le m\sigma.
\]
 For simplicity, we use the equivalent condition 
\begin{equation}
\vec{s}\le0\mbox{ and }-\sum\vec{s}\le m\sigma,\label{con:unbalancedCondition}
\end{equation}
 which can always be obtained from the previous condition by using
$\vec{s}-\max\vec{s}$ as the new shift. 

In the balanced shift case, a central problem is to find a way to
handle unbalanced column degrees of the output order basis. In this
section, the unbalanced shift makes row degrees of the output also
unbalanced, which is a major problem that needs to be resolved. Here
we note a second transformation by \citet{Storjohann:2006} which
converts the input in such a way that each high degree row of the
output becomes multiple rows of lower degrees. We refer to this as
Storjohann's second transformation to distinguish it from that described
in \prettyref{sub:storjohannTransformation}. The transformed problem
can then be computed efficiently using \prettyref{alg:mab}. After
the computation, rows can then be combined appropriately to form a
basis of the original problem. The method is computationally efficient.

Unfortunately, the bases computed this way are not minimal and hence
do not in general produce our reduced order bases. In the following,
we describe a transformation that incorporates Storjohann's second
transformation and guarantees the minimality of some columns of the
output, hence providing a partial order basis. We can then work on
the remaining columns iteratively as done in the balanced shift case
to compute a full order basis.

Condition (\ref{con:unbalancedCondition}) essentially allows us to
locate the potential high degree rows that need to be balanced. In
more general cases, we may not know in advance which are the high
degree rows that need to be balanced, so our approach given in this
section does not work directly. This suggests that one possible future
direction to pursue is to find an effective way to estimate the row
degree of the result pivot entries. Such an estimate may allow us
to apply the method given in this section efficiently for general
unbalanced shifts. 


\subsection{Transform to Balanced Shifts}

We now describe the transformation for balancing the high degree rows
of the resulting basis. Consider the problem of computing a $\left(\mathbf{F},\sigma,\vec{s}\right)$-basis,
where the input shift $\vec{s}$ satisfies the conditions (\ref{con:unbalancedCondition}).
Let $\alpha,\beta\in\mathbb{Z}_{>0}$ be two parameters. For each
shift entry $s_{i}$ in $\vec{s}$ with $-s_{i}>\alpha+\beta$, let
\[
r_{i}=\mbox{rem}\left(-s_{i}-\alpha-1,\beta\right)+1
\]
 be the remainder when $-s_{i}-\alpha$ is divided by $\beta$, and
where $r_{i}=\beta$ in the case where the remainder is $0$, and
set 
\[
q_{i}=\begin{cases}
1 & \mbox{if }-s_{i}\le\alpha+\beta\\
1+\left(-s_{i}-\alpha-r_{i}\right)/\beta & \mbox{otherwise}
\end{cases}
\]
 Then, for each $q_{i}>1$, we expand the corresponding $i$th column
$\mathbf{f}_{i}$ of $\mathbf{F}$ and shift $s_{i}$ to 
\begin{eqnarray*}
\tilde{\mathbf{F}}^{\left(i\right)} & = & \left[~\mathbf{f}_{i},~x^{r_{i}}\mathbf{f}_{i},~x^{r_{i}+\beta}\mathbf{f}_{i},~\dots~,~x^{r_{i}+(q_{i}-2)\beta}\mathbf{f}_{i}\right],~~\tilde{s}_{i}=\left[-\alpha-\beta,~\dots~,-\alpha-\beta\right]
\end{eqnarray*}
 with $q_{i}$ entries in each case. When $q_{i}=1$, the corresponding
shift entry and input column remain the same, that is, $\tilde{s}_{i}=s_{i}$,
and $\tilde{\mathbf{F}}^{\left(i\right)}=\mathbf{f}_{i}$. Then for
the transformed problem, the new shift becomes $\bar{s}=[\tilde{s}_{1},\dots,\tilde{s}_{n}]\in\mathbb{Z}_{\le0}^{\bar{n}}$,
and the new input matrix becomes $\bar{\mathbf{F}}=[\tilde{\mathbf{F}}^{\left(1\right)},\dots,\tilde{\mathbf{F}}^{\left(n\right)}]\in\mathbb{K}\left[x\right]^{m\times\bar{n}}$,
with the new column dimension $\bar{n}$ satisfies $\bar{n}=\sum_{i=1}^{n}q_{i}$.
Note that every entry of the new shift $\bar{s}$ is an integer from
$-\alpha-\beta$ to $0$. Let 
\[
\mathbf{E}=\left[\begin{array}{ccccc|ccc|ccccccc}
1 & x^{r_{1}} & x^{r_{1}+\beta} & \cdots & x^{r_{1}+(q_{1}-2)\beta} &  &  & \\
\hline  &  &  &  &  & \ddots &  & \\
 &  &  &  &  &  & \  & \ddots\\
\hline  &  &  &  &  &  &  &  & 1 & x^{r_{n}} & x^{r_{n}+\beta} & \cdots & x^{r_{n}+(q_{n}-2)\beta}
\end{array}\right]_{n\times\bar{n}}.
\]
 Then $\bar{\mathbf{F}}=\mathbf{F}\mathbf{E}$. Storjohann's second
transformation is determined by setting $\alpha=-1$, a value not
allowed in our transformation (we show later in \prettyref{thm:correctHighDegreeElements}
that this value is not useful in our case). One can verify that the
new dimension 
\[
\bar{n}=\sum_{i=1}^{n}q_{i}\le n+\sum_{i=1}^{n}-s_{i}/\beta\le m\sigma/\beta+n.
\]
 Thus by setting $\beta\in\Theta\left(m\sigma/n\right)=\Theta\left(d\right)$,
we can make $\bar{n}\in\Theta\left(n\right)$. Furthermore, by also
setting $\alpha\in\Theta\left(d\right)$, we have a balanced shift
problem since 
\[
\max\bar{s}-\min\bar{s}\le-\min\bar{s}\le\alpha+\beta\in\Theta(d).
\]
 Hence \prettyref{alg:mab} can compute a $\left(\bar{\mathbf{F}},\sigma,\bar{s}\right)$-basis
with cost $O^{\sim}\left(n^{\omega}d\right)$ in this case.

With a $\left(\bar{\mathbf{F}},\sigma,\bar{s}\right)$-basis $\bar{\mathbf{P}}\in\mathbb{K}\left[x\right]^{\bar{n}\times\bar{n}}$
computed, let us now consider $\mathbf{E}\bar{\mathbf{P}}\in\mathbb{K}\left[x\right]^{n\times\bar{n}}$.
While it is easy to see that $\mathbf{E}\bar{\mathbf{P}}$ has order
$\left(\mathbf{F},\sigma\right)$ since \textbf{$\mathbf{F}\mathbf{E}\bar{\mathbf{P}}=\bar{\mathbf{F}}\bar{\mathbf{P}}\equiv0\mod x^{\sigma}$},
in general it is not a minimal basis (in fact, $\mathbf{E}\bar{\mathbf{P}}$
is not even square). However, our transformation does guarantee that
the highest degree columns of $\mathbf{E}\bar{\mathbf{P}}$ having
$\vec{s}$-degrees exceed $-\alpha$ are minimal. That is, the columns
of $\mathbf{E}\bar{\mathbf{P}}$ whose $\vec{s}$-degrees exceed $-\alpha$
are exactly the columns of a $\left(\mathbf{F},\sigma,\vec{s}\right)$-basis
whose $\vec{s}$-degrees exceed $-\alpha$. We have therefore correctly
computed a partial $\left(\mathbf{F},\sigma,\vec{s}\right)$-basis. 
\begin{example}
\label{exm:unbalancedShift} Let us use the same input as in \prettyref{exm:StorjohannTransformation},
but with shift $\vec{s}=[0,-3,-5,-6]$, and parameters $\alpha=\beta=1$.
Then we get the transformed input 
\begin{align*}
\bar{\mathbf{F}}=[ & x+x^{2}+x^{3}+x^{4}+x^{5}+x^{6},\,~1+x+x^{5}+x^{6}+x^{7},\,~x+x^{2}+x^{6}+x^{7}+x^{8},\\
 & 1+x^{2}+x^{4}+x^{5}+x^{6}+x^{7},\,~x+x^{3}+x^{5}+x^{6}+x^{7}+x^{8},\,~x^{2}+x^{4}+x^{6}+x^{7}+x^{8}+x^{9},\\
 & x^{3}+x^{5}+x^{7}+x^{8}+x^{9}+x^{10},\,~1+x+x^{3}+x^{7},\,~x+x^{2}+x^{4}+x^{8},\\
 & x^{2}+x^{3}+x^{5}+x^{9},\,~x^{3}+x^{4}+x^{6}+x^{10},\,~x^{4}+x^{5}+x^{7}+x^{11}]
\end{align*}
 having $12$ components, and $\bar{s}=[0,-2,-2,-2,-2,-2,-2,-2,-2,-2,-2,-2]$.
In this case $r_{1}=r_{2}=r_{3}=r_{4}=1$, $q_{1}=1$, $q_{2}=2$,
$q_{3}=4$, $q_{4}=5$ and the transformation matrix is 
\[
\mathbf{E}=\left[{\begin{array}{c|cc|cccc|ccccc}
~1~ & ~0~ & ~0~ & ~0~ & ~0~ & ~0~ & ~0~ & ~0~ & ~0~ & ~0~ & ~0~ & ~0~\\
\hline 0 & 1 & x & 0 & 0 & 0 & 0 & 0 & 0 & 0 & 0 & 0\\
\hline 0 & 0 & 0 & 1 & x & x^{2} & x^{3} & 0 & 0 & 0 & 0 & 0\\
\hline 0 & 0 & 0 & 0 & 0 & 0 & 0 & 1 & x & x^{2} & x^{3} & x^{4}
\end{array}}\right].
\]
 Using the earlier algorithm for balanced shift, we compute a $(\bar{\mathbf{F}},8,\bar{s})$-basis
\[
\bar{\mathbf{P}}=\left[{\begin{array}{cccccccccccc}
0 & ~0~ & ~0~ & ~0~ & 0 & 0 & 0 & 0 & 0 & 0 & 0 & 1\\
\hline ~x~ & 1 & 0 & 0 & 1 & 0 & ~x~ & 0 & 0 & 0 & ~x~ & ~0~\\
0 & 0 & 1 & 0 & 0 & ~x~ & 1+~x~ & ~x~ & ~x~ & ~x~ & 1 & 0\\
\hline ~x~ & 1 & 0 & 1 & 1+x & 1 & ~x~ & 0 & 0 & 0 & 0 & 1\\
~x~ & 0 & 1 & 1 & 1+x & 1+x & 1 & ~x~ & ~x~ & 0 & 0 & 0\\
~x~ & 0 & 0 & 1 & 1+x & 1+x & 1 & ~x~ & 0 & 1 & 0 & 0\\
~x~ & 0 & 0 & 1 & 1 & 0 & 0 & 1 & 0 & 0 & 0 & 0\\
\hline 0 & 0 & 0 & 1 & ~x~ & 1 & 0 & 0 & 0 & 0 & 0 & 1\\
0 & 1 & 0 & 0 & 0 & 0 & 0 & 0 & 0 & 0 & 0 & 0\\
0 & 0 & 1 & 0 & 0 & 0 & 0 & 0 & 0 & 0 & 0 & 0\\
0 & 0 & 0 & 1 & 0 & ~x~ & 1 & 1 & 1 & 1 & 0 & 0\\
0 & 0 & 0 & 1 & 0 & 0 & 0 & 0 & 0 & 0 & 0 & 0
\end{array}}\right]
\]
 with $\bar{s}$-degrees $[-1,-2,-2,-2,-1,-1,-1,-1,-1,-1,-1,0]$.
Only the last column has $\bar{s}$-degree exceeding $-\alpha=-1$
and so is the only column guaranteed to give a correct $(\mathbf{F},8,\vec{s})$-basis
element. Comparing 
\[
\mathbf{E}\bar{\mathbf{P}}=\left[{\begin{array}{cccccccccccc}
0 & ~0~ & ~0~ & 0 & ~0~ & 0 & ~0~ & ~0~ & ~0~ & ~0~ & ~0~ & ~1~\\
x & 1 & x & 0 & 1 & x^{2} & x^{2} & x^{2} & x^{2} & x^{2} & 0 & 0\\
x+x^{2}+x^{3}+x^{4} & 1 & x & 1+x+x^{2}+x^{3} & 1 & 1+x+x^{3} & x^{2} & x^{2} & x^{2} & x^{2} & 0 & 1\\
0 & x & x^{2} & 1+x^{3}+x^{4} & x & 1+x^{4} & x^{3} & x^{3} & x^{3} & x^{3} & 0 & 1
\end{array}}\right]
\]
 to a $(\mathbf{F},8,\vec{s})$-basis 
\[
\mathbf{P}=\left[{\begin{array}{cccc}
~0~ & 0 & 0 & ~1~\\
1 & 0 & 0 & 0\\
1 & x^{2}+x^{3}+x^{4} & 1+x+x^{2}+x^{3} & 1\\
x & x^{2} & 1+x^{3}+x^{4} & 1
\end{array}}\right]
\]
 with $\vec{s}$-degrees $[-3,-1,-2,~0~]$, we see that the last column
of $\mathbf{E}\bar{\mathbf{P}}$ is a element of a $(\mathbf{F},8,\vec{s})$-basis.

If we set $\alpha=2,\beta=1$, then the new transformed problem gives
\[
\bar{\mathbf{P}}=\left[{\begin{array}{ccccccccc}
~0~ & ~0~ & ~0~ & ~0~ & ~0~ & ~0~ & ~0~ & ~0~ & ~1~\\
1 & 0 & 0 & x & 1+x & x & x & \, x\, & \,0\,\\
1 & x^{2} & 1 & x & 1 & x & \, x\, & 0 & 1\\
0 & x^{2} & 1 & \, x\, & 1 & \, x\, & 0 & 1 & 0\\
0 & x^{2} & 1+x & 1 & 0 & 1 & 0 & 0 & 0\\
0 & x^{2} & 1 & 0 & \, x\, & 0 & 0 & 0 & 1\\
1 & 0 & 0 & 0 & 0 & 0 & 0 & 0 & 0\\
\,0\, & \,0\, & \, x\, & 1+x & 1 & 1 & 1 & 1 & 0\\
0 & 0 & x & 1 & 0 & 0 & 0 & 0 & 0
\end{array}}\right]
\]
 with $\bar{s}$-degrees $[-3,-1,-2,-2,-2,-2,-2,-2,~0~]$. In this
case the second column also has $\bar{s}$-degree exceeding $-\alpha=-2$,
and so it is guaranteed to produce another element of a $(\mathbf{F},8,\vec{s})$-basis.
Computing 
\[
\mathbf{E}\bar{\mathbf{P}}=\left[{\begin{array}{ccccccccc}
~0~ & ~0~ & ~0~ & ~0~ & ~0~ & ~0~ & ~0~ & ~0~ & ~1~\\
1 & 0 & 0 & x & 1+x & x & x & x & 0\\
1 & x^{2}+x^{3}+x^{4} & 1+x+x^{2}+x^{3} & x & 1+x & x & x & x & 1\\
x & x^{2} & 1+x^{3}+x^{4} & x^{2} & x+x^{2} & ~x^{2} & ~x^{2} & ~x^{2} & 1
\end{array}}\right],
\]
 we notice the second column is indeed an element of a $(\mathbf{F},8,\vec{s})$-basis. 
\end{example}

\subsection{Correspondence Between the Original Problem and the Transformed Problem}

We now work towards establishing the correspondence between the high
degree columns of a $\left(\bar{\mathbf{F}},\sigma,\bar{s}\right)$-basis
whose $\bar{s}$-degrees exceed $-\alpha$ and those of a $\left(\mathbf{F},\sigma,\vec{s}\right)$-basis
whose $\vec{s}$-degrees exceed $-\alpha$. A useful link is provided
by the following a matrix .

Set 
\[
\mathbf{A}_{i}=\begin{bmatrix}~~x^{r_{i}}\\
-1 & ~~x^{\beta}\\
 & -1 & \ddots\\
 &  & \ddots & ~~x^{\beta}\\
 &  &  & -1
\end{bmatrix}_{q_{i}\times(q_{i}-1)}\mbox{ and \qquad}\mathbf{A}=\left[\begin{array}{ccc}
\mathbf{A}_{1}\\
 & \ddots\\
 &  & \mathbf{A}_{n}
\end{array}\right]_{\bar{n}\times(\bar{n}-n)}.
\]
 If $q_{i}=1$, $\mathbf{A}_{i}$ has dimension $1\times0$, which
just adds a zero row and no column in $\mathbf{A}$.

We now show that for any $\bar{\mathbf{w}}\in\left\langle \left(\bar{\mathbf{F}},\sigma,\bar{s}\right)\right\rangle $,
$\bar{\mathbf{w}}$ can be transformed by $\mathbf{A}$ to one of
the two forms that correspond to the original problem and transformed
problem. This is made more precise in the following lemma. We then
use unimodular equivalence of these two forms to show the equivalence
between the high degree part of the result from the transformed problem
and that of the original problem. 
\begin{lem}
Let 
\[
\bar{\mathbf{w}}=\begin{bmatrix}\bar{\mathbf{w}}_{1}\\
\vdots\\
\bar{\mathbf{w}}_{n}
\end{bmatrix}\in\langle(\bar{\mathbf{F}},\sigma,\bar{s})\rangle\mbox{ with }\bar{\mathbf{w}}_{i}=\begin{bmatrix}\bar{w}_{i,0}\\
\vdots\\
\bar{w}_{i,q_{i}-1}
\end{bmatrix}_{q_{i}\times1}.
\]
 Then there exists a vector $\mathbf{u}\in\mathbb{K}\left[x\right]^{\left(\bar{n}-n\right)\times1}$
such that $\bar{\mathbf{w}}+\mathbf{A}\mathbf{u}$ has one of the
following two forms. \end{lem}
\begin{description}
\item [{{{{(a)}}}}] The first form is 
\[
\mathbf{w}^{[1]}=\begin{bmatrix}\mathbf{w}_{1}^{[1]}\\
\vdots\\
\mathbf{w}_{n}^{[1]}
\end{bmatrix}\mbox{ with }\mathbf{w}_{i}^{[1]}=\begin{bmatrix}w_{i}\\
0\\
\vdots\\
0
\end{bmatrix}_{q_{i}\times1},
\]
  where $w_{i}=\bar{w}_{i,0}+\bar{w}_{i,1}x^{r_{i}}+\bar{w}_{i,2}x^{r_{i}+\beta}+\cdots+\bar{w}_{i,q_{i}-1}x^{r_{i}+(q_{i}-2)\beta}$. 

\begin{description}
\item [{{{{(b)}}}}] The second form is 
\[
\mathbf{w}^{[2]}=\begin{bmatrix}\mathbf{w}_{1}^{[2]}\\
\vdots\\
\mathbf{w}_{n}^{[2]}
\end{bmatrix}\mbox{ with }\mathbf{w}_{i}^{[2]}=\begin{bmatrix}w_{i,0}\\
\vdots\\
w_{i,q_{i}-1}
\end{bmatrix},
\]
 where $\deg w_{i,j}<r_{i}\le\beta$ when $j=0$ and $\deg w_{i,j}<\beta$
when $j\in\{1,\dots,q_{i}-2\}$. There is no degree restriction on
$w_{i,q_{i}-1}$. 
\end{description}
\end{description}
\begin{proof}
The first form is obtained by setting 
\[
\mathbf{u}^{[1]}=\begin{bmatrix}\mathbf{u}_{1}^{[1]}\\
\vdots\\
\mathbf{u}_{n}^{[1]}
\end{bmatrix}\mbox{ with }\mathbf{u}_{i}^{[1]}=\left[\begin{array}{r}
\bar{w}_{i,1}+\bar{w}_{i,2}x^{\beta}+\bar{w}_{i,3}x^{2\beta}+\cdots+\bar{w}_{i,q_{i}-1}x^{(q_{i}-2)\beta}\\
\bar{w}_{i,2}+\bar{w}_{i,3}x^{\beta}+\cdots+\bar{w}_{i,q_{i}-1}x^{(q_{i}-3)\beta}\\
\vdots~~~~~\\
\bar{w}_{i,q_{i}-1}
\end{array}\right].
\]
 Then $\bar{\mathbf{w}}+\mathbf{A}\mathbf{u}^{[1]}$ gives the first
form. Note that $\mathbf{u}_{i}^{[1]}$ is empty if $q_{i}=1$ and
$\bar{\mathbf{w}}_{i}=\mathbf{w}_{i}^{[1]}=[\bar{w}_{i,0}]$ is not
changed by the transformation.

The second form can be obtained based on the first form. Let 
\[
t_{i,j}=\begin{cases}
r_{i} & \mbox{if }j=0\\
\beta & \mbox{if }j\in\{1,\dots,q_{i}-2\}
\end{cases}
\]
 and write $w_{i}$ from the first form as 
\begin{equation}
w_{i}=w_{i,0}+w_{i,1}x^{r_{i}}+w_{i,2}x^{r_{i}+\beta}+\cdots+w_{i,q_{i}-1}x^{r_{i}+(q_{i}-2)\beta}\label{eq:wiSeparatedForm}
\end{equation}
 with $\deg w_{i,j}<t_{i,j}$ for $j<q_{i}-1$. Note that in general
$w_{i,j}\ne\bar{w}_{i,j}$, as $\deg\bar{w}_{i,j}$ may not be less
than $t_{i,j}$. Now set 
\[
\mathbf{v}=\begin{bmatrix}\mathbf{v}_{1}\\
\vdots\\
\mathbf{v}_{n}
\end{bmatrix}\mbox{ with }\mathbf{v}_{i}=\left[\begin{array}{r}
w_{i,1}+w_{i,2}x^{\beta}+w_{i,3}x^{2\beta}+\cdots+w_{i,q_{i}-1}x^{(q_{i}-2)\beta}\\
w_{i,2}+w_{i,3}x^{\beta}+\cdots+w_{i,q_{i}-1}x^{(q_{i}-3)\beta}\\
\vdots~~~~~\\
w_{i,q_{i}-1}
\end{array}\right]
\]
 and $\mathbf{u}^{[2]}=\mathbf{u}^{[1]}-\mathbf{v}$, which comes
from the unimodular transformation 
\[
\left[\bar{\mathbf{w}},\mathbf{A}\right]\left[\begin{array}{c|c}
1\\
\hline \mathbf{\mathbf{u}}^{[1]} & \mathbf{I}
\end{array}\right]\left[\begin{array}{c|c}
1\\
\hline -\mathbf{v} & \mathbf{I}
\end{array}\right]=\left[\bar{\mathbf{w}},\mathbf{A}\right]\left[\begin{array}{c|c}
1\\
\hline \mathbf{u}^{[1]}-\mathbf{v} & \mathbf{I}
\end{array}\right].
\]
 Then $\mathbf{w}^{[2]}=\bar{\mathbf{w}}+\mathbf{A}\mathbf{u}^{[2]}$
is in the second form. Again note that $\mathbf{v}_{i}$ and $\mathbf{u}_{i}^{[2]}$
are empty if $q_{i}=1$ and $\mathbf{w}_{i}^{[2]}=\bar{\mathbf{w}}_{i}=[\bar{w}_{i,0}]$.\end{proof}
\begin{lem}
\label{lem:degreeCorrespondence}Let $\bar{\mathbf{w}}\in\left\langle \left(\bar{\mathbf{F}},\sigma,\bar{s}\right)\right\rangle $
and $\mathbf{w}^{[2]}$ be in the second form. If $\deg_{\vec{s}}\mathbf{E}\bar{\mathbf{w}}>-\alpha$
or $\deg_{\bar{s}}\mathbf{w}^{[2]}>-\alpha$, then $\deg_{\vec{s}}\mathbf{E}\bar{\mathbf{w}}=\deg_{\bar{s}}\mathbf{w}^{[2]}$. \end{lem}
\begin{proof}
Consider the $i$th entry $w_{i}$ of $\mathbf{E}\bar{\mathbf{w}}$
and the %corresponding 
entries $\mathbf{w}_{i}^{[2]}=\left[w_{i,0},\dots,w_{i,q_{i}-1}\right]^{T}$
in $\mathbf{w}^{[2]}$. If $q_{i}=1$, then $w_{i}=w_{i,0}$ and the
corresponding shifts satisfies $s_{i}=\bar{s}_{\ell(i)}$, where $\ell(i)=\sum_{k=1}^{i}q_{k}$.
Hence $\deg w_{i}+s_{i}=\deg w_{i,0}+\bar{s}_{\ell(i)}$. Thus we
only need to consider the case where $q_{i}>1$. Write $w_{i}$ as
in Equation \prettyref{eq:wiSeparatedForm}. Note that $\deg w_{i,q_{i}-1}=\deg w_{i}-r_{i}-\beta\left(q_{i}-2\right)$
and hence $\deg w_{i,q_{i}-1}-\alpha-\beta=\deg w_{i}-r_{i}-\alpha-\beta\left(q_{i}-1\right)$,
that is, $\deg w_{i,q_{i}-1}+\bar{s}_{\ell(i)}=\deg w_{i}+s_{i}$.
It follows that 
\begin{eqnarray*}
\deg_{\vec{s}}\mathbf{E}\bar{\mathbf{w}} & = & \max_{i}(\deg w_{i}+s_{i})=\max_{i}(\deg w_{i,q_{i}-1}+\bar{s}_{\ell(i)})\\
 & \le & \max_{i,j}\left(\deg w_{i,j}+\bar{s}_{\ell(i-1)+j+1}\right)=\deg_{\bar{s}}\mathbf{w}^{[2]}.
\end{eqnarray*}
 The only possible indices $j$ where the inequality can be strict
occur when $j<q_{i}-1$. But $\deg w_{i,j}<\beta$ for all $j<q_{i}-1$,
which implies $\deg w_{i,j}+\bar{s}_{\ell(i-1)+j+1}=\deg w_{i,j}-\alpha-\beta<-\alpha$,
and so it follows that the entries at these indices $j$ do not contribute
to $\deg_{\bar{s}}\mathbf{w}^{[2]}$ when $\deg_{\bar{s}}\mathbf{w}^{[2]}>-\alpha$
or $\deg_{\vec{s}}\mathbf{E}\bar{\mathbf{w}}=\max_{i}(\deg w_{i,q_{i}-1}+\bar{s}_{\ell(i)})>-\alpha$.
In other words, if one of them exceeds $-\alpha$, then $\deg_{\bar{s}}\mathbf{w}^{[2]}$
and $\deg_{\vec{s}}\mathbf{E}\bar{\mathbf{w}}$ are determined only
by entries at indices $j=q_{i}-1$, but the equality always holds
for these entries. \end{proof}
\begin{rem}
Notice that the first form $\mathbf{w}^{\left[1\right]}$ of $\bar{\mathbf{w}}$
has nonzero entries only at indices $I=[1,q_{1}+1,\dots,\sum_{k=1}^{n-1}q_{k}+1]$.
Let $\mathbf{B}$ be a $\bar{n}\times n$ matrix with $1$'s at position
$(\sum_{k=1}^{n-1}q_{k}+1,i)$ and 0's everywhere else. Then the first
form satisfies $\mathbf{w}^{[1]}=\mathbf{B}\mathbf{E}\bar{\mathbf{w}}$.
Hence \prettyref{lem:degreeCorrespondence} provides the degree correspondence
between the degrees of the first form $\mathbf{B}\mathbf{E}\bar{\mathbf{w}}$,
which is just $\mathbf{E}\bar{\mathbf{w}}$ with zero rows added,
and the second form $\bar{\mathbf{w}}^{[2]}$ of $\bar{\mathbf{w}}$.\end{rem}
\begin{cor}
\label{cor:degreeCorrespondence}Let $\bar{\mathbf{w}}\in\left\langle \left(\bar{\mathbf{F}},\sigma,\bar{s}\right)\right\rangle $
and $\mathbf{w}^{[2]}$ be its second form. Then $\deg_{\vec{s}}\mathbf{E}\bar{\mathbf{w}}>-\alpha$
if and only if $\deg_{\bar{s}}\mathbf{w}^{[2]}>-\alpha$.\end{cor}
\begin{proof}
The proof follows directly from \prettyref{lem:degreeCorrespondence}.\end{proof}
\begin{lem}
\label{lem:degEwLessEqDegw}Let $\bar{\mathbf{w}}\in\left\langle \left(\bar{\mathbf{F}},\sigma,\bar{s}\right)\right\rangle $.
Then $\deg_{\vec{s}}\mathbf{E}\bar{\mathbf{w}}\le\deg_{\bar{s}}\bar{\mathbf{w}}$. \end{lem}
\begin{proof}
As in \prettyref{lem:degreeCorrespondence}, consider the $i$th entry
$w_{i}$ of $\mathbf{E}\bar{\mathbf{w}}$ and the corresponding entries
$\bar{\mathbf{w}}_{i}=\left[\bar{w}_{i,0},\dots,\bar{w}_{i,q_{i}-1}\right]^{T}$
in $\bar{\mathbf{w}}$. If $q_{i}=1$, then $\deg w_{i}+s_{i}=\deg w_{i,0}+\bar{s}_{\ell(i)}$
as before. Thus we just need to consider the case $q_{i}>1$, where
the shifts for $\bar{\mathbf{w}}_{i}$ are $-\alpha-\beta$. Since
$w_{i}=\bar{w}_{i,0}+\bar{w}_{i,1}x^{r_{i}}+\bar{w}_{i,2}x^{r_{i}+\beta}+\cdots+\bar{w}_{i,q_{i}-1}x^{r_{i}+(q_{i}-2)\beta},$
we get 
\[
\deg w_{i}=\max\left\{ \deg\bar{w}_{i,0},\deg\bar{w}_{i,1}+r_{i},\deg\bar{w}_{i,2}+r_{i}+\beta,\dots,\deg\bar{w}_{i,q_{i}-2}+r_{i}+(q_{i}-2)\beta\right\} .
\]
 Then 
\begin{eqnarray*}
\deg w_{i}+s_{i} & = & \deg w_{i}-r_{i}-\alpha-\beta(q_{i}-1)\\
 & = & \max\left\{ \deg\bar{w}_{i,0}-r_{i}-\alpha-\beta(q_{i}-1),~\deg\bar{w}_{i,1}-\alpha-\beta(q_{i}-1),~\dots,\right.\\
 &  & \left.~~~~~~~~~~~~~~\dots,\deg\bar{w}_{i,q_{i}-2}-\alpha-\beta\right\} \\
 & \le & \max\left\{ \deg\bar{w}_{i,0}-\alpha-\beta,\deg\bar{w}_{i,1}-\alpha-\beta,\dots,\deg\bar{w}_{i,q_{i}-2}-\alpha-\beta\right\} ,
\end{eqnarray*}
 and so $\deg_{\vec{s}}\mathbf{E}\bar{\mathbf{w}}\le\deg_{\bar{s}}\bar{\mathbf{w}}$.\end{proof}
\begin{cor}
\label{cor:P2Degree}Let $\bar{\mathbf{P}}=[\bar{\mathbf{P}}_{1},\bar{\mathbf{P}}_{2}]$
be a $\left(\bar{\mathbf{F}},\sigma,\bar{s}\right)$-basis, where
$\deg_{\bar{s}}\bar{\mathbf{P}}_{1}\le-\alpha$ and $\deg_{\bar{s}}\bar{\mathbf{P}}_{2}>-\alpha$.
Let $\bar{\mathbf{P}}_{2}^{[2]}$ be the second form of $\bar{\mathbf{P}}_{2}$.
Then $\deg_{\bar{s}}\bar{\mathbf{P}}_{2}=\deg_{\bar{s}}\bar{\mathbf{P}}_{2}^{[2]}=\deg_{\vec{s}}\mathbf{E}\bar{\mathbf{P}}_{2}$.
Hence $[\bar{\mathbf{P}}_{1},\bar{\mathbf{P}}_{2}^{[2]}]$ is also
a $(\bar{\mathbf{F}},\sigma,\bar{s})$-basis.\end{cor}
\begin{proof}
Since any column $\bar{\mathbf{p}}$ of $\bar{\mathbf{P}}_{2}$ satisfies
$\deg_{\bar{s}}\bar{\mathbf{p}}>-\alpha,$ from \prettyref{lem:degreeCorrespondence}
and \prettyref{lem:degEwLessEqDegw}, we get 
\[
\deg_{\bar{s}}\bar{\mathbf{p}}^{[2]}=\deg_{\vec{s}}\mathbf{E}\bar{\mathbf{p}}\le\deg_{\bar{s}}\bar{\mathbf{p}}.
\]
 The inequality is in fact an equality, since otherwise, $\bar{\mathbf{p}}$
in $\bar{\mathbf{P}}$ can be replaced by $\bar{\mathbf{p}}^{[2]}$
to get a basis of lower degree, contradicting the minimality of $\bar{\mathbf{P}}$.
Note that $\bar{\mathbf{P}}$ with its column $\bar{\mathbf{p}}$
replaced by $\bar{\mathbf{p}}^{[2]}$ remains to be a $\left(\bar{\mathbf{F}},\sigma,\bar{s}\right)$-basis,
since $\bar{\mathbf{p}}^{[2]}=\bar{\mathbf{p}}+\mathbf{A}\mathbf{u}$
involves column operations with only columns in $\bar{\mathbf{P}}_{1}$
as $\mathbf{A}$ has $\bar{s}$-degrees bounded by $-\alpha$ and
hence is generated by $\bar{\mathbf{P}}_{1}$.\end{proof}
\begin{lem}
\label{lem:PtoBPA}If $\mathbf{P}$ is a $\left(\mathbf{F},\sigma,\vec{s}\right)$-basis,
then $\left[\mathbf{B}\mathbf{P},\mathbf{A}\right]$ is a basis for
$\left\langle \left(\bar{\mathbf{F}},\sigma,\bar{s}\right)\right\rangle $.\end{lem}
\begin{proof}
Any $\bar{\mathbf{w}}\in\left\langle \left(\bar{\mathbf{F}},\sigma,\bar{s}\right)\right\rangle $
can be transformed by $\mathbf{A}$ to the first form 
\[
\mathbf{w}^{[1]}=\bar{\mathbf{w}}+\mathbf{A}\mathbf{u}^{[1]}=\mathbf{B}\mathbf{E}\bar{\mathbf{w}},
\]
 where $\mathbf{E}\bar{\mathbf{w}}\in\left\langle \left(\mathbf{F},\sigma,\vec{s}\right)\right\rangle $
is generated by $\mathbf{P}$. That is, 
\[
\bar{\mathbf{w}}=\mathbf{w}^{[1]}-\mathbf{A}\mathbf{u}^{[1]}=\mathbf{B}\mathbf{E}\bar{\mathbf{w}}-\mathbf{A}\mathbf{u}^{[1]}=\mathbf{B}\mathbf{P}\mathbf{v}-\mathbf{A}\mathbf{u}^{[1]}=\left[\mathbf{B}\mathbf{P},\mathbf{A}\right][\mathbf{v},-\mathbf{u}^{[1]}]^{T}.
\]
 One can also see that the columns of $\mathbf{A}$ and the columns
of $\mathbf{B}\mathbf{P}$ are linearly independent, as each zero
row of $\mathbf{B}\mathbf{P}$ has a $-1$ from a column of $\mathbf{A}$. 
\end{proof}
\begin{comment}
\begin{lem}
If $\mathbf{P}_{1}$ is a $\left(\mathbf{F},\sigma,\vec{s}\right)_{-\alpha}$-basis
, then $\left[\mathbf{B}\mathbf{P}_{1},\mathbf{A}\right]$ is a basis
for $\left\langle \left(\bar{\mathbf{F}},\sigma,\bar{s}\right)\right\rangle _{-\alpha}$.\end{lem}
\begin{proof}
We know that if $\bar{\mathbf{w}}$ has order $\left(\bar{\mathbf{F}},\sigma\right)$
then $\mathbf{E}\bar{\mathbf{w}}$ has order $\left(\vec{\mathbf{F}},\sigma\right)$.
Also if $\deg_{\bar{s}}\bar{\mathbf{w}}\le-\alpha$, then $\deg_{\vec{s}}\mathbf{E}\bar{\mathbf{w}}\le-\alpha$
by \prettyref{lem:degEwLessEqDegw}. Therefore, if $\bar{\mathbf{w}}\in\left\langle \left(\bar{\mathbf{F}},\sigma,\bar{s}\right)\right\rangle _{-\alpha}$
, then $\mathbf{E}\bar{\mathbf{w}}\in\left\langle \left(\mathbf{F},\sigma,\vec{s}\right)\right\rangle _{-\alpha}$.
Now apply the same procedure as in \prettyref{lem:PtoBPA}, we get
$\bar{\mathbf{w}}=\mathbf{w}^{[1]}-\mathbf{A}\mathbf{u}^{[1]}=\mathbf{B}\mathbf{E}\bar{\mathbf{w}}-\mathbf{A}\mathbf{u}^{[1]}=\mathbf{B}\mathbf{P}_{1}\mathbf{v}-\mathbf{A}\mathbf{u}^{[1]}=\left[\mathbf{B}\mathbf{P}_{1},\mathbf{A}\right][\mathbf{v},-\mathbf{u}^{[1]}]^{T}.$
Also as before, the columns of $\mathbf{A}$ and the columns of $\mathbf{B}\mathbf{P}_{1}$
are linearly independent. \end{proof}
\end{comment}
{} 
\begin{lem}
\label{lem:EPgeneration}If $\bar{\mathbf{P}}$ is a $\left(\bar{\mathbf{F}},\sigma,\bar{s}\right)$-basis,
then $\mathbf{E}\bar{\mathbf{P}}$ generates $\left\langle \left(\mathbf{F},\sigma,\vec{s}\right)\right\rangle $.
That is, for any $\mathbf{w}\in\left\langle \left(\mathbf{F},\sigma,\vec{s}\right)\right\rangle $,
there is an $\mathbf{u}\in\mathbb{K}\left[x\right]^{\bar{n}\times1}$
such that $\mathbf{w}=\mathbf{E}\bar{\mathbf{P}}\mathbf{u}$.\end{lem}
\begin{proof}
For any $\left(\mathbf{F},\sigma,\vec{s}\right)$-basis $\mathbf{P}$,
the columns of $\mathbf{B}\mathbf{P}$ are in $\langle(\bar{\mathbf{F}},\sigma,\bar{s})\rangle$
generated by $\bar{\mathbf{P}}$, that is, $\mathbf{B}\mathbf{P}=\bar{\mathbf{P}}\mathbf{U}$
for some $\mathbf{U}\in\mathbb{K}[x]^{\bar{n}\times n}$. Hence $\mathbf{E}\mathbf{B}\mathbf{P}=\mathbf{P}$
is generated by $\mathbf{E}\bar{\mathbf{P}}$. That is, $\mathbf{P}=\mathbf{E}\bar{\mathbf{P}}\mathbf{U}$.
Then any $\mathbf{w}\in\left\langle \left(\mathbf{F},\sigma,\vec{s}\right)\right\rangle $,
which satisfies $\mathbf{w}=\mathbf{P}\mathbf{v}$ for some $\mathbf{v}\in\mathbb{K}[x]^{n\times1}$,
satisfies $\mathbf{w}=\mathbf{E}\bar{\mathbf{P}}\mathbf{U}\mathbf{v}$. 
\end{proof}
\begin{comment}
\begin{lem}
\label{lem:EP1generation}If $\bar{\mathbf{P}}_{1}$ is a $\left(\bar{\mathbf{F}},\sigma,\bar{s}\right)_{-\alpha}$-basis,
then $\mathbf{E}\bar{\mathbf{P}}_{1}$ generates $\left\langle \left(\mathbf{F},\sigma,\vec{s}\right)\right\rangle _{-\alpha}$. \end{lem}
\begin{proof}
Suppose $\mathbf{P}_{1}$ is a $\left(\mathbf{F},\sigma,\vec{s}\right)_{-\alpha}$-basis,
then the columns of $\mathbf{B}\mathbf{P}_{1}$ can be transformed
to the second form with $\bar{s}$-degrees less than or equal to$-\alpha$
by \prettyref{cor:degreeCorrespondence}, hence generated by $\bar{\mathbf{P}}_{1}$.
That is, $\mathbf{B}\mathbf{P}_{1}+\mathbf{A}\mathbf{U}$ is generated
by $\bar{\mathbf{P}}_{1}$. Therefore, $\mathbf{E}(\mathbf{B}\mathbf{P}_{1}+\mathbf{A}\mathbf{U})=\mathbf{E}\mathbf{B}\mathbf{P}_{1}=\mathbf{P}_{1}$
is generated by $\mathbf{E}\bar{\mathbf{P}}_{1}$. \end{proof}
\end{comment}


We are now ready to prove the main result on the correspondence between
a high degree part of a basis of the transformed problem and that
of the original problem. 
\begin{thm}
\label{thm:correctHighDegreeElements}Let $\bar{\mathbf{P}}=[\bar{\mathbf{P}}_{1},\bar{\mathbf{P}}_{2}]$
be a $\left(\bar{\mathbf{F}},\sigma,\bar{s}\right)$-basis, where
$\deg_{\bar{s}}\bar{\mathbf{P}}_{1}\le-\alpha$ and $\deg_{\bar{s}}\bar{\mathbf{P}}_{2}>-\alpha$.
Then $\mathbf{E}\bar{\mathbf{P}}_{2}$ is the matrix of the columns
of a $\left(\mathbf{F},\sigma,\vec{s}\right)$-basis whose $\vec{s}$-degrees
exceed $-\alpha$. \end{thm}
\begin{proof}
We want to show that $[\mathbf{P}_{1},\mathbf{E}\bar{\mathbf{P}}_{2}]$
is a $\left(\mathbf{F},\sigma,\vec{s}\right)$-basis for any $\left(\mathbf{F},\sigma,\vec{s}\right)_{-\alpha}$-basis
$\mathbf{P}_{1}$. First, $\mathbf{E}\bar{\mathbf{P}}$ has order
$(\mathbf{F},\sigma)$ since $\bar{\mathbf{F}}\bar{\mathbf{P}}=\mathbf{F}\mathbf{E}\bar{\mathbf{P}}$
and $\bar{\mathbf{P}}$ has order $\left(\bar{\mathbf{F}},\sigma\right)$.
Also, since $\mathbf{E}\bar{\mathbf{P}}$ generates $\left\langle \left(\mathbf{F},\sigma,\vec{s}\right)\right\rangle $
by \prettyref{lem:EPgeneration}, and from \prettyref{cor:degreeCorrespondence}
$\mathbf{E}\bar{\mathbf{P}}_{1}$ has $\vec{s}$-degree bounded by
$-\alpha$ hence is generated by $\mathbf{P}_{1}$, it follows that
$\left[\mathbf{P}_{1},\mathbf{E}\bar{\mathbf{P}}_{2}\right]$ generates
$\left\langle \left(\mathbf{F},\sigma,\vec{s}\right)\right\rangle $.

It only remains to show that the $\vec{s}$-degrees of $\mathbf{E}\bar{\mathbf{P}}_{2}$
are minimal. Suppose not, then $[\mathbf{P}_{1},\mathbf{E}\bar{\mathbf{P}}_{2}]$
can be reduced to $[\mathbf{P}_{1},\tilde{\mathbf{P}}_{2}]$ where
$\tilde{\mathbf{P}}_{2}$ has a column having lower $\vec{s}$-degree
than that of the corresponding column in $\mathbf{E}\bar{\mathbf{P}}_{2}$.
That is, assuming the columns of $\tilde{\mathbf{P}}_{2}$ and $\mathbf{E}\bar{\mathbf{P}}_{2}$
are in non-decreasing $\vec{s}$-degrees order, then we can find the
first index $i$ where the $\vec{s}$-degree of $i$th column of $\tilde{\mathbf{P}}_{2}$
is lower than the $\vec{s}$-degree of the $i$th column of $\mathbf{E}\bar{\mathbf{P}}_{2}$.
It follows that $[\mathbf{B}\mathbf{P}_{1},\mathbf{B}\mathbf{E}\bar{\mathbf{P}}_{2}]$
can be reduced to $[\mathbf{B}\mathbf{P}_{1},\mathbf{B}\tilde{\mathbf{P}}_{2}]$
and $[\mathbf{B}\mathbf{P}_{1},\mathbf{B}\mathbf{E}\bar{\mathbf{P}}_{2},\mathbf{A}]$
can be reduced to $[\mathbf{B}\mathbf{P}_{1},\mathbf{B}\tilde{\mathbf{P}}_{2},\mathbf{A}]$.
Since $[\mathbf{B}\mathbf{P}_{1},\mathbf{B}\tilde{\mathbf{P}}_{2},\mathbf{A}]$
generates $\langle(\bar{\mathbf{F}},\sigma,\bar{s})\rangle$ by \prettyref{lem:PtoBPA},
it can be reduced to $\bar{\mathbf{P}}=[\bar{\mathbf{P}}_{1},\bar{\mathbf{P}}_{2}]$.
But it can also be reduced to $[\bar{\mathbf{P}}_{1},\tilde{\mathbf{P}}_{2}^{[2]},\mathbf{A}]$
with $\tilde{\mathbf{P}}_{2}^{[2]}$ the second form of $\mathbf{B}\tilde{\mathbf{P}}_{2}$,
and to $[\bar{\mathbf{P}}_{1},\tilde{\mathbf{P}}_{2}^{[2]}]$ as the
columns of $\mathbf{A}$ are generated by the $\left(\bar{\mathbf{F}},\sigma,\bar{s}\right)_{-\alpha}$-basis
$\bar{\mathbf{P}}_{1}$.

In order to reach a contradiction we just need to show that $\tilde{\mathbf{P}}_{2}^{[2]}$
has a column with $\bar{s}$-degree less than that of the corresponding
column in $\bar{\mathbf{P}}_{2}$. Let $\tilde{\mathbf{w}}$ be the
first column of $\tilde{\mathbf{P}}_{2}$ with $\vec{s}$-degree less
than that of the corresponding column $\mathbf{w}$ in $\mathbf{E}\bar{\mathbf{P}}_{2}$
and let $\bar{\mathbf{w}}$ be the corresponding column in $\bar{\mathbf{P}}_{2}$.
By \prettyref{cor:P2Degree} $\deg_{\vec{s}}\mathbf{w}=\deg_{\bar{s}}\bar{\mathbf{w}}$.
Let $\tilde{\mathbf{w}}^{[2]}$ be the second form of $\mathbf{B}\tilde{\mathbf{w}}$,
which is a column in $\tilde{\mathbf{P}}_{2}^{[2]}$ corresponding
to the column $\bar{\mathbf{w}}$ in $\bar{\mathbf{P}}_{2}$. We know
that either $\deg_{\bar{s}}\tilde{\mathbf{w}}^{[2]}\le-\alpha$ or
$\deg_{\bar{s}}\tilde{\mathbf{w}}^{[2]}=\deg_{\vec{s}}\tilde{\mathbf{w}}$
by \prettyref{lem:degreeCorrespondence}, as $\mathbf{E}\tilde{\mathbf{w}}^{[2]}=\mathbf{E}(\mathbf{B}\tilde{\mathbf{w}}+\mathbf{A}\mathbf{u})=\tilde{\mathbf{w}}$.
In either case, $\deg_{\bar{s}}\tilde{\mathbf{w}}^{[2]}<\deg_{\bar{s}}\bar{\mathbf{w}}$,
as $\deg_{\bar{s}}\bar{\mathbf{w}}$ is greater than both $-\alpha$
and $\deg_{\vec{s}}\tilde{\mathbf{w}}$. Hence we have $[\bar{\mathbf{P}}_{1},\tilde{\mathbf{P}}_{2}^{[2]}]$
is another $\left(\bar{\mathbf{F}},\sigma,\bar{s}\right)$-basis with
lower $\bar{s}$-degrees than $\mathbf{\bar{P}}$, contradicting with
the minimality of $\bar{\mathbf{P}}$. 
\end{proof}

\subsection{Achieving Efficient Computation}

\prettyref{thm:correctHighDegreeElements} essentially tells us that
a high degree part of a $\left(\mathbf{F},\sigma,\vec{s}\right)$-basis
can be determined by computing a $\left(\bar{\mathbf{F}},\sigma,\bar{s}\right)$-basis,
something we know can be done efficiently. Notice the parallel between
the situation here and in the earlier balanced shift case, where the
transformed problem also allows us to compute a partial $\left(\mathbf{F},\sigma,\vec{s}\right)$-basis,
albeit a low degree part, in each iteration.

After a $\left(\bar{\mathbf{F}},\sigma,\bar{s}\right)$-basis, or
equivalently a high degree part of a $\left(\mathbf{F},\sigma,\vec{s}\right)$-basis,
is computed, for the remaining problem of computing the remaining
basis elements, we can in fact reduce the dimension of the input $\mathbf{F}$
by removing some of its columns corresponding to the high shift entries. 
\begin{thm}
\label{thm:zeroHighShiftEntries}Suppose without loss of generality
that the entries of $\vec{s}$ are in non-decreasing order. Let $I$
be the index set containing the indices of entries $s_{i}$ in $\vec{s}$
such that $s_{i}\le-\alpha$. Let $\mathbf{F}_{I}$ be the columns
of $\mathbf{F}$ indexed by $I$. Then a $\left(\mathbf{F}_{I},\sigma,\vec{s}\right)_{-\alpha}$-basis
\textbf{$\mathbf{P}_{1}$} gives a $\left(\mathbf{F},\sigma,\vec{s}\right)_{-\alpha}$-basis
$\left[\mathbf{P}_{1}^{T},\mathbf{0}\right]^{T}$. 
\end{thm}
%The theorem follows from the following claim.
%\end{pf}
%\begin{claim}
%\label{cla:zeroHighShiftEntries}

\begin{proof}
For any $\mathbf{p}\in\mathbb{K}\left[x\right]^{n\times1}$ and $\deg_{\vec{s}}\mathbf{p}\le-\alpha$,
note that if the $i$th entry of the shift satisfies $s_{i}\le-\alpha$,
then the corresponding entry $p_{i}$ of $\mathbf{p}$ is zero. %\end{claim}
%\begin{pf}
Otherwise, if $p_{i}\ne0$ then the $\vec{s}$-degree of $\mathbf{p}$
is at least $s_{i}>-\alpha$, contradicting the assumption that the
$\vec{s}$-degree of $\mathbf{p}$ is lower than or equal to $-\alpha$. 
\end{proof}
Thus, these zero entries do not need to be considered in the remaining
problem of computing a $\left(\mathbf{F},\sigma,\vec{s}\right)_{-\alpha}$-basis.
As such the corresponding columns from the input matrix $\mathbf{F}$
can be removed. 
\begin{example}
Let us return to \prettyref{exm:unbalancedShift}. When the parameters
$\alpha=\beta=1$, after computing an element of a $(\mathbf{F},8,\vec{s})$-basis
with $\vec{s}$-degree $0$ that exceeds $-\alpha=-1$, the first
row of any $\left(\mathbf{F},\sigma,\vec{s}\right)_{-1}$-basis must
be zero by \prettyref{thm:zeroHighShiftEntries} (since the first
entry of $\vec{s}=[0,-3,-5,-6]$ is $0>-\alpha$). This is illustrated
by the $(\mathbf{F},8,\vec{s})$-basis $\mathbf{P}$ given in \prettyref{exm:unbalancedShift}.
This implies that the first column of $\mathbf{F}$ is not needed
in the subsequent computation of the remaining basis elements.\end{example}
\begin{cor}
\label{cor:numberBasisElements}If the shift $\vec{s}$ satisfies
condition (\ref{con:unbalancedCondition}) and $c$ is a constant
greater than or equal to $1$, then a $\left(\mathbf{F},\sigma,\vec{s}\right)_{-cd}$-basis
has at most $n/c$ basis elements.\end{cor}
\begin{proof}
Since $d=m\sigma/n\ge-\sum_{i=1}^{n}s_{i}/n$ under condition (\ref{con:unbalancedCondition}),
there cannot be more than $n/c$ entries of $\vec{s}$ less than or
equal to $-cd$. By \prettyref{thm:zeroHighShiftEntries}, the only
possible nonzero rows of a $\left(\mathbf{F},\sigma,\vec{s}\right)_{-cd}$-basis
are the ones corresponding to (with the same indices as) the shift
entries that are less than or equal to $-cd$. Hence there cannot
be more than $n/c$ nonzero rows and at most $n/c$ columns, as the
columns are linearly independent. 
\end{proof}
We now have a situation similar to that found in the balanced shift
case. Namely, for each iteration we transform the problem using appropriate
parameters $\alpha$ and $\beta$ to efficiently compute the basis
elements with degrees greater than $-\alpha$. Then we can remove
columns from the input matrix $\mathbf{F}$ corresponding to the shift
entries that are greater than $-\alpha$. We can then repeat the same
process again, with a larger $\alpha$ and $\beta$, in order to compute
more basis elements. 
\begin{thm}
\label{thm:unbalancedOrderBasisCost}If the shift $\vec{s}$ satisfies
condition (\ref{con:unbalancedCondition}), then a $\left(\mathbf{F},\sigma,\vec{s}\right)$-basis
can be computed with cost $O\left(n^{\omega}\bar{\M}(d)\log\sigma\right)=O(n^{\omega}d\log d\log\log d\log\sigma)\subset O^{\sim}(n^{\omega}d)$. \end{thm}
\begin{proof}
We give the following constructive proof. Initially, we set transformation
parameters $\alpha_{1}=\beta_{1}=2d$ with $d=m\sigma/n\ge-\sum_{i=1}^{n}s_{i}/n$.
\prettyref{alg:mab} works efficiently on the transformed problem
as the shift $\bar{s}^{\left(1\right)}$ is balanced and the dimension
of $\bar{\mathbf{F}}_{1}$ remains $O\left(n\right)$. By \prettyref{thm:correctHighDegreeElements}
this gives the basis elements of $\left(\mathbf{F},\sigma,\vec{s}\right)$-basis
with $\vec{s}$-degree exceeding $-\alpha_{1}=-2d$. By \prettyref{cor:numberBasisElements},
the number of basis elements remaining to be computed is at most $n/2$,
hence the number of elements correctly computed is at least $n/2$.
By \prettyref{thm:zeroHighShiftEntries}, this also allows us to remove
at least half of the columns from the input $\mathbf{F}$ and correspondingly
at least half of the rows from the output for the remaining problem.
Thus the new input matrix $\mathbf{F}_{2}$ has a new column dimension
$n_{2}\le n/2$ and the corresponding shift $\vec{s}^{\left(2\right)}$
has $n_{2}$ entries. The average degree of the new problem is $d_{2}~=~m\sigma/n_{2}$.

For the second iteration, we set $\alpha_{2}$ and $\beta_{2}$ to
$2d_{2}$. Since 
\[
\alpha_{2}=2m\sigma/n_{2}\ge-2\sum_{i=1}^{n}s_{i}/n_{2}\ge-2\sum_{i=1}^{n_{2}}s_{i}^{\left(2\right)}/n_{2},
\]
 this allows us to reduce the dimension $n_{3}$ of $\mathbf{F}_{3}$
to at most $n_{2}/2$ after finishing computing a $\left(\bar{\mathbf{F}}_{2},\sigma,\bar{s}^{\left(2\right)}\right)_{-\alpha_{1}}$-basis.
Again, this can be done using \prettyref{alg:mab} with a cost of
$O\left(n_{2}^{\omega}\bar{\M}(d_{2})\log\sigma\right)$ as the shift
$\bar{a}_{2}$ is balanced and the dimension of $\bar{\mathbf{F}}_{2}$
is $O\left(n_{2}\right)$. Repeating this process, at iteration $i$,
we set $\alpha_{i}=\beta_{i}=2d_{i}=2m\sigma/n_{i}$. The transformed
problem has a balanced shift $\bar{a}_{i}$ and column dimension $O\left(n_{i}\right)$.
So a $\left(\bar{\mathbf{F}}_{i},\sigma,\bar{s}^{\left(i\right)}\right)_{-\alpha_{i-1}}$-basis
can be computed with a cost of 
\[
O\left(n_{i}^{\omega}\bar{\M}\left(d_{i}\right)\log\sigma\right)\subset O\left(\left(2^{-i}n\right)^{\omega}\bar{\M}\left(2^{i}d\right)\log\sigma\right)\subset O\left(2^{-i}n^{\omega}\bar{\M}(d)\log\sigma\right).
\]
 Since 
\[
\alpha_{i}=2m\sigma/n_{i}\ge-2\sum_{i=1}^{n}s_{i}/n_{i}\ge-2\sum_{i=1}^{n_{i}}s_{i}^{\left(i\right)}/n_{i},
\]
 the column dimension $n_{i+1}$ of the next problem can again be
reduced by a half. After iteration $i$, at most $n/2^{i}$ $\left(\mathbf{F},\sigma,\vec{s}\right)$-basis
elements remain to be computed. We can stop this process when the
column dimension $n_{i}$ of the input matrix $\mathbf{F}_{i}$ reaches
the row dimension $m$, as an order basis can be efficiently computed
in such case. Therefore, a complete $\left(\mathbf{F},\sigma,\vec{s}\right)$-basis
can be computed in at most $\log(n/m)$ iterations, so the overall
cost is 
\[
O\left(\sum_{i=1}^{\log(n/m)}\left(2^{-i}n^{\omega}\bar{\M}(d)\log\sigma\right)\right)=O\left(n^{\omega}\bar{\M}(d)\log\sigma\sum_{i=1}^{\log(n/m)}2^{-i}\right)\subset O\left(n^{\omega}\bar{\M}(d)\log\sigma\right)
\]
 field operations. 
\end{proof}
\begin{algorithm}
\caption{$\umab\left(\mathbf{F},\sigma,\vec{s}\right)$ }


\label{alg:umab} 
\begin{algor}
\item [{{{*}}}] Input: $\mathbf{F}\in K\left[x\right]^{m\times n}$,
$\sigma\in\mathbb{Z}_{\ge0}$, $\vec{s}$ satisfies condition (\ref{con:unbalancedCondition}).
\item [{{{*}}}] Output: $\mathbf{P}\in K\left[x\right]^{n\times n}$,
an $\left(\mathbf{F},\sigma,\vec{s}\right)$-basis.
\item [{{{*}}}] Uses:
\item [{{*}}] (a) $\TransformUnbalanced$ : converts an unbalanced shift
problem to a balanced one using the transformation described in \prettyref{chap:Unbalanced-Shift}.
Returns transformed input matrix, transformed shift, and transformation
matrix.
\item [{{*}}] (b) %{\em OrderBasis} 
$\mmab$ : computes order basis with balanced shift. 
\item [{{*}}]~\end{algor}
\begin{algor}[1]
\item [{{{*}}}]  $i:=1;$ $\mathbf{P}=[\,]$;
\item [{{{*}}}] $\mathbf{F}^{\left(i\right)}:=\mathbf{F}$, $\vec{s}^{\left(i\right)}:=\vec{s}$;
\item [{{while}}] $\columnDimension(\mathbf{P})\ne n$
\item [{{{*}}}] $d_{i}=\left\lceil m\sigma/\columnDimension(\mathbf{F}^{\left(i\right)})\right\rceil $;
\item [{{{*}}}] $\alpha_{i}:=\beta_{i}:=2d_{i};$
\item [{{{*}}}] $\left[\bar{\mathbf{F}}^{\left(i\right)},\bar{s}^{\left(i\right)},\mathbf{E}\right]:=\TransformUnbalanced\left(\mathbf{F}^{\left(i\right)},\vec{s}^{\left(i\right)},\alpha_{i},\beta_{i}\right)$;
\item [{{{*}}}] $\bar{\mathbf{P}}^{\left(i\right)}:=\mmab\left(\bar{\mathbf{F}}^{\left(i\right)},\sigma,\bar{s}^{\left(i\right)}\right)$;
\item [{{{*}}}] Set $\mathbf{P}^{\left(i\right)}$ to be the columns
of $\mathbf{E}\bar{\mathbf{P}}^{\left(i\right)}$ with $\bar{s}_{i}$-column
degrees in $(-\alpha_{i},-\alpha_{i-1}]$;
\item [{{{*}}}] $\mathbf{P}:=\left[\mathbf{P}^{\left(i\right)},\mathbf{P}\right]$;
\item [{{{*}}}] Set $I$ as the set of indices $i$ satisfying $s_{i}\le-\alpha_{i}$;
\item [{{{*}}}] $\mathbf{F}^{\left(i+1\right)}:=\mathbf{F}_{I}^{\left(i\right)}$,
$\vec{s}^{\left(i+1\right)}:=\vec{s}_{I}^{\left(i\right)}$;
\item [{{{*}}}] $i:=i+1$;
\item [{{endwhile}}] ~
\item [{{{*}}}] return $\mathbf{P}$ ; \end{algor}
\end{algorithm}

 Finally, we remark that when the condition
(\ref{con:unbalancedCondition}) is relaxed to $\sum_{i=1}^{n}-s_{i}\in O\left(m\sigma\right)$,
so that $\sum_{i=1}^{n}-s_{i}\le cm\sigma$ for a constant $c$, we
can still compute a $\left(\mathbf{F},\sigma,\vec{s}\right)$-basis
with the same complexity, by setting $\alpha_{i}=\beta_{i}=2cm\sigma/n_{i}$
at each iteration $i$ and following the same procedure as above.
The cost at each iteration $i$ remains $O^{\sim}\left(n^{\omega}d\right)$,
and the entire computation still uses at most $\log(n/m)$ iterations. 




\chapter{Kernel Basis\label{chap:NullspaceBasis}}

In this chapter we discuss the computation of minimal kernel bases.

Minimal kernel bases can be directly computed via order basis computation.
Indeed if the order $\sigma$ of a $\left(\mathbf{F},\sigma,\vec{s}\right)$-basis
$\mathbf{P}$ is high enough, then $\mathbf{P}$ contains a $\vec{s}$-minimal
kernel basis $\mathbf{N}$. However, this approach may require the
order $\sigma$ to be quite high. For example, if $\mathbf{F}$ has
degree $d$ and $\vec{s}$ is uniform, then its minimal kernel bases
can have degree up to $md$. In that case the order $\sigma$ would
need to be set to $d+md$ in the order basis computation in order
to fully compute a minimal kernel basis. The fastest method of computing
such a $\left(\mathbf{F},d+md\right)$-basis would cost $O^{\sim}\left(n^{\omega-1}m^{2}d\right)$
using the algorithm from \citep{za2009}. We can see from this cost
that there is room for improvement when $m$ is large. For example,
in the worst case when $m\in\Theta\left(n\right)$, this cost would
be $O^{\sim}\left(n^{\omega+1}d\right)$. This points to a root cause
for the inefficiency in this approach. Namely, when $m$ is large,
the computed kernel basis, which can have a column dimension of $n-m$,
is a much smaller subset of the order basis computed. Hence considerable
effort is put in the computation of order basis elements that are
not part of a kernel basis. A key to reducing the cost is therefore
to reduce such computation of unneeded order basis elements, which
is achieved in our algorithm by only using order basis computation
to compute partial kernel bases of low degrees.


\section{Minimal Kernel Basis Computation}

\label{sec:Nullspace-Basis-Computation}

In this section, we describe a new, efficient algorithm for computing
a shifted minimal kernel basis. The algorithm uses two computation
processes recursively. The first process, described in \prettyref{sub:continueComputingNullspaceBasisByColumns},
uses an order basis computation to compute a subset of kernel basis
elements of lower degree, and results in a new problem of lower column
dimension. The second process, described in \prettyref{sub:continueComputingNullspaceBasisByRows},
reduces the row dimension of the problem by computing a kernel basis
of a submatrix formed by a subset of the rows of the input matrix.

We assume that the row dimension $m$ is bounded by the column dimension
$n$ in this chapter. But this assumption is later removed in \prettyref{sub:removeDimensionAssumption}
with results from \prettyref{chap:rank}.

We require that the shift $\vec{s}$ bounds the column degrees of
$\mathbf{F}$ component-wise. For example, we can set each entry of
$\vec{s}$ to be the corresponding column degree of $\mathbf{F}$,
or we can simply set each entry of $\vec{s}$ to be the maximum column
degree of $\mathbf{F}$. %As it may become evident later in this section, this condition on
%the shift is very useful, as it provides a simple bound on the column
%degrees of $\mathbf{F} \cdot \mathbf{A}$ for any polynomial matrix $\mathbf{A}$. 
 This is a very useful condition as it helps us to keep track of and
bound the shifted degrees throughout the kernel basis computation,
as we will see in \prettyref{sub:boundsBasedOnShift}. In the remainder
of this thesis, when we say a list $\vec{s}=\left[s_{1},\dots,s_{n}\right]$
bounds another list $\vec{t}=\left[t_{1},\dots,t_{n}\right]$ or if
we write $\vec{s}\ge\vec{t}$, we mean $\vec{s}$ bound $\vec{t}$
component-wise, that is, each $s_{i}\ge t_{i}$ for each $i$ in $\left\{ 1,\dots,n\right\} $. 

For simplicity, we will also assume without loss of generality that
the columns of $\mathbf{F}$ and the corresponding entries of $\vec{s}=\left[s_{1},\dots,s_{n}\right]$
are arranged so that the entries of $\vec{s}$ are in increasing order.

Let $\rho=\sum_{n-m+1}^{n}s_{i}$ be the sum of $m$ largest entries
of $\vec{s}$, and $s=\rho/m$ be their average. The algorithm we
present in this section computes a $\vec{s}$-minimal null space basis
$\mathbf{N}$ with a cost of $O^{\sim}(n^{\omega}s)$ field operations.
For uniform shift $\vec{s}=\left[s,\dots,s\right]$, we improve this
later to $O^{\sim}\left(n^{\omega-1}ms\right)=O^{\sim}\left(n^{\omega-1}\rho\right)$.


\subsection{\label{sub:boundsBasedOnShift}Bounds based on the shift}

A key requirement for efficient computation is making sure that the
intermediate computations do not blow up in size. We will see that
this requirement is satisfied by the existence of a bound, $\xi=\sum\vec{s}=\sum_{i=1}^{n}{s}_{i},$
\begin{comment}
the sum of all entries of the initial input shift $\vec{s}$, 
\end{comment}
{} on the sum of all entries of the input shift of all subproblems throughout
the computation.%
\begin{comment}
, as we will see later in \prettyref{lem:boundOfSumOfShiftedDegreesOfOrderBasis},
\prettyref{thm:boundOfSumOfShiftedDegreesOfKernelBasis}, and \prettyref{lem:boundOnShiftedDegrees}
that this quantity bounds the sum of all entries of the input shift
of all subproblems throughout the computation,. 
\end{comment}
{} 



The following lemma shows that any $(\mathbf{F},\sigma,\vec{s})$-basis
contains a partial $\vec{s}$-minimal kernel basis of $\mathbf{F}$,
and as a result, any $(\mathbf{F},\sigma,\vec{s})$-basis with high
enough $\sigma$ contains a $\vec{s}$-minimal kernel basis of $\mathbf{F}$. 
\begin{lem}
\label{lem:orderBasisContainsNullspaceBasis}Let $\mathbf{P}=\left[\mathbf{P}_{1},\mathbf{P}_{2}\right]$
be any $\left(\mathbf{F},\sigma,\vec{s}\right)$-basis and $\mathbf{N}=\left[\mathbf{N}_{1},\mathbf{N}_{2}\right]$
be any $\vec{s}$-minimal kernel basis of $\mathbf{F}$, where $\mathbf{P}_{1}$
and $\mathbf{N}_{1}$ contain all columns from $\mathbf{P}$ and $\mathbf{N}$,
respectively, whose $\vec{s}$-column degrees are less than $\sigma$.
Then $\left[\mathbf{P}_{1},\mathbf{N}_{2}\right]$ is a $\vec{s}$-minimal
kernel basis of $\mathbf{F}$, and $\left[\mathbf{N}_{1},\mathbf{P}_{2}\right]$
is a $\left(\mathbf{F},\sigma,\vec{s}\right)$-basis.\end{lem}
\begin{proof}
From \prettyref{lem:boundOnDegreesOfFA}, any column $\mathbf{p}$
of $\mathbf{P}_{1}$ satisfies $\deg\mathbf{F}\mathbf{p}\le\deg_{\vec{s}}\mathbf{p}<\sigma$.
Combining this with the fact that $\mathbf{F}\mathbf{p}\equiv0\mod x^{\sigma}$
we get $\mathbf{F}\mathbf{p}=0$. Thus $\mathbf{P}_{1}$ is generated
by $\mathbf{N}_{1}$, that is, $\mathbf{P}_{1}=\mathbf{N}_{1}\mathbf{U}$
for some polynomial matrix $\mathbf{U}$. On the other hand, $\mathbf{N}_{1}$
has order $\left(\mathbf{F},\sigma\right)$ and %
\begin{comment}
is therefore generated by $\mathbf{P}_{1}$, i.e. 
\end{comment}
{} therefore satisfies $\mathbf{N}_{1}=\mathbf{P}_{1}\mathbf{V}$ for
some polynomial matrix $\mathbf{V}$. We now have $\mathbf{P}_{1}=\mathbf{P}_{1}\mathbf{V}\mathbf{U}$
and $\mathbf{N}_{1}=\mathbf{N}_{1}\mathbf{U}\mathbf{V}$, implying
both $\mathbf{U}$ and $\mathbf{V}$ are unimodular. The result then
follows from the unimodular equivalence of $\mathbf{P}_{1}$ and $\mathbf{N}_{1}$
and the fact that they are $\vec{s}$-column reduced. %reducedness of .
\end{proof}
We can now provide a simple bound on the $\vec{s}$-minimal kernel
basis of $\mathbf{F}$. 
\begin{thm}
\label{thm:boundOfSumOfShiftedDegreesOfKernelBasis}Suppose $\mathbf{F}\in\mathbb{K}\left[x\right]^{m\times n}$
and $\vec{s}\in\mathbb{Z}_{\ge0}^{n}$ is a shift with entries bounding
the corresponding column degrees of $\mathbf{F}$. Then the sum of
the $\vec{s}$-column degrees of any $\vec{s}$-minimal kernel basis
of $\mathbf{F}$ is bounded by $\xi=\sum\vec{s}$.\end{thm}
\begin{proof}
Let $\mathbf{P}$ be a $(\mathbf{F},\sigma,\vec{s})$-basis with high
enough order $\sigma$ so that $\mathbf{P}=\left[\mathbf{N},\bar{\mathbf{N}}\right]$
contains a complete kernel basis, $\mathbf{N}$, of $\mathbf{F}$.
By \prettyref{lem:orderBasisContainsNullspaceBasis} we just need
$\sigma$ to be greater than the $\vec{s}$-column degree %
\begin{comment}
$\delta$ 
\end{comment}
of a $\vec{s}$-minimal kernel basis of $\mathbf{F}$. %
\begin{comment}
To see this, let $k$ be the column dimension of any kernel basis
of $\mathbf{F}$, and $\mathbf{P}'$ be the submatrix of $\mathbf{P}$
consists of the columns with the $k$ lowest $\vec{s}$-column degrees.
Note that the minimality of $\mathbf{P}$ implies that the $\vec{s}$-column
degrees of $\mathbf{P}'$ are no more that of any $\vec{s}$-minimal
kernel basis of $\mathbf{F}$. For any column $\mathbf{p}$ of $\mathbf{P}'$
we also have $\deg\mathbf{F}\mathbf{p}\le\deg_{\vec{s}}\mathbf{p}$
by \prettyref{lem:boundOnDegreesOfFA}. Hence we have $\deg\mathbf{F}\mathbf{p}\le\deg_{\vec{s}}\mathbf{p}\le\delta<\sigma$.
Combining this with the fact that $\mathbf{F}\mathbf{p}\equiv0\mod x^{\sigma}$
we get $\mathbf{F}\mathbf{p}=0$, which means $\mathbf{P}'$, whose
$k$ columns are linearly independent, is also a kernel basis of $\mathbf{F}$.
In fact, although not needed for the proof of this theorem, it is
not difficult to show from here that $\mathbf{P}'$ is in fact a $\vec{s}$-minimal
kernel basis because its $\vec{s}$-column degrees are no more than
that of a $\vec{s}$-minimal kernel basis of $\mathbf{F}$. 
\end{comment}
Let $r$ be the column dimension of $\bar{\mathbf{N}}$. Note that
this is the same as the rank of $\mathbf{F}$. By \prettyref{lem:boundOfSumOfShiftedDegreesOfOrderBasis}
the sum of the $\vec{s}$-column degrees of $\mathbf{P}$ is at most
$\xi+r\sigma$. By \prettyref{lem:boundOnDegreesOfFA} the sum of
the $\vec{s}$-column degrees of $\bar{\mathbf{N}}$ is greater than
or equal to the sum of the column degrees of $\mathbf{F}\cdot\bar{\mathbf{N}}$,
which is at least $r\sigma$, since every column of $\mathbf{F}\bar{\mathbf{N}}$
is nonzero and has order $\sigma$. So the sum of the $\vec{s}$-column
degrees of $\mathbf{N}$ is bounded by $\xi+r\sigma-r\sigma=\xi$. 
\end{proof}
\prettyref{thm:boundOfSumOfShiftedDegreesOfKernelBasis} specializes
to the following well-known results in the case of uniform shift:
\begin{cor}
Given $\mathbf{F}\in\mathbb{K}\left[x\right]^{m\times n}$ with degree
$d$ . The sum of the column degree of its minimal kernel basis is
bounded by $md$.\end{cor}
\begin{proof}
From \prettyref{thm:boundOfSumOfShiftedDegreesOfKernelBasis} and
the definition of shifted column degrees we have 
\[
nd\ge\sum\cdeg_{\left[d,\dots d\right]}\mathbf{N}\ge(n-m)d+\sum\cdeg\mathbf{N},
\]
 which gives 
\[
\sum\cdeg\mathbf{N}\le nd-(n-m)d=md.
\]

\end{proof}
\begin{comment}
Our intermediate computations should not blow up in size but should
also not be too expensive. Our algorithm will reduce a single null
space computation to a set of null space computations of smaller size.
These smaller problems are constructed by the computations of residuals
which in all cases involves the multiplication of two polynomial matrices
having unbalanced degrees. The following lemma, whose proof we defer
until a later section, says that this can be done efficiently.
\begin{lem}
\label{lem:multiplyUnbalancedMatrices-1} Let $\vec{s}$ be a shift
ordered in terms of increasing values and $\xi$, a bound on the sum
of the entries of $\vec{s}$. Let $\mathbf{A}\in\mathbb{K}\left[x\right]^{m\times n}$,
with column degrees bounded by $\vec{s}$ and $\mathbf{B}\in\mathbb{K}\left[x\right]^{n\times k}$
with $k\in O\left(m\right)$ and the sum $\theta$ of its $\vec{s}$-column
degrees satisfying $\theta\in O\left(\xi\right)$. Then we can multiply
$\mathbf{A}$ and $\mathbf{B}$ with a cost of $O^{\sim}(nm^{\omega-2}\xi)$. \end{lem}
\end{comment}


%For simplicity, we also assume without loss of generality that the
%columns of $\mathbf{F}$ and the corresponding entries of $\vec{s}=\left[s_{1},\dots,s_{n}\right]$
%are arranged so that the entries of $\vec{s}$ are in increasing order.
%%\begin{comment}
%A convenient bound throughout the computation is provided by $\xi=\sum\vec{s}$
%be.
%%\end{comment}
%{} 
%Let $\rho=\sum_{n-m+1}^{n}s_{i}$ be the sum of $m$ largest entries
%of $\vec{s}$, and $s=\rho/m$ be their average. The algorithm we
%present in this section computes a $\vec{s}$-minimal null space basis
%$\mathbf{N}$ with a cost of $O^{\sim}(n^{\omega}s)$ field operations.
%For uniform shift $\vec{s}=\left[s,\dots,s\right]$, we improve this
%later to $O^{\sim}\left(n^{\omega}\left\lceil ms/n\right\rceil \right)=O^{\sim}\left(n^{\omega}\left\lceil \rho/n\right\rceil \right)=O^{\sim}\left(n^{\omega-1}\left\lceil m\xi/n\right\rceil \right)$.



\subsection{\label{sub:continueComputingNullspaceBasisByColumns}Reducing the
column dimension via order basis computation}

In this subsection we look at how an order basis computation can be
used to reduce the column dimension of our problem. While order basis
computations were also used in \citep{storjohann-villard:2005} to
reduce the column dimensions of their problems, here order basis computations
are used in a more comprehensive way. In particular, \prettyref{thm:continueComputingNullspaceBasisByColumns}
given later in this section, allows us to maintain the minimality
of the bases with the use of the shifted degrees and the residuals.

We begin by computing a $\left(\mathbf{F},3s,\vec{s}\right)$-basis
$\mathbf{P}$, which can be done with a cost of $O^{\sim}\left(n^{\omega}s\right)$
using the algorithm from \citet{Giorgi2003}. Note that if $\vec{s}$
is balanced, then we can compute this with a cost of $O^{\sim}\left(n^{\omega-1}\rho\right)$
using the algorithm from \prettyref{chap:OrderBasis}. We will show
that at most %$3m/2$ 
$\frac{3m}{2}$ columns of \textbf{$\mathbf{P}$} are not elements
of a kernel basis of $\mathbf{F}$. 
\begin{rem}
Note that it is not essential to choose $3s$ for the order. The order
can be set to $\ell s$ for any constant $\ell>1$. A smaller $\ell$
means less work to compute a $(\mathbf{F},\ell s,\vec{s})$-basis,
but also results in fewer kernel basis elements computed and leaves
more work for computing the remaining basis elements. On the other
hand, a larger $\ell$ means more work is needed for order basis computation,
but leaves less remaining work. It may be possible to better balance
these computations with a better choice of $\ell$. However, as we
will see later, the resulting complexity given in this paper would
remain the same for any $\ell>1$ as long as we use the big $O$ notation
and do not care about the constant factors in the cost. \end{rem}
\begin{thm}
\label{thm:dimensionOfPartialNullspaceBasisBasedOnOrder} Let $\mathbf{P}=[\mathbf{P}_{1},\mathbf{P}_{2}]$
be a $(\mathbf{F},\sigma,\vec{s})$-basis with $\sigma>s$ and $\mathbf{P}_{1}$
containing all columns $\mathbf{n}$ of $\mathbf{P}$ satisfying $\mathbf{F}\mathbf{n}=0$.
Then for $\ell=\sigma/s$ the column dimension $\kappa$ of $\mathbf{P}_{2}$
is bounded by $\frac{\ell m}{(\ell-1)}.$ \end{thm}
\begin{proof}
Any column $\mathbf{p}$ of $\mathbf{P}_{2}$ has order $\sigma$
but also satisfies $\mathbf{F}\mathbf{p}\ne0$. Thus the degree of
$\mathbf{F}\mathbf{p}$ must be at least $\sigma$ and, by \prettyref{lem:boundOnDegreesOfFA},
$\mathbf{p}$ must have $\vec{s}$-column degree at least $\sigma$.
It follows that the sum of the $\vec{s}$-column degrees of the columns
of $\mathbf{P}_{2}$ must satisfy $\sum\deg_{\vec{s}}\mathbf{P}_{2}\ge\kappa\sigma.$
From \prettyref{lem:boundOfSumOfShiftedDegreesOfOrderBasis} we know
that the sum of the $\vec{s}$-column degrees of the columns of $\mathbf{P}$
satisfies $\sum\deg_{\vec{s}}\mathbf{P}\le\sum\vec{s}+m\sigma,$ and
hence the sum of $\vec{s}$-column degrees of the columns of $\mathbf{P}_{1}$
must satisfy 
\[
\sum\deg_{\vec{s}}\mathbf{P}_{1}=\sum\deg_{\vec{s}}\mathbf{P}-\sum\deg_{\vec{s}}\mathbf{P}_{2}\le\sum\vec{s}+m\sigma-\kappa\sigma.
\]
 On the other hand, the lowest possible value of $\sum\deg_{\vec{s}}\mathbf{P}_{1}$
is $\sum_{i=1}^{n-\kappa}s_{i}$, the sum of the $n-\kappa$ smallest
entries of $\vec{s}$ (which occurs when $\mathbf{P}_{1}=\left[\mathbf{I},0\right]^{T}$).%
\begin{comment}
based on the assumption that the entries of $\vec{s}=\left[s_{1},\dots,s_{n}\right]$
are arranged in increasing order. 
\end{comment}
{} It follows that 
\[
\sum\vec{s}+m\sigma-\kappa\sigma\ge\sum_{i=1}^{n-\kappa}s_{i},
\]
 or, after rearrangement, 
\[
m\sigma\ge\kappa\sigma-\left(\sum\vec{s}-\sum_{i=1}^{n-\kappa}s_{i}\right).
\]
 Combining this with the fact that for $\kappa\ge m$ the average
of the $\kappa$ largest entries of $\vec{s}$ is no more than the
average of the $m$ largest entries of $\vec{s}$, that is, 
\[
\left(\sum\vec{s}-\sum_{i=1}^{n-\kappa}s_{i}\right)/\kappa\le s,\mbox{ or }\sum\vec{s}-\sum_{i=1}^{n-\kappa}s_{i}\le\kappa s,
\]
 we get $m\sigma\ge\kappa\sigma-\kappa s,$ which gives $\kappa\le m\sigma/(\sigma-s)$
for $\sigma>s$. Substituting in $\sigma=\ell s$, we get $\kappa\le\frac{\ell m}{(\ell-1)}$
as required. 
\end{proof}
Let $\left[\mathbf{P}_{1},\mathbf{P}_{2}\right]=\mathbf{P}$ with
$\mathbf{P}_{1}$ consisting of the kernel basis elements computed.
Then the residual $\mathbf{F}\mathbf{P}=\left[\mathbf{0},\mathbf{F}\mathbf{P}_{2}\right]$
can be used to compute the remaining kernel basis elements. Before
showing this can be correctly done, let us first make sure that the
matrix multiplication $\mathbf{F}\mathbf{P}_{2}$ can be done efficiently,
which may not be obvious since $\mathbf{F}$, $\mathbf{P}_{2}$, and
their product $\mathbf{F}\mathbf{P}_{2}$ can all have degrees up
to $\Theta(\xi)$. But we do have the sum of the column degrees of
$\mathbf{F}$, that of $\mathbf{F}\mathbf{P}_{2}$, and the sum of
the $\vec{s}$-column degrees of $\mathbf{P}_{2}$ all bounded by
$O(\xi)$, which means their total size are not too big but their
column degrees can be quite unbalanced. We will encounter this type
of multiplication again multiple times, for computing residuals and
combining results. In fact, almost all of the matrices in our kernel
basis computation can have such unbalanced degrees. To efficiently
multiply these matrices, we provide the following theorem, whose proof
we defer until the end of this section. 
\begin{thm}
\label{thm:multiplyUnbalancedMatrices} Let $\mathbf{A}\in\mathbb{K}\left[x\right]^{m\times n}$,
$\vec{s}$ a shift with entries bounding the column degrees of $\mathbf{A}$
and $\xi$, a bound on the sum of the entries of $\vec{s}$. Let $\mathbf{B}\in\mathbb{K}\left[x\right]^{n\times k}$
with $k\in O\left(m\right)$ and the sum $\theta$ of its $\vec{s}$-column
degrees satisfying $\theta\in O\left(\xi\right)$. Then we can multiply
$\mathbf{A}$ and $\mathbf{B}$ with a cost of $O^{\sim}(nm^{\omega-2}\xi)$.
\begin{comment}
\begin{proof}
For simplicity we assume $m$ is a power of 2, something which can
be achieved by appending zero rows to $\mathbf{F}$. Divide the matrix
$\mathbf{B}$ into $\log m$ column blocks according to the $\vec{s}$-column
degrees of its columns. Let 
\[
\mathbf{B}=\left[\begin{array}{cccc}
\mathbf{B}^{\left(\log m\right)} & \mathbf{B}^{\left(\log m-1\right)} & \cdots & \mathbf{B}^{\left(1\right)}\end{array}\right],
\]
 with $\mathbf{B}^{\left(\log m\right)}$, $\mathbf{B}^{\left(\log m-1\right)},$
$\mathbf{B}^{\left(\log m-2\right)}$, ... , $\mathbf{B}^{\left(2\right)}$,
$\mathbf{B}^{\left(1\right)}$ having $\vec{s}$-column degrees in
the range $\left[0,2\xi/m\right]$, $(2\xi/m,4\xi/m]$, $(4\xi/m,8\xi/m]$,
...,$(\xi/4,\xi/2]$, $(\xi/2,\theta]$, respectively. We will multiply
$\mathbf{A}$ with each $\mathbf{B}^{\left(i\right)}$ separately.

We also divide the matrix $\mathbf{A}$ into $\log m$ column blocks
and each matrix $\mathbf{B}^{\left(i\right)}$ into $\log m$ row
blocks according to the size of the corresponding entries in $\vec{s}$.
Set 
\begin{eqnarray}
~~~~~~~~~~~~~~~~~~~~~~~~~~\vec{s} & = & \left[\begin{array}{cccc}
\vec{s}_{\log m} & \vec{s}_{\log m-1} & \cdots & \vec{s}_{1}\end{array}\right]\nonumber \\
\mathbf{A} & = & \left[\begin{array}{cccc}
\mathbf{A}_{\log m} & \mathbf{A}_{\log m-1} & \cdots & \mathbf{A}_{1}\end{array}\right]\nonumber \\
\mathbf{B} & = & \left[\begin{array}{cccc}
\mathbf{B}^{\left(\log m\right)} & \mathbf{B}^{\left(\log m-1\right)} & \cdots & \mathbf{B}^{\left(1\right)}\end{array}\right]\nonumber \\
 & = & \left[\begin{array}{cccc}
\mathbf{B}_{\log m}^{\left(\log m\right)} & \mathbf{B}_{\log m}^{\left(\log m-1\right)} & \cdots & \mathbf{B}_{\log m}^{\left(1\right)}\\
\vdots &  &  & \vdots\\
\mathbf{B}_{1}^{\left(\log m\right)} & \mathbf{B}_{1}^{\left(\log m-1\right)} & \cdots & \mathbf{B}_{1}^{\left(1\right)}
\end{array}\right]
\end{eqnarray}
 with $\vec{s}_{\log m},\vec{s}_{\log m-1},\dots,\vec{s}_{1}$ having
entries in the range $\left[0,2\xi/m\right]$, $(2\xi/m,4\xi/m]$,
$(4\xi/m,8\xi/m]$, ..., $(\xi/2,\xi]$ respectively. Also the column
dimension of $\mathbf{A}_{j}$ and the row dimension of $\mathbf{B}_{j}^{\left(i\right)}$
match that of $\vec{s}_{j}$ for $j$ from $1$ to $\log m$.

Notice that $\mathbf{B}_{(j)}^{(i)}$ for $i>j$ must be zero. Otherwise,
as $\vec{s}_{j}>\xi/2^{j}\ge\xi/2^{i-1}$, the $\vec{s}$-column degree
of $\mathbf{B}^{(i)}$ would exceed $\xi/2^{i-1}$, a contradiction
since by definition the $\vec{s}$-column degree of $\mathbf{B}^{(i)}$
is bounded by $\xi/2^{i-1}$ when $i>1$. So $\mathbf{B}$ in fact
has a block triangular shape 
\[
\mathbf{B}=\left[\begin{array}{cccc}
\mathbf{B}_{\log m}^{\left(\log m\right)} & \mathbf{B}_{\log m}^{\left(\log m-1\right)} & \cdots & \mathbf{B}_{\log m}^{\left(1\right)}\\
 & \mathbf{B}_{\log m-1}^{\left(\log m-1\right)} &  & \vdots\\
 &  & \ddots\\
 &  &  & \mathbf{B}_{1}^{\left(1\right)}
\end{array}\right]
\]
 (while remembering that the blocks have varying sizes).

First consider the multiplication 
\[
\mathbf{A}\mathbf{B}^{\left(1\right)}=\left[\begin{array}{cccc}
\mathbf{A}_{\log m} & \mathbf{A}_{\log m-1} & \cdots & \mathbf{A}_{1}\end{array}\right]\left[\begin{array}{l}
\mathbf{B}_{\log m}^{\left(1\right)}\\
\mathbf{B}_{\log m-1}^{\left(1\right)}\\
\vdots\\
\mathbf{B}_{1}^{\left(1\right)}
\end{array}\right].
\]
 Note that there are $O\left(1\right)$ columns in $\mathbf{B}^{(1)}$
since $\theta\in O\left(\xi\right)$. We do this in $\log m$ steps.
At step $j$ for $j$ from $1$ to $\log m$ we multiply $\mathbf{A}_{j}$
and $\mathbf{B}_{j}^{(1)}$. The column dimension of $\mathbf{A}_{j}$,
which is the same as the row dimension of $\mathbf{B}_{j}^{(1)}$,
is less than $2^{j}$. The degree of $\mathbf{B}_{j}^{(1)}$ is $O\left(\xi\right)$.
To use fast multiplication, we expand $\mathbf{B}_{j}^{(1)}$ to a
matrix $\bar{\mathbf{B}}_{j}^{(1)}$ with degree less than $\delta\in\Theta(\xi/2^{j})$
and column dimension $q\in O(2^{j})$ as follows. Write 
\[
\mathbf{B}_{j}^{(1)}=\mathbf{B}_{j,0}^{(1)}+\mathbf{B}_{j,1}^{(1)}x^{\delta}+\dots+\mathbf{B}_{j,q-1}^{(1)}x^{\delta(q-1)}=\sum_{k=0}^{q-1}\mathbf{B}_{j,k}^{(1)}x^{\delta k}
\]
 with each $\mathbf{B}_{j,k}^{(1)}$ having degree less than $\delta.$
Set $\bar{\mathbf{B}}_{j}^{(1)}=\left[\mathbf{B}_{j,0}^{(1)},\mathbf{B}_{j,1}^{(1)},\dots,\mathbf{B}_{j,q-1}^{(1)}\right]$.
We can then multiply $\mathbf{A}_{j}$, which has dimension $m\times O(2^{j})$
for $j<\log m$, and $\bar{\mathbf{B}}_{j}^{(1)}$, which has dimension
$O(2^{j})\times O(2^{j})$ for $j<\log m$, with a cost of 
\[
O^{\sim}\left((m/2^{j})\left(2^{j}\right)^{\omega}\xi/2^{j}\right)=O^{\sim}\left(\left(2^{j}\right)^{\omega-2}m\xi\right)\subset O^{\sim}\left(m^{\omega-1}\xi\right)\subset O^{\sim}(nm^{\omega-2}\xi).
\]
 For $j=\log m$, $\mathbf{A}_{j}$ has dimension $m\times O\left(n\right)$,
$\bar{\mathbf{B}}_{j}^{\left(1\right)}$ has dimension $O\left(n\right)\times O(m)$,
and their degrees are $O\left(\xi/m\right)$. Hence they can be multiplied
with a cost of $O^{\sim}\left((n/m)m^{\omega}(\xi/m)\right)=O^{\sim}\left(nm^{\omega-2}\xi\right)$.
Adding up the columns of $\mathbf{A}_{j}\bar{\mathbf{B}}_{j}^{(1)}$
gives $\mathbf{A}_{j}\mathbf{B}_{j}^{(1)}$ and costs $O(m\xi)$.
Therefore, multiplying $\mathbf{A}$ and $\mathbf{B}^{(1)}$ over
$\log(m)$ steps costs $O^{\sim}\left(nm^{\omega-2}\xi\right)$.

Next we multiply $\mathbf{A}$ with $\mathbf{B}^{(2)}$. We proceed
in the same way as before, but notice that $\mathbf{A}_{1}\mathbf{B}_{1}^{(2)}$
is no longer needed since $\mathbf{B}_{1}^{(2)}=0$. Multiplying $\mathbf{A}$
and $\mathbf{B}^{(2)}$ also costs $O^{\sim}\left(nm^{\omega-2}\xi\right)$.

Continue doing this, it costs $O^{\sim}\left(nm^{\omega-2}\xi\right)$.
to multiply $\mathbf{A}$ with the columns $\mathbf{B}^{(i)}$ for
$i$ from $1$ to $\log m$. As before, remember that $\mathbf{B}_{(j)}^{(i)}=0$
for $j>i$. The overall cost for $i$ from 1 to $\log m$ is $O^{\sim}\left(nm^{\omega-2}\xi\right)$
to multiply $\mathbf{A}$ and $\mathbf{B}$. \end{proof}
\end{comment}

\end{thm}
With \prettyref{thm:multiplyUnbalancedMatrices}, we can now do the
multiplication $\mathbf{F}\mathbf{P}_{2}$ efficiently. 
\begin{cor}
\label{cor:multiplyingFP2}The multiplication of $\mathbf{F}$ and
$\mathbf{P}_{2}$ can be done with a cost of $O^{\sim}\left(nm^{\omega-2}\xi\right)$.\end{cor}
\begin{proof}
Since $\mathbf{P}=[\mathbf{P}_{1},\mathbf{P}_{2}]$ is a $(\mathbf{F},3s,\vec{s})$-basis,
we have from \prettyref{lem:boundOfSumOfShiftedDegreesOfOrderBasis}
that the sum of the $\vec{s}$-column degrees of $\mathbf{P}_{2}$
satisfies $\sum\deg_{\vec{s}}\mathbf{P}_{2}\le3sm+\xi\leq4\xi$. Hence
\prettyref{thm:multiplyUnbalancedMatrices} applies. 
\end{proof}
It remains to show that the residual $\mathbf{F}\mathbf{P}_{2}$ can
be used to compute the remaining kernel basis elements. 
\begin{thm}
\label{thm:continueComputingNullspaceBasisByColumns}Let $\mathbf{P}=\left[\mathbf{P}_{1},\mathbf{P}_{2}\right]$
be a $\left(\mathbf{F},\sigma,\vec{s}\right)$-basis such that $\mathbf{P}_{1}$
consists of all the kernel basis elements of $\mathbf{F}$ in $\mathbf{P}$.
Let $\vec{b}=[\vec{b}_{1},\vec{b}_{2}]$ be the $\vec{s}$-column
degrees of $\mathbf{P}$, where $\vec{b}_{1},\vec{b}_{2}$ are the
$\vec{s}$-column degrees of $\mathbf{P}_{1},$ $\mathbf{P}_{2}$
respectively. Let $\mathbf{Q}$ be a $\vec{b}_{2}$-minimal kernel
basis of $\mathbf{F}\mathbf{P}_{2}$ with $\vec{b}_{2}$-column degrees
$\vec{b}_{2}'$. Then $\left[\mathbf{P}_{1},\mathbf{P}_{2}\mathbf{Q}\right]$
is a $\vec{s}$-minimal kernel basis of $\mathbf{F}$ with $\vec{s}$-column
degrees $[\vec{b}_{1},\vec{b}_{2}']$.\end{thm}
\begin{proof}
Let $\mathbf{Q}'=\diag(\left[I,\mathbf{Q}\right])$, where the dimension
of the identity matrix $I$ matches that of $\mathbf{P}_{1}.$ Then
$\mathbf{Q}'$ is a $\vec{b}$-minimal kernel basis of $\mathbf{F}\mathbf{P}$
since %
\begin{comment}
$\mathbf{F}\mathbf{P}=\left[0,\mathbf{F}\mathbf{P}_{2}\right]$ and 
\end{comment}
{} $\mathbf{F}\mathbf{P}\mathbf{Q}'=\left[\mathbf{F}\mathbf{P}_{1},\mathbf{F}\mathbf{P}_{2}\mathbf{Q}\right]=0$.
It follows that $\mathbf{P}\mathbf{Q}'=\left[\mathbf{P}_{1},\mathbf{P}_{2}\mathbf{Q}\right]$
is a kernel basis of $\mathbf{F}$. We now show that $\mathbf{P}\mathbf{Q}'$
is $\vec{s}$-column reduced and has $\vec{s}$-column degrees $[\vec{b}_{1},\vec{b}_{2}']$,
or equivalently, $x^{\vec{s}}\mathbf{P}\mathbf{Q}'$ is column reduced
and has column degrees $[\vec{b}_{1},\vec{b}_{2}']$. Notice that
$x^{\vec{s}}\mathbf{P}$ has column degrees $[\vec{b}_{1},\vec{b}_{2}]$
and a full rank leading column coefficient matrix $P$. Hence $x^{\vec{s}}\mathbf{P}x^{-[\vec{b}_{1},\vec{b}_{2}]}$
has column degrees $\left[0,\dots0\right]$. (If one is concerned
about the entries not being polynomials, one can simply multiply the
matrix by $x^{\xi}$ to shift the degrees up.) Similarly, $x^{\vec{b}_{2}}\mathbf{Q}x^{-\vec{b}_{2}'}$
has column degrees $[0,\dots,0]$, and so $x^{[\vec{b}_{1},\vec{b}_{2}]}\mathbf{Q}'x^{-[\vec{b}_{1},\vec{b}_{2}']}$
also has column degrees $[0,\dots,0]$ and a full rank leading column
coefficient matrix $Q'$. Putting these together, we see that $x^{\vec{s}}\mathbf{P}x^{-[\vec{b}_{1},\vec{b}_{2}]}x^{[\vec{b}_{1},\vec{b}_{2}]}\mathbf{Q}'x^{-[\vec{b}_{1},\vec{b}_{2}']}=x^{\vec{s}}\mathbf{P}\mathbf{Q}'x^{-[\vec{b}_{1},\vec{b}_{2}']}$
has column degrees $[0,\dots,0]$ and a full rank leading column coefficient
matrix $PQ'$. It follows that $x^{\vec{s}}\mathbf{P}\mathbf{Q}'$
has column degrees $[\vec{b}_{1},\vec{b}_{2}']$, or equivalently,
the $\vec{s}$-column degrees of $\mathbf{PQ}'$ is $[\vec{b}_{1},\vec{b}_{2}']$.

It remains to show that any $\mathbf{n}$ satisfying $\mathbf{F}\mathbf{n}=0$
must be a linear combination of the columns of $\mathbf{PQ}'$. Since
$\mathbf{n}\in\left\langle \left(\mathbf{F},\sigma\right)\right\rangle $,
it is generated by the $\left(\mathbf{F},\sigma\right)$-basis $\mathbf{P}$,
that is, $\mathbf{n}=\mathbf{P}\mathbf{a}$ with $\mathbf{a}=\mathbf{P}^{-1}\mathbf{n}\in\mathbb{K}\left[x\right]^{n}$.
Also, $\mathbf{F}\mathbf{n}=0$ implies $\mathbf{F}\mathbf{P}\mathbf{a}=0$,
hence $\mathbf{a}=\mathbf{Q}'\mathbf{b}$ for some vector $\mathbf{b}$
as $\mathbf{Q}'$ is a kernel basis of $\mathbf{F}\mathbf{P}$. We
now have $\mathbf{n}=\mathbf{P}\mathbf{Q}'\mathbf{b}$ as required.\end{proof}
\begin{example}
\label{exm:continueComputingNullspaceBasisByColumns}Let us look at
an example of computing kernel basis using \prettyref{thm:continueComputingNullspaceBasisByColumns}.
Let $\mathbf{F}$ be given by 
\[
\left[\begin{array}{cccc}
x+x^{2}+x^{3} & 1+x & 0 & 1+x\\
1+x^{2}+x^{3} & x+x^{2}+x^{3} & x+x^{2} & x^{3}
\end{array}\right]\in\mathbb{Z}_{2}\left[x\right]^{2\times4}.
\]
 Let $\sigma=3$, $\vec{s}=\left[3,3,3,3\right]$. We first compute
a $\left(\mathbf{F},\sigma,\vec{s}\right)$-basis 
\[
\mathbf{P}=\left[\begin{array}{cccc}
~0~ & ~0~ & x^{2} & x\\
1 & 0 & 0 & x^{2}\\
1 & x^{2} & x+x^{2} & 1+x\\
1 & 0 & 0 & 0
\end{array}\right],
\]
 with the $\vec{s}$-column degrees $\vec{b}=\left[3,5,5,5\right]$
and the residual 
\[
\mathbf{F}\mathbf{P}=\left[\begin{array}{cccc}
~0~ & 0 & x^{3}+x^{4}+x^{5} & x^{4}\\
0 & x^{3}+x^{4} & x^{5} & x^{3}+x^{5}
\end{array}\right].
\]
 Thus $\mathbf{P}_{1}=[0,1,1,1]^{T}$, %\[
%\mathbf{P}_{1}=\begin{bmatrix}0\\
%1\\
%1\\
%1
%\end{bmatrix}
%\]
with $\vec{s}$-column degree $3$, is the only kernel basis element
computed. Let $\mathbf{P}_{2}$ contain the remaining columns of $\mathbf{P}$
and $\vec{b}_{2}=\left[5,5,5\right]$ be its $\vec{s}$-column degrees.
Next we compute a $\vec{b}_{2}$-minimal kernel basis of $\mathbf{F}\mathbf{P}_{2}$
\[
\mathbf{Q}=[1+x+x^{4},\ x+x^{2},\ 1+x^{3}]^{T}
\]
 which has $\vec{b}_{2}$-column degree 9. Then 
\[
\left[\mathbf{P}_{1},\mathbf{P}_{2}\mathbf{Q}\right]=\left[\begin{array}{cc}
0 & x+x^{3}\\
1 & x^{2}+x^{5}\\
1 & 1+x+x^{6}\\
1 & 0
\end{array}\right]
\]
 is a complete $\vec{s}$-minimal kernel basis of $\mathbf{F}$ with
$\vec{s}$-column degrees $\left[3,9\right]$. 
\end{example}
\prettyref{thm:continueComputingNullspaceBasisByColumns} shows that
the remaining $\vec{s}$-minimal kernel basis elements $\mathbf{P}_{2}\mathbf{Q}$
can be correctly computed from the residual $\mathbf{F}\mathbf{P}_{2}$.
\begin{comment}
We still need to show it can be done efficiently. 
\end{comment}
{} Before discussing the computation of a $\vec{b}_{2}$-minimal kernel
basis $\mathbf{Q}$ of $\mathbf{F}\mathbf{P}_{2}$, let us first note
that the multiplication $\mathbf{P}_{2}\mathbf{Q}$ can be done efficiently,
which again follows from \prettyref{thm:multiplyUnbalancedMatrices}. 
\begin{lem}
\label{lem:multiplyingP2Q}The multiplication of $\mathbf{P}_{2}$
and $\mathbf{Q}$ can be done with a cost of $O^{\sim}\left(nm^{\omega-2}\xi\right)$.\end{lem}
\begin{proof}
Note that the dimension of $\mathbf{P}_{2}$ is $n\times O(m)$ from
\prettyref{thm:dimensionOfPartialNullspaceBasisBasedOnOrder} and
the dimension of $\mathbf{Q}$ is $O\left(m\right)\times O\left(m\right)$.
The column degrees of $\mathbf{P}_{2}$ are bounded by the $\vec{s}$-column
degrees $\vec{b}_{2}$ of $\mathbf{P}_{2}$ since $\vec{s}$ is non-negative.
Also recall that $\sum\vec{b}_{2}\le4\xi$ from the proof of \prettyref{cor:multiplyingFP2}.
By \prettyref{lem:boundOnDegreesOfFA} the column degrees of $\mathbf{F}\mathbf{P}_{2}$
are bounded by the $\vec{s}$-column degrees $\vec{b}_{2}$ of $\mathbf{P}_{2}$.
By \prettyref{thm:boundOfSumOfShiftedDegreesOfKernelBasis}, the sum
of the $\vec{b}_{2}$-column degrees of $\mathbf{Q}$ is %
\begin{comment}
bounded by the sum of the column degrees of $\mathbf{F}\mathbf{P}_{2}$,
which is 
\end{comment}
{} also bounded by $\sum\vec{b}_{2}\le4\xi$. Now if we separate $\mathbf{P}_{2}$
to $n/m$ blocks rows each with no more than $m$ rows, \prettyref{thm:multiplyUnbalancedMatrices}
can be used to multiply each block row with $\mathbf{Q}$. Each multiplication
involves matrices of dimension $O\left(m\right)\times O\left(m\right)$.
In addition, both the sum of the column degrees of $\mathbf{P}_{2}$
and the sum of the $\vec{b}_{2}$-column degrees of $\mathbf{Q}$
are bounded by $4\xi$. So each multiplication costs $O^{\sim}(m^{\omega-1}\xi)$.
Hence doing this for all $n/m$ block rows costs $O^{\sim}\left(nm^{\omega-2}\xi\right)$. 
\end{proof}

\subsection{Reducing the degrees}

Our next task is computing a $\vec{b}_{2}$-minimal kernel basis of
the residual $\mathbf{F}\mathbf{P}_{2}$. It is useful to note that
the lower degree terms of $\mathbf{F}\mathbf{P}_{2}$ are zero since
it has order $\sigma$. Hence we can use $\mathbf{G}=\mathbf{F}\mathbf{P}_{2}/x^{\sigma}$
instead to compute the remaining basis elements. In the following,
we show that just like the original input matrix $\mathbf{F}$, this
new input matrix $\mathbf{G}$ has column degrees bounded by the corresponding
entries of $\vec{s}$.%
\begin{comment}
However we still need to verify that the size of $\mathbf{G}$ is
small enough to allow for the efficient computation of the remaining
basis elements. 
\end{comment}
\begin{comment}
\prettyref{lem:boundOnShiftedDegrees} and \prettyref{cor:boundOnFPAfterLowerTermsRemoved}
shows that $\vec{b}-\sigma$, which bound the column degrees of $\mathbf{F}\mathbf{P}/x^{\sigma}$
by \prettyref{lem:boundOnDegreesOfFA}, are bounded by the corresponding
entries of $\vec{s}$. 
\end{comment}

\begin{lem}
\label{lem:boundOnShiftedDegrees}%Let $\vec{s}$, $\mathbf{F}$ be as before. 
If an $(\mathbf{F},\sigma,\vec{s})$-basis has columns arranged in
increasing $\vec{s}$-column degrees with $\vec{s}$-column degrees
$\vec{b}$, then the entries of $\vec{b}-[\sigma,\dots,\sigma]=\left[b_{1}-\sigma,\dots,b_{n}-\sigma\right]$
are bounded component-wise by $\vec{s}$. \end{lem}
\begin{proof}
A $(\mathbf{F},0,\vec{s})$-basis of order $0$ has $\vec{s}$-column
degrees given by $\vec{s}$. For each order increase, any column of
the basis has its $\vec{s}$-column degree increases by at most one,
which occurs when its order is increased by multiplying the column
by $x$. Hence at order $\sigma$, the $\vec{s}$-column degree increase
for each column is at most $\sigma$. \end{proof}
\begin{cor}
\label{cor:boundOnFPAfterLowerTermsRemoved}The column degrees of
$\mathbf{F}\mathbf{P}/x^{\sigma}$ are \\
bounded component-wise by $\vec{s}$.\end{cor}
\begin{proof}
From \prettyref{lem:boundOnDegreesOfFA}, the column degrees of $\mathbf{F}\mathbf{P}$
are bounded component-wise by $\vec{b}$, the $\vec{s}$-column degrees
of $\mathbf{P}$. Hence the column degrees of $\mathbf{F}\mathbf{P}/x^{\sigma}$
are bounded component-wise by $\vec{b}-[\sigma,\dots,\sigma]$. The
result then follows from \prettyref{lem:boundOnShiftedDegrees}. 
\end{proof}
From \prettyref{cor:boundOnFPAfterLowerTermsRemoved}, the column
degrees of $\mathbf{F}\mathbf{P}_{2}/x^{\sigma}$ are bounded by the
entries of the corresponding subset $\vec{t}$ of $\vec{b}-[\sigma,\dots,\sigma]$,
which is in turn bounded by the entries of the corresponding subset
of $\vec{s}$. 
\begin{example}
\label{exm:reducingDegree}From \prettyref{exm:continueComputingNullspaceBasisByColumns},
note that instead of using the residual 
\[
\mathbf{F}\mathbf{P}_{2}=\left[\begin{array}{ccc}
0 & x^{3}+x^{4}+x^{5} & x^{4}\\
x^{3}+x^{4} & x^{5} & x^{3}+x^{5}
\end{array}\right]
\]
 to compute a $[5,5,5]$-minimal kernel basis of $\mathbf{F}$,%
\begin{comment}
$\left(\mathbf{F}\mathbf{P}_{2},\tau,[5,5,5]\right)$-basis,
\end{comment}
{} we can instead use 
\[
\mathbf{G}=\mathbf{F}\mathbf{P}_{2}/x^{3}=\left[\begin{array}{ccc}
0 & 1+x+x^{2} & x\\
1+x & x^{2} & 1+x^{2}
\end{array}\right]
\]
 to compute a $[2,2,2]$-minimal kernel basis of $\mathbf{G}$%
\begin{comment}
$\left(\mathbf{G},\tau,[2,2,2]\right)$-basis
\end{comment}
. The column degrees of $\mathbf{G}$ are bounded by the new shift
$\left[2,2,2\right]$, which is in turn bounded by the corresponding
entries $\left[3,3,3\right]$ of $\vec{s}$. 
\end{example}
At this point, using \prettyref{thm:continueComputingNullspaceBasisByColumns}
and \prettyref{cor:boundOnFPAfterLowerTermsRemoved}, the problem
is reduced to computing a $\vec{t}$-minimal kernel basis of $\mathbf{G}=\mathbf{F}\mathbf{P}_{2}/x^{3s}$,
which still has row dimension $m$. But its column dimension is now
bounded by $3m/2$. Also notice that as in the original problem, the
column degrees of the new input matrix $\mathbf{G}$ are bounded by
the corresponding entries of the new shift $\vec{t}$. In addition,
as the new shift $\vec{t}$ is bounded component-wise by a subset
of the old shift $\vec{s}$, the new problem is no more difficult
than the original problem.


\subsection{\label{sub:continueComputingNullspaceBasisByRows}Reducing the row
dimension }

We now turn to the new problem of computing a $\vec{t}$-minimal kernel
basis of $\mathbf{G}$. Let 
\[
\mathbf{G}=\left[\begin{array}{c}
\mathbf{G}_{1}\\
\mathbf{G}_{2}
\end{array}\right]
\]
 %[\mathbf{G}_{1},\mathbf{G}_{2}]^{T} \]
 with $\mathbf{G}_{1}$ having $\left\lfloor m/2\right\rfloor $ rows
and $\mathbf{G}_{2}$ having $\left\lceil m/2\right\rceil $ rows.
If we compute a $\vec{t}$-minimal kernel basis $\mathbf{N}_{1}$
of $\mathbf{G}_{1}$, where $\mathbf{N}_{1}$ has $\vec{t}$-column
degrees $\vec{u}$, then compute a $\vec{u}$-minimal kernel basis
$\mathbf{N}_{2}$ of $\mathbf{G}_{2}\mathbf{N}_{1}$, then the next
theorem %\prettyref{thm:continueComputingkernelBasisByRows} 
shows that $\mathbf{N}_{1}\mathbf{N}_{2}$ is a $\vec{t}$-minimal
kernel basis of $\mathbf{G}$. 
\begin{thm}
\label{thm:continueComputingNullspaceBasisByRows}Let %an input matrix
$\mathbf{G}=\left[\mathbf{G}_{1}^{T},\mathbf{G}_{2}^{T}\right]^{T}\in\mathbb{K}\left[x\right]^{m\times n}$
and $\vec{t}$ a shift vector. If $\mathbf{N}_{1}$ is a $\vec{t}$-minimal
kernel basis of $\mathbf{G}_{1}$ with $\vec{t}$-column degrees $\vec{u}$,
and $\mathbf{N}_{2}$ is a $\vec{u}$-minimal kernel basis of $\mathbf{G}_{2}\mathbf{N}_{1}$
with $\vec{u}$-column degrees $\vec{v}$, then $\mathbf{N}_{1}\mathbf{N}_{2}$
is a $\vec{t}$-minimal kernel basis of $\mathbf{G}$ with $\vec{t}$-column
degrees $\vec{v}$.\end{thm}
\begin{proof}
The proof is very similar to the proof of \prettyref{thm:continueComputingNullspaceBasisByColumns}.
It is clear that $\mathbf{G}\mathbf{N}_{1}\mathbf{N}_{2}=0$ hence
$\mathbf{N}_{1}\mathbf{N}_{2}$ is a kernel basis of $\mathbf{G}$.
We now show that $\mathbf{N}_{1}\mathbf{N}_{2}$ is $\vec{t}$-column
reduced and has $\vec{t}$-column degrees $\vec{v}$, or equivalently,
$x^{\vec{t}}\mathbf{N}_{1}\mathbf{N}_{2}$ is column reduced. Notice
that $x^{\vec{t}}\mathbf{N}_{1}$ has column degrees $\vec{u}$ and
a full rank leading column coefficient matrix $N_{1}$. Hence $x^{\vec{t}}\mathbf{N}_{1}x^{-\vec{u}}$
has column degrees $\left[0,\dots,0\right]$. Again, if one is concerned
about the entries not being polynomials, one can simply multiply the
matrix by $x^{\xi}$ to shift the degrees up. Similarly, $x^{\vec{u}}\mathbf{N}_{2}x^{\vec{v}}$
has column degrees $\left[0,\dots,0\right]$ and a full rank leading
column coefficient matrix $N_{2}$. Putting them together, $x^{\vec{t}}\mathbf{N}_{1}x^{-\vec{u}}x^{\vec{u}}\mathbf{N}_{2}x^{-\vec{v}}=x^{\vec{t}}\mathbf{N}_{1}\mathbf{N}_{2}x^{-\vec{v}}$
has column degrees $[0,\dots,0]$ and a full rank leading column coefficient
matrix $N_{1}N_{2}$. It follows that $x^{\vec{t}}\mathbf{N}_{1}\mathbf{N}_{2}$
has column degrees $\vec{v}$, or equivalently, the $\vec{t}$-column
degrees of $\mathbf{N}_{1}\mathbf{N}_{2}$ is $\vec{v}$.

It remains to show that any $\mathbf{n}$ satisfying $\mathbf{G}\mathbf{n}=0$
must be a linear combination of the columns of $\mathbf{N}_{1}\mathbf{N}_{2}$.
First notice that $\mathbf{n}=\mathbf{N}_{1}\mathbf{a}$ for some
polynomial vector $\mathbf{a}$ since $\mathbf{N}_{1}$ is a kernel
basis of $\mathbf{G}_{1}$. Also, $\mathbf{G}\mathbf{n}=0$ implies
that $\mathbf{G}_{2}\mathbf{N}_{1}\mathbf{a}=0$, hence $\mathbf{a}=\mathbf{N}_{2}\mathbf{b}$
for some vector $\mathbf{b}$ as $\mathbf{N}_{2}$ is a kernel basis
of $\mathbf{G}_{2}\mathbf{N}_{1}$. We now have $\mathbf{n}=\mathbf{N}_{1}\mathbf{N}_{2}\mathbf{b}$
as required.\end{proof}
\begin{example}
Let us compute a $\vec{t}$-minimal kernel basis of 
\[
\mathbf{G}=\left[\begin{array}{ccc}
0 & 1+x+x^{2} & x\\
1+x & x^{2} & 1+x^{2}
\end{array}\right]
\]
 from \prettyref{exm:reducingDegree}, where $\vec{t}=\left[2,2,2\right]$.
Then 
\begin{align*}
\mathbf{G}_{1} & =\left[\begin{array}{ccc}
0 & 1+x+x^{2} & x\end{array}\right]\mbox{and }\mathbf{G}_{2}=\left[\begin{array}{ccc}
1+x & x^{2} & 1+x^{2}\end{array}\right].
\end{align*}


We first compute a $\vec{t}$-minimal kernel basis $\mathbf{N}_{1}$
of $\mathbf{G}_{1}$: 
\[
\mathbf{N}_{1}=\left[\begin{array}{cc}
1 & 0\\
0 & x\\
0 & 1+x+x^{2}
\end{array}\right]
\]
 with its $\vec{t}$-column degrees $\vec{u}=\left[2,4\right]$. Next,
we compute a $\vec{u}$-minimal kernel basis $\mathbf{N}_{2}$ of
$\mathbf{G}_{2}\mathbf{N}_{1}=\left[\begin{array}{cc}
1+x & 1+x+x^{4}\end{array}\right]$: 
\[
\mathbf{N}_{2}=[1+x+x^{4},\ 1+x]^{T}.
\]
 Then 
\[
\mathbf{N}_{1}\mathbf{N}_{2}=[1+x+x^{4},\ x+x^{2},\ 1+x^{3}]^{T}
\]
 is a $\vec{t}$-minimal kernel basis of $\mathbf{G}$. 
\end{example}
While \prettyref{thm:continueComputingNullspaceBasisByColumns} allows
us to compute kernel bases by columns, which then reduces the column
dimensions, \prettyref{thm:continueComputingNullspaceBasisByRows}
shows that that the kernel bases can also be computed by rows, which
then reduces the row dimensions. Again, we need to check that these
computations can be done efficiently. In the following, %
\begin{comment}
\prettyref{lem:sizeOfG2N1} shows that the size of $\mathbf{G}_{2}\mathbf{N}_{1}$
is within a required bound. Then 
\end{comment}
{} \prettyref{lem:mutiplyingG2N1} and \prettyref{lem:multiplyingN1N2}
show that the multiplication $\mathbf{G}_{2}\mathbf{N}_{1}$ and the
multiplication $\mathbf{N}_{1}\mathbf{N}_{2}$ can be done efficiently,
which are again consequences of \prettyref{thm:multiplyUnbalancedMatrices}.%
\begin{comment}
\begin{lem}
\label{lem:sizeOfG2N1}The sum of the column degrees of $\mathbf{G}_{2}\mathbf{N}_{1}$
is bounded by $\xi$.\end{lem}
\begin{proof}
This follows from \prettyref{thm:boundOfSumOfShiftedDegreesOfKernelBasis}
and \prettyref{lem:boundOnDegreesOfFA}.\end{proof}
\end{comment}

\begin{lem}
\label{lem:mutiplyingG2N1}The multiplication of $\mathbf{G}_{2}$
and $\mathbf{N}_{1}$ can be done with a cost of $O^{\sim}(m^{\omega-1}\xi)$.\end{lem}
\begin{proof}
\prettyref{thm:multiplyUnbalancedMatrices} applies directly here.\end{proof}
\begin{lem}
\label{lem:multiplyingN1N2}The multiplication of $\mathbf{N}_{1}$
and $\mathbf{N}_{2}$ can be done with a cost of $O^{\sim}(m^{\omega-1}\xi)$.\end{lem}
\begin{proof}
\prettyref{thm:multiplyUnbalancedMatrices} applies because the sum
of the column degrees of $\mathbf{N}_{1}$ is bounded by the sum of
the $\vec{t}$-column degrees of $\mathbf{N}_{1}$, which is $\sum\vec{u}\le\xi$,
and by \prettyref{thm:boundOfSumOfShiftedDegreesOfKernelBasis} the
sum of $\vec{u}$-column degrees of $\mathbf{N}_{2}$ is also bounded
by $\xi$. 
\end{proof}

\subsection{Recursive computation}

The computation of $\mathbf{N}_{1}$ and $\mathbf{N}_{2}$ is identical
to the original problem, only the dimension has decreased. For computing
$\mathbf{N}_{1}$, the dimension of the input matrix $\mathbf{G}_{1}$
is bounded by $\left\lfloor m/2\right\rfloor \times\left(3m/2\right)$.
For computing $\mathbf{N}_{2}$ , the dimension of input matrix $\mathbf{G}_{2}\mathbf{N}_{1}$
is bounded by $\left\lceil m/2\right\rceil \times(3m/2)$. The column
degrees of $\mathbf{G}_{1}$ are bounded by the entries of $\vec{t}$,
with $\sum\vec{t}\le\xi$. Similarly, the column degrees of $\mathbf{G}_{2}\mathbf{N}_{1}$
are bounded by the entries of $\vec{u}$, with $\sum\vec{u}\le\xi$.
Hence, the same computation process can be repeated on these two smaller
problems. This gives a recursive algorithm, shown in \prettyref{alg:minimalNullspaceBasisWithRankProfile}.

\begin{algorithm}[t]
\caption{$\mnb(\mathbf{F},\vec{s})$}
\label{alg:minimalNullspaceBasis}

\begin{algorithmic}[1]
\REQUIRE{$\mathbf{F}\in\mathbb{K}\left[x\right]^{m\times n}$, $\vec{s}=[s_{1},\dots,s_{n}]\in\mathbb{Z}^{n}$
with entries arranged in non-decreasing order and bounding the corresponding
column degrees of $\mathbf{F}$. }

\ENSURE{A $\vec{s}$-minimal kernel basis of $\mathbf{F}$.}

\STATE{$\xi:=\sum_{i=1}^{n}s_{i}$; $\rho:=\sum_{i=n-m+1}^{n}s_{i};$ $s:=\rho/m$; }

\STATE{$\left[\mathbf{P},\vec{b}\right]:=\mab\left(\mathbf{F},3s,\vec{s}\right)$,
a $\left(\mathbf{F},3s,\vec{s}\right)$-basis with the columns of
$\mathbf{P}$ and the entries of is $\vec{s}$-column degrees $\vec{b}$
arranged so that the entries of $\vec{b}$ are in non-decreasing order;}\label{line:orderBasis}

\STATE{$\left[\mathbf{P}_{1},\mathbf{P}_{2}\right]:=\mathbf{P}$ where $\mathbf{P}_{1}$
consists of all columns $\mathbf{p}$ of $\mathbf{P}$ satisfying
$\mathbf{F}\mathbf{p}=0$;}

\IF{$m=1$}

\ifbody{\RETURN $\mathbf{P}_{1}$ 

\ELSE{}

\STATE{ $\vec{t}:=\deg_{\vec{s}}\mathbf{P}_{2}-\left[3s,3s,\dots,3s\right];$}

\STATE{$\mathbf{G}:=\mathbf{F}\mathbf{P}_{2}/x^{3s}$;}\label{line:multiplyFP2}

\STATE{$\left[\mathbf{G}_{1}^{T},\mathbf{G}_{2}^{T}\right]^{T}:=\mathbf{G}$,
with $\mathbf{G}_{1}$ having $\left\lfloor m/2\right\rfloor $ rows
and $\mathbf{G}_{2}$ having $\left\lceil m/2\right\rceil $ rows;}

\STATE{$\mathbf{N}_{1}:=\mnb\left(\mathbf{G}_{1},\vec{t}\right);$ }

\STATE{$\mathbf{N}_{2}:=\mnb\left(\mathbf{G}_{2}\mathbf{N}_{1},\cdeg_{\vec{t}}\mathbf{N}_{1}\right);$}\label{line:multiplyG2N1}

\STATE{$\mathbf{Q}:=\mathbf{N}_{1}\mathbf{N}_{2}$;}\label{line:multiplyN1N2}

\RETURN $\left[\mathbf{P}_{1},\mathbf{P}_{2}\mathbf{Q}\right]$\label{line:multiplyP2Q}}
\end{algorithmic}
\end{algorithm}



Before analyzing the computational complexity of \prettyref{alg:minimalNullspaceBasisWithRankProfile}
in the following section, we provide a proof of \prettyref{thm:multiplyUnbalancedMatrices},
which is needed to efficiently multiply matrices with unbalanced degrees
in the algorithm.


\subsection{Proof of \prettyref{thm:multiplyUnbalancedMatrices}}

%The recursive procedure described in the next section reduces a single null space
%computation to a set of null space computations of smaller size. These smaller problems
%are constructed by the computations of residuals which in all cases involves the multiplication
%of two polynomial matrices having unbalanced degrees. In this section, we present a method for 
%multiplying polynomial matrices that occur in the algorithm given in the following section.  These matrices have unbalanced degrees so it would be inefficient to directly apply the 
%standard matrix multiplication.


%\begin{lem}
%\label{lem:multiplyUnbalancedMatrices} Let $\vec{s}$ be a shift
%ordered in terms of increasing values and $\xi$, a bound on the sum
%of the entries of $\vec{s}$. Let $\mathbf{A}\in\mathbb{K}\left[x\right]^{m\times n}$,
%with column degrees bounded by $\vec{s}$ and $\mathbf{B}\in\mathbb{K}\left[x\right]^{n\times k}$
%with $k\in O\left(m\right)$ and the sum $\theta$ of its $\vec{s}$-column
%degrees satisfying $\theta\in O\left(\xi\right)$. Then we can multiply
%$\mathbf{A}$ and $\mathbf{B}$ with a cost of $O^{\sim}(nm^{\omega-2}\xi)$. \end{lem}


In this subsection we give a proof of \prettyref{thm:multiplyUnbalancedMatrices}.
\begin{proof}
Recall that $\vec{s}$ is a shift with entries ordered in terms of
increasing values and $\xi$ is a bound on the sum of the entries
of $\vec{s}$. We wish to determine the cost of multiplying the two
polynomials matrices $\mathbf{A}\in\mathbb{K}\left[x\right]^{m\times n}$
and $\mathbf{B}\in\mathbb{K}\left[x\right]^{n\times k}$ where $\mathbf{A}$
has column degrees bounded by $\vec{s}$ and where $k\in O\left(m\right)$
and the sum $\theta$ of its $\vec{s}$-column degrees satisfies $\theta\in O\left(\xi\right)$.
The goal is to show that these polynomial matrices can be multiplied
%Then we can multiply %$\mathbf{A}$ and $\mathbf{B}$ 
with a cost of $O^{\sim}(nm^{\omega-2}\xi)$.

For simplicity we assume $m$ is a power of $2$, something which
can be achieved by appending zero rows to $\mathbf{F}$. We divide
the matrix $\mathbf{B}$ into $\log m$ column blocks according to
the $\vec{s}$-column degrees of its columns. Let 
\[
\mathbf{B}=\left[\begin{array}{ccccc}
\mathbf{B}^{\left(\log m\right)} & \mathbf{B}^{\left(\log m-1\right)} & \cdots & \mathbf{B}^{\left(2\right)} & \mathbf{B}^{\left(1\right)}\end{array}\right],
\]
 with $\mathbf{B}^{\left(\log m\right)}$, $\mathbf{B}^{\left(\log m-1\right)},$
$\mathbf{B}^{\left(\log m-2\right)}$, ... , $\mathbf{B}^{\left(2\right)}$,
$\mathbf{B}^{\left(1\right)}$ having $\vec{s}$-column degrees in
the range $\left[0,2\xi/m\right]$, $(2\xi/m,4\xi/m]$, $(4\xi/m,8\xi/m]$,
...,$(\xi/4,\xi/2]$, $(\xi/2,\theta]$, respectively. We will multiply
$\mathbf{A}$ with each $\mathbf{B}^{\left(i\right)}$ separately.

We also divide the matrix $\mathbf{A}$ into $\log m$ column blocks
and each matrix $\mathbf{B}^{\left(i\right)}$ into $\log m$ row
blocks according to the size of the corresponding entries in $\vec{s}$.
Set 
\begin{eqnarray*}
~~~~\vec{s} & = & \left[\begin{array}{cccc}
\vec{s}_{\log m} & \vec{s}_{\log m-1} & \cdots & \vec{s}_{1}\end{array}\right]\\
\mathbf{A} & = & \left[\begin{array}{cccc}
\mathbf{A}_{\log m} & \mathbf{A}_{\log m-1} & \cdots & \mathbf{A}_{1}\end{array}\right]\\
\mathbf{B} & = & \left[\begin{array}{cccc}
\mathbf{B}^{\left(\log m\right)} & \mathbf{B}^{\left(\log m-1\right)} & \cdots & \mathbf{B}^{\left(1\right)}\end{array}\right]\\
 & = & \left[\begin{array}{cccc}
\mathbf{B}_{\log m}^{\left(\log m\right)} & \mathbf{B}_{\log m}^{\left(\log m-1\right)} & \cdots & \mathbf{B}_{\log m}^{\left(1\right)}\\
\vdots &  &  & \vdots\\
\mathbf{B}_{1}^{\left(\log m\right)} & \mathbf{B}_{1}^{\left(\log m-1\right)} & \cdots & \mathbf{B}_{1}^{\left(1\right)}
\end{array}\right]
\end{eqnarray*}
 %
\begin{comment}
\wei{Shouldn't they be reversed? We are not assign to $\vec{s},\mathbf{A},\mathbf{B},$
but to $\vec{s}_{i},\mathbf{A}_{i},\mathbf{B}_{i}$. Same issue with
other assignments as well} \george{I think it is more readable this
way.} 
\end{comment}
with $\vec{s}_{\log m},\vec{s}_{\log m-1},\dots,\vec{s}_{1}$ having
entries in the range $\left[0,2\xi/m\right]$, $(2\xi/m,4\xi/m]$,
$(4\xi/m,8\xi/m]$, ..., $(\xi/2,\xi]$ respectively. Also the column
dimension of $\mathbf{A}_{j}$ and the row dimension of $\mathbf{B}_{j}^{\left(i\right)}$
match that of $\vec{s}_{j}$ for $j$ from $1$ to $\log m$.

Notice that $\mathbf{B}_{(j)}^{(i)}$ for $i>j$ must be zero. Otherwise,
as $\vec{s}_{j}>\xi/2^{j}\ge\xi/2^{i-1}$, the $\vec{s}$-column degree
of $\mathbf{B}^{(i)}$ would exceed $\xi/2^{i-1}$, a contradiction
since by definition the $\vec{s}$-column degree of $\mathbf{B}^{(i)}$
is bounded by $\xi/2^{i-1}$ when $i>1$. So $\mathbf{B}$ in fact
has a block triangular shape 
\[
\mathbf{B}=\left[\begin{array}{cccc}
\mathbf{B}_{\log m}^{\left(\log m\right)} & \mathbf{B}_{\log m}^{\left(\log m-1\right)} & \cdots & \mathbf{B}_{\log m}^{\left(1\right)}\\
 & \mathbf{B}_{\log m-1}^{\left(\log m-1\right)} &  & \vdots\\
 &  & \ddots\\
 &  &  & \mathbf{B}_{1}^{\left(1\right)}
\end{array}\right]
\]
 (while remembering that the blocks have varying sizes).

First consider the multiplication 
\[
\mathbf{A}\mathbf{B}^{\left(1\right)}=\left[\begin{array}{ccc}
\mathbf{A}_{\log m} & \cdots & \mathbf{A}_{1}\end{array}\right]\left[\begin{array}{l}
\mathbf{B}_{\log m}^{\left(1\right)}\\
\vdots\\
\mathbf{B}_{1}^{\left(1\right)}
\end{array}\right].
\]
 Note that there are $O\left(1\right)$ columns in $\mathbf{B}^{(1)}$
since $\theta\in O\left(\xi\right)$. We do this in $\log m$ steps.
At step $j$ for $j$ from $1$ to $\log m$ we multiply $\mathbf{A}_{j}$
and $\mathbf{B}_{j}^{(1)}$. The column dimension of $\mathbf{A}_{j}$,
which is the same as the row dimension of $\mathbf{B}_{j}^{(1)}$,
is less than $2^{j}$. The degree of $\mathbf{B}_{j}^{(1)}$ is $O\left(\xi\right)$.
To use fast multiplication, we expand $\mathbf{B}_{j}^{(1)}$ to a
matrix $\bar{\mathbf{B}}_{j}^{(1)}$ with degree less than $\delta\in\Theta(\xi/2^{j})$
and column dimension $q\in O(2^{j})$ as follows. Write 
\[
\mathbf{B}_{j}^{(1)}=\mathbf{B}_{j,0}^{(1)}+\mathbf{B}_{j,1}^{(1)}x^{\delta}+\dots+\mathbf{B}_{j,q-1}^{(1)}x^{\delta(q-1)}=\sum_{k=0}^{q-1}\mathbf{B}_{j,k}^{(1)}x^{\delta k}
\]
 with each $\mathbf{B}_{j,k}^{(1)}$ having degree less than $\delta.$
Set 
\[
\bar{\mathbf{B}}_{j}^{(1)}=\left[\mathbf{B}_{j,0}^{(1)},\mathbf{B}_{j,1}^{(1)},\dots,\mathbf{B}_{j,q-1}^{(1)}\right]
\]
. We can then multiply $\mathbf{A}_{j}$, which has dimension $m\times O(2^{j})$
for $j<\log m$, and $\bar{\mathbf{B}}_{j}^{(1)}$, which has dimension
$O(2^{j})\times O(2^{j})$ for $j<\log m$, with a cost of 
\begin{align*}
O^{\sim}\left((m/2^{j})\left(2^{j}\right)^{\omega}\xi/2^{j}\right) & =O^{\sim}\left(\left(2^{j}\right)^{\omega-2}m\xi\right).\\
 & \subset O^{\sim}\left(m^{\omega-1}\xi\right)\subset O^{\sim}(nm^{\omega-2}\xi)
\end{align*}
 For $j=\log m$, $\mathbf{A}_{j}$ has dimension $m\times O\left(n\right)$,
$\bar{\mathbf{B}}_{j}^{\left(1\right)}$ has dimension $O\left(n\right)\times O(m)$,
and their degrees are $O\left(\xi/m\right)$. Hence they can be multiplied
with a cost of $O^{\sim}\left((n/m)m^{\omega}(\xi/m)\right)=O^{\sim}\left(nm^{\omega-2}\xi\right)$.
Adding up the columns of $\mathbf{A}_{j}\bar{\mathbf{B}}_{j}^{(1)}$
gives $\mathbf{A}_{j}\mathbf{B}_{j}^{(1)}$ and costs $O(m\xi)$.
Therefore, multiplying $\mathbf{A}$ and $\mathbf{B}^{(1)}$ over
$\log(m)$ steps costs $O^{\sim}\left(nm^{\omega-2}\xi\right)$.

Next we multiply $\mathbf{A}$ with $\mathbf{B}^{(2)}$. We proceed
in the same way as before, but notice that $\mathbf{A}_{1}\mathbf{B}_{1}^{(2)}$
is no longer needed since $\mathbf{B}_{1}^{(2)}=0$. Multiplying $\mathbf{A}$
and $\mathbf{B}^{(2)}$ also costs $O^{\sim}\left(nm^{\omega-2}\xi\right)$.

Continuing to doing this, gives a costs of $O^{\sim}\left(nm^{\omega-2}\xi\right)$
to multiply $\mathbf{A}$ with the columns $\mathbf{B}^{(i)}$ for
$i$ from $1$ to $\log m$. As before, we recall that $\mathbf{B}_{(j)}^{(i)}=0$
for $j>i$. The overall cost for $i$ from 1 to $\log m$ is therefore
$O^{\sim}\left(nm^{\omega-2}\xi\right)$ to multiply $\mathbf{A}$
and $\mathbf{B}$. \end{proof}




\section{Computational Complexity}

\label{sec:complexityNullspaceBasis}

\begin{comment}
The result of the previous section is a recursive algorithm, shown
in \prettyref{alg:minimalNullspaceBasisWithRankProfile} for the computation
of a minimal kernel. It remains to determine the complexity.

%\input{algorithmkernelBasis.tex}



\subsection{Cost Analysis of Algorithm $\mnb$}
\end{comment}


For the cost analysis we first consider the case where the column
dimension $n$ is not much bigger than the row dimension $m$. 
\begin{thm}
\label{thm:costLowColDimension}If $n\in O\left(m\right)$, then the
cost of \prettyref{alg:minimalNullspaceBasisWithRankProfile} is $O^{\sim}\left(m^{\omega-1}\xi\right)=O^{\sim}\left(m^{\omega-1}\rho\right)$
field operations.%
\begin{comment}
to compute a $\vec{s}$-minimal kernel basis of $\mathbf{F}$. 
\end{comment}
\end{thm}
\begin{proof}
We may assume $m$ is a power of $2$, which can be achieved by appending
zero rows to $\mathbf{F}$. Note that $\rho\in\Theta\left(\xi\right)$
when $n\in O\left(m\right)$. Then the order basis computation at
\prettyref{line:orderBasis} costs $O^{\sim}\left(n^{\omega}s\right)=O^{\sim}\left(m^{\omega-1}\rho\right)$.
The multiplications at \prettyref{line:multiplyFP2} and \prettyref{line:multiplyP2Q}
cost $O^{\sim}\left(nm^{\omega-2}\xi\right)=O^{\sim}\left(m^{\omega-1}\xi\right)$.
The remaining operations including multiplications at \prettyref{line:multiplyG2N1}
and \prettyref{line:multiplyN1N2} cost $O^{\sim}\left(m^{\omega-1}\xi\right)$.
Let $g(m,\xi)$ be the computational cost of the original problem.
Then we have the recurrence relation 
\[
g(m,\xi)\in O^{\sim}(m^{\omega-1}\xi)+g(m/2,\xi)+g(m/2,\xi),
\]
 with the base case $g(1,\xi)\in O^{\sim}\left(\xi\right)$, the cost
of just an order basis computation at $m=1.$ This gives $g(m,\xi)\in O^{\sim}(m^{\omega-1}\xi)$
field operations as the cost of the algorithm. 
\end{proof}
We now consider the general case where the column dimension $n$ can
be much bigger than the row dimension $m$. 
\begin{thm}
\label{thm:costGeneral}\prettyref{alg:minimalNullspaceBasisWithRankProfile}
costs $O^{\sim}\left(n^{\omega}s\right)$ field operations in general.\end{thm}
\begin{proof}
The order basis computation at \prettyref{line:orderBasis} costs
$O^{\sim}\left(n^{\omega}s\right)$ in general, which dominates the
cost of other operations. The problem is then reduced to one where
we have column dimension $O\left(m\right)$, which is handled by \prettyref{thm:costLowColDimension}
with a cost of $O^{\sim}\left(m^{\omega-1}\xi\right)\in O^{\sim}\left(n^{\omega}s\right)$. 
\end{proof}
When we have the important special case where the shift $\vec{s}=\left[s,\dots,s\right]$
is uniform then \prettyref{alg:minimalNullspaceBasisWithRankProfile}
has a lower cost. Indeed we notice that the order basis computation
at \prettyref{line:orderBasis} costs $O^{\sim}\left(n^{\omega-1}ms\right)$
using the algorithm from \prettyref{chap:OrderBasis}. In addition,
the multiplication of $\mathbf{F}$ and $\mathbf{P}_{2}$ at \prettyref{line:multiplyFP2}
and the multiplication of $\mathbf{P}_{2}$ and $\mathbf{Q}$ at \prettyref{line:multiplyP2Q}
both cost $O^{\sim}\left(nm^{\omega-1}s\right)$ as shown in \prettyref{lem:multiplyFP2WithUniformShift}
and \prettyref{lem:multiplyP2QWithUniformShift}. 
\begin{lem}
\label{lem:multiplyFP2WithUniformShift}If the degree of $\mathbf{F}$
is bounded by $s$, then the multiplication of $\mathbf{F}$ and $\mathbf{P}_{2}$
at \prettyref{line:multiplyFP2} costs $O^{\sim}\left(nm^{\omega-1}s\right)$.\end{lem}
\begin{proof}
Since $\mathbf{P}_{2}$ is a part of a $\left(\mathbf{F},3s,\vec{s}\right)$-basis,
its degree is bounded by $3s$. It has dimension $n\times O\left(m\right)$
from \prettyref{thm:dimensionOfPartialNullspaceBasisBasedOnOrder}.
Multiplying $\mathbf{F}$ and $\mathbf{P}_{2}$ therefore costs $(n/m)O^{\sim}\left(m^{\omega}s\right)=O^{\sim}\left(nm^{\omega-1}s\right)$.\end{proof}
\begin{lem}
\label{lem:multiplyP2QWithUniformShift}If $\mathbf{F}$ has degree
$s$, then the multiplication of $\mathbf{P}_{2}$ and $\mathbf{Q}$
at \prettyref{line:multiplyP2Q} costs $O^{\sim}\left(nm^{\omega-1}s\right)$.\end{lem}
\begin{proof}
First note that the dimension of $\mathbf{Q}$ is $O\left(m\right)\times O\left(m\right)$
since it is a $\vec{t}$-minimal kernel basis of $\mathbf{G}=\mathbf{F}\mathbf{P}_{2}/x^{3s}$,
which has dimension $m\times O\left(m\right)$. In addition, by \prettyref{thm:boundOfSumOfShiftedDegreesOfKernelBasis},
the sum of the $\vec{t}$-column degrees of $\mathbf{Q}$ is bounded
by $\sum\vec{t}$, which is bounded by $O\left(ms\right)$ since $\vec{t}$
has $O\left(m\right)$ entries all bounded by $s$.

Now \prettyref{thm:multiplyUnbalancedMatrices} and its proof still
work. The current situation is even simpler as we do not need to subdivide
the columns of $\mathbf{P}_{2}$, which has degree bounded by $3s$
and dimension $n\times O\left(m\right)$. We just need to separate
the columns of $\mathbf{Q}$ to $O\left(\log m\right)$ groups with
degree ranges $\left[0,2s\right],$ $(2s,4s],$ $(4s,8s],$ $\dots$,
and multiply $\mathbf{P}_{2}$ with each group in the same way as
in \prettyref{thm:multiplyUnbalancedMatrices}, with each of these
$O\left(\log m\right)$ multiplications costs $(n/m)O^{\sim}\left(m^{\omega}s\right)=O^{\sim}\left(nm^{\omega-1}s\right)$.\end{proof}
\begin{thm}
\label{thm:costOfMinimalNullspaceBasisWithUniformShift}If $\vec{s}=\left[s,\dots,s\right]$
is uniform, then \prettyref{alg:minimalNullspaceBasisWithRankProfile}
costs $O^{\sim}\left(n^{\omega-1}ms\right)$. \end{thm}
\begin{proof}
After the initial order basis computation, which costs $O^{\sim}\left(n^{\omega-1}ms\right),$
and the multiplication of $\mathbf{F}$ and $\mathbf{P}_{2}$, which
costs $O^{\sim}\left(nm^{\omega-1}s\right)$ from \prettyref{lem:multiplyFP2WithUniformShift},
the column dimension is reduced to $O\left(m\right)$, allowing \prettyref{thm:costLowColDimension}
to apply for computing a $\vec{t}$-minimal kernel basis of $\mathbf{F}\mathbf{P}_{2}/x^{3s}$.
Hence the remaining work costs $O^{\sim}\left(m^{\omega}s\right)$.
The overall cost is therefore dominated by the cost $O^{\sim}\left(n^{\omega-1}ms\right)$
of the initial order basis computation.\end{proof}
\begin{cor}
\label{cor:costOfMinimalNullspaceBasis}If the input matrix $\mathbf{F}$
has degree $d$, then a minimal kernel basis of $\mathbf{F}$ can
be computed with a cost of $O^{\sim}\left(n^{\omega-1}md\right)$. \end{cor}
\begin{proof}
We can just set the shift $\vec{s}$ to $\left[d,\dots,d\right]$
and apply \prettyref{thm:costOfMinimalNullspaceBasisWithUniformShift}. \end{proof}





\chapter{\label{chap:Matrix-inverse}Matrix inverse}

In this chapter, we consider the problem of computing the inverse
of a $n\times n$ polynomial matrix with degree $d$. \citet{jeannerod-villard:05}
gave a deterministic algorithm for this problem that costs $O^{\sim}\left(n^{3}d\right)$
field operations. But their algorithm only works well on input matrices
that are generic with dimension a power of 2. \citet{storjohann:2008}
gave another algorithm with a similar cost, but the algorithm is randomized
Las Vegas. In the following, we show that Jeannerod and Villard's
algorithm can be improved to handle any matrix with a cost of $O^{\sim}\left(n^{3}d\right)$
using new results from this thesis. The algorithm given here is still
deterministic. In fact, an additional improvement of our algorithm
is that it is more general, as its computational cost is given in
terms of the sum of the column (or row) degrees. If $\xi$ is the
minimum of the sum of the column degrees and the sum of the row degrees
of the input matrix, then the inverse can be computed with $O^{\sim}\left(n^{2}\xi\right).$
In the following, we assume without loss of generality that the sum
of the column degrees is the minimum sum.

\begin{comment}
Note that rank sensitive computation for order basis is not as natural,
since for order basis we work with power series in general, whose
rank may never be truly computed. In addition, the computed basis
does not correspond to the rank. For these reasons, we do not pursue
rank sensitive computations for order basis. 
\end{comment}
\begin{algorithm}[t]
\caption{$\inv(\mathbf{F},\vec{s})$}
\label{alg:matrixInverse}

\begin{algorithmic}
[1]\REQUIRE{$\mathbf{F}\in\mathbb{K}\left[x\right]^{n\times n}$; $\vec{s}$ is
initially set to the column degrees of $\mathbf{F}$. It keeps track
of the degrees.}

\ENSURE{$\mathcal{A}=\left[\mathbf{A}_{1},\dots,\mathbf{A}_{\left\lceil \log n\right\rceil }\right],\mathbf{B}$
with $\mathbf{A}_{1},\dots,\mathbf{A}_{\left\lceil \log n\right\rceil },\mathbf{B}\in\mathbb{K}\left[x\right]^{n\times n}$
such that $\mathbf{A}_{1}\dots\mathbf{A}_{\left\lceil \log n\right\rceil }\mathbf{B}^{-1}=\mathbf{F}^{-1}$
if $\mathbf{F}$ is nonsingular, or fail if $\mathbf{F}$ is singular.}

\begin{comment}
\STATE{$\vec{s}:=$column degrees of $\mathbf{F}$;}
\end{comment}


\STATE{$\left[\mathbf{F}_{1}^{T},\mathbf{F}_{2}^{T}\right]:=\mathbf{F}^{T}$
with $\mathbf{F}_{1}$ consists of the top $\left\lceil n\right\rceil $
rows of $\mathbf{F}$;}

\STATE{\textbf{if }$\mathbf{F}=0$ \textbf{then} fail \textbf{endif};}

\begin{comment}
\IF{$\mathbf{F}=0$ }

\ifbody{fail; //$\mathbf{F}$ is singular}
\end{comment}


\STATE{\textbf{if }$n=1$ \textbf{then} \textbf{return} $\left\{ 1,\mathbf{F}\right\} $;
\textbf{endif};}

\begin{comment}
\IF{$n=1$ }

\ifbody{\RETURN $1,\mathbf{F}$;}
\end{comment}


\label{line:nullspaceBasisComputation}\STATE{$\mathbf{N}_{1}:=\mnb(\mathbf{F}_{1},\vec{s})$;$\mathbf{N}_{2}:=\mnb(\mathbf{F}_{2},\vec{s})$;}

\begin{comment}
\IF{$\columnDimension(\mathbf{N}_{1})\ne\left\lfloor n\right\rfloor $
\OR{} $\columnDimension(\mathbf{N}_{2})\ne\left\lceil n\right\rceil $ }

\ifbody{ fail; //$\mathbf{F}$ is singular}
\end{comment}


\STATE{\textbf{if }$\columnDimension(\mathbf{N}_{1})\ne\left\lfloor n\right\rfloor $
\OR{} $\columnDimension(\mathbf{N}_{2})\ne\left\lceil n\right\rceil $
\textbf{then} fail; \textbf{endif};}

\label{line:multiplyFN}\STATE{$\mathbf{R}_{1}:=\mathbf{F}_{1}\mathbf{N}_{2}$;$\mathbf{R}_{2}:=\mathbf{F}_{2}\mathbf{N}_{1}$;}

\STATE{$\left\{ \mathcal{A}^{(1)},\mathbf{H}_{1}\right\} :=\inv(\mathbf{R}_{1},\deg_{\vec{s}}\mathbf{N}_{2})$;
$\left\{ \mathcal{A}^{(2)},\mathbf{H}_{2}\right\} :=\inv(\mathbf{R}_{2},\deg_{\vec{s}}\mathbf{N}_{1})$;}

\STATE{$\mathcal{A}:=\left[\left[\mathbf{N}_{2},\mathbf{N}_{1}\right],\diag(\mathcal{A}_{1}^{(1)},\mathcal{A}_{1}^{(2)}),\dots,\diag(\mathcal{A}_{\left\lceil \log n\right\rceil -1}^{(1)},\mathcal{A}_{\left\lceil \log n\right\rceil -1}^{(2)})\right]$}

\label{line:multiplyNG}\STATE{\textbf{return} $\left\{ \mathcal{A},\diag\left([\mathbf{H}_{1},\mathbf{H}_{2}]\right)\right\} $;}
\end{algorithmic}
\end{algorithm}



\prettyref{alg:matrixInverse} is a recursive version of the algorithm
from \citet{jeannerod-villard:05}, except that we replace the kernel
basis computation at \prettyref{line:nullspaceBasisComputation} and
the matrix multiplications at \prettyref{line:multiplyFN} with the
new algorithms from this thesis. The algorithm also returns a list
of matrices $\mathbf{A}_{1},\dots,\mathbf{A}_{\left\lceil \log n\right\rceil },\mathbf{B}$
satisfying $\mathbf{A}_{1}\dots\mathbf{A}_{\left\lceil \log n\right\rceil }\mathbf{B}^{-1}=\mathbf{F}^{-1}$,
instead of just two matrices $\mathbf{A},\mathbf{B}$ satisfying $\mathbf{A}\mathbf{B}^{-1}=\mathbf{F}^{-1}$.
We can then compute the product $\mathbf{A}=\mathbf{A}_{1}\dots\mathbf{A}_{\left\lceil \log n\right\rceil }$
with a cost of $O^{\sim}\left(n^{2}\xi\right)$. It is interesting
to note that the output $\mathbf{A}_{1},\dots,\mathbf{A}_{\left\lceil \log n\right\rceil },\mathbf{B}$
takes only $O(n\xi\log n)$ space, but the product $\mathbf{A}=\mathbf{A}_{1}\dots\mathbf{A}_{\left\lceil \log n\right\rceil }$
takes $O(n^{2}\xi)$ space.

Let us first look at the cost of the kernel basis computation and
matrix multiplications, since they dominate the cost of \prettyref{alg:matrixInverse}. 
\begin{lem}
The kernel basis computation at \prettyref{line:nullspaceBasisComputation}
costs $O^{\sim}(n^{\omega-1}\xi)$.\end{lem}
\begin{proof}
Just use the earlier kernel basis algorithm with the shift set to
the column degrees of the input matrix.\end{proof}
\begin{lem}
The multiplications $\mathbf{R}_{1}:=\mathbf{F}_{1}\mathbf{N}_{2}$
and $\mathbf{R}_{2}:=\mathbf{F}_{2}\mathbf{N}_{1}$at \prettyref{line:multiplyFN}
cost $O^{\sim}(n^{\omega-1}\xi)$.\end{lem}
\begin{proof}
From \prettyref{thm:boundOfSumOfShiftedDegreesOfKernelBasis} we know
that the sum of the $\vec{s}$-column degrees of $\mathbf{N}_{1}$
and that of $\mathbf{N}_{2}$ are both bounded by $\xi$. Now \prettyref{thm:multiplyUnbalancedMatrices}
can be applied.\end{proof}
\begin{thm}
\label{thm:inverseCost}\prettyref{alg:matrixInverse} costs $O^{\sim}\left(n^{\omega-1}\xi\right)$
field operations to compute an inverse of a nonsingular matrix $\mathbf{F}\in\mathbb{K}\left[x\right]^{n\times n}$,
where $\xi$ is the minimum of the sum of the column degrees and the
sum of the row degrees of the input matrix.\end{thm}
\begin{proof}
If the sum of the row degrees is smaller, we can just transpose the
matrix. Let the cost be $g(n,\xi)$. Then we have the following recurrence
relation:
\begin{eqnarray*}
g(n,\xi) & \in & O^{\sim}(n^{\omega-1}\xi)+g(\left\lceil n/2\right\rceil ,\xi)+g(\left\lfloor n/2\right\rfloor ,\xi)\\
 & \in & O^{\sim}(n^{\omega-1}\xi)+2g(\left\lceil n/2\right\rceil ,\xi)\\
 & \in & O^{\sim}(n^{\omega-1}\xi).
\end{eqnarray*}
 Note that always rounding up $n/2$ to $\left\lceil n/2\right\rceil $
is no worse than assuming $n$ is a power of $2$. In other words,
the entries in the sequence $\left[\left\lceil n/2\right\rceil ,\left\lceil n/4\right\rceil ,\dots,1\right]$
is no larger than the corresponding entries in the sequence $\left[n'/2,n'/4,\dots,1\right]$,
where $n'$ is the smallest power of $2$ that is no less than $n$,
that is, $n'=2^{\left\lceil \log_{2}n\right\rceil }$. \end{proof}
\begin{lem}
The multiplications $\mathbf{A}=\mathbf{A}_{1}\dots\mathbf{A}_{\left\lceil \log n\right\rceil }$
can be done with a cost of $O^{\sim}(n^{2}\xi)$ .\end{lem}
\begin{proof}
Note that $\mathbf{A}_{i}$ for $i\le\log n$ has $2^{i}$ blocks
on the diagonal. Each block of $\mathbf{A}_{i}$ is used to compute
two corresponding blocks of $\mathbf{A}_{i+1}$. Let us first look
at $\mathbf{A}_{1}=\left[\mathbf{N}_{2},\mathbf{N}_{1}\right]$ and
\[
\mathbf{A}_{2}=\begin{bmatrix}\mathbf{N}'_{2} & \mathbf{N}'_{1}\\
 &  & \mathbf{M}'_{2} & \mathbf{M}'_{1}
\end{bmatrix},
\]
 where $\mathbf{N}'_{1},$ $\mathbf{N}'_{2}$ are the kernel bases
of the submatrices $\mathbf{F}'_{1},$ $\mathbf{F}'_{2}$ contained
in 
\[
\mathbf{R}_{1}=\begin{bmatrix}\mathbf{F}'_{1}\\
\mathbf{F}'_{2}
\end{bmatrix}=\mathbf{F}_{1}\mathbf{N}_{2}.
\]
 When multiplying $\mathbf{A}_{1}$ and $\mathbf{A}_{2}$, the submatrix
$\mathbf{N}_{2}$ of $\mathbf{A}_{1}$ is multiplied with the block
$\left[\mathbf{N}'_{2},\mathbf{N}'_{1}\right]$ in $\mathbf{A}_{2}$.
Let $\vec{s}'$ be the list of the $\vec{s}$-column degrees of $\mathbf{N}_{2}$,
where $\vec{s}$ is list of the column degrees of the input matrix
$\mathbf{F}$. Then $\sum\vec{s}'\le\sum\vec{s}=\xi$ by \prettyref{thm:boundOfSumOfShiftedDegreesOfKernelBasis}.
From \prettyref{lem:boundOnDegreesOfFA}, we know the column degrees
of $\mathbf{R}_{1}=\mathbf{F}_{1}\mathbf{N}_{2}$ are bounded component-wise
by the $\vec{s}$-column degrees $\vec{s}'$ of $\mathbf{N}_{2}$,
hence the sum of the column degrees of $\mathbf{R}_{1}$ is also bounded
by $\xi$. It follows that the sum of $\vec{s}'$-column degrees of
$\mathbf{N}'_{1}$ and that of $\mathbf{N}'_{2}$ are each bounded
by $\xi$. We can therefore use \prettyref{thm:multiplyUnbalancedMatrices}
to multiply $\mathbf{N}_{2}$ and $\left[\mathbf{N}'_{2},\mathbf{N}'_{1}\right]$
with a cost of $O^{\sim}\left(n^{\omega-1}\xi\right)$. From \prettyref{lem:boundOnDegreesOfFA},
the $\vec{s}$-column degrees of the product $\mathbf{N}_{2}\left[\mathbf{N}'_{2},\mathbf{N}'_{1}\right]$
are bounded by the $\vec{s}'$-column degrees of $\left[\mathbf{N}'_{2},\mathbf{N}'_{1}\right]$,
hence the sum of the $\vec{s}$-column degrees of each column block
in $\mathbf{N}_{2}\left[\mathbf{N}'_{2},\mathbf{N}'_{1}\right]=\left[\mathbf{N}_{2}\mathbf{N}'_{2},\mathbf{N}_{2}\mathbf{N}'_{1}\right]$
is still bounded by $\xi$. The multiplication involving $\mathbf{N}_{1}$
and the second block of $\mathbf{A}_{2}$ is done in the same way
as the multiplication $\mathbf{N}_{2}\left[\mathbf{N}'_{2},\mathbf{N}'_{1}\right]$,
hence the multiplication $\mathbf{A}_{1}\mathbf{A}_{2}$ cost $O^{\sim}\left(n^{\omega-1}\xi\right)$,
with the sum of $\vec{s}$-column degrees of each of the four column
blocks in $\mathbf{A}_{1}\mathbf{A}_{2}=\left[\mathbf{N}_{2}\mathbf{N}'_{2},\mathbf{N}_{2}\mathbf{N}'_{1},\mathbf{N}_{1}\mathbf{M}'_{2},\mathbf{N}_{1}\mathbf{M}'_{1}\right]$
bounded by $\xi$.

Next, we multiply $\mathbf{A}_{1}\mathbf{A}_{2}$ with $\mathbf{A}_{3}$.
The matrix $\mathbf{A}_{3}$ now has four blocks on the diagonal.
Consider $\mathbf{N}_{2}\mathbf{N}'_{2}$ , the first column block
of $\mathbf{A}_{1}\mathbf{A}_{2}$, multiplied with the first block
$\left[\mathbf{N}"_{2},\mathbf{N}"_{1}\right]$ on the diagonal of
$\mathbf{A}_{3}$. Let $\vec{s}"$ be the $\vec{s}'$-column degrees
of $\mathbf{N}'_{2}$, which bound the $\vec{s}$-column degrees of
$\mathbf{N}_{2}\mathbf{N}'_{2}$. Then $\sum\vec{s}"\le\sum\vec{s}'\le\sum\vec{s}=\xi$.
Following the same reasoning as before, the sum of the $\vec{s}"$-column
degrees of $\mathbf{N}"_{2}$ is still bounded by $\xi$. We can therefore
again use \prettyref{thm:multiplyUnbalancedMatrices} to multiply
$\mathbf{N}_{2}\mathbf{N}'_{2}$ and $\mathbf{N}"_{2}$. The multiplication
of the remaining blocks are done in the same way. The product $\mathbf{A}_{1}\mathbf{A}_{2}\mathbf{A}_{3}$
now has 8 column blocks, with the sum of the $\vec{s}$-column degrees
of each column block bounded by $\xi$.

Repeating this process, we multiply $\mathbf{A}_{1}\cdots\mathbf{A}_{i}$
with $\mathbf{A}_{i+1}$ at step $i$ for $i$ from 1 to $\left\lfloor \log n\right\rfloor $.
Each of the $2^{i}$ column blocks of $\mathbf{A}_{1}\cdots\mathbf{A}_{i}$
has dimension $n\times O(n/2^{i})$. Each of the $O(2^{i})$ column
blocks on the diagonal of $\mathbf{A}_{i+1}$ has dimension $O(n/2^{i})\times O(n/2^{i})$.
(Big $O$ notation is used here because $n/2^{i}$ may not be an integer.)
Let $\vec{u}_{j}$ be the shift used to compute the $j$th in $\mathbf{A}_{j+1}$,
then as before, the $\vec{s}$-column degrees of the $j$th column
block in $\mathbf{A}_{1}\cdots\mathbf{A}_{i}$ are bounded by $\vec{u}_{j}$,
with $\sum\vec{u}_{j}\le\xi$. The sum of the $\vec{u}$-column degrees
of the $j$th block in $\mathbf{A}_{j+1}$ is bounded by $2\xi$.
(Each of the left half and the right half has the sum bounded by $\xi$.)
Therefore, multiplying $\mathbf{A}_{1}\cdots\mathbf{A}_{i}$ with
$\mathbf{A}_{i+1}$ cost $O^{\sim}\left(2^{j}2^{j}\left(n/2^{j}\right)^{\omega-1}\xi\right)$.
Take $\omega=3,$ we get $O^{\sim}\left(n^{2}\xi\right)$ as desired.
\end{proof}
Again, it is interesting to note that \prettyref{alg:matrixInverse}
costs only $O^{\sim}\left(n^{\omega-1}\xi\right)$ to compute the
inverse and it represents the inverse with $O\left(n\xi\log n\right)$
space. It is possible that this representation is useful in some applications.
For example, if we wish to multiply another low degree matrix or a
row vector $\mathbf{H}$ by $\mathbf{F}^{-1}$, representing $\mathbf{F}^{-1}=\mathbf{A}\mathbf{B}^{-1}$
requires us to multiply $\mathbf{H}$ with a high degree matrix $\mathbf{A}$.
This can be more expensive than the multiplication using the representation
$\mathbf{F}^{-1}=\mathbf{A}_{1}\mathbf{A}_{2}\cdots\mathbf{A}_{\left\lceil \log n\right\rceil }\mathbf{B}^{-1}$,
then $\mathbf{H}\mathbf{F}^{-1}=\mathbf{H}\mathbf{A}_{1}\mathbf{A}_{2}\cdots\mathbf{A}_{\left\lceil \log n\right\rceil }\mathbf{B}^{-1}$,
which is less expensive. It may be interesting to look for other applications
where this smaller representation is useful.



\chapter{\label{chap:Matrix-GCD}Column Basis}

In this Chapter, we consider the problem of computing a column basis
of an input matrix $\mathbf{F}\in\mathbb{K}\left[x\right]^{m\times n}$
with $n\ge m$ and column degrees bounded by a shift $\vec{s}$. An
efficient way of computing a column basis immediately leads to efficient
computations of matrix GCDs, column reduced forms and Popov forms
of $\mathbf{F}$ with any dimension and rank. Note that the existing
algorithms from \citet{Giorgi2003}, \citet{GSSV2012} and \citet{SS2011}
for computing column reduced forms and Popov forms require that the
input matrices be square nonsingular or full-column rank. As in the
previous chapters, $\mathbf{F}$ will always be our input matrix in
this chapter, and $\vec{s}$ will be a shift with entries that bound
the column degrees of $\mathbf{F}$.

Column basis is a generalization of polynomial GCDs. Recall that a
column basis $\mathbf{T}\in\mathbb{K}\left[x\right]^{m\times r}$
of $\mathbf{F}$ is a full rank matrix that generates the same $\mathbb{F}\left[x\right]$-module
as generated by the columns of $\mathbf{F}$. From \prettyref{sec:minimality},
we know any matrix $\mathbf{F}\in\mathbb{K}\left[x\right]^{m\times n}$
can be unimodularly transformed to its column basis, as it can be
unimodularly transformed to a column reduced form that contains a
column basis.
\begin{lem}
\label{lem:unimodularlyReduceToColumnBasis}Any matrix $\mathbf{F}\in\mathbb{K}\left[x\right]^{m\times n}$
can be unimodularly transformed to $\left[0,\mathbf{T}\right]\in\mathbb{K}\left[x\right]^{m\times n}$
with a full rank matrix $\mathbf{T}\in\mathbb{K}\left[x\right]^{m\times r}$,
that is, $\mathbf{F}\mathbf{U}=\left[0,\mathbf{T}\right]$ for some
unimodular matrix $\mathbf{U}$. Any such matrix $\mathbf{T}$ is
a column basis of $\mathbf{F}$.\end{lem}
\begin{proof}
This follows from \prettyref{sec:minimality}. We can just repeatedly
apply \prettyref{lem:columnOperationToReduceDegree}. From \prettyref{lem:columnReducedLeadingCoefficient},
this eventually gives a column reduced form that contains a column
basis. The matrix $\mathbf{T}$ is a column basis of $\mathbf{F}$
since its columns are linearly independent and the $m\times n$ matrix
$\left[0,\mathbf{T}\right]$ is unimodularly equivalent with $\mathbf{F}$,
implying $\mathbf{T}$ and $\mathbf{F}$ each has columns that generate
the same $\mathbb{F}\left[x\right]$-module.
\end{proof}


Before discussing the computation of column basis, it is useful to
look at following relationship between column basis, kernel basis,
and unimodular matrix.
\begin{lem}
\label{lem:unimodular_kernel_columnBasis}Given $\mathbf{F}\in\mathbb{K}\left[x\right]^{m\times n}$.
If $\mathbf{U}$ is a unimodular matrix such that $\mathbf{F}\mathbf{U}=\left[0,\mathbf{T}\right]$
gives a full column rank $\mathbf{T}$, then $\mathbf{U}$ can be
separated into two submatrices $\mathbf{U}=\left[\mathbf{U}_{L},\mathbf{U}_{R}\right]$,
where $\mathbf{U}_{L}$ is a kernel basis of $\mathbf{F}$ and $\mathbf{F}\mathbf{U}_{R}=\mathbf{T}$
is a column basis of $\mathbf{F}$. In addition, the kernel basis
$\mathbf{U}_{L}$ can be replaced by any other kernel basis $\mathbf{N}$
of $\mathbf{F}$ and still gives a unimodular matrix $\left[\mathbf{N},\mathbf{U}_{R}\right]$,
which can also be used to unimodularly transform $\mathbf{F}$ to
$\left[0,\mathbf{T}\right]$. \end{lem}
\begin{proof}
Note that $\mathbf{T}$ is a column basis of $\mathbf{F}$ by \prettyref{lem:unimodularlyReduceToColumnBasis}.
It remains to show that $\mathbf{U}_{L}$ is a kernel basis of $\mathbf{F}$.
Since $\mathbf{F}\mathbf{U}_{L}=0$, $\mathbf{U}_{L}$ is generated
by any kernel basis $\mathbf{N}$, that is, $\mathbf{U}_{L}=\mathbf{N}\mathbf{C}$
for some polynomial matrix $\mathbf{C}$. Let $r$ be the rank of
$\mathbf{F}$, which is also the column dimension of $\mathbf{T}$
and $\mathbf{U}_{R}$. Then both $\mathbf{N}$ and $\mathbf{U}_{L}$
have column dimension $n-r$. Hence $\mathbf{C}$ is a square $(n-r)\times(n-r)$
matrix. Now the unimodular matrix $\mathbf{U}$ can be factored as
\[
\mathbf{U}=\left[\mathbf{N}\mathbf{C},\mathbf{U}_{R}\right]=\left[\mathbf{N},\mathbf{U}_{R}\right]\begin{bmatrix}\mathbf{C}\\
 & I
\end{bmatrix},
\]
 implying that both factors $\left[\mathbf{N},\mathbf{U}_{R}\right]$
and $\begin{bmatrix}\mathbf{C}\\
 & I
\end{bmatrix}$ is unimodular. Therefore, $\mathbf{C}$ is unimodular and $\mathbf{U}_{L}=\mathbf{N}\mathbf{C}$
is also a kernel basis. Notice that the unimodular matrix $\left[\mathbf{N},\mathbf{U}_{R}\right]$
also transforms $\mathbf{F}$ to $\left[0,\mathbf{T}\right]$.


\end{proof}
\prettyref{lem:unimodular_kernel_columnBasis} gives the following
result for a unimodular matrix and its inverse.
\begin{cor}
\label{cor:unimodular_kernel_columnBasis2}Let $\mathbf{U}=\left[\mathbf{U}_{L},\mathbf{U}_{R}\right]$
be any unimodular matrix with columns separated arbitrarily to $\mathbf{U}_{L}$
and $\mathbf{U}_{R}$. Let its inverse $\mathbf{V}=\begin{bmatrix}\mathbf{V}_{U}\\
\mathbf{V}_{D}
\end{bmatrix}$, where the row dimension of $\mathbf{V}_{U}$ matches the column
dimension of $\mathbf{U}_{L}$. So we have 
\[
\mathbf{V}\mathbf{U}=\begin{bmatrix}\mathbf{V}_{U}\\
\mathbf{V}_{D}
\end{bmatrix}\begin{bmatrix}\mathbf{U}_{L},\mathbf{U}_{R}\end{bmatrix}=\begin{bmatrix}\mathbf{V}_{U}\mathbf{U}_{L} & \mathbf{V}_{U}\mathbf{U}_{R}\\
\mathbf{V}_{D}\mathbf{U}_{L} & \mathbf{V}_{D}\mathbf{U}_{R}
\end{bmatrix}=\begin{bmatrix}I & 0\\
0 & I
\end{bmatrix}.
\]
Then $\mathbf{V}_{U}\mathbf{U}_{L}=I$ is a column basis of $\mathbf{V}_{U}$
and a row basis of $\mathbf{U}_{L}$, while $\mathbf{V}_{D}\mathbf{U}_{R}=I$
is a column basis of $\mathbf{V}_{D}$ and a row basis of of $\mathbf{U}_{R}$.
In addition, \textup{$\mathbf{V}_{D}$ and $\mathbf{U}_{L}$ are kernel
bases of each other, while $\mathbf{V}_{U}$ and $\mathbf{U}_{R}$
are kernel bases of each other.}\end{cor}
\begin{proof}
This follows directly from \prettyref{lem:unimodular_kernel_columnBasis},
by taking $\mathbf{F}$ from \prettyref{lem:unimodular_kernel_columnBasis}
to be $\mathbf{V}_{U}$, $\mathbf{V}_{D}$, $\mathbf{U}_{L}^{T}$,
and $\mathbf{U}_{R}^{T}$ here.
\end{proof}
To compute a column basis of $\mathbf{F}$, we use the following procedure.
We first compute a right  kernel basis $\mathbf{N}$ of $\mathbf{F}$.
Then we compute a left  kernel basis $\mathbf{G}$ of $\mathbf{N}$.
This matrix $\mathbf{G}$ is a right factor of $\mathbf{F}$, that
is, $\mathbf{F}=\mathbf{T}\mathbf{G}$ for some $\mathbf{T}\in\mathbb{K}\left[x\right]^{m\times r}$.
Then we can compute the left factor $\mathbf{T}$, which is in fact
a column basis of $\mathbf{F}$.
\begin{lem}
\label{lem:matrixGCD}Given $\mathbf{F}\in\mathbb{K}\left[x\right]^{m\times n}$%
\begin{comment}
 and has full row rank
\end{comment}
. Let $\mathbf{N}\in\mathbb{K}\left[x\right]^{n\times(n-r)}$ be any
right kernel basis of $\mathbf{F}$, and $\mathbf{G}\in\mathbb{K}\left[x\right]^{r\times n}$
be any left kernel basis of $\mathbf{N}$, where $r$ is the rank
of $\mathbf{F}$. Then $\mathbf{F}=\mathbf{T}\mathbf{G}$ for some
$\mathbf{T}\in\mathbb{K}\left[x\right]^{m\times r}$ and $\mathbf{T}$
is a column basis of $\mathbf{F}$.\end{lem}
\begin{proof}
Let the matrix $\mathbf{U}=\begin{bmatrix}\mathbf{U}_{L},\mathbf{U}_{R}\end{bmatrix}$
from \prettyref{cor:unimodular_kernel_columnBasis2} be a unimodular
matrix that transforms $\mathbf{F}$ to a column basis $\mathbf{B}\in\mathbb{K}\left[x\right]^{m\times r}$
of $\mathbf{F}$, where $\mathbf{U}_{L}$ is any right kernel basis
$\mathbf{F}$. From $\mathbf{F}\mathbf{U}=\left[0,\mathbf{B}\right]$,
we get\textbf{ $\mathbf{F}=\left[0,\mathbf{B}\right]\mathbf{U}^{-1}=\mathbf{B}\left[0,I\right]\mathbf{V}=\mathbf{B}\mathbf{V}_{D}$}.
Since $\mathbf{V}_{D}$ is a left kernel basis of\textbf{ $\mathbf{U}_{L}$},
any other left kernel basis $\mathbf{G}$ of $\mathbf{U}_{L}$ is
unimodularly equivalent to $\mathbf{V}_{D}$, that is, $\mathbf{V}_{D}=\mathbf{W}\mathbf{G}$
for some unimodular matrix $\mathbf{W}$. Now $\mathbf{F}=\mathbf{B}\mathbf{W}\mathbf{G}$,
where $\mathbf{BW}=\mathbf{T}$ a column basis of $\mathbf{F}$ since
it is unimodularly equivalent to the column basis $\mathbf{B}$.
\end{proof}


\prettyref{lem:matrixGCD} outlines a procedure for computing a column
basis of $\mathbf{F}$ with three main steps. The first step is to
compute a $\left(\mathbf{F},\vec{s}\right)$-kernel basis $\mathbf{N}$,
which can be efficiently done using \prettyref{alg:minimalNullspaceBasisWithRankProfile}.
However, we still need to work on the second step of computing a $\left(\mathbf{N}^{T},-\vec{s}\right)$-kernel
basis $\mathbf{G}^{T}$ and the third step of computing the column
basis $\mathbf{T}$ from $\mathbf{F}$ and $\mathbf{G}$. Note that
while \prettyref{lem:matrixGCD} does not require the bases computed
to be minimal, working with minimal bases keeps the degrees well-managed
and helps to make the computation efficient.


\section{\label{sec:computeRightFactor}Computing a Right Factor}

Let us now look at the computation of a $\left(\mathbf{N}^{T},-\vec{s}\right)$-kernel
basis $\mathbf{G}^{T}$. For this problem, \prettyref{alg:minimalNullspaceBasisWithRankProfile}
does not work well directly, since the input matrix $\mathbf{N}^{T}$
has nonuniform row degrees and negative shift. Comparing to the earlier
problem of computing a $\left(\mathbf{F},\vec{s}\right)$-kernel basis
$\mathbf{N}$, it is interesting to note that the old output $\mathbf{N}$
now becomes the new input matrix $\mathbf{N}^{T}$, while the new
output matrix $\mathbf{G}$ has size bounded by $\mathbf{F}$. In
other words, the new input has degrees that matches the old output,
while the new output has degrees bounded by the old input. It is
therefore reasonable to expect that the new problem can be computed
efficiently. However, we need to find some way to work with the more
complicated input degree structure. On the other hand, the simpler
output degree structure makes it easier to apply order basis computation
to compute a $\left(\mathbf{N}^{T},-\vec{s}\right)$-kernel basis. 

To see how order basis computations can be applied here, let us first
extend \prettyref{lem:orderBasisContainsNullspaceBasis}, which provides
a relationship between order bases and kernel bases, to accommodate
our situation here.
\begin{lem}
\label{lem:orderbasisContainsNullspacebasisGeneralized}Given a matrix
$\mathbf{A}\in\mathbb{K}\left[x\right]^{m\times n}$ and a degree
shift $\vec{u}$ with $\rdeg_{\vec{u}}\mathbf{A}\le\vec{v}$, or equivalently,
$\cdeg_{-\vec{v}}\mathbf{A}\le-\vec{u}$. Let $\mathbf{P}=\left[\mathbf{P}_{1},\mathbf{P}_{2}\right]$
be any $\left(\mathbf{A},\vec{v}+1,-\vec{u}\right)$-basis and $\mathbf{Q}=\left[\mathbf{Q}_{1},\mathbf{Q}_{2}\right]$
be any $(\mathbf{A},-\vec{u})$-kernel basis, where $\mathbf{P}_{1}$
and $\mathbf{Q}_{1}$ contain all columns from $\mathbf{P}$ and $\mathbf{Q}$,
respectively, whose $-\vec{u}$-column degrees are no more than $0$.
Then $\left[\mathbf{P}_{1},\mathbf{Q}_{2}\right]$ is an $(\mathbf{A},-\vec{u})$-kernel
basis, and $\left[\mathbf{Q}_{1},\mathbf{P}_{2}\right]$ is an $\left(\mathbf{A},\vec{v}+\left[1,\dots,1\right],-\vec{u}\right)$-basis.\end{lem}
\begin{proof}
We know $\cdeg_{-\vec{v}}\mathbf{A}\mathbf{P}_{1}\le\cdeg_{-\vec{u}}\mathbf{P}_{1}\le0$,
or equivalently, $\rdeg\mathbf{A}\mathbf{P}_{1}\le\vec{v}$, but it
has order greater than $\vec{v}$, hence $\mathbf{A}\mathbf{P}_{1}=0$.
The result then follows the same reasoning as in the proof of \prettyref{lem:orderBasisContainsNullspaceBasis}.%
\begin{comment}
We know $\cdeg\mathbf{P}_{1}^{T}\le\vec{u}$ from $\cdeg_{-\vec{u}}\mathbf{P}_{1}\le0$,
hence $\cdeg\mathbf{P}_{1}^{T}\mathbf{A}^{T}\le\cdeg_{\vec{u}}\mathbf{A}^{T}$
by \prettyref{lem:boundOnDegreesOfFA}. Now for each row $\mathbf{a}_{i}$
in $\mathbf{A}$ and its $\vec{u}$-column degree $v_{i}$, we have
\[
\rdeg\mathbf{a}_{i}\mathbf{P}_{1}=\cdeg\mathbf{P}_{1}^{T}\mathbf{a}_{i}^{T}\le\cdeg_{\vec{u}}\mathbf{a}_{i}^{T}=v_{i},
\]
 and $\mathbf{a}_{i}\mathbf{P}_{1}\equiv0\mod x^{v_{i}+1}$, hence
$\mathbf{A}\mathbf{P}_{1}=0$. The result then follows the same reasoning
as in the proof of \prettyref{lem:orderBasisContainsNullspaceBasis}.
\end{comment}

\end{proof}
Now with the help of \prettyref{lem:orderbasisContainsNullspacebasisGeneralized},
let us get back to our problem of computing a $(\mathbf{F},\vec{s})$-kernel
basis.  In fact, we just need to use a special case of \prettyref{lem:orderbasisContainsNullspacebasisGeneralized},
where all the elements of the kernel basis have shifted degrees bounded
by $0$, making the partial kernel basis a complete kernel basis%
\begin{comment}
, which follows from our requirement of using a shift $\vec{s}\ge\cdeg\mathbf{F}$
\end{comment}
.
\begin{lem}
\label{lem:nullspaceBasisInOrderBasis}Let $\mathbf{N}$ be a $(\mathbf{F},\vec{s})$-kernel
basis with $\cdeg_{\vec{s}}\mathbf{N}=\vec{b}$. Let $\mathbf{P}=\left[\mathbf{P}_{1},\mathbf{P}_{2}\right]$
be a $\left(\mathbf{N}^{T},\vec{b}+1,-\vec{s}\right)$-basis, where
$\mathbf{P}_{1}$ consists of all columns $\mathbf{p}$ with $\cdeg_{-\vec{s}}\mathbf{p}\le0$.
\begin{comment}
of $\mathbf{P}$ satisfying $\mathbf{N}^{T}\mathbf{p}=0$. 
\end{comment}
Then $\mathbf{P}_{1}$ is a $(\mathbf{N}^{T},-\vec{s})$-kernel basis. \end{lem}
\begin{proof}
Let the rank of $\mathbf{F}$ be $r$, which is also the column dimension
of any $(\mathbf{N}^{T},-\vec{s})$-kernel basis $\mathbf{G}^{T}$.
Since both $\mathbf{F}$ and $\mathbf{G}$ are in the left kernel
of $\mathbf{N}$, we know $\mathbf{F}$ is generated by $\mathbf{G}$,
and the $-\vec{s}$-row degrees of $\mathbf{G}$ are bounded by the
corresponding $r$ largest $-\vec{s}$-row degrees of $\mathbf{F}$,
which are in turn bounded by $0$ since $\cdeg\mathbf{F}\le\vec{s}$.
Therefore, any $(\mathbf{N}^{T},-\vec{s})$-kernel basis $\mathbf{G}^{T}$
satisfies $\cdeg_{-\vec{s}}\mathbf{G}^{T}\le0$. The result now follows
from \prettyref{lem:orderbasisContainsNullspacebasisGeneralized}.
\end{proof}
We can use \prettyref{thm:continueComputingNullspaceBasisByRows}
to compute a $\left(\mathbf{N}^{T},-\vec{s}\right)$-kernel basis
by rows. If we separate $\mathbf{N}$ to $\mathbf{N}=\left[\mathbf{N}_{1},\mathbf{N}_{2}\right]$
with $\vec{s}$-column degrees $\vec{b}_{1}$, $\vec{b}_{2}$ respectively,
and first compute a $\left(\mathbf{N}_{1}^{T},-\vec{s}\right)$-kernel
basis $\mathbf{Q}_{1}$ with $-\vec{s}$-column degrees $-\vec{s}_{2}$,
and then compute a $\left(\mathbf{N}_{2}^{T}\mathbf{Q}_{1},-\vec{s}_{2}\right)$-kernel
basis $\mathbf{Q}_{2}$, then $\mathbf{Q}_{1}\mathbf{Q}_{2}$ is a
$\left(\mathbf{N}^{T},-\vec{s}\right)$-kernel basis. To compute kernel
bases $\mathbf{Q}_{1}$ and $\mathbf{Q}_{2}$, we can use order basis
computation. However, we need to make sure that the order bases we
compute do contain these kernel bases.
\begin{lem}
\label{lem:nullspaceBasisOfSubsetOfRowsContainedInOrderBasis}Let
$\mathbf{N}=\left[\mathbf{N}_{1},\mathbf{N}_{2}\right]$, with $\vec{s}$-column
degrees $\vec{b}_{1}$, $\vec{b}_{2}$ respectively. Then a $\left(\mathbf{N}_{1}^{T},\vec{b}_{1}+1,-\vec{s}\right)$-basis
contains a $\left(\mathbf{N}_{1}^{T},-\vec{s}\right)$-kernel basis
whose $-\vec{s}$-row degrees are bounded by 0. Let $\mathbf{Q}_{1}$
be this kernel basis, and $-\vec{s}_{2}=\cdeg_{-\vec{s}}\mathbf{Q}_{1}$.
Then a $\left(\mathbf{N}_{2}^{T}\mathbf{Q}_{1},\vec{b}_{2}+1,-\vec{s}_{2}\right)$-basis
contains a $\left(\mathbf{N}_{2}^{T}\mathbf{Q}_{1},-\vec{s}_{2}\right)$-kernel
basis $\mathbf{Q}_{2}$ whose $-\vec{s}$-row degrees are bounded
by 0. The product $\mathbf{Q}_{1}\mathbf{Q}_{2}$ is then a $\left(\mathbf{N}^{T},-\vec{s}\right)$-kernel
basis.\end{lem}
\begin{proof}
To see that a $\left(\mathbf{N}_{1}^{T},\vec{b}_{1}+1,-\vec{s}\right)$-basis
contains a $\left(\mathbf{N}_{1}^{T},-\vec{s}\right)$-kernel basis
whose $-\vec{s}$-row degrees are bounded by 0, we just need to show
that $\rdeg_{-\vec{s}}\mathbf{\bar{Q}}_{1}\le0$ for any $\left(\mathbf{N}_{1}^{T},-\vec{s}\right)$-kernel
basis $\mathbf{\bar{Q}}_{1}$ and then apply \prettyref{lem:orderbasisContainsNullspacebasisGeneralized}.
Note that there exists a polynomial matrix $\bar{\mathbf{Q}}_{2}$
such that $\mathbf{\bar{Q}}_{1}\mathbf{\bar{Q}}_{2}=\bar{\mathbf{G}}$
for any $\left(\mathbf{N}^{T},-\vec{s}\right)$-kernel basis $\bar{\mathbf{G}}$,
as $\bar{\mathbf{G}}$ satisfies $\mathbf{N}_{1}^{T}\bar{\mathbf{G}}=0$
and is therefore generated by the $\left(\mathbf{N}_{1}^{T},-\vec{s}\right)$-kernel
basis $\bar{\mathbf{Q}}_{1}$. If $\rdeg_{-\vec{s}}\mathbf{\bar{Q}}_{1}\nleq0$,
then \prettyref{lem:predictableDegree} forces $\rdeg_{-\vec{s}}\left(\bar{\mathbf{Q}}_{1}\bar{\mathbf{Q}}_{2}\right)=\rdeg_{-\vec{s}}\bar{\mathbf{G}}\nleq0$,
a contradiction. 

As before, to see that a $\left(\mathbf{N}_{2}^{T}\mathbf{Q}_{1},\vec{b}_{2}+1,-\vec{s}_{2}\right)$-basis
contains a $\left(\mathbf{N}_{2}^{T}\mathbf{Q}_{1},-\vec{s}_{2}\right)$-kernel
basis whose $-\vec{s}$-row degrees are no more than 0, we can just
show $\rdeg_{-\vec{s}_{2}}\hat{\mathbf{Q}}_{2}\le0$ for any $\left(\mathbf{N}_{2}^{T}\mathbf{Q}_{1},-\vec{s}_{2}\right)$-kernel
basis $\hat{\mathbf{Q}}_{2}$ and then apply \prettyref{lem:orderbasisContainsNullspacebasisGeneralized}.
Note that $\rdeg_{-\vec{b}_{2}}\mathbf{N}_{2}^{T}\mathbf{Q}_{1}\le-\vec{s}_{2}$.
Let $\hat{\mathbf{G}}=\mathbf{Q}_{1}\hat{\mathbf{Q}}_{2}$, a $\left(\mathbf{N}^{T},-\vec{s}\right)$-kernel
basis. Note that $\rdeg_{-\vec{s}_{2}}\hat{\mathbf{Q}}_{2}=\rdeg_{-\vec{s}}\mathbf{Q}_{1}\hat{\mathbf{Q}}_{2}=\rdeg_{-\vec{s}}\hat{\mathbf{G}}\le0$. 
\end{proof}
Now that we can correctly compute a $\left(\mathbf{N}^{T},-\vec{s}\right)$-kernel
basis by rows with the help of order basis computation, we need to
look at how to do it efficiently. One major difficulty is that the
order, or equivalently, the $\vec{s}$-row degrees are nonuniform
and can have degree as large as $\xi=\sum\vec{s}$. To overcome this,
we separate the rows of $\mathbf{N}^{T}$ into blocks according to
their $\vec{s}$-row degrees, and then work with these blocks one
by one consecutively using \prettyref{thm:continueComputingNullspaceBasisByRows}. 

\begin{comment}
Note that rank sensitive computation for order basis is not as natural,
since for order basis we work with power series in general, whose
rank may never be truly computed. In addition, the computed basis
does not correspond to the rank. For these reasons, we do not pursue
rank sensitive computations for order basis. 
\end{comment}
\begin{algorithm}[t]
\caption{$\mnbr(\mathbf{M},\vec{s})$}
\label{alg:minimalNullspaceBasisReverse}

\begin{algorithmic}
[1]\REQUIRE{$\mathbf{M}\in\mathbb{K}\left[x\right]^{m\times n}$ such that $\sum\rdeg_{\vec{s}}\mathbf{M}\le\sum\vec{s}$
and a $\left(\mathbf{M},-\vec{s}\right)$-nullspace basis has row
degrees bounded by $\vec{s}$ (equivalently, $-\vec{s}$-column degrees
bounded by 0).}

\ENSURE{$\mathbf{G}\in\mathbb{K}\left[x\right]^{n\times k}$, a $\left(\mathbf{M},-\vec{s}\right)$-nullspace
basis.}

\STATE{\textbf{$\left[\mathbf{M}_{1}^{T},\mathbf{M}_{2}^{T},\cdots,\mathbf{M}_{\log k-1}^{T},\mathbf{M}_{\log k}^{T}\right]:=\mathbf{M}^{T}$},
with $\mathbf{M}_{\log k},\mathbf{M}_{\log k-1},\cdots,\mathbf{M}_{2},\mathbf{M}_{1}$
having $\vec{s}$-row degrees in the range $\left[0,2\xi/k\right],(2\xi/k,4\xi/k],...,(\xi/4,\xi/2],(\xi/2,\xi].$\textbf{ }}

\FOR{$i$ \textbf{from $1$ to $\log k$ }} 

\forbody{\STATE{$\vec{\sigma}_{i}:=\left[\xi/2^{i-1}+1,\dots,\xi/2^{i-1}+1\right]$,
with the number of entries matches the row dimension of $\mathbf{M}_{i};$}}

\STATE{$\vec{\sigma}:=\left[\vec{\sigma}_{\log k},\vec{\sigma}_{\log k-1},\dots,\vec{\sigma}_{1}\right]$;}

\STATE{$\hat{\mathbf{M}}:=x^{\vec{\sigma}-\vec{b}-1}\mathbf{M};$}

\STATE{$\mathbf{N}_{0}:=I_{n}$; $\tilde{\mathbf{N}}_{0}:=I_{n};$}

\FOR{$i$ \textbf{from $1$ to $\log k$ }} 

\forbody{\STATE{$\vec{s}_{i}:=-\cdeg_{-\vec{s}}\mathbf{N}_{i-1};$ (note $\vec{s}_{1}=\vec{s}$)}

\STATE{$\mathbf{P}_{i}:=\umab\left(\hat{\mathbf{M}}_{i}\tilde{\mathbf{N}}_{i-1},\vec{\sigma}_{i},-\vec{s}_{i}\right)$; }

\STATE{$\left[\mathbf{N}_{i},\mathbf{Q}_{i}\right]:=\mathbf{P}_{i}$, where
$\mathbf{N}_{i}$ is a $\left(\hat{\mathbf{M}}_{i},-\vec{s}_{i}\right)$-nullspace
basis;}

\STATE{$\tilde{\mathbf{N}_{i}}:=\tilde{\mathbf{N}}_{i-1}\cdot\mathbf{N}_{i};$ }

}



\RETURN $\tilde{\mathbf{N}_{i}}$ 
\end{algorithmic}
\end{algorithm}



Let $k$ be the column dimension of $\mathbf{N}$. Since $\sum\cdeg_{\vec{s}}\mathbf{N}=\sum\vec{b}\le\sum\vec{s}\le\xi$,
at most $k/c$ columns of $\mathbf{N}$ have $\vec{s}$-column degrees
greater than or equal to $c\xi/k$ for any $c\ge1$. We assume without
loss of generality that the rows of $\mathbf{N}^{T}$ are arranged
in decreasing $\vec{s}$-row degrees. We divide $\mathbf{N}^{T}$
into $\log k$ row blocks according to the $\vec{s}$-row degrees
of its rows, or equivalently, divide $\mathbf{N}$ to blocks of columns
according to the $\vec{s}$-column degrees. Let 
\[
\mathbf{N}=\left[\mathbf{N}_{1},\mathbf{N}_{2},\cdots,\mathbf{N}_{\log k-1},\mathbf{N}_{\log k}\right]
\]
with $\mathbf{N}_{1},\mathbf{N}_{2},\cdots,\mathbf{N}_{\log k-1},\mathbf{N}_{\log k}$
having $\vec{s}$-column degrees in the range $\left[0,2\xi/k\right]$,
$(2\xi/k,4\xi/k],$ $(4\xi/k,8\xi/k],\ ...,$ $(\xi/4,\xi/2],$ $(\xi/2,\xi].$
Let $\vec{\sigma}_{i}=\left[\xi/2^{i-1}+1,\dots,\xi/2^{i-1}+1\right]$
with the same dimension as the row dimension of $\mathbf{N}_{i}$.
Let $\vec{\sigma}=\left[\vec{\sigma}_{\log k},\vec{\sigma}_{\log k-1},\dots,\vec{\sigma}_{1}\right]$
be the order in the order basis computation.

To further simply our task, we also make the order of our problem
in each block uniform. Rather than of using $\mathbf{N}^{T}$ as the
input matrix, we use 
\begin{eqnarray*}
\hat{\mathbf{N}} & =\begin{bmatrix}\hat{\mathbf{N}}_{1}\\
\vdots\\
\hat{\mathbf{N}}_{\log k}
\end{bmatrix}= & x^{\vec{\sigma}-\vec{b}-1}\begin{bmatrix}\mathbf{N}_{1}^{T}\\
\vdots\\
\mathbf{N}_{\log k}^{T}
\end{bmatrix}=x^{\vec{\sigma}-\vec{b}-1}\mathbf{N}^{T}
\end{eqnarray*}
 instead, so that a $\left(\hat{\mathbf{N}},\vec{\sigma},-\vec{s}\right)$-basis
is a $\left(\mathbf{N}^{T},\vec{b}+1,-\vec{s}\right)$-basis.

We are now ready to compute a $\left(\mathbf{N}^{T},-\vec{s}\right)$-kernel
basis, which is done by a series of order basis computations that
computes a series of kernel bases as follows.

Let $\vec{s}_{1}=\vec{s}$. First we compute an $\left(\hat{\mathbf{N}}_{1},\vec{\sigma}_{1},-\vec{s}_{1}\right)$-basis
$\mathbf{P}_{1}=\left[\mathbf{G}_{1},\mathbf{Q}_{1}\right]$, where
$\mathbf{G}_{1}$ is a $\left(\hat{\mathbf{N}}_{1},-\vec{s}_{1}\right)$-kernel
basis%
\begin{comment}
 with $\cdeg_{-\vec{s}_{1}}\mathbf{N}_{1}\le0$
\end{comment}
. This computation can be done using \prettyref{alg:umab} with a
cost of $O^{\sim}\left(n^{\omega}d\right)$, where $d=\xi/n$. 

Let $\tilde{\mathbf{G}}_{1}=\mathbf{G}_{1}$. Let $\vec{s}_{2}=-\cdeg_{-\vec{s}}\mathbf{G}_{1}$.
\begin{comment}
Note that $-\vec{s}_{1}\le-[\vec{s}_{2},\vec{t}_{2}]\le\left[0,\dots,0,1,\dots1\right]$
component-wise, since $\mathbf{P}_{1}$ has lower order than any $\left(\mathbf{M}^{T},\vec{b}+\left[1,\dots,1\right],-\vec{s}\right)$-basis
$\mathbf{P}$ hence generates $\mathbf{P}$. Therefore, $\cdeg_{-\vec{s}}\mathbf{P}_{1}\le\cdeg_{-\vec{s}}\mathbf{P}\le\left[0,\dots,0,1,\dots1\right]$. 
\end{comment}
{} We then compute an $\left(\hat{\mathbf{N}}_{2}\tilde{\mathbf{G}}_{1},\vec{\sigma}_{2},-\vec{s}_{2}\right)$-basis
$\mathbf{P}_{2}=\left[\mathbf{G}_{2},\mathbf{Q}_{2}\right]$ with
$\vec{s}_{3}=-\cdeg_{-\vec{s}_{2}}\mathbf{G}_{2}$. Let $\tilde{\mathbf{G}}_{2}=\tilde{\mathbf{G}}_{1}\mathbf{G}_{2}$.
 %
\begin{comment}
Let $\mathbf{R}_{1}=\left[\mathbf{N}_{1}\mathbf{Q}_{2},\mathbf{Q}_{1}\right]$
and $\mathbf{R}_{1}^{r}=\revCol\left(\mathbf{R}_{1},-\vec{s},\cdeg_{-\vec{s}}\mathbf{R}_{1}\right)$.
Then from \prettyref{lem:unimodularComputationByRows} we know $\left[\mathbf{F}^{T},\mathbf{R}_{1}^{r}\right]$
is a unimodular matrix.
\end{comment}


Continuing this process, at step $i$ we compute an $\left(\hat{\mathbf{N}}_{i}\tilde{\mathbf{G}}_{i-1},\vec{\sigma}_{i},-\vec{s}_{i}\right)$-basis
$\mathbf{P}_{i}=\left[\mathbf{G}_{i},\mathbf{Q}_{i}\right]$. Let
$\tilde{\mathbf{G}}_{i}=\prod_{j=1}^{i}\mathbf{G}_{i}=\tilde{\mathbf{G}}_{i-1}\mathbf{G}_{i}$.
Note that $\tilde{\mathbf{G}}_{\log k}$ is a $\left(\mathbf{N}^{T},-\vec{s}\right)$-kernel
basis. 

This process of computing a $\left(\mathbf{N}^{T},-\vec{s}\right)$-kernel
basis gives \prettyref{alg:minimalNullspaceBasisReverse}.

We still need to check the cost of the multiplications $\hat{\mathbf{N}}_{i}\tilde{\mathbf{G}}_{i-1}$
and $\tilde{\mathbf{G}}_{i-1}\mathbf{G}_{i}$.
\begin{lem}
The multiplications $\hat{\mathbf{N}}_{i}\tilde{\mathbf{G}}_{i-1}$
can be done with a cost of $O^{\sim}\left(n^{\omega}d\right)$.\end{lem}
\begin{proof}
The dimension of $\hat{\mathbf{N}}_{i}$ is bounded by $2^{i-1}\times n$
and $\sum\rdeg_{\vec{s}}\hat{\mathbf{N}}_{i}\le2^{i-1}\cdot\xi/2^{i-1}=\xi$.
We also have $\cdeg_{-\vec{s}}\tilde{\mathbf{G}}_{i-1}\le0$, or equivalently,
$\rdeg\tilde{\mathbf{G}}_{i-1}\le\vec{s}$. We can now use \prettyref{thm:multiplyUnbalancedMatrices}
to multiply $\tilde{\mathbf{G}}_{i-1}^{T}$ and $\hat{\mathbf{N}}_{i}^{T}$
with a cost of $O^{\sim}\left(n^{\omega-1}\xi\right)=O^{\sim}\left(n^{\omega}d\right)$.\end{proof}
\begin{lem}
The multiplication $\tilde{\mathbf{G}}_{i-1}\mathbf{G}_{i}$ can be
done with a cost of $O^{\sim}\left(n^{\omega}d\right)$.\end{lem}
\begin{proof}
We know $\cdeg_{-\vec{s}}\tilde{\mathbf{G}}_{i-1}=-\vec{s}_{i}$,
and $\cdeg_{-\vec{s}_{i}}\mathbf{G}_{i}=-\vec{s}_{i+1}\le0.$ In other
words, $\rdeg\mathbf{G}_{i}\le\vec{s}_{i}$, and $\rdeg_{\vec{s}_{i}}\tilde{\mathbf{G}}_{i-1}\le\vec{s}$,
hence we can again use \prettyref{thm:multiplyUnbalancedMatrices}
to multiply $\mathbf{G}_{i}^{T}$ and $\tilde{\mathbf{G}}_{i-1}^{T}$
with a cost of $O^{\sim}\left(n^{\omega}d\right)$.\end{proof}
\begin{thm}
A right factor $\mathbf{G}$ satisfying $\mathbf{TG}=\mathbf{F}$
for a column basis $\mathbf{T}$ can be computed with a cost of $O^{\sim}\left(n^{\omega-1}\xi\right)=O^{\sim}\left(n^{\omega}d\right)$.
\end{thm}

\section{Computing a Column Basis}

With a right factor $\mathbf{G}$ of $\mathbf{F}$ computed, we are
now ready to compute a column basis $\mathbf{T}$, where $\mathbf{F}=\mathbf{T}\mathbf{G}$.
To do so efficiently, the degree of $\mathbf{T}$ cannot be too big,
which is indeed the case as shown by the following lemmas.
\begin{lem}
\label{lem:colBasisdegreeBoundByRdegOfRightFactor}The column degrees
of $\mathbf{T}$ are bounded by the corresponding entries of $\vec{t}=-\rdeg_{-\vec{s}}\mathbf{G}$.\end{lem}
\begin{proof}
Since $\mathbf{G}$ is $-\vec{s}$-row reduced, and $\rdeg_{-\vec{s}}\mathbf{F}\le0$,
by \prettyref{lem:predictableDegree} $\rdeg_{-\vec{t}}\mathbf{T}\le0$,
or equivalently, $\mathbf{T}$ has column degrees bounded by $\vec{t}$.\end{proof}
\begin{lem}
\label{lem:colBasisDegreeBoundByInputDegrees}Let $\vec{t}=-\rdeg_{-\vec{s}}\mathbf{G}$,
a vector with $r$ entries and bounds $\cdeg\mathbf{T}$ from \prettyref{lem:colBasisdegreeBoundByRdegOfRightFactor}.
Let $\vec{s}'$ be the list of the $r$ largest entries of $\vec{s}$.
Then $\vec{t}\le\vec{s}'$.\end{lem}
\begin{proof}
Let $\mathbf{G}'$ be the $-\vec{s}$-row Popov form of $\mathbf{G}$,
and the square matrix $\mathbf{G}"$ consist of only the columns of
$\mathbf{G}'$ that contains pivot entries, and has the rows permuted
so the pivots are in the diagonal. Let $\vec{s}"$ be the list of
the entries in $\vec{s}$ that correspond to the columns of $\mathbf{G}"$
in $\mathbf{G}'$. Note that $\rdeg_{-\vec{s}"}\mathbf{G}"=-\vec{t}"$
is just a permutation of $-\vec{t}$ with the same entries. By the
definition of shifted row degree, $-\vec{t}"$ is the sum of $-\vec{s}"$
and the list of the diagonal pivot degrees, which are nonnegative.
Therefore, $-\vec{t}"\ge-\vec{s}"$. The result then follows as $\vec{t}$
is a permutation of $\vec{t}"$ and $\vec{s}'$ has the largest entries
of $\vec{s}$.
\end{proof}
With the bound on the column degrees of $\mathbf{T}$ determined,
we are now ready to compute $\mathbf{T}$. This is done again using
an order basis computation.
\begin{lem}
Let $\vec{t}'=\left[0,\dots,0,\vec{t}\right]\in\mathbb{Z}^{m+r}$.
Any $\left(\left[\mathbf{F}^{T},\mathbf{G}^{T}\right],-\vec{t}'\right)$-kernel
basis has the form $\begin{bmatrix}V\\
\bar{\mathbf{T}}
\end{bmatrix}$, where $V\in\mathbb{K}^{m\times m}$ is a unimodular matrix and $\left(\bar{\mathbf{T}}V^{-1}\right)^{T}$
is a column basis of $\mathbf{F}$.\end{lem}
\begin{proof}
First, the matrix $\begin{bmatrix}I\\
\mathbf{T}^{T}
\end{bmatrix}$ is a kernel basis of $\left[\mathbf{F}^{T},\mathbf{G}^{T}\right]$
and is unimodularly equivalent to any other kernel basis. Hence we
must have $\begin{bmatrix}V\\
\bar{\mathbf{T}}
\end{bmatrix}=\begin{bmatrix}I\\
\mathbf{T}^{T}
\end{bmatrix}V$, which gives $\mathbf{T}=\left(\bar{\mathbf{T}}V^{-1}\right)^{T}$.
Also note that the $-\vec{t}'$-minimality forces the unimodular matrix
$V$ in any $\left(\left[\mathbf{F}^{T},\mathbf{G}^{T}\right],-\vec{t}'\right)$-kernel
basis to be the same degree as $I$.
\end{proof}
To compute a $\left(\left[\mathbf{F}^{T},\mathbf{G}^{T}\right],-\vec{t}'\right)$-kernel
basis, we can again use order basis computation.
\begin{lem}
Any $\left(\left[\mathbf{F}^{T},\mathbf{G}^{T}\right],\vec{s}+\left[1,\dots,1\right],-\vec{t}'\right)$-basis
contain a $\left(\left[\mathbf{F}^{T},\mathbf{G}^{T}\right],-\vec{t}'\right)$-kernel
basis whose $-\vec{t}'$-row degrees are bounded by 0.\end{lem}
\begin{proof}
As before, \prettyref{lem:orderbasisContainsNullspacebasisGeneralized}
can be used here. We just need to show that a $\left(\left[\mathbf{F}^{T},\mathbf{G}^{T}\right],-\vec{t}'\right)$-kernel
basis has $-\vec{t}'$-row degrees no more than 0, which is true since
$\rdeg_{-\vec{t}'}\begin{bmatrix}I\\
\mathbf{T}^{T}
\end{bmatrix}\le0$.
\end{proof}
Now to compute a $\left(\left[\mathbf{F}^{T},\mathbf{G}^{T}\right],-\vec{t}'\right)$-kernel
basis efficiently, notice we have the same type of problem as in \prettyref{sec:computeRightFactor},
and so \prettyref{alg:minimalNullspaceBasisReverse} works here as
well. With a $\left(\left[\mathbf{F}^{T},\mathbf{G}^{T}\right],-\vec{t}'\right)$-kernel
basis $\begin{bmatrix}V\\
\bar{\mathbf{T}}
\end{bmatrix}$ computed, a column basis can now be easily computed by $\mathbf{T}=\left(\bar{\mathbf{T}}V^{-1}\right)^{T}$.

We now have a complete algorithm for computing a column basis, given
in \prettyref{alg:colBasis}.

\begin{comment}
Note that rank sensitive computation for order basis is not as natural,
since for order basis we work with power series in general, whose
rank may never be truly computed. In addition, the computed basis
does not correspond to the rank. For these reasons, we do not pursue
rank sensitive computations for order basis. 
\end{comment}
\begin{algorithm}[t]
\caption{$\colBasis(\mathbf{F})$}
\label{alg:colBasis}

\begin{algorithmic}[1]
\REQUIRE{$\mathbf{F}\in\mathbb{K}\left[x\right]^{m\times n}$.}

\ENSURE{a column basis of $\mathbf{F}$.}

\STATE{$\vec{s}:=\cdeg\mathbf{F}$;}

\STATE{$\mathbf{N}:=\mnb(\mathbf{F},\vec{s})$;}

\STATE{$\mathbf{G}:=\left(\mnbr(\mathbf{N}^{T},\vec{s})\right)^{T}$; }

\STATE{$\vec{t}':=\left[0,\dots,0,-\rdeg_{-\vec{s}}\mathbf{G}\right]$, with
$\rowDimension(\mathbf{G})$ number of 0's ; }

\STATE{$\left[V^{T},\bar{\mathbf{T}}^{T}\right]^{T}:=\mnbr(\left[\mathbf{F}^{T},\mathbf{G}^{T}\right],\vec{t}')$
with a square $V$; }

\STATE{\textbf{$\mathbf{T}=\left(\bar{\mathbf{T}}V^{-1}\right)^{T}$};}

\RETURN $\mathbf{T}$; 
\end{algorithmic}
\end{algorithm}


\begin{thm}
\label{thm:columnBasisCost1}A column basis $\mathbf{T}$ of $\mathbf{F}$
can be computed with a cost of $O^{\sim}\left(n^{\omega-1}\xi\right)$.
\end{thm}
\begin{comment}
This can be done by computing a left $\left[0,\dots,0,d,\dots,d\right]$-minimal
kernel basis $\left[\mathbf{T}',V\right]$ of $\left[\mathbf{G}^{T},\mathbf{F}^{T}\right]^{T}$,
where $d$ is the degree of $\mathbf{F},$ $V$ is a unimodular matrix
and $\mathbf{T}'$ is a $m\times m$ matrix. Note that $\left[\mathbf{T}',V\right]$
has $m$ rows since the rank of $\left[\mathbf{G}^{T},\mathbf{F}^{T}\right]^{T}$
is $m$. Also note that since $\left[\mathbf{T},I\right]$ is a left
kernel basis with $\left[0,\dots,0,d,\dots,d\right]$-row degrees
bounded by $d$, the $\left[0,\dots,0,d,\dots,d\right]$-minimal kernel
basis $\left[\mathbf{T}',V\right]$ must also has its $\left[0,\dots,0,d,\dots,d\right]$-row
degrees bounded by $d$, hence the degree of $V$ must be 0. We can
then easily compute $\mathbf{T}=\mathbf{T}'V^{-1}$. 
\end{comment}



\section{\label{sec:successiveColBasisComputation}A Simple Improvement}

When the input matrix $\mathbf{F}$ has column dimension much larger
$n$ than the row dimension $m$, we can separate $\mathbf{F}=\left[\mathbf{F}_{1},\mathbf{F}_{2},\dots,\mathbf{F}_{n/m}\right]$
to $n/m$ blocks, each with dimension $m\times m$, assuming without
loss of generality $n$ is a multiple of $m$, and the columns are
arranged in increasing degrees. We then do a series of column basis
computations. First we compute a column basis $\mathbf{T}_{1}$ of
$\left[\mathbf{F}_{1},\mathbf{F}_{2}\right]$. Then compute a column
basis $\mathbf{T}_{2}$ of $\left[\mathbf{T}_{1},\mathbf{F}_{3}\right]$.
Repeating this process, at step $i$, we compute a column basis $\mathbf{T}_{i}$
of $\left[\mathbf{T}_{i-1},\mathbf{F}_{i+1}\right]$, until $i=n/m-1$,
when a column basis of $\mathbf{F}$ is computed.
\begin{lem}
At step $i$, computing a column basis $\mathbf{T}_{i}$ of $\left[\mathbf{T}_{i-1},\mathbf{F}_{i+1}\right]$
can be done with a cost of $O^{\sim}\left(m^{\omega-1}\left(\sum\cdeg\mathbf{F}_{i}+\sum\cdeg\mathbf{F}_{i+1}\right)\right)$
field operations.\end{lem}
\begin{proof}
From \prettyref{lem:colBasisDegreeBoundByInputDegrees}, the column
basis $\mathbf{T}_{i-1}$ of $\left[\mathbf{F}_{1},\dots,\mathbf{F}_{i}\right]$
has column degrees bounded by the largest column degrees of $\mathbf{F}_{i}$,
hence $\sum\cdeg\mathbf{T}_{i-1}\le\sum\cdeg\mathbf{F}_{i}$. The
lemma then follows by combining this with the result that a column
basis $\mathbf{T}_{i}$ of $\left[\mathbf{T}_{i-1},\mathbf{F}_{i+1}\right]$
can be computed with a cost of $O^{\sim}\left(n^{\omega-1}\left(\sum\cdeg\mathbf{T}_{i-1}+\sum\cdeg\mathbf{F}_{i+1}\right)\right)$
from \prettyref{thm:columnBasisCost1}.\end{proof}
\begin{thm}
\label{thm:finalCollBasisCost}A column basis of $\mathbf{F}$ can
be computed with a cost of $O^{\sim}\left(m^{\omega-1}\xi\right)$.\end{thm}
\begin{proof}
Summing up the cost of all the column basis computations, 
\begin{eqnarray*}
 &  & \sum_{i=1}^{n/m-1}O^{\sim}\left(m^{\omega-1}\left(\sum\cdeg\mathbf{F}_{i}+\sum\cdeg\mathbf{F}_{i+1}\right)\right)\\
 & \subset & O^{\sim}\left(m^{\omega-1}\left(2\sum\cdeg\mathbf{F}\right)\right)\\
 & = & O^{\sim}\left(m^{\omega-1}\xi\right)
\end{eqnarray*}
\end{proof}
\begin{rem}
In this section, the computational efficiency is improved by reducing
the original problem to about $n/m$ subproblems whose column dimensions
are close to the row dimension $m$. This is done by successive column
basis computations. Note that we can also reduce the column dimension
by using successive order basis computations, and only do a column
basis computation at the very last step. The computational complexity
of using order basis computation to reduce the column dimension would
remain the same, but in practice it maybe more efficient since order
basis computations are simpler.\end{rem}






\chapter{\label{chap:Unimodular-Completion}Unimodular Completion}

Given a matrix $\mathbf{F}\in\mathbb{K}\left[x\right]^{m\times n}$
with $n>m$ and its column degrees bounded by the entries of a vector
$\vec{s}$, we consider the problem of efficiently computing a matrix
$\mathbf{G}\in\mathbb{K}\left[x\right]^{(n-m)\times n}$ %
\begin{comment}
with $\left(-\vec{s}\right)$-minimal rows 
\end{comment}
such that $\begin{bmatrix}\mathbf{F}\\
\mathbf{G}
\end{bmatrix}$ is unimodular.%
\begin{comment}
let us call the product of the nonzero entries of its smith normal
form the \emph{generalized determinant} of $\mathbf{F}$.

Suppose $\mathbf{F}$ is full-rank with column degrees bounded by
the entries of a shift $\vec{s}\in\mathbb{Z}_{\ge0}^{n}$. We consider
the problem of finding a matrix $\mathbf{G}\in\mathbb{K}\left[x\right]^{(n-m)\times n}$
with $\left(-\vec{s}\right)$-minimal rows such that $\left[\mathbf{F}^{T},\mathbf{G}^{T}\right]^{T}$
has the same determinant as the generalized determinant of $\mathbf{F}$.
In the special case where the generalized determinant of $\mathbf{F}$
is $1$, the problem specializes to the standard unimodular completion
problem, where $\left[\mathbf{F}^{T},\mathbf{G}^{T}\right]^{T}$ is
unimodular. Note that the $\left(-\vec{s}\right)$ shift is chosen
to make the degrees consistent with the degrees of the input matrix
$\mathbf{F}$.
\begin{example}
If $\mathbf{F}=\left[1,0\right]$, $\vec{s}=\left[0,0\right]$. Then
the generalized determinant of $\mathbf{F}$ is $1$. A $\left(-\vec{s}\right)$-minimal
unimodular completion of $\mathbf{F}$ is then $\mathbf{G}=\left[0,1\right]$.
A unimodular completion that is not minimal is $\left[x^{9},1\right]$.
If $\mathbf{F}=\left[x,x^{2}\right]$, then a $\left(-\vec{s}\right)$-minimal
completion that maintains the generalized determinant is again $\left[0,1\right]$.\end{example}
\end{comment}
{} As before, our goal is to do this with a cost of $O^{\sim}\left(n^{\omega-1}\xi\right)$
field operations, where $\xi=\sum\vec{s}$. 

Unimodular completion also provides another way of computing column
bases. To compute a column basis of $\mathbf{A}$, we can compute
a right kernel basis $\mathbf{B}$ of $\mathbf{A}$. If $\mathbf{C}^{T}$
is a unimodular completion of $\mathbf{B}^{T}$, then \prettyref{lem:unimodular_kernel_columnBasis}
tells us that $\mathbf{A}\mathbf{C}$ is a column basis of $\mathbf{A}$.
Note that the kernel basis computation and the multiplication $\mathbf{AC}$
can be done efficiently using the algorithms from earlier chapters,
while an efficient way of computing unimodular completion is provided
in this chapter.

Before discussing the computation of a unimodular completion, we need
to check the existence of unimodular completion for a given matrix.
In fact, a unimodular completion does not exist for some input matrices,
as in the case of $\mathbf{F}=\left[0,x\right]$. So we need to know
what type of input matrices admit a unimodular completion.
\begin{lem}
\label{lem:unimodularCompletionCondition}A unimodular completion
of $\mathbf{F}$ exists if and only if $\mathbf{F}$ has unimodular
column bases. \end{lem}
\begin{proof}
If $\mathbf{F}$ has a non-unimodular column basis $\mathbf{A}$,
then $\diag\left(\left[\mathbf{A},I\right]\right)$ is always a factor
of $\begin{bmatrix}\mathbf{F}\\
\mathbf{B}
\end{bmatrix}$ for any polynomial matrix $\mathbf{B}$, implying that the matrix
$\begin{bmatrix}\mathbf{F}\\
\mathbf{B}
\end{bmatrix}$ is non-unimodular. On the other hand, if $\mathbf{F}$ has a unimodular
column basis, then there exists a unimodular matrix $\mathbf{U}$
such that $\mathbf{F}\mathbf{U}=\left[I_{m},0\right]$, or $\mathbf{F}=\left[I_{m},0\right]\mathbf{U}^{-1}$
after rearranging, that is, $\mathbf{F}$ must be consists of the
top $m$ rows of $\mathbf{U}^{-1}$. The matrix $\mathbf{U}^{-1}$
is therefore a unimodular completion of the matrix $\mathbf{F}$.
\end{proof}
Since a unimodular completion is only possible for input matrices
with unimodular column bases, we assume for simplicity this is the
case with our input matrix $\mathbf{F}$. This also requires $\mathbf{F}$
to be full rank. For other matrices without unimodular column bases,
we will see later that our method computes a matrix completion for
a row basis of $\mathbf{F}$ with its column basis factor removed.

The proof of \prettyref{lem:unimodularCompletionCondition} shows
that a unimodular completion of $\mathbf{F}$ can be obtained from
the unimodular matrix $\mathbf{U}$ that transforms $\mathbf{F}$
to its column bases. However, we may not be able to compute this $\mathbf{U}$
efficiently since its degree might be too large. More specifically,
$\mathbf{U}$ contains a kernel basis of $\mathbf{F}$ that may have
degree $\xi$, while each of the remaining columns of $\mathbf{U}$
may also have degree $\xi$. 

Before discussing the actual matrix completion, let us look at the
operations that reverses the coefficients of a polynomial, the coefficients
of the polynomial entries of a vector, and the coefficients of the
polynomial entries of a polynomial matrix. These operations are needed
in the computation of our matrix completion.


\section{Reversing polynomial coefficients}

First let us look at the operation that reverses the coefficients
of a polynomial.
\begin{defn}
For a polynomial $p=p_{0}+p_{1}x+\dots+p_{u}x^{u}\in\mathbb{K}\left[x\right]$
with degree bounded by $u$, we define the operation 
\[
\rev(p,u)=\left(p(x^{-1})\right)x^{u}=p_{u}+p_{u-1}x+\cdots+p_{1}x^{u-1}+p_{0}x^{u}.
\]

\end{defn}
We now extend this definition to column vectors and row vectors with
shifted degrees.
\begin{defn}
Let $\vec{u}=\left[u_{1},\dots u_{n}\right]\in\mathbb{Z}^{n}$ be
a degree shift, and a column vector $\mathbf{a}\in\mathbb{K}\left[x\right]^{n\times1}$
with $\vec{u}$-column degree bounded by $v$. We define
\[
\colRev(\mathbf{a},\vec{u},v)=x^{-\vec{u}}\left(\mathbf{a}(x^{-1})\right)x^{v}=\begin{bmatrix}x^{-u_{1}}\\
 & \ddots\\
 &  & x^{-u_{n}}
\end{bmatrix}\left(\mathbf{a}(x^{-1})\right)x^{v}=\begin{bmatrix}\rev(p,v-u_{1})\\
\vdots\\
\rev(p,v-u_{n})
\end{bmatrix}.
\]
Similarly for a row vector $\mathbf{b}\in\mathbb{K}\left[x\right]^{1\times n}$
with $\vec{u}$-row degree bounded by $v$, where $\vec{u}=\left[u_{1},\dots u_{n}\right]\in\mathbb{Z}^{n}$
is a degree shift, we define
\[
\rowRev(\mathbf{b},\vec{u},v)=\colRev(\mathbf{b}^{T},\vec{u},v)^{T}=x^{v}\left(\mathbf{b}(x^{-1})\right)x^{-\vec{u}}=x^{v}\left(\mathbf{b}(x^{-1})\right)\begin{bmatrix}x^{-u_{1}}\\
 & \ddots\\
 &  & x^{-u_{n}}
\end{bmatrix}.
\]
\end{defn}
\begin{example}
If $\mathbf{f}=\left[10+x,5+x+2x^{2}\right]$, $\vec{u}=\left[-1,-2\right]$,
and $v=0$, then 
\[
\rowRev(\mathbf{f},\vec{u},v)=x^{0}\left[10+x^{-1},5+x^{-1}+2x^{-2}\right]\begin{bmatrix}x\\
 & x^{2}
\end{bmatrix}=\left[10x+1,5x^{2}+x+2\right].
\]

\end{example}
We can extend the reverse operation further to polynomial matrices.
\begin{defn}
Let $\vec{u}=\left[u_{1},\dots u_{n}\right]\in\mathbb{Z}^{n}$ be
a degree shift. Let $\mathbf{A}\in\mathbb{K}\left[x\right]^{n\times k}$
with $\vec{u}$-column degrees bounded component-wise by $\vec{v}=\left[v_{1},\dots,v_{k}\right]$,
we define 
\[
\colRev(\mathbf{A},\vec{u},\vec{v})=x^{-\vec{u}}\left(\mathbf{A}(1/x)\right)x^{\vec{v}}
\]
 Similarly, for $\vec{u}=\left[u_{1},\dots,u_{n}\right]$ and $\mathbf{B}\in\mathbb{K}\left[x\right]^{k\times n}$
with $\vec{u}$-row degrees bounded component-wise by $\vec{v}=\left[v_{1},\dots,v_{k}\right]$,
\[
\rowRev(\mathbf{B},\vec{u},\vec{v})=\colRev(\mathbf{B}^{T},\vec{u},\vec{v})^{T}=x^{\vec{v}}\left(\mathbf{B}(1/x)\right)x^{-\vec{u}}
\]
 Note that we also have $\rowRev(\mathbf{B},\vec{u},\vec{v})=\colRev(\mathbf{B},-\vec{v},-\vec{u})$.
\end{defn}
\begin{comment}
Again, we assume a shift $\vec{s}$ bounds the column degrees of $\mathbf{F}$
component-wise with $\sum\vec{s}=\xi$. First, we compute a $\vec{s}$-minimal
kernel basis $\mathbf{N}$ of $\mathbf{F}$. We then reverse the coefficients
of $\mathbf{N}$ based on its $\vec{s}$-column degrees as follows:
To reverse the coefficients of column $\mathbf{n}$ that has $\vec{s}$-column
degrees bounded by $t$, let
\[
\reverse(\mathbf{n},\vec{s},t)=x^{-\vec{s}}\left(\mathbf{n}(1/x)\right)x^{t}=\begin{bmatrix}x^{-s_{1}}\\
 & \ddots\\
 &  & x^{-s_{n}}
\end{bmatrix}\left(\mathbf{n}(1/x)\right)x^{t}.
\]
 I.e., for the $i$th entry $\mathbf{n}_{i}$ of $\mathbf{n}$, where
$\mathbf{n}_{i}=p_{0}+p_{1}x+\dots+p_{t-s_{i}}x^{t-s_{i}}$, the reversed
$\mathbf{n}$ becomes $\reverse(\mathbf{n},\vec{s},t)=p_{t-s_{i}}+p_{t-s_{i}-1}x+\cdots+p_{1}x^{t-s_{i}-1}+p_{0}x^{t-s_{i}}.$
Each column of $\mathbf{N}$ is reversed in this way to get a new
matrix $\bar{\mathbf{N}}$. That is, for matrix $\mathbf{N}\in\mathbb{K}\left[x\right]^{n\times k}$
with $\vec{s}$-column degrees bounded component-wise by $\vec{t}$,
we define 
\[
\reverse(\mathbf{N},\vec{s},\vec{t})=x^{-\vec{s}}\left(\mathbf{N}(1/x)\right)x^{\vec{t}}=\begin{bmatrix}x^{-s_{1}}\\
 & \ddots\\
 &  & x^{-s_{n}}
\end{bmatrix}\left(\mathbf{N}(1/x)\right)\begin{bmatrix}x^{t_{1}}\\
 & \ddots\\
 &  & x^{t_{k}}
\end{bmatrix}.
\]


We then compute a $\left(\bar{\mathbf{N}}^{T},\sigma,-\vec{s}\right)$-basis
$\bar{\mathbf{P}}'$ with $\sigma$ big enough to contain a complete
kernel basis $\bar{\mathbf{F}}'$ of $\bar{\mathbf{N}}^{T}$. Let
$\mathbf{P}$ and $\mathbf{F'}$ be the matrices $\bar{\mathbf{P}}'$
and $\mathbf{\bar{F}'}$ with coefficients reversed based on their
$(-\vec{s})$-column degrees respectively. Then it is not difficult
to see that $\mathbf{F}'$ is a kernel basis of $\mathbf{N}'$.
\end{comment}


It is useful to note that any degree bound remains the same after
the reverse operations.
\begin{lem}
If $\mathbf{A}\in\mathbb{K}\left[x\right]^{n\times k}$ has $\vec{u}$-column
degrees bounded by the corresponding entries of $\vec{v}$, then $\colRev(\mathbf{A},\vec{u},\vec{v})$
also has $\vec{u}$-column degrees bounded by the corresponding entries
of $\vec{v}$.
\end{lem}
As one would expect, applying two reverse operations gives back the
original input.
\begin{lem}
The following equalities holds:%
\begin{comment}
\begin{eqnarray*}
\rev\left(\rev(p,u),u\right) & = & p\\
\colRev\left(\colRev(\mathbf{a},\vec{u},v),\vec{u},v\right) & = & \mathbf{a}\\
\rowRev\left(\rowRev(\mathbf{b},\vec{u},v),\vec{u},v\right) & = & \mathbf{b}
\end{eqnarray*}
\end{comment}
\begin{eqnarray*}
\colRev\left(\colRev(\mathbf{A},\vec{u},\vec{v}),\vec{u},\vec{v}\right) & = & \mathbf{A}\\
\rowRev\left(\rowRev(\mathbf{B},\vec{u},\vec{v}),\vec{u},\vec{v}\right) & = & \mathbf{B}
\end{eqnarray*}

\end{lem}
Let us look at a degree bound on the product of a row vector and a
column vector, based on their shifted degrees, when opposite shifts
are used.
\begin{lem}
\label{lem:vectorProductBound}If $\mathbf{a}\in\mathbb{K}\left[x\right]^{1\times n}$
and $\mathbf{a}^{T}$ has $\left(-\vec{u}\right)$-column degree bounded
by $\alpha$ (or equivalently, $\mathbf{a}$ has $\left(-\vec{u}\right)$-row
degree bounded by $\alpha$) and $\mathbf{b}\in\mathbb{K}\left[x\right]^{n\times1}$
has $\vec{u}$-column degree bounded by $\beta$, then $\mathbf{a}\mathbf{b}$
has degree bounded by $\alpha+\beta$.\end{lem}
\begin{proof}
Since $\mathbf{a}x^{-\vec{u}}$ has degree bounded by $\alpha$ and
$x^{\vec{u}}\mathbf{b}$ has degree bounded by $\beta$, $\mathbf{a}x^{-\vec{u}}x^{\vec{u}}\mathbf{b}=\mathbf{a}\mathbf{b}$
has degree bounded by $\alpha+\beta$.
\end{proof}
The following lemma shows that the reverse operation and the multiplication
are commutative when we use the opposite shifts.
\begin{lem}
\label{lem:reverseProduct}If $\mathbf{a}\in\mathbb{K}\left[x\right]^{1\times n}$
has $\left(-\vec{u}\right)$-row degree bounded by $\alpha$ and $\mathbf{b}\in\mathbb{K}\left[x\right]^{n\times1}$
has $\vec{u}$-column degree bounded by $\beta$, then 
\[
\rowRev(\mathbf{a},-\vec{u},\alpha)\cdot\colRev(\mathbf{b},\vec{u},\beta)=\rev(\mathbf{a}\mathbf{b},\alpha+\beta).
\]
\end{lem}
\begin{proof}
\ 
\begin{eqnarray*}
 &  & \rowRev(\mathbf{a},-\vec{u},\alpha)\cdot\colRev(\mathbf{b},\vec{u},\beta)\\
 & = & x^{\alpha}\left(\mathbf{a}(1/x)\right)x^{\vec{u}}x^{-\vec{u}}\left(\mathbf{b}(1/x)\right)x^{\beta}\\
 & = & \left(\mathbf{a}(1/x)\right)x^{\vec{u}}x^{-\vec{u}}\left(\mathbf{b}(1/x)\right)x^{\alpha+\beta}\\
 & = & \left(\mathbf{a}(1/x)\right)\left(\mathbf{b}(1/x)\right)x^{\alpha+\beta}\\
 & = & \left(\left(\mathbf{a}\mathbf{b}\right)(1/x)\right)x^{\alpha+\beta}\\
 & = & \rev(\mathbf{a}\mathbf{b},\alpha+\beta)
\end{eqnarray*}
\begin{comment}
\begin{lem}
If $\mathbf{F}\in\mathbb{K}\left[x\right]^{m\times n}$ has $\left(-\vec{s}\right)$-row
degree bounded component-wise by $\vec{a}=\left[a_{1},\dots,a_{m}\right]$
and $\mathbf{G}\in\mathbb{K}\left[x\right]^{n\times k}$ has $\vec{s}$-column
degree bounded component-wise by $\vec{b}=[b_{1},\dots,b_{m}]$, then
\[
\reverse(\mathbf{F},-\vec{s},\vec{a})\cdot\reverse(\mathbf{G},\vec{s},\vec{b})=\reverse(\mathbf{F}\mathbf{G},\vec{c}),
\]
 where 
\[
\vec{c}=\begin{bmatrix}a_{1}+b_{1} & a & \cdots & a_{1}+b_{k}\\
\\
\\
\end{bmatrix}\in\mathbb{Z}^{m\times k}
\]
\end{lem}
\begin{proof}
sd$\reverse(\mathbf{n},\vec{s},t)\cdot\reverse(\mathbf{g},\vec{s},b)$\end{proof}
\end{comment}


We also have the following similar result on the reverse operation
and matrix multiplication\end{proof}
\begin{lem}
\label{lem:reverseMatrixProduct}If $\mathbf{A}\in\mathbb{K}\left[x\right]^{m\times n}$
has $\vec{u}$-column degrees bounded by $\vec{v}$, and $\mathbf{B}\in\mathbb{K}\left[x\right]^{n\times k}$
has $\vec{v}$-column degrees bounded by $\vec{w}$, then 
\[
\colRev(\mathbf{A},\vec{u},\vec{v})\colRev(\mathbf{B},\vec{v},\vec{w})=\colRev(\mathbf{A}\mathbf{B},\vec{u},\vec{w})
\]
 has $\vec{u}$-column degrees bounded by $\vec{w}$. \end{lem}
\begin{proof}
\ 
\begin{eqnarray*}
 &  & \colRev(\mathbf{A},\vec{u},\vec{v})\colRev(\mathbf{B},\vec{v},\vec{w})\\
 & = & x^{-\vec{u}}\left(\mathbf{A}(1/x)\right)x^{\vec{v}}x^{-\vec{v}}\left(\mathbf{B}(1/x)\right)x^{\vec{w}}\\
 & = & x^{-\vec{u}}\left(\mathbf{A}\mathbf{B}\right)(1/x)x^{\vec{w}}.
\end{eqnarray*}
\end{proof}
\begin{lem}
\label{lem:reverseMatrixProduct2}If $\mathbf{A}\in\mathbb{K}\left[x\right]^{m\times n}$
has $\vec{u}$-row degrees bounded by $\vec{v}$, and $\mathbf{B}\in\mathbb{K}\left[x\right]^{n\times k}$
has $-\vec{u}$-column degrees bounded by $\vec{w}$, then 
\[
\rowRev(\mathbf{A},\vec{u},\vec{v})\colRev(\mathbf{B},-\vec{u},\vec{w})=\colRev(\mathbf{A}\mathbf{B},-\vec{v},\vec{w}).
\]
.\end{lem}
\begin{proof}
\begin{eqnarray*}
 &  & \rowRev(\mathbf{A},\vec{u},\vec{v})\colRev(\mathbf{B},-\vec{u},\cdeg_{-\vec{u}}\mathbf{B})\\
 & = & x^{\vec{v}}\left(\mathbf{A}(1/x)\right)x^{-\vec{u}}x^{\vec{u}}\left(\mathbf{B}(1/x)\right)x^{\cdeg_{-\vec{u}}\mathbf{B}}\\
 & = & x^{\vec{v}}\left(\mathbf{A}(1/x)\right)\left(\mathbf{B}(1/x)\right)x^{\cdeg_{-\vec{u}}\mathbf{B}}\\
 & = & x^{\vec{v}}\left(\mathbf{A}\mathbf{B}(1/x)\right)x^{\cdeg_{-\vec{u}}\mathbf{B}}.
\end{eqnarray*}

\end{proof}

\section{Unimodular completion}

In this section, we look at how a unimodular completion can be done
using a combination of kernel basis computations, order basis computations,
and reverse operations. First, we have the following natural relationship
between a kernel basis and the reverse operation.
\begin{lem}
\label{lem:reverseNullspaceBasis}Let $\vec{u}\in\mathbb{Z}^{n}$,
$\mathbf{A}\in\mathbb{K}\left[x\right]^{m\times n}$ with $(-\vec{u})$-row
degrees bounded component-wise by $\vec{a}$, and $\mathbf{A}^{r}=\rowRev\left(\mathbf{A},-\vec{u},\vec{a}\right)$.
Then a matrix $\mathbf{N}\in\mathbb{K}\left[x\right]^{n\times k}$
with $\vec{u}$-column degrees $\vec{b}$ is a $(\mathbf{A},\vec{u})$-kernel
basis %
\begin{comment}
$\vec{u}$-minimal kernel basis of $\mathbf{A}$ 
\end{comment}
if and only if $\mathbf{N}^{r}=\colRev\left(\mathbf{N},\vec{u},\vec{b}\right)$
is a $(\mathbf{A}^{r},\vec{u})$-kernel basis%
\begin{comment}
$\vec{u}$-minimal kernel basis of $\rowRev\left(\mathbf{A},-\vec{u},\vec{a}\right)$
\end{comment}
.\end{lem}
\begin{proof}
If $\mathbf{N}$ is a kernel basis of $\mathbf{A}$, then we know
from \prettyref{lem:reverseProduct} that 
\[
\rowRev\left(\mathbf{A},-\vec{u},\vec{a}\right)\cdot\colRev\left(\mathbf{N},\vec{u},\vec{b}\right)=0,
\]
so $\colRev\left(\mathbf{N},\vec{u},\vec{b}\right)$ is a kernel basis
of $\rowRev\left(\mathbf{A},-\vec{u},\vec{a}\right)$. Suppose $\colRev\left(\mathbf{N},\vec{u},\vec{b}\right)$
is not $\vec{u}$-minimal and we have another kernel basis $\mathbf{M}$
of $\rowRev\left(\mathbf{A},-\vec{u},\vec{a}\right)$ with $\vec{u}$-column
degrees $\vec{c}$ that has some entry lower than the corresponding
entry in $\vec{b}$. Then $\colRev\left(\mathbf{M},\vec{u},\vec{c}\right)$
is also a kernel of $\mathbf{A}$ with lower $\vec{u}$-column degrees
than $\vec{b}$, contradicting the $\vec{u}$-minimality of $\mathbf{N}$.
\end{proof}


The following lemma shows the unimodular equivalence between any matrix
$\mathbf{A}$ that has a unimodular column basis, and a left kernel
basis of any right kernel basis of $\mathbf{A}$.
\begin{lem}
\label{lem:unimodularEquivalenceNullspaceBasisOfNullspaceBasis}Given
a matrix $\mathbf{A}\in\mathbb{K}\left[x\right]^{m\times n}$ with
unimodular column basis. Let $\mathbf{N}\in\mathbb{K}\left[x\right]^{n\times\left(n-m\right)}$
be a right kernel basis of $\mathbf{A}$. Let $\mathbf{B}$ be a left
kernel basis of $\mathbf{N}$. Then $\mathbf{A}=\mathbf{U}\mathbf{B}$
for a unimodular matrix $\mathbf{U}$.\end{lem}
\begin{proof}
This follows from \prettyref{lem:matrixGCD}, which tells us that
$\mathbf{U}$ is just a column basis of $\mathbf{A}$.
\end{proof}
Now let us look at how an order basis can lead to a unimodular matrix. 
\begin{lem}
\label{lem:reverseOrderBasisToUnimodular}Let $\vec{u}=\left[u_{1},\dots u_{n}\right]\in\mathbb{Z}^{n}$
be a degree shift. Any $\left(\mathbf{A},\sigma,\vec{u}\right)$-basis
$\mathbf{P}$ with $\cdeg_{\vec{u}}\mathbf{P}=\vec{v}=\left[v_{1},\dots,v_{k}\right]$
has $\det\left(\mathbf{P}\right)=cx^{\sum\vec{v}-\sum\vec{u}}$ and
$\det\left(\colRev(\mathbf{P},\vec{u},\vec{v})\right)=c$ for some
constant $c\in\mathbb{K}$. In other words, $\colRev(\mathbf{P},\vec{u},\vec{v})$
is unimodular.\end{lem}
\begin{proof}
To see that $\det\left(\mathbf{P}\right)=cx^{\sum\vec{v}-\sum\vec{u}}$,
note that an identity matrix is an $\left(\mathbf{A},0,\vec{u}\right)$-basis,
which has $\vec{u}$-column degrees $\vec{u}$ and determinant $1$.
Then the $\vec{u}$-column degrees only increases by multiplying some
column of $\mathbf{P}$ by $x$. The second property $\det\left(\colRev(\mathbf{P},\vec{u},\vec{v})\right)=c$
follows from the definition 
\[
\colRev(\mathbf{P},\vec{u},\vec{v})=x^{-\vec{u}}\left(\mathbf{P}(1/x)\right)x^{\vec{v}}.
\]

\end{proof}
\prettyref{lem:reverseOrderBasisToUnimodular} suggests that a unimodular
completion of $\mathbf{F}$ can be computed by embedding $\mathbf{F}$
in a reversed order basis, or equivalently, embedding a reversed $\mathbf{F}$
in an order basis. The next question is therefore how to embed a matrix
in an order basis. Recall that \prettyref{lem:nullspaceBasisInOrderBasis}
shows how kernel bases can be embedded in order bases. Therefore,
if we can make the reversed $\mathbf{F}$ a kernel basis of some matrix
$\mathbf{M}$, then there is an order basis of $\mathbf{M}$ that
contains the reversed $\mathbf{F}$. A natural choice for $\mathbf{M}$
is a kernel basis of the reversed $\mathbf{F}$.  We actually have
two choices here. We can either reverse the coefficients of $\mathbf{F}$,
as we do in \prettyref{lem:unimodularComputation} below, or we can
reverse the coefficients of a kernel basis of $\mathbf{F}$.
\begin{lem}
\label{lem:unimodularComputation}Let $\mathbf{F}^{r}=\rowRev\left(\mathbf{F},-\vec{s},0\right)$
and $\mathbf{M}$ be a $(\mathbf{F}^{r},\vec{s})$-kernel basis with
$\cdeg_{\vec{s}}\mathbf{M}=\vec{b}$. Let $\mathbf{P}=\left[\mathbf{P}_{1},\mathbf{P}_{2}\right]$
be a $\left(\mathbf{M}^{T},\vec{b}+\left[1,\dots,1\right],-\vec{s}\right)$-basis,
where $\mathbf{P}_{1}$ consists of all columns $\mathbf{p}$ with
$\cdeg_{-\vec{s}}\mathbf{p}\le0$. If $\mathbf{P}_{2}^{r}=\colRev\left(\mathbf{P}_{2},-\vec{s},\cdeg_{-\vec{s}}\mathbf{P}_{2}\right)$,
then $\left[\mathbf{F}^{T},\mathbf{P}_{2}^{r}\right]$ is a unimodular
matrix.\end{lem}
\begin{proof}
Let $\mathbf{P}_{1}^{r}=\colRev\left(\mathbf{P}_{1},-\vec{s},\cdeg_{-\vec{s}}\mathbf{P}_{1}\right)$.
We know from \prettyref{lem:reverseOrderBasisToUnimodular} that $\left[\mathbf{P}_{1}^{r},\mathbf{P}_{2}^{r}\right]$
is unimodular. Let $\mathbf{M}^{r}=\colRev\left(\mathbf{M},\vec{s},\vec{b}\right)$.
Then from \prettyref{lem:reverseNullspaceBasis} we know $\mathbf{M}^{r}$
is a $\left(\mathbf{F},\vec{s}\right)$-kernel basis and $\mathbf{P}_{1}^{r}$
is a $\left(\left(\mathbf{M}^{r}\right)^{T},-\vec{s}\right)$-kernel
basis, hence by \prettyref{lem:unimodularEquivalenceNullspaceBasisOfNullspaceBasis}
$\mathbf{F}=\mathbf{U}\left(\mathbf{P}_{1}^{r}\right)^{T}$ for some
unimodular matrix $\mathbf{U}$. Now $\left[\mathbf{F}^{T},\mathbf{P}_{2}^{r}\right]^{T}=\diag\left(\left[\mathbf{U},I\right]\right)\left[\mathbf{P}_{1}^{r},\mathbf{P}_{2}^{r}\right]^{T}$.
\end{proof}
\prettyref{lem:unimodularComputation} provides a way to correctly
compute a unimodular completion of $\mathbf{F}$. To improve the computational
efficiency, we can in fact separate the rows of $\mathbf{M}^{T}$
and just work with one subset of rows at a time. 
\begin{lem}
\label{lem:unimodularComputationByRows}Let $\mathbf{F}^{r}=\rowRev\left(\mathbf{F},-\vec{s},0\right)$.
Let $\mathbf{M}$ be a $(\mathbf{F}^{r},\vec{s})$-kernel basis with
$\cdeg_{\vec{s}}\mathbf{M}=\vec{b}$. Let $\mathbf{M}=\left[\mathbf{M}_{1},\mathbf{M}_{2}\right]$.
Let $\mathbf{P}_{1}=\left[\mathbf{N}_{1},\mathbf{Q}_{1}\right]$ be
a $\left(\mathbf{M}_{1}^{T},\cdeg_{\vec{s}}\mathbf{M}_{1}+\left[1,\dots,1\right],-\vec{s}\right)$-basis,
where $\mathbf{N}_{1}$ consists of all columns $\mathbf{p}$ of $\mathbf{P}_{1}$
with $\cdeg_{-\vec{s}}\mathbf{p}\le0$. Let $\vec{t}=\cdeg_{-\vec{s}}\mathbf{N}_{1}$
and $\mathbf{P}_{2}=\left[\mathbf{N}_{2},\mathbf{Q}_{2}\right]$ be
a $\left(\mathbf{M}_{2}^{T}\mathbf{N}_{1},\cdeg_{\vec{s}}\mathbf{M}_{2}+\left[1,\dots,1\right],\vec{t}\right)$-basis,
where $\mathbf{N}_{2}$ consists of all columns $\mathbf{p}$ of $\mathbf{P}_{2}$
with $\cdeg_{-\vec{t}}\mathbf{p}\le0$. Let $\mathbf{R}=\left[\mathbf{N}_{1}\mathbf{Q}_{2},\mathbf{Q}_{1}\right]$
and $\mathbf{R}^{r}=\colRev\left(\mathbf{R},-\vec{s},\cdeg_{-\vec{s}}\mathbf{R}\right)$.
Then $\left[\mathbf{F}^{T},\mathbf{R}^{r}\right]$ is a unimodular
matrix.\end{lem}
\begin{proof}
We know from \prettyref{lem:reverseOrderBasisToUnimodular} that $\mathbf{P}_{1}^{r}=\colRev\left(\mathbf{P}_{1},-\vec{s},\cdeg_{-\vec{s}}\mathbf{P}_{1}\right)$
and $\mathbf{P}_{2}^{r}=\colRev\left(\mathbf{P}_{1},\vec{t},\cdeg_{\vec{t}}\mathbf{P}_{2}\right)$
are both unimodular. Hence $\mathbf{P}_{1}^{r}\cdot\diag\left(\left[\mathbf{P}_{2}^{r},I\right]\right)=\left[\mathbf{N}_{1}^{r}\mathbf{N}_{2}^{r},\mathbf{N}_{1}^{r}\mathbf{Q}_{2}^{r},\mathbf{Q}_{1}\right]=\left[\mathbf{N}_{1}^{r}\mathbf{N}_{2}^{r},\mathbf{R}^{r}\right]$
is unimodular, where\textbf{ $\mathbf{N}_{1}\mathbf{N}_{2}$ }is a
kernel basis of $\mathbf{M}$. The result follows by the same reasoning
as in \prettyref{lem:unimodularComputation}.
\end{proof}

\section{Efficient Computation}

\prettyref{lem:unimodularComputationByRows} provides a way to correctly
compute a unimodular completion of $\mathbf{F}$. Our next task is
to make sure it can be computed efficiently and analyze its computational
cost. We already know that a $(\mathbf{F}^{r},\vec{s})$-kernel basis
can be computed with a cost of $O^{\sim}\left(n^{\omega-1}\xi\right)$.
Therefore, it only remains to check the cost of the order basis computations.
Note that the non-uniform order makes our problem here a little more
difficult. But on the other hand, the output basis has its $-\vec{s}$-column
degrees bounded by $1$, which is a consequence of the fact $\mathbf{M}$
is a $\vec{s}$-minimal kernel basis, as shown in \prettyref{lem:nullspaceOrderbasisDegree}
below. But we first need a few general lemmas on the degree bounds
of order bases and kernel bases.

First, the following lemma is a simple extension of \prettyref{lem:boundOfSumOfShiftedDegreesOfOrderBasis}
for dealing with nonuniform orders.
\begin{lem}
\label{lem:boundOfSumOfShiftedDegreesOfOrderBasisWithNonuniformOrder}Given
an input matrix $\mathbf{A}\in\mathbb{K}^{m\times n}[x]$, a shift
$\vec{u}\in\mathbb{Z}^{n}$, and an order list $\vec{\sigma}\in\mathbb{Z}^{m}$.
Let $\vec{v}$ be the $\vec{u}$-column degrees of a $\left(\mathbf{A},\vec{\sigma},\vec{u}\right)$-basis.
Then $\sum\vec{t}~\le~\sum\vec{s}+\sum\vec{\sigma}$%
\begin{comment}
 and $\max_{i}\left(\vec{t}_{i}-\vec{s}_{i}\right)\le\sigma$
\end{comment}
\textup{}%
\begin{comment}
need to permute the columns to put the pivots on the diagonal.
\end{comment}
. \end{lem}
\begin{proof}
\begin{comment}
For example, for a input matrix with two rows, instead of increasing
the order of both rows to $\left[1,1\right]$ at once, we can work
on the second row first to increase the order to $\left[0,1\right]$,
and then compute to order $\left[1,1\right]$.  
\end{comment}
The sum of the $\vec{s}$-column degrees is $\sum\vec{s}$ at order
$\left[0,\dots,0\right]$, since the identity matrix is a $\left(\mathbf{A},\left[0,\dots,0\right],\vec{s}\right)$-basis.
This sum increases by $1$ for each order increase of each row. The
total number of order increases required for all rows is at most $\sum\vec{\sigma}$.
Note that from \prettyref{thm:combineOrderBases}, we can work with
just one row at a time to increase its order in the order basis computation. 
\end{proof}
The following lemma extends \prettyref{thm:boundOfSumOfShiftedDegreesOfKernelBasis}
to give a bound based on the shifted column degrees or shifted row
degrees, instead of just the column degrees of the input matrix.
\begin{lem}
\label{lem:generalNullspaceBasisDegreeBound}If $\mathbf{A}\in\mathbb{K}^{m\times n}[x]$
has $\rdeg_{\vec{u}}\mathbf{A}\le\vec{v}$ or equivalently $\cdeg_{-\vec{v}}\mathbf{A}\le-\vec{u}$,
then any $(\mathbf{A},-\vec{u})$-kernel basis has $-\vec{u}$-column
degrees bounded by $\sum\vec{v}-\sum\vec{u}$. \end{lem}
\begin{proof}
Let $\mathbf{P}=\left[\mathbf{B},\bar{\mathbf{B}}\right]$ be a $\left(\mathbf{A},\vec{v}+\left[\sigma,\dots,\sigma\right],-\vec{u}\right)$-basis
containing a kernel basis, $\mathbf{B}$, of $\mathbf{A}$. Then $\sum\cdeg_{-\vec{u}}\mathbf{P}$
is at least $m\sigma+\sum\vec{v}-\sum\vec{u}$. We also know that
$\sum\cdeg_{-\vec{u}}\bar{\mathbf{B}}\ge\sum\cdeg_{-\vec{v}}\mathbf{A}\bar{\mathbf{B}}$,
but $\cdeg\mathbf{A}\bar{\mathbf{B}}\ge\vec{v}+\left[\sigma,\dots,\sigma\right]$
or $\sum\cdeg_{-\vec{v}}\mathbf{A}\bar{\mathbf{B}}\ge m\sigma$, therefore
$\sum\cdeg_{-\vec{u}}\bar{\mathbf{B}}\ge m\sigma$. It follows that
$\sum\cdeg_{-\vec{u}}\mathbf{B}\le m\sigma+\sum\vec{v}-\sum\vec{u}-m\sigma=\sum\vec{v}-\sum\vec{u}$.
\end{proof}
When the matrix $\mathbf{A}$ is also a $\left(\mathbf{B}^{T},\vec{u}\right)$-kernel
basis, as in our case, the bound in fact becomes tight.
\begin{lem}
\label{lem:mutualMinimalNullspaceBasisDegrees}Let $\mathbf{A}\in\mathbb{K}^{m\times n}[x]$
and $\mathbf{B}\in\mathbb{K}^{n\times(n-m)}\left[x\right]$. If $\mathbf{B}$
is a $(\mathbf{A},-\vec{u})$-kernel basis with $\cdeg_{-\vec{u}}\mathbf{B}=\vec{w}$
and $\mathbf{A}^{T}$ is a $\left(\mathbf{B}^{T},\vec{u}\right)$-kernel
basis with $\rdeg_{\vec{u}}\mathbf{A}=\vec{v}$, then $\sum\vec{w}=\sum\vec{v}-\sum\vec{u}$.\end{lem}
\begin{proof}
This follows from \prettyref{lem:generalNullspaceBasisDegreeBound},
which gives $\sum\vec{w}\le\sum\vec{v}-\sum\vec{u}$ and also $\sum\vec{v}\le\sum\vec{w}+\sum\vec{u}$
in the reverse direction.
\end{proof}
From \prettyref{lem:nullspaceBasisInOrderBasis}, we know that any
$\left(\mathbf{M}^{T},\vec{b}+1,-\vec{s}\right)$-basis contains a
$\left(\mathbf{M}^{T},-\vec{s}\right)$-kernel basis whose $-\vec{s}$-column
degrees bounded by 0. The following lemma shows that the remaining
part of the $\left(\mathbf{M}^{T},\vec{b}+1,-\vec{s}\right)$-basis
has degrees bounded by 1.
\begin{lem}
\label{lem:nullspaceOrderbasisDegree}Let $\mathbf{F}^{r}=\rowRev\left(\mathbf{F},-\vec{s},0\right)$
and $\mathbf{M}$ be a $(\mathbf{F}^{r},\vec{s})$-kernel basis with
$\cdeg_{\vec{s}}\mathbf{M}=\vec{b}$ as before. Let $\mathbf{P}$
be a $\left(\mathbf{M}^{T},\vec{b}+1,-\vec{s}\right)$-basis. Then
$\cdeg_{-\vec{b}-1}\mathbf{M}^{T}\mathbf{P}_{2}=\left[0,\dots,0\right]$
and $\cdeg_{-\vec{s}}\mathbf{P}_{2}=\left[1,\dots,1\right]$.\end{lem}
\begin{proof}
We already know that $\mathbf{P}$ contains a $\left(\mathbf{M}^{T},-\vec{s}\right)$-kernel
basis. Let this kernel basis be $\mathbf{P}_{1}$ in $\mathbf{P}=\left[\mathbf{P}_{1},\mathbf{P}_{2}\right]$.
We know that $\sum\cdeg_{-\vec{s}}\mathbf{P}=-\sum\vec{s}+\sum\vec{b}+n-m$
and for the kernel basis $\mathbf{P}_{1}$ in $\mathbf{P}$, we know
$\sum\cdeg_{-\vec{s}}\mathbf{P}_{1}=\sum\vec{b}-\sum\vec{s}$ from
\prettyref{lem:mutualMinimalNullspaceBasisDegrees}. Therefore, $\sum\cdeg_{-\vec{s}}\mathbf{P}_{2}=n-m$.
It follows that $\sum\cdeg_{-\vec{b}}\mathbf{M}^{T}\mathbf{P}_{2}\le\sum\cdeg_{-\vec{s}}\mathbf{P}_{2}=n-m$,
or $\sum\cdeg_{-\vec{b}-\left[1,\dots,1\right]}\mathbf{M}^{T}\mathbf{P}_{2}=0$.
But since $\mathbf{P}_{2}$ is nonzero and has order $\left(\mathbf{F},\vec{b}+\left[1,\dots,1\right]\right)$,
we have $\cdeg_{-\vec{b}-\left[1,\dots,1\right]}\mathbf{M}^{T}\mathbf{P}_{2}\ge\left[0,\dots,0\right]$,
implying $\sum\cdeg_{-\vec{b}-\left[1,\dots,1\right]}\mathbf{M}^{T}\mathbf{P}_{2}\ge0$.
It follows that $\sum\cdeg_{-\vec{b}-\left[1,\dots,1\right]}\mathbf{M}^{T}\mathbf{P}_{2}=0$,
hence $\cdeg_{-\vec{b}-\left[1,\dots,1\right]}\mathbf{M}^{T}\mathbf{P}_{2}=\left[0,\dots,0\right]$
or $\cdeg_{-\vec{b}}\mathbf{M}^{T}\mathbf{P}_{2}=\left[1,\dots,1\right]$.
Combining this with $\sum\cdeg_{-\vec{b}}\mathbf{M}^{T}\mathbf{P}_{2}\le\sum\cdeg_{-\vec{s}}\mathbf{P}_{2}=n-m$
we then get $\cdeg_{-\vec{s}}\mathbf{P}_{2}=\left[1,\dots,1\right]$.
\end{proof}
\begin{comment}
\begin{lem}
Let $\mathbf{F}^{r}=\rowRev\left(\mathbf{F},-\vec{s},0\right)$ and
$\mathbf{M}$ be a $(\mathbf{F}^{r},\vec{s})$-kernel basis with $\cdeg_{\vec{s}}\mathbf{M}=\vec{b}$
as before. Let $\mathbf{P}$ be a $\left(\mathbf{M}^{T},\vec{b}+\left[1,\dots,1\right],-\vec{s}\right)$-basis.
Then $\mathbf{M}^{T}\mathbf{P}$ has row degrees $\vec{b}+[1,\dots,1]$.
In other words, the row degrees are the same as the order.\end{lem}
\begin{proof}
From \prettyref{lem:nullspaceBasisOrderBasisContainsNullspaceBasis},
$\mathbf{P}$ contains a $\left(\mathbf{M}^{T},-\vec{s}\right)$-kernel
basis. Let this kernel basis be $\mathbf{P}_{1}$ in $\mathbf{P}=\left[\mathbf{P}_{1},\mathbf{P}_{2}\right]$.
We know that $\sum\cdeg_{-\vec{s}}\mathbf{P}=-\sum\vec{s}+\sum\vec{b}+n-m$
and for the kernel basis $\mathbf{P}_{1}$ in $\mathbf{P}$, we know
$\sum\cdeg_{-\vec{s}}\mathbf{P}_{1}=\sum\vec{b}-\sum\vec{s}$. Therefore,
$\sum\cdeg_{-\vec{s}}\mathbf{P}_{2}=n-m$. It follows that $\sum\cdeg_{-\vec{b}}\mathbf{M}^{T}\mathbf{P}_{2}\le\sum\cdeg_{-\vec{s}}\mathbf{P}_{2}=n-m$,
or $\sum\cdeg_{-\vec{b}-\left[1,\dots,1\right]}\mathbf{M}^{T}\mathbf{P}_{2}=0$.
But since $\mathbf{P}_{2}$ is nonzero and has order $\left(\mathbf{F},\vec{b}+\left[1,\dots,1\right]\right)$,
we have $\cdeg_{-\vec{b}-\left[1,\dots,1\right]}\mathbf{M}^{T}\mathbf{P}_{2}\ge\left[0,\dots,0\right]$,
implying $\sum\cdeg_{-\vec{b}-\left[1,\dots,1\right]}\mathbf{M}^{T}\mathbf{P}_{2}\ge0$.
It follows that $\sum\cdeg_{-\vec{b}-\left[1,\dots,1\right]}\mathbf{M}^{T}\mathbf{P}_{2}=0$,
hence $\cdeg_{-\vec{b}-\left[1,\dots,1\right]}\mathbf{M}^{T}\mathbf{P}_{2}=\left[0,\dots,0\right]$. 

Since 

or equivalently, $\rdeg\mathbf{M}^{T}\mathbf{P}_{2}=\vec{b}+\left[1,\dots,1\right]$.

We know that $\mathbf{M}^{T}$ has an identity matrix GCD, hence by
\prettyref{lem:CoprimeFullRankConstantCoefficientMatrix} $\mathbf{M}^{T}\mod x$
is full-rank, and by \prettyref{cor:fullRankConstantCoefficientMatrix}
$x^{-\vec{b}-[1,\dots,]}\mathbf{M}^{T}\mathbf{P}\mod x$ is also full-rank.
Now since 
\begin{eqnarray*}
 &  & \rowRev(\mathbf{M}^{T},\vec{s},\vec{b})\colRev(\mathbf{P},-\vec{s},\cdeg_{-\vec{s}}\mathbf{P})\\
 & = & x^{\vec{b}}\left(\mathbf{M}^{T}(1/x)\right)x^{-\vec{s}}x^{\vec{s}}\left(\mathbf{P}(1/x)\right)x^{\cdeg_{-\vec{s}}\mathbf{P}}\\
 & = & x^{\vec{b}}\left(\mathbf{M}^{T}(1/x)\right)\left(\mathbf{P}(1/x)\right)x^{\cdeg_{-\vec{s}}\mathbf{P}}\\
 & = & x^{\vec{b}}\left(\mathbf{M}^{T}\mathbf{P}(1/x)\right)x^{\cdeg_{-\vec{s}}\mathbf{P}}
\end{eqnarray*}


We know that $\mathbf{P}$ contains a $(\mathbf{M}^{T},-\vec{s})$-kernel
basis since any $(\mathbf{M}^{T},-\vec{s})$-kernel basis has $-\vec{s}$-column
degrees bounded by that of $\mathbf{F}^{rT}$, which is the same as
that of $\mathbf{F}^{T}$ and no more than $0$.\end{proof}
\end{comment}


We are now ready to look at the algorithm for computing a $\left(\mathbf{M}^{T},\vec{b}+\left[1,\dots,1\right],-\vec{s}\right)$-basis,
given in \prettyref{alg:unimodularCompletion}.  We follow the same
process as in \prettyref{sec:computeRightFactor}. We assume without
loss of generality that the rows of $\mathbf{M}^{T}$ are arranged
in decreasing $\vec{s}$-row degrees. We divide $\mathbf{M}^{T}$
into $\log k$ row blocks according to the $\vec{s}$-row degrees
of its rows. Let 
\[
\mathbf{M}^{T}=\left[\mathbf{M}_{1}^{T},\mathbf{M}_{2}^{T},\cdots,\mathbf{M}_{\log k-1}^{T},\mathbf{M}_{\log k}^{T}\right]
\]
 with $\mathbf{M}_{\log k},\mathbf{M}_{\log k-1},\cdots,\mathbf{M}_{2},\mathbf{M}_{1}$
having $\vec{s}$-row degrees in the range $\left[0,2\xi/k\right]$,
$(2\xi/k,4\xi/k],$ $(4\xi/k,8\xi/k],\ ...,$ $(\xi/4,\xi/2],$ $(\xi/2,\xi].$Let
$\vec{\sigma}_{i}=\left[\xi/2^{i-1}+1,\dots,\xi/2^{i-1}+1\right]$
with the same dimension as the row dimension of $\mathbf{M}_{i}$.
Let $\vec{\sigma}=\left[\vec{\sigma}_{\log k},\vec{\sigma}_{\log k-1},\dots,\vec{\sigma}_{1}\right]$
be the order in the order basis computation. For simplicity, instead
of using $\mathbf{M}^{T}$ as the input matrix, we use 
\begin{eqnarray*}
\hat{\mathbf{M}} & =\begin{bmatrix}\hat{\mathbf{M}}_{1}\\
\vdots\\
\hat{\mathbf{M}}_{\log k}
\end{bmatrix}= & x^{\vec{\sigma}-\vec{b}-1}\begin{bmatrix}\mathbf{M}_{1}\\
\vdots\\
\mathbf{M}_{\log k}
\end{bmatrix}=x^{\vec{\sigma}-\vec{b}-1}\mathbf{M}
\end{eqnarray*}
 instead, so that a $\left(\hat{\mathbf{M}},\vec{\sigma},-\vec{s}\right)$-basis
is a $\left(\mathbf{M},\vec{b}+1,-\vec{s}\right)$-basis.



\begin{comment}
kjh
\begin{lem}
If $\mathbf{A}$ is $\vec{u}$-row reduced and has an identity matrix
GCD, and if $\mathbf{P}$ is a $\left(\mathbf{A},\vec{v},-\vec{u}\right)$-basis
that contains a complete kernel basis of $\mathbf{A}$, then $\det\mathbf{P}=x^{\sum\vec{v}}$
or equivalently, $\sum\cdeg_{-\vec{u}}\mathbf{P}=\sum\vec{v}-\sum\vec{u}$.\end{lem}
\begin{proof}
In general, we have $\det\mathbf{P}\le x^{\sum\vec{v}}$, since each
increase of the $-\vec{u}$ column degree of some column of $\mathbf{P}$
also increases the order of at least one row by one. Therefore, to
show the equality holds in our special case here, we just need to
show that $\det\mathbf{P}\nless x^{\sum\vec{v}}$. If $\mathbf{A}=\left[I,0\right]$,
then it is not difficult to see that the lemma is true, since $\diag\left(\left[x^{\vec{v}},I\right]\right)$
is a $\left(\mathbf{A},\vec{v},-\vec{u}\right)$-basis.

Note that there is a $\left(\mathbf{A},\vec{v},-\vec{u}\right)$-basis
$\mathbf{Q}$ such that $\mathbf{A}\mathbf{Q}=\left[x^{\vec{v}},0\right]$.
\end{proof}
But the equality hold in the special case here.
\begin{lem}
Let $\mathbf{P}=\left[\mathbf{P}_{1},\mathbf{P}_{2}\right]$ be a
$\left(\mathbf{M}^{T},\vec{b}+\left[1,\dots,1\right],-\vec{s}\right)$-basis
as before. 
\end{lem}
The only component we need to consider 

To compute a unimodular completion of $\mathbf{F}$ with column degrees
bounded component-wise by $\vec{s}$ (or equivalently, $-\vec{s}$-row
degrees bounded by 0), let us first compute a $\vec{s}$-minimal kernel
basis $\mathbf{N}$ of $\mathbf{F}$. Let $\vec{b}$ be the $\vec{s}$-column
degrees of $\mathbf{N}$. Let $\bar{\mathbf{N}}=\colRev\left(\mathbf{N},\vec{s},\vec{b}\right)^{T}$.
We then compute a $\left(\bar{\mathbf{N}},-\vec{s},\vec{b}+1\right)$-basis
$\bar{\mathbf{P}}$, which would contain a complete kernel basis $\bar{\mathbf{F}}$
of $\bar{\mathbf{N}}$ since the row degrees of $\bar{\mathbf{N}}\bar{\mathbf{F}}$
are bounded by the corresponding entries of $\vec{b}$ and $\order(\bar{\mathbf{N}},\bar{\mathbf{P}})$
is greater than $\vec{b}$ component-wise . Note that the $-\vec{s}$-column
degrees of $\bar{\mathbf{P}}$ are bounded by $\left[1,\dots,1\right]$
since the $-\vec{s}$-row degrees of $\bar{\mathbf{F}}^{T}$ are bounded
by that of $\mathbf{F}$, which are bounded by 0.
\begin{lem}
Given $\mathbf{F}\in\mathbb{K}\left[x\right]^{m\times n}$ with $\cdeg\mathbf{F}\le\vec{s}$
(or equivalently, $\rdeg_{-\vec{s}}\mathbf{F}\le0$). If $\mathbf{N}$
is a $\left(\mathbf{F},\infty,\vec{s}\right)$-basis with $\cdeg_{\vec{s}}\mathbf{N}=\vec{b}$, \end{lem}
\end{comment}


We now do a series of order basis computations in order to compute
a unimodular completion of $\mathbf{F}$ based on \prettyref{lem:unimodularComputationByRows}.

Let $\vec{s}_{1}=\vec{s}$. First we compute an $\left(\hat{\mathbf{M}}_{1},\vec{\sigma}_{1},-\vec{s}_{1}\right)$-basis
$\mathbf{P}_{1}=\left[\mathbf{N}_{1},\mathbf{Q}_{1}\right]$, where
$\mathbf{N}_{1}$ is a $\left(\hat{\mathbf{M}}_{1},-\vec{s}_{1}\right)$-kernel
basis%
\begin{comment}
 with $\cdeg_{-\vec{s}_{1}}\mathbf{N}_{1}\le0$
\end{comment}
. This computation can be done using \prettyref{alg:umab} with a
cost of $O^{\sim}\left(n^{\omega}d\right)$, where $d=\xi/n$. 

Let $\tilde{\mathbf{N}}_{1}=\mathbf{N}_{1}$. Let $\vec{s}_{2}=-\cdeg_{-\vec{s}}\mathbf{N}_{1}$
and $\vec{t}_{2}=-\cdeg_{-\vec{s}}\mathbf{Q}_{1}$. %
\begin{comment}
Note that $-\vec{s}_{1}\le-[\vec{s}_{2},\vec{t}_{2}]\le\left[0,\dots,0,1,\dots1\right]$
component-wise, since $\mathbf{P}_{1}$ has lower order than any $\left(\mathbf{M}^{T},\vec{b}+\left[1,\dots,1\right],-\vec{s}\right)$-basis
$\mathbf{P}$ hence generates $\mathbf{P}$. Therefore, $\cdeg_{-\vec{s}}\mathbf{P}_{1}\le\cdeg_{-\vec{s}}\mathbf{P}\le\left[0,\dots,0,1,\dots1\right]$. 
\end{comment}
{} We then compute an $\left(\hat{\mathbf{M}}_{2}\tilde{\mathbf{N}}_{1},\vec{\sigma}_{2},-\vec{s}_{2}\right)$-basis
$\mathbf{P}_{2}=\left[\mathbf{N}_{2},\mathbf{Q}_{2}\right]$ with
$\vec{s}_{3}=-\cdeg_{-\vec{s}_{2}}\mathbf{N}_{2}$ and $\vec{t}_{3}=-\cdeg_{-\vec{s}_{2}}\mathbf{Q}_{2}$.
Let $\tilde{\mathbf{N}}_{2}=\tilde{\mathbf{N}}_{1}\mathbf{N}_{2}$
 %
\begin{comment}
Let $\mathbf{R}_{1}=\left[\mathbf{N}_{1}\mathbf{Q}_{2},\mathbf{Q}_{1}\right]$
and $\mathbf{R}_{1}^{r}=\colRev\left(\mathbf{R}_{1},-\vec{s},\cdeg_{-\vec{s}}\mathbf{R}_{1}\right)$.
Then from \prettyref{lem:unimodularComputationByRows} we know $\left[\mathbf{F}^{T},\mathbf{R}_{1}^{r}\right]$
is a unimodular matrix.
\end{comment}


Continue this process, at step $i$, we compute an $\left(\hat{\mathbf{M}}_{i}\tilde{\mathbf{N}}_{i-1},\vec{\sigma}_{i},-\vec{s}_{i}\right)$-basis
$\mathbf{P}_{i}=\left[\mathbf{N}_{i},\mathbf{Q}_{i}\right]$. Let
$\tilde{\mathbf{N}}_{i}=\prod_{j=1}^{i}\mathbf{N}_{i}=\tilde{\mathbf{N}}_{i-1}\mathbf{N}_{i}$.
Note that $\tilde{\mathbf{N}}_{\log k}$ is a $\left(\mathbf{M},-\vec{s}\right)$-kernel
basis. Let $\mathbf{R}=\left[\mathbf{Q}_{1},\tilde{\mathbf{N}}_{1}\mathbf{Q}_{2},\dots,\tilde{\mathbf{N}}_{\log k-2}\mathbf{Q}_{\log k-1},\tilde{\mathbf{N}}_{\log k-1}\mathbf{Q}_{\log k}\right]$,
and $\mathbf{R}^{r}=\colRev\left(\mathbf{R},-\vec{s},\cdeg_{-\vec{s}}\mathbf{R}\right)$,
then from \prettyref{lem:unimodularComputationByRows} we can conclude
that $\left[\mathbf{F}^{T},\mathbf{R}^{r}\right]$ is a unimodular
matrix. 

\begin{comment}
Note that rank sensitive computation for order basis is not as natural,
since for order basis we work with power series in general, whose
rank may never be truly computed. In addition, the computed basis
does not correspond to the rank. For these reasons, we do not pursue
rank sensitive computations for order basis. 
\end{comment}
\begin{algorithm}[t]
\caption{$\unimodularCompletion(\mathbf{F})$}
\label{alg:unimodularCompletion}

\begin{algorithmic}[1]
\REQUIRE{$\mathbf{F}\in\mathbb{K}\left[x\right]^{m\times n}$ with full row
rank; $\vec{s}$ is initially set to the column degrees of $\mathbf{F}$.
It keeps track of the degrees.}

\ENSURE{$\mathbf{G}\in\mathbb{K}\left[x\right]^{\left(n-m\right)\times n}$
such that $\begin{bmatrix}\mathbf{F}\\
\mathbf{G}
\end{bmatrix}$ is unimodular.}

\STATE{$\vec{s}:=\cdeg\mathbf{F}$;}

\STATE{$\mathbf{F}^{r}:=\rowRev\left(\mathbf{F},-\vec{s},0\right)$;}

\STATE{$\mathbf{M}:=\mnb(\mathbf{F}^{r},\vec{s})$; $\vec{b}:=\cdeg_{\vec{s}}\mathbf{M}$;
$k:=n-m;$}

\STATE{\textbf{$\left[\mathbf{M}_{1}^{T},\mathbf{M}_{2}^{T},\cdots,\mathbf{M}_{\log k-1}^{T},\mathbf{M}_{\log k}^{T}\right]:=\mathbf{M}$},
with $\mathbf{M}_{\log k},\mathbf{M}_{\log k-1},\cdots,\mathbf{M}_{2},\mathbf{M}_{1}$
having $\vec{s}$-row degrees in the range $\left[0,2\xi/k\right],(2\xi/k,4\xi/k],...,(\xi/4,\xi/2],(\xi/2,\xi].$\textbf{ }}

\FOR{$i$ \textbf{from $1$ to $\log k$ }} 

\forbody{\STATE{$\vec{\sigma}_{i}:=\left[\xi/2^{i-1}+1,\dots,\xi/2^{i-1}+1\right]$,
with the number of entries matches the row dimension of $\mathbf{M}_{i};$}}

\STATE{$\vec{\sigma}:=\left[\vec{\sigma}_{\log k},\vec{\sigma}_{\log k-1},\dots,\vec{\sigma}_{1}\right]$;}

\STATE{$\hat{\mathbf{M}}:=x^{\vec{\sigma}-\vec{b}-1}\mathbf{M};$}

\STATE{$\mathbf{N}_{0}:=I_{n}$; $\tilde{\mathbf{N}}_{0}:=I_{n};$}

\FOR{$i$ \textbf{from $1$ to $\log k$ }} 

\forbody{\STATE{$\vec{s}_{i}:=-\cdeg_{-\vec{s}}\mathbf{N}_{i-1};$ (note $\vec{s}_{1}=\vec{s}$)}

\STATE{$\mathbf{P}_{i}:=\umab\left(\hat{\mathbf{M}}_{i}\tilde{\mathbf{N}}_{i-1},\vec{\sigma}_{i},-\vec{s}_{i}\right)$; }

\STATE{$\left[\mathbf{N}_{i},\mathbf{Q}_{i}\right]:=\mathbf{P}_{i}$, where
$\mathbf{N}_{i}$ is a $\left(\hat{\mathbf{M}}_{i},-\vec{s}_{i}\right)$-nullspace
basis;}

\STATE{$\tilde{\mathbf{N}_{i}}:=\tilde{\mathbf{N}}_{i-1}\cdot\mathbf{N}_{i};$ }

\STATE{$\mathbf{R}:=\left[\mathbf{R},\tilde{\mathbf{N}}_{i-1}\mathbf{Q}_{i}\right]$;}}

\STATE{$\mathbf{R}^{r}:=\colRev\left(\mathbf{R},-\vec{s},\cdeg_{-\vec{s}}\mathbf{R}\right);$}

\RETURN $\left(\mathbf{R}^{r}\right)^{T}$ 
\end{algorithmic}
\end{algorithm}



We still need to check the cost of the multiplications $\hat{\mathbf{M}}_{i}\tilde{\mathbf{N}}_{i-1}$,
$\tilde{\mathbf{N}}_{i-1}\mathbf{N}_{i}$, and $\tilde{\mathbf{N}}_{i-1}\mathbf{Q}_{i}$. 
\begin{lem}
The multiplications $\hat{\mathbf{M}}_{i}\tilde{\mathbf{N}}_{i-1}$
can be done with a cost of $O^{\sim}\left(n^{\omega}d\right)$.\end{lem}
\begin{proof}
The dimension of $\hat{\mathbf{M}}_{i}$ is bounded by $2^{i-1}\times n$
and $\sum\rdeg_{\vec{s}}\hat{\mathbf{M}}_{i}\le2^{i-1}\cdot\xi/2^{i-1}=\xi$.
We also have $\cdeg_{-\vec{s}}\tilde{\mathbf{N}}_{i-1}\le0$, or equivalently,
$\rdeg\tilde{\mathbf{N}}_{i-1}\le\vec{s}$. We can now use \prettyref{thm:multiplyUnbalancedMatrices}
to multiply $\tilde{\mathbf{N}}_{i-1}^{T}$ and $\hat{\mathbf{M}}_{i}^{T}$
with a cost of $O^{\sim}\left(n^{\omega-1}\xi\right)=O^{\sim}\left(n^{\omega}d\right)$.\end{proof}
\begin{lem}
The multiplication $\tilde{\mathbf{N}}_{i-1}\mathbf{N}_{i}$ can be
done with a cost of $O^{\sim}\left(n^{\omega}d\right)$.\end{lem}
\begin{proof}
We know $\cdeg_{-\vec{s}}\tilde{\mathbf{N}}_{i-1}=-\vec{s}_{i}$,
and $\cdeg_{-\vec{s}_{i}}\mathbf{N}_{i}=-\vec{s}_{i+1}\le0.$ In other
words, $\rdeg\mathbf{N}_{i}\le\vec{s}_{i}$, and $\rdeg_{\vec{s}_{i}}\tilde{\mathbf{N}}_{i-1}\le\vec{s}$,
hence we can again use \prettyref{thm:multiplyUnbalancedMatrices}
to multiply $\mathbf{N}_{i}^{T}$ and $\tilde{\mathbf{N}}_{i-1}^{T}$
with a cost of $O^{\sim}\left(n^{\omega}d\right)$.\end{proof}
\begin{lem}
The multiplication $\tilde{\mathbf{N}}_{i-1}\mathbf{Q}_{i}$ can be
done with a cost of $O^{\sim}\left(n^{\omega}d\right)$.\end{lem}
\begin{proof}
We know $\cdeg_{-\vec{s}_{i}}\mathbf{Q}_{i}\le\max\cdeg_{\vec{s}}\mathbf{P}=1$,
or equivalently, $\rdeg\mathbf{Q}_{i}\le\vec{s}_{i}+\left[1,\dots,1\right]$.
But we also know that this $\mathbf{Q}_{i}$ from the order basis
computation has a factor $xI$. Therefore, $\rdeg\left(\mathbf{Q}_{i}/x\right)\le\vec{s}_{i}$.
In addition, $\rdeg_{\vec{s}_{i}}\tilde{\mathbf{N}}_{i-1}\le\vec{s}$
as before. So we can again use \prettyref{thm:multiplyUnbalancedMatrices}
to multiply $\mathbf{Q}_{i}^{T}$ and $\tilde{\mathbf{N}}_{i-1}^{T}$
with a cost of $O^{\sim}\left(n^{\omega-1}\xi\right)$.\end{proof}
\begin{thm}
A unimodular completion of $\mathbf{F}$ can be computed with a cost
of $O^{\sim}\left(n^{\omega-1}\xi\right)$ field operations.\end{thm}




\chapter{\label{chap:determinant}Diagonal Entries of Hermite Normal Form
and Determinant}

In this Chapter, we consider the problem of computing a determinant
of a nonsingular input matrix $\mathbf{F}\in\mathbb{K}\left[x\right]^{n\times n}$
with column degrees bounded by a shift $\vec{s}$. The computation
is done by using the column basis and kernel basis computation to
compute the diagonal entries of the Hermite form of $\mathbf{F}$,
and then multiply these diagonal entries.

Consider unimodularly transforming $\mathbf{F}$ to 
\begin{equation}
\mathbf{F}\mathbf{U}=\mathbf{G}=\begin{bmatrix}\mathbf{G}_{1} & 0\\
* & \mathbf{G}_{2}
\end{bmatrix},\label{eq:step1HermiteDiagonal}
\end{equation}


 After this unimodular transformation, which eliminated the top
right block of $\mathbf{G}$ , the matrix is now closer to the Hermite
normal form of $\mathbf{F}$. This procedure can then be applied recursively
to $\mathbf{G}_{1}$ and $\mathbf{G}_{2}$, until the matrices reaching
dimension 1, which then gives the diagonal entries of the Hermite
normal form of $\mathbf{F}$.

Although this procedure correctly computes the diagonal entries of
the Hermite normal form of $\mathbf{F}$, a major problem is that
the degree of the unimodular $\mathbf{U}$ can be too large for $\mathbf{U}$
to be efficiently computed. However, with the tools we have developed
in the earlier chapters, we can efficiently compute $\mathbf{G}_{1}$
and $\mathbf{G}_{2}$ without computing $\mathbf{U}$.

If we separate $\mathbf{F}$ to $\mathbf{F}=\begin{bmatrix}\mathbf{F}_{U}\\
\mathbf{F}_{D}
\end{bmatrix}$, each has full-rank as $\mathbf{F}$ is assumed to be nonsingular,
and also separate $\mathbf{U}$ to $\mathbf{U}=\begin{bmatrix}\mathbf{U}_{L} & \mathbf{U}_{R}\end{bmatrix}$,
where the column dimension of $\mathbf{U}_{L}$ matches the row dimension
of $\mathbf{F}_{U}$, then 
\[
\mathbf{F}\mathbf{U}=\begin{bmatrix}\mathbf{F}_{U}\\
\mathbf{F}_{D}
\end{bmatrix}\begin{bmatrix}\mathbf{U}_{L} & \mathbf{U}_{R}\end{bmatrix}=\mathbf{G}=\begin{bmatrix}\mathbf{G}_{1} & 0\\
* & \mathbf{G}_{2}
\end{bmatrix}.
\]
 Notice that the matrix $\mathbf{G}_{1}$ is nonsingular and is therefore
just a column basis of $\mathbf{F}_{U}$, and can be efficiently computed
using \prettyref{alg:colBasis}. To compute $\mathbf{G}_{2}=\mathbf{F}_{D}\mathbf{U}_{R}$,
notice that the matrix $\mathbf{U}_{R}$ is a right kernel basis\textbf{
}of $\mathbf{F}$, which makes the top right block of $\mathbf{G}$
zero. As we have seen from \prettyref{lem:unimodular_kernel_columnBasis},
the kernel basis $\mathbf{U}_{R}$ can be replaced by any other kernel
basis of $\mathbf{F}$ to give another unimodular matrix that also
works. 
\begin{lem}
\label{lem:oneStepHermiteDiagonal}Given a polynomial matrix $\mathbf{F}=\begin{bmatrix}\mathbf{F}_{U}\\
\mathbf{F}_{D}
\end{bmatrix}$. If $\mathbf{G}_{1}$ is a column basis of $\mathbf{F}_{U}$ and
$\mathbf{N}$ is a kernel basis of $\mathbf{F}_{U}$, then there is
a unimodular matrix $\mathbf{U}=\left[*,\mathbf{N}\right]$ such that
\[
\mathbf{F}\mathbf{U}=\begin{bmatrix}\mathbf{G}_{1}\\
* & \mathbf{G}_{2}
\end{bmatrix},
\]
 where $\mathbf{G}_{2}=\mathbf{F}_{D}\mathbf{N}$.  If $\mathbf{F}$
is square nonsingular, then $\mathbf{G}_{1}$ and $\mathbf{G}_{2}$
are also square nonsingular.
\end{lem}
Note that we do not compute the blocks represented by the symbol $*$,
which may have very large degrees and cannot be computed efficiently.

\prettyref{lem:oneStepHermiteDiagonal} allows us to compute $\mathbf{G}_{1}$
and $\mathbf{G}_{2}$ independently without computing the unimodular
matrix. $\mathbf{G}_{1}$ can be computed using the method from \prettyref{chap:Matrix-GCD},
while the kernel basis computation from \prettyref{chap:NullspaceBasis}
can be used to compute a kernel basis $\mathbf{N}_{1}$ of $\mathbf{F}_{u}$,
which can then be used to compute $\mathbf{G}_{2}=\mathbf{F}_{d}\mathbf{N}_{1}$.

After $\mathbf{G}_{1}$ and $\mathbf{G}_{2}$ are computed, we can
repeat the same process on each of these two matrices, which now have
lower dimensions, until the dimension becomes one. This procedure
of computing the diagonal entries gives \prettyref{alg:hermiteDiagonal}

\begin{algorithm}[t]
\caption{$\hermiteDiagonal(\mathbf{F})$}
\label{alg:hermiteDiagonal}

\begin{algorithmic}
[1]\REQUIRE{$\mathbf{F}\in\mathbb{K}\left[x\right]^{n\times n}$ is nonsingular.
}

\ENSURE{$\mathbf{d}\in\mathbb{K}\left[x\right]^{n}$ a list of diagonal entries
of the Hermite normal form of $\mathbf{F}$.}

\STATE{$\begin{bmatrix}\mathbf{F}_{U}\\
\mathbf{F}_{D}
\end{bmatrix}:=\mathbf{F}$, with $\mathbf{F}_{U}$ consists of the top $\left\lceil n/2\right\rceil $
rows of $\mathbf{F}$;}



\STATE{\textbf{if }$n=1$ \textbf{then} \textbf{return} $\mathbf{F}$; \textbf{endif};}

\STATE{$\mathbf{G}_{1}:=\colBasis(\mathbf{F}_{U})$;}



\STATE{$\mathbf{N}:=\mnb(\mathbf{F}_{U},\cdeg\mathbf{F})$;}

\STATE{$\mathbf{G}_{2}:=\mathbf{F}_{D}\mathbf{N}$;}

\STATE{$\mathbf{d}_{1}:=\hermiteDiagonal(\mathbf{G}_{1});$$\mathbf{d}_{2}:=\hermiteDiagonal(\mathbf{G}_{2});$}

\STATE{\textbf{return} $\left[\mathbf{d}_{1},\mathbf{d}_{2}\right]$;}
\end{algorithmic}
\end{algorithm}




\section{Computational Cost}

Let us look at the computational cost of \prettyref{alg:hermiteDiagonal}. 
\begin{thm}
\prettyref{alg:hermiteDiagonal} costs $O^{\sim}\left(n^{\omega-1}\xi\right)$
field operations to compute the diagonal entries for the Hermite normal
form of a nonsingular matrix $\mathbf{F}\in\mathbb{K}\left[x\right]^{n\times n}$,
where $\xi$ is the sum of the column degrees \end{thm}
\begin{proof}
The three main operations are computing a column basis of $\mathbf{F}_{U}$,
computing a kernel basis $\mathbf{N}$ of $\mathbf{F}_{U}$, and the
matrix multiplication $\mathbf{F}_{d}\mathbf{N}$.

For the column basis computation, by \prettyref{thm:columnBasisCost1}
we know that a column basis $\mathbf{G}_{1}$ of $\mathbf{F}_{U}$
can be computed with a cost of $O^{\sim}\left(n^{\omega-1}\xi\right).$
By \prettyref{lem:colBasisDegreeBoundByInputDegrees} the column degrees
of the computed column basis $\mathbf{G}_{1}$ are also bounded by
the original column degrees $\vec{s}$.

For the kernel basis computation, it also costs $O^{\sim}\left(n^{\omega-1}\xi\right)$
to compute a $\vec{s}$-minimal kernel basis $\mathbf{N}$ of $\mathbf{F}_{U}$
from \prettyref{thm:costLowColDimension}. The sum of the $\vec{s}$-column
degrees of the output kernel basis $\mathbf{N}$ is bounded by $\xi$
by \prettyref{thm:boundOfSumOfShiftedDegreesOfKernelBasis}.

Finally for the matrix multiplication $\mathbf{F}_{d}\mathbf{N}$,
since the sum of the column degrees of $\mathbf{F}_{d}$ and the sum
of the $\vec{s}$-column degrees of $\mathbf{N}$ are both bounded
by $\xi$, \prettyref{thm:multiplyUnbalancedMatrices} applies and
the multiplication can be done with a cost of $O^{\sim}\left(n^{\omega-1}\xi\right)$.

Now if we let the cost of \prettyref{alg:hermiteDiagonal} be $g(n,\xi)$
for a input matrix of dimension $n$ with $\xi$ bounding the sum
of its column degrees, then we have the recurrence relation
\[
g(n,\xi)\in O^{\sim}(n^{\omega-1}\xi)+g(\left\lceil n/2\right\rceil ,\xi)+g(\left\lfloor n/2\right\rfloor ,\xi),
\]
 which is the same as in \prettyref{thm:inverseCost} for computing
the matrix inverse. Therefore, we also get $g(n,\xi)=O^{\sim}(n^{\omega-1}\xi)$
as in the inverse computation.\end{proof}
\begin{cor}
The determinant of a nonsingular matrix $\mathbf{F}\in\mathbb{K}\left[x\right]^{n\times n}$
can be computed with a cost of $O^{\sim}(n^{\omega-1}\xi)$ field
operations, where $\xi$ is the minimum of the sum of the column degrees
and the sum of the row degrees of the input matrix.\end{cor}
\begin{proof}
Just use \prettyref{alg:hermiteDiagonal} to compute the diagonal
entries of the Hermite normal form of either $\mathbf{F}$ or $\mathbf{F}^{T}$,
and then multiply the diagonal entries.\end{proof}




\chapter{\label{chap:hermite} Hermite Normal Form}

In \prettyref{chap:determinant}, we have shown how the diagonal entries
of the Hermite normal form of a nonsingular input matrix $\mathbf{F}\in\mathbb{K}\left[x\right]^{n\times n}$
can be computed efficiently. In this Chapter, we consider the problem
of computing the complete Hermite normal form $\mathbf{H}$ of $\mathbf{F}$.
\citet{GS2011,G2011} gave a randomized Las Vegas algorithm that costs
$O^{\sim}\left(n^{\omega}d\right)$ to compute the Hermite normal
form. We make use of some of their ideas and follow a similar path.
But we do not use Smith normal form and our algorithm is deterministic.

For simplicity, we assume $\mathbf{F}$ is already column reduced
and has degrees $\vec{d}=\left[d_{1},\dots,d_{n}\right]$ and let
$d=\max\vec{d}$. Let $\vec{s}=\left[s_{1},\dots,s_{n}\right]$ be
the degrees of the diagonal entries of the Hermite form $\mathbf{H}$.
Then if $\vec{u}=\left[\max\vec{s},\dots,\max\vec{s}\right]$ with
$n$ entries, we can obtain the Hermite normal form from a $\left[-\vec{u},-\vec{s}\right]$-minimal
kernel basis of $\left[\mathbf{F},I\right]$.
\begin{lem}
If $\begin{bmatrix}\mathbf{V}\\
\mathbf{G}
\end{bmatrix}$ is a $\left[-\vec{u},-\vec{s}\right]$-minimal kernel basis of $\left[\mathbf{F},-I\right]$,
where each block is $n\times n$ square, then $\mathbf{G}$ is unimodularly
equivalent with the Hermite normal form $\mathbf{H}$ of $\mathbf{F}$
and has row degrees $\vec{s}$, the same as the row degrees of $\mathbf{H}$.\end{lem}
\begin{proof}
Notice that the unimodular matrix $\mathbf{U}$ satisfying $\mathbf{F}\mathbf{U}=\mathbf{H}$
has $\vec{d}$-column degrees bounded by $\max\vec{s}$ from the predictable-degree
property \prettyref{lem:predictableDegree}, which means $\mathbf{U}$
has degree bounded by $\max\vec{s}$, or equivalently, $\cdeg_{-\vec{u}}\mathbf{U}\le0$,
hence $\cdeg_{\left[-\vec{u},-\vec{s}\right]}\begin{bmatrix}\mathbf{U}\\
\mathbf{H}
\end{bmatrix}=\cdeg_{-\vec{s}}\mathbf{H}=0$, making $\begin{bmatrix}\mathbf{U}\\
\mathbf{H}
\end{bmatrix}$ $\left[-\vec{u},-\vec{s}\right]$-column reduced and a $\left[-\vec{u},-\vec{s}\right]$-minimal
kernel basis of $\left[\mathbf{F},-I\right]$. If we compute a $\left[-\vec{u},-\vec{s}\right]$-minimal
kernel basis $\begin{bmatrix}\mathbf{V}\\
\mathbf{G}
\end{bmatrix}$ of $\left[\mathbf{F},-I\right]$, we know that $\begin{bmatrix}\mathbf{V}\\
\mathbf{G}
\end{bmatrix}$ is unimodularly equivalent to $\begin{bmatrix}\mathbf{U}\\
\mathbf{H}
\end{bmatrix}$, implying that $\mathbf{V}$ is also unimodular. The minimality also
ensures that $\cdeg_{\left[-\vec{u},-\vec{s}\right]}\begin{bmatrix}\mathbf{V}\\
\mathbf{G}
\end{bmatrix}=\cdeg_{\left[-\vec{u},-\vec{s}\right]}\begin{bmatrix}\mathbf{U}\\
\mathbf{H}
\end{bmatrix}=0$, implying $\cdeg_{-\vec{s}}\mathbf{G}\le0=\cdeg_{-\vec{s}}\mathbf{H}$.
But the minimality of $\mathbf{H}$ ensures that $\cdeg_{-\vec{s}}\mathbf{G}=\cdeg_{-\vec{s}}\mathbf{H}=0$,
or equivalently, $\rdeg\mathbf{G}=\rdeg\mathbf{H}=\vec{s}$.
\end{proof}
Knowing that $\mathbf{G}$ has the same row degrees as $\mathbf{H}$
and is unimodularly equivalent with $\mathbf{H}$, the Hermite form
$\mathbf{H}$ can then be obtained from $\mathbf{G}$ using Lemma
8 from \citep{GS2011}, restated as follows:
\begin{lem}
\label{lem:recoverH}If the Hermite normal form $\mathbf{H}$ of $\mathbf{F}$
is a column basis of a matrix $\mathbf{A}\in\mathbb{K}\left[x\right]^{n\times k}$,
and has the same row degrees as $\mathbf{A}$, then the matrix $U\in\mathbb{K}^{n\times n}$
putting $\lcoeff\left(x^{-\vec{s}}\mathbf{A}\right)U$ in reduced
column echelon form also gives the Hermite normal form $\mathbf{H}$
as the principal $n\times n$ submatrix of $\mathbf{A}U$.\end{lem}
\begin{proof}
This follows from the fact that $\mathbf{H}$ and $\mathbf{A}$ all
have uniform $-\vec{s}$ column degrees 0, which allows their relationship
to be completely determined by $\lcoeff\left(x^{-\vec{s}}\mathbf{A}\right)$
and the diagonal matrix $\lcoeff\left(x^{-\vec{s}}\mathbf{H}\right)$
alone.
\end{proof}
Although the Hermite normal form $\mathbf{H}$ of $\mathbf{F}$ can
be computed from a $\left[-\vec{u},-\vec{s}\right]$-minimal kernel
basis of $\left[\mathbf{F},I\right]$, a major problem here is that
$\max\vec{s}$ can be very large. So the existing algorithms would
be inefficient if applied directly. 

However, since we know the row degrees of $\mathbf{H}$, we can expand
each of the high degree rows of $\mathbf{H}$ to multiple rows with
lower degrees, as done in \citep{GS2011,G2011} and also in the computation
of order basis with unbalanced shift from \prettyref{chap:Unbalanced-Shift},
which then allows to compute an alternative matrix $\mathbf{H}'$
with lower degrees but a higher row dimension that is still in $O(n)$,
such that $\mathbf{H}'$ can be easily transformed to $\mathbf{H}$.
Our task here is in fact easier than in \prettyref{chap:Unbalanced-Shift}
as we already know the exact row degrees of $\mathbf{H}$.

For each entry $s_{i}$ of the shift $\vec{s}$, let $q_{i}$ and
$r_{i}$ be the quotient and remainder of $s_{i}$ divided by $d$.
Then, we expand the $i$th column $e_{i}$ of the identity matrix
$I$ in $[\mathbf{F},I]$ and shift $s_{i}$ to
\[
\tilde{\mathbf{E}}^{(i)}=\left[e_{i},x^{s_{i}-q_{i}d}e_{i},\dots,x^{s_{i}-d}e_{i}\right]\mbox{ and }\tilde{s}_{i}=\left[r_{i},d,\dots,d,\right],
\]
 %
\begin{comment}
\[
\tilde{\mathbf{E}}^{(i)}=\left[e_{i},x^{d}e_{i},\dots,x^{q_{i}d}e_{i},x^{q_{i}d+r_{i}}e_{i}\right]\mbox{ and }\tilde{s}_{i}=\left[d,\dots,d,r_{i}\right],
\]
\end{comment}
{} where $\tilde{s}_{i}$ has with $q_{i}+1$ entries in each case.
For the transformed problem, the shift $\vec{s}$ becomes $\bar{s}=[\tilde{s}_{1},\dots,\tilde{s}_{n}]\in\mathbb{Z}_{\le0}^{\bar{n}}$,
and the identity matrix becomes
\begin{eqnarray*}
\mathbf{E} & = & [\tilde{\mathbf{E}},\dots,\tilde{\mathbf{E}}^{\left(n\right)}]\\
 & = & \left[\begin{array}{ccccc|ccc|ccccccc}
1 & x^{s_{1}-q_{1}d} & \cdots & x^{s_{1}-2d} & x^{s_{1}-d} &  &  & \\
\hline  &  &  &  &  & \ddots &  & \\
 &  &  &  &  &  & \  & \ddots\\
\hline  &  &  &  &  &  &  &  & 1 & x^{s_{n}-q_{n}d} & \cdots & x^{s_{n}-2d} & x^{s_{n}-d}
\end{array}\right]_{n\times\bar{n}},
\end{eqnarray*}
 with the new column dimension $\bar{n}$ satisfying 
\[
\bar{n}=n+\sum_{i=1}^{n}q_{i}\le n+\sum_{i=1}^{n}s_{i}/d\le n+nd/d=2n.
\]


We can now recover $\mathbf{H}$ from a $\left[-\vec{u},-\bar{s}\right]$-minimal
kernel basis of $\left[\mathbf{F},-\mathbf{E}\right]$.
\begin{lem}
\label{lem:expandH}Let $\mathbf{B}=\begin{bmatrix}\mathbf{V}'\\
\mathbf{G}'
\end{bmatrix}$ be a $\left(\left[\mathbf{F},-\mathbf{E}\right],\left[-\vec{u},-\bar{s}\right]\right)$-kernel
basis, where $\mathbf{G}'$ has dimension $\bar{n}\times\bar{n}$.
Let $\bar{\mathbf{B}}_{0}$ be the matrix consisting of the columns
of $\mathbf{G}'$ whose $-\bar{s}$-column degrees are bounded by
0. Then $\mathbf{H}$ is a column basis of $\mathbf{E}\bar{\mathbf{B}}_{0}$
and the nonzero columns of $\mathbf{E}\bar{\mathbf{B}}_{0}$ have
$-\vec{s}$-column degrees 0, allowing us to recover $\mathbf{H}$
from $\mathbf{E}\bar{\mathbf{B}}_{0}$ using \prettyref{lem:recoverH}.\end{lem}
\begin{proof}
Note that we can construct a $\left(\left[\mathbf{F},-\mathbf{E}\right],\left[-\vec{u},-\bar{s}\right]\right)$-kernel
basis $\mathbf{A}=\begin{bmatrix}\mathbf{U} & 0\\
\mathbf{H}_{\mathbf{E}} & \mathbf{N}_{\mathbf{E}}
\end{bmatrix}$, where $\mathbf{U}$ is the unimodular matrix satisfying $\mathbf{F}\mathbf{U}=\mathbf{H}$,
$\mathbf{H}_{\mathbf{E}}$ is the matrix $\mathbf{H}$ expanded according
to $\mathbf{E}$, so that $\mathbf{H}_{\mathbf{E}}$ has row degrees
$\bar{s}$ and $\mathbf{H}_{\mathbf{E}}\mathbf{E}=\mathbf{H}$, and
$\mathbf{N}_{\mathbf{E}}$ is a $\left(\mathbf{E},-\bar{s}\right)$-kernel
basis. Let $\mathbf{B}=\begin{bmatrix}\mathbf{V}'\\
\mathbf{G}'
\end{bmatrix}$ be another $\left(\left[\mathbf{F},-\mathbf{E}\right],\left[-\vec{u},-\bar{s}\right]\right)$-kernel
basis. Then the matrix $\mathbf{A}_{0}$ and $\mathbf{B}_{0}$ consists
of the columns from $\mathbf{A}$ and $\mathbf{B}$, respectively,
whose $\left[-\vec{u},-\bar{s}\right]$-column degrees are bounded
by $0$, are unimodularly equivalent, that is, $\mathbf{A}_{0}\mathbf{U}_{0}=\mathbf{B}_{0}$
for some unimodular matrix $\mathbf{U}_{0}$. As a result, the matrices
$\bar{\mathbf{A}}_{0}$ and $\bar{\mathbf{B}}_{0}$ consists of the
bottom $\bar{n}$ rows of $\mathbf{A}_{0}$ and $\mathbf{B}_{0}$
respectively, also satisfies $\bar{\mathbf{A}}_{0}\mathbf{U}_{0}=\bar{\mathbf{B}}_{0}$.
Therefore, we also get $\mathbf{E}\bar{\mathbf{A}}_{0}\mathbf{U}_{0}=\mathbf{E}\bar{\mathbf{B}}_{0}$,
with $\cdeg_{-\vec{s}}\mathbf{E}\bar{\mathbf{A}}_{0}\le0$ and $\cdeg_{-\vec{s}}\mathbf{E}\bar{\mathbf{B}}_{0}\le0$,
since $\cdeg_{-\bar{s}}\bar{\mathbf{A}}_{0}\le0$ and $\cdeg_{-\bar{s}}\bar{\mathbf{B}}_{0}\le0$.
Let $\bar{\mathbf{A}}_{0}=\left[\mathbf{H}_{\mathbf{E}},\mathbf{N}'\right]$,
where $\mathbf{N}'$ consists of the columns of $\mathbf{N}_{E}$
with $-\bar{s}$-column degrees bounded by 0. Then $\mathbf{E}\bar{\mathbf{A}}_{0}\mathbf{U}_{0}=\mathbf{E}\left[\mathbf{H}_{\mathbf{E}},\mathbf{N}'\right]\mathbf{U}_{0}=\left[\mathbf{H},0\right]\mathbf{U}_{0}=\mathbf{E}\bar{\mathbf{B}}_{0}$.
Now since $\mathbf{H}$ is $-\vec{s}$-column reduced and has $-\vec{s}$-column
degrees $0$, the nonzero columns of $\mathbf{E}\bar{\mathbf{B}}_{0}$
must have $-\vec{s}$-column degrees no less than $0$, hence their
$-\vec{s}$-column degrees are equal to 0.
\end{proof}
The problem of computing a $\left(\left[\mathbf{F},I\right],\left[-\vec{u},-\vec{s}\right]\right)$-kernel
basis of is now reduced to computing a $\left(\left[\mathbf{F},\mathbf{E}\right],\left[-\vec{u},-\bar{s}\right]\right)$-kernel
basis. However, the degree of $\mathbf{E}$ and the shift $\vec{u}$
are still too big to make the computation efficient. To lower these,
we can reduce $\mathbf{E}$ against $\mathbf{F}$ in a way similar
to \citep{GS2011,G2011}, where the authors used the Smith normal
form to reduce the degrees. 
\begin{lem}
\label{lem:reduceToRemainder}Let $\mathbf{R}=\mathbf{E}-\mathbf{F}\mathbf{Q}$
for some polynomial matrix $\mathbf{Q}$ such that $\mathbf{R}$ has
degree less than $d$ and $\bar{u}=\left[2d,\dots,2d\right]\in\mathbb{Z}^{n}$.
Let $\mathbf{D}=\begin{bmatrix}\bar{\mathbf{V}}\\
\mathbf{G}'
\end{bmatrix}$ be a $\left(\left[\mathbf{F},-\mathbf{R}\right],\left[-\bar{u},-\bar{s}\right]\right)$-kernel
basis, where the block $\mathbf{G}'$ has dimension $n\times n$,
and $\bar{\mathbf{D}}_{0}$ be the matrix consisting of the columns
of $\mathbf{G}'$ whose $-\bar{s}$-column degrees are bounded by
0. Then $\mathbf{H}$ is a column basis of $\mathbf{E}\bar{\mathbf{D}}_{0}$
and the nonzero columns of $\mathbf{E}\bar{\mathbf{D}}_{0}$ have
$-\vec{s}$-column degrees 0, allowing us to recover $\mathbf{H}$
from $\mathbf{E}\bar{\mathbf{D}}_{0}$ using \prettyref{lem:recoverH}.\end{lem}
\begin{proof}
First note that from the kernel basis $\mathbf{A}=\begin{bmatrix}\mathbf{U} & 0\\
\mathbf{H}_{\mathbf{E}} & \mathbf{N}_{\mathbf{E}}
\end{bmatrix}$ of $\left[\mathbf{F},-\mathbf{E}\right]$ constructed in \prettyref{lem:expandH},
we can construct a kernel basis 
\[
\mathbf{C}=\begin{bmatrix}I & -\mathbf{Q}\\
0 & I
\end{bmatrix}\mathbf{A}=\begin{bmatrix}\mathbf{U}-\mathbf{Q}\mathbf{H}_{\mathbf{E}} & -\mathbf{Q}\mathbf{N}_{E}\\
\mathbf{H}_{\mathbf{E}} & \mathbf{N}_{\mathbf{E}}
\end{bmatrix}
\]
 of $\left[\mathbf{F},-\mathbf{R}\right]=\left[\mathbf{F},-\mathbf{E}\right]\begin{bmatrix}I & \mathbf{Q}\\
0 & I
\end{bmatrix}$. Now if $\mathbf{D}$ is a $\left(\left[\mathbf{F},-\mathbf{R}\right],\left[-\bar{u},-\bar{s}\right]\right)$-kernel
basis, it satisfies $\mathbf{C}\mathbf{V}=\mathbf{D}$ for a unimodular
$\mathbf{V}$. Also, the matrix $\mathbf{C}_{0}$ and $\mathbf{D}_{0}$
consist of the columns from $\mathbf{C}$ and $\mathbf{D}$, respectively,
whose $\left[-\bar{u},-\bar{s}\right]$-column degrees are bounded
by $0$, satisfy $\mathbf{C}_{0}=\mathbf{D}_{0}\mathbf{V}_{0}$ for
some polynomial matrix $\mathbf{V}_{0}$. Then the matrices $\bar{\mathbf{C}}$,
$\bar{\mathbf{D}}$, $\bar{\mathbf{C}}_{0}$, $\bar{\mathbf{D}}_{0}$
consist of the bottom $\bar{n}$ rows of $\mathbf{C}$, $\mathbf{D}$,
$\mathbf{C}_{0}$, $\mathbf{D}_{0}$ respectively, satisfy $\bar{\mathbf{C}}\mathbf{V}=\bar{\mathbf{D}}$
and $\bar{\mathbf{C}}_{0}=\bar{\mathbf{D}}_{0}\mathbf{V}_{0}$. It
then follows that $\mathbf{E}\bar{\mathbf{C}}\mathbf{V}=\mathbf{E}\left[\mathbf{H}_{\mathbf{E}},\mathbf{N}_{\mathbf{E}}\right]\mathbf{V}=\left[\mathbf{H},0\right]\mathbf{V}=\mathbf{E}\bar{\mathbf{D}}$
and $\mathbf{E}\bar{\mathbf{C}}_{0}=\mathbf{E}\left[\mathbf{H}_{\mathbf{E}},\mathbf{N}'\right]=\left[\mathbf{H},0\right]=\mathbf{E}\bar{\mathbf{D}}_{0}\mathbf{V}_{0}$,
where $\mathbf{N}'$ consists of the columns of $\mathbf{N}_{E}$
with $-\bar{s}$-column degrees bounded by 0. From $\left[\mathbf{H},0\right]\mathbf{V}=\mathbf{E}\bar{\mathbf{D}}$
we know that the nonzero columns of $\mathbf{E}\bar{\mathbf{D}}$
has $-\vec{s}$-column degrees no less than $\cdeg_{-\vec{s}}\mathbf{H}=0$.
On the other hand, we know that $\cdeg_{-\vec{s}}\mathbf{E}\bar{\mathbf{D}}_{0}\le0$
since $\cdeg_{-\bar{s}}\bar{\mathbf{D}}_{0}\le0$, therefore the nonzero
columns of $\mathbf{E}\bar{\mathbf{D}}_{0}$ has $-\vec{s}$-column
degrees equal $0$. Also from $\left[\mathbf{H},0\right]\mathbf{V}=\mathbf{E}\bar{\mathbf{D}}$
and $\left[\mathbf{H},0\right]=\mathbf{E}\bar{\mathbf{D}}_{0}\mathbf{V}_{0}$
we know that $\mathbf{H}$ is a column basis of $\mathbf{E}\bar{\mathbf{D}}_{0}$.
\end{proof}
A $\left(\left[\mathbf{F},-\mathbf{R}\right],\left[-\bar{u},-\bar{s}\right]\right)$-kernel
basis from \prettyref{lem:reduceToRemainder} can now be efficiently
computed, and can then be used to recover the Hermite normal form.
A big question remaining, however, is how to efficiently compute the
remainder $\mathbf{R}$ from $\mathbf{E}$ and $\mathbf{F}$. For
this, we can use the series expansion of the inverse of 
\[
\mathbf{F}^{r}=\colRev(\mathbf{F},0,\vec{d})=\mathbf{F}(1/x)\begin{bmatrix}x^{d_{1}}\\
 & \ddots\\
 &  & x^{d_{n}}
\end{bmatrix},
\]
 as noted in the proof of Lemma 3.4 from \citep{Giorgi2003}. The
series expansion can be done using the series solution algorithm from
\citet{storjohann:2003}. Note that since $\mathbf{F}$ is assumed
to be column reduced, $\deg\det\mathbf{F}=\sum\vec{d}$ exactly, and
therefore $x$ is not a factor of $\deg\det\mathbf{F}^{r}$, which
means the series expansion of $\left(\mathbf{F}^{r}\right)^{-1}$
always exists. It also means that using $x^{d}$-adic lifting always
works, and the series solution algorithm from \citet{storjohann:2003}
becomes deterministic. 

Let us now look at how the series expansion of $\left(\mathbf{F}^{r}\right)^{-1}$
gives a remainder of $x^{k}I$ divide by $\mathbf{F}$.
\begin{lem}
\label{lem:remainder}Let \textup{the series expansion of $\left(\mathbf{F}^{r}\right)^{-1}$
be }$\bar{\mathbf{F}}=F_{0}+F_{1}x+F_{2}x^{2}+\dots$. Then for any
integer $k\ge d$, we have 
\begin{equation}
I=\mathbf{F}^{r}\left(\bar{\mathbf{F}}\mod x^{k-\vec{d}}\right)+x^{k-d}\mathbf{C},\label{eq:division}
\end{equation}
where $\bar{\mathbf{F}}\mod x^{k-\vec{d}}$ denotes the $i$th row
of $\bar{\mathbf{F}}$ $\mod$ $x^{k-d_{i}}$ for each row $i$. Then
$\mathbf{C}^{r}=\colRev\left(x^{k-d}\mathbf{C},0,k\right)$ has degree
less than $d$ and satisfies \textup{$x^{k}I=\mathbf{F}\cdot\left(*\right)+\mathbf{C}^{r}$.}\end{lem}
\begin{proof}
Since the first term of Equation \eqref{eq:division} has degree less
than $k$, the the degree of $x^{k-d}\mathbf{C}$ must be also less
than $k$, or equivalently, the degree of $\mathbf{C}$ must be less
than $d$. If we now reverse the coefficients, we get 
\[
\colRev\left(I,0,k\right)=\colRev(\mathbf{F}^{r},0,\vec{d})\cdot\colRev\left(\left(\bar{\mathbf{F}}\mod x^{k-\vec{d}}\right),\vec{d},k\right)+\colRev\left(x^{k-d}\mathbf{C},0,k\right),
\]
 that is, 
\[
x^{k}I=\mathbf{F}\cdot\left(*\right)+\mathbf{C}^{r},
\]
 which gives us $\mathbf{C}^{r}$ as a remainder of $x^{k}I$ divided
by $\mathbf{F}$, where $\mathbf{C}^{r}$ has degree less than $d$. 
\end{proof}
\prettyref{lem:remainder} shows how the series expansion $\bar{\mathbf{F}}$
can be used to compute a remainder of $x^{k}I$ divided by $\mathbf{F}$
for any $k\ge d$. Similarly, the $i$th column $\bar{\mathbf{F}}_{i}=\bar{\mathbf{F}}e_{i}$
of $\bar{\mathbf{F}}$ allows us to compute a remainder $\mathbf{r}$
of $x^{k}e_{i}$ divided by $\mathbf{F}$, with $\deg\mathbf{r}<d$.
Note that the degrees of columns correspond to $e_{i}$ are bounded
by $s_{i}$, so we need to compute the series expansion $\bar{\mathbf{F}}_{i}$
to at least order $s_{i}$. Now let us look how these series expansions
can be computed efficiently.


\begin{lem}
Computing the series expansions $\bar{\mathbf{F}}_{i}$ to order $s_{i}$
for all $i$'s where $s_{i}\ge d$ can be done with a cost of \textup{$O^{\sim}\left(n^{\omega}d\right)$
field operations.}\end{lem}
\begin{proof}
As before, we assume without loss of generality that the columns of
$\mathbf{F}$ and the corresponding entries of $\vec{s}=\left[s_{1},\dots,s_{n}\right]$
are arranged so that the entries of $\vec{s}$ are in increasing order.
We separate $\vec{s}$ to $\left\lceil \log n\right\rceil +1$ disjoint
lists  $\vec{s}_{\bar{j}^{\left(0\right)}},\vec{s}_{\bar{j}^{\left(1\right)}},\vec{s}_{\bar{j}^{\left(2\right)}},\dots,\vec{s}_{\bar{j}^{\left(\left\lceil \log n\right\rceil \right)}}$
with entries in the ranges $[0,d)$, $[d,2d)$, $[2d,4d)$, $[4d,8d)$,
...,$[2^{\left\lceil \log n\right\rceil -2}d,2^{\left\lceil \log n\right\rceil -1}d)$,
$[2^{\left\lceil \log n\right\rceil -1}d,nd]$ respectively, where
each $\bar{j}^{(i)}$ consists a list of indices of the entries of
$\vec{s}$ that belong to $\vec{s}_{\bar{j}^{\left(i\right)}}$. Note
that $\bar{j}^{\left(i\right)}$ has at most $n/2^{i-1}$ entries,
otherwise, the sum of the entries of $\vec{s}_{\bar{j}^{(i)}}$ would
exceed $\sum\vec{s}=nd$. Then we compute series expansions $\bar{\mathbf{F}}_{\bar{j}^{\left(1\right)}},\bar{\mathbf{F}}_{\bar{j}^{\left(2\right)}},\dots,\bar{\mathbf{F}}_{\bar{j}^{\left(\left\lceil \log n\right\rceil \right)}}$
separately, to order $2d,4d,\dots,2^{\left\lceil \log n\right\rceil }d/2,nd$
respectively, where again $\bar{\mathbf{F}}_{\bar{j}^{(i)}}$ consists
of the columns of $\bar{\mathbf{F}}$ that are indexed by the entries
in $\bar{j}^{\left(i\right)}$. We can use the series solution algorithm
from \citet{storjohann:2003} to do these computations. For $\bar{\mathbf{F}}{}_{\bar{j}^{\left(i\right)}}$,
there are at most $n/2^{i-1}$ columns, so computing the series expansion
to order $2^{i}d$ cost $O^{\sim}\left(n^{\omega}d\right)$. Then
doing this for $i$ from $1$ to $\left\lceil \log n\right\rceil $
costs $O^{\sim}\left(n^{\omega}d\right)$ field operations. 
\end{proof}
With the series expansions computed, we can now compute a remainder
$\mathbf{R}$ of $\mathbf{E}$ divide by $\mathbf{F}$.
\begin{lem}
A remainder $\mathbf{R}$ of $\mathbf{E}$ divide by $\mathbf{F}$,
where $\deg\mathbf{R}<d$, can be computed with a cost of $O^{\sim}\left(n^{\omega}d\right)$
field operations.\end{lem}
\begin{proof}
The remainder $\mathbf{r}$ of $x^{k}e_{i}$ divide by $\mathbf{F}$
can be obtained by 
\[
\left(e_{i}-\mathbf{F}^{r}\left(\bar{\mathbf{F}}_{i}\mod x^{k-\vec{d}}\right)\right)/x^{k-d}.
\]
Note that only the terms from $\bar{\mathbf{F}}_{i}$ with degrees
in the range $[k-2d,k)$ are needed for this computation, which means
we are just multiplying $\mathbf{F}^{r}$ with a polynomial vector
with degree bounded by $2d$. To make the multiplication more efficient,
we can compute all the remainder vectors at once. Since there at most
$n$ columns with degrees no less than $d$, the cost is just the
multiplication of matrices of dimension $n$ and degrees bounded by
$2d$, which costs $O^{\sim}\left(n^{\omega}d\right)$ field operations.
\end{proof}
With the remainder $\mathbf{R}$ computed, we can now compute a $\left(\left[\mathbf{F},-\mathbf{R}\right],\left[-\bar{u},-\bar{s}\right]\right)$-kernel
basis that can be used to recover the Hermite normal form using \prettyref{lem:reduceToRemainder}.
\begin{thm}
A Hermite normal form of $\mathbf{F}$ can be computed with a cost
of \textup{$O^{\sim}\left(n^{\omega}d\right)$ field operations.}\end{thm}




\chapter{\label{chap:rank}Rank Profile and Rank Sensitive Computation of
Kernel Basis}

In this chapter, we consider the problems of computing the row rank
profile of an input matrix $\mathbf{F}\in\mathbb{K}\left[x\right]^{m\times n}$,
which also immediately gives us the rank. If $n\ge m$, the rank can
already be computed by either kernel basis computation or column basis
computation from the earlier chapters. The column basis computation
(\prettyref{alg:colBasis}) can compute the rank with a cost of $O^{\sim}(m^{\omega-1}\xi)$
field operations, where $\xi$ is the sum of the column degrees of
$\mathbf{F}$. However, we would like to refine this cost to $O^{\sim}\left(mr^{\omega-2}\xi\right)$,
where $r$ is the rank of $\mathbf{F}$. We also would like to compute
a row rank profile with the same cost.

We use the following approach to achieve the desired cost. We first
modify our kernel basis algorithm, \prettyref{alg:minimalNullspaceBasisWithRankProfile},
slightly to allow the rank profile to be computed along with a kernel
basis. Then we do a series of computations with increasing number
of rows from $\mathbf{F}$. For each set of rows we do successive
column basis computation (or order basis computation) as in \prettyref{sec:successiveColBasisComputation}
to reduce the column dimension of the problem, so the modified \prettyref{alg:minimalNullspaceBasisWithRankProfile}
can work efficiently to compute the rank profile of this set of rows.


\section{Rank Profile from Kernel Basis Computation}

Recall that the row rank profile of $\mathbf{F}$ is the lexicographically
smallest list of row indices $\left[i_{1},i_{2},\dots i_{r}\right]$
such that these rows of $\mathbf{F}$ are linearly independent, where
$r$ is the rank of $\mathbf{F}$. Let us see how the row rank profile
can be computed by our kernel basis algorithm.  The following lemma
provides a key to the rank profile computation using \prettyref{alg:minimalNullspaceBasis}.
\begin{lem}
\label{lem:rankProfileIndicator}At any base case of running \prettyref{alg:minimalNullspaceBasis}
on the input matrix\textbf{ $\mathbf{F}$}, we work with an input
matrix $\mathbf{g}$ consisting of a single row. Let $\mathbf{f}$
be the original row in $\mathbf{F}$ corresponding to $\mathbf{g}$
and $\mathbf{F}'$ be the submatrix of $\mathbf{F}$ consists of the
rows above $\mathbf{f}$. Then $\mathbf{g}=0$ if and only if $\mathbf{f}$
is linearly dependent with the rows of $\mathbf{F}'$.\end{lem}
\begin{proof}
When the algorithm has reached the base case involving the single
row matrix $\mathbf{g}$, it has finished processing $\mathbf{F}'$
and has produced a number of order bases and kernel bases from the
earlier subproblems, where the kernel bases computed only involved
all the rows of $\mathbf{F}'$. The matrix $\mathbf{g}$ is the residual
from multiplying $\mathbf{f}$ with these order bases and kernel bases.
Note that such multiplications do not change the linear dependency
of $\mathbf{g}$ with the rows of $\mathbf{F}'$. But if $\mathbf{f}$
is linearly dependent with the rows of $\mathbf{F}'$, the residual
$\mathbf{g}$ becomes 0 after multiplying with kernel bases of the
rows of $\mathbf{F}'$.
\end{proof}
\prettyref{lem:rankProfileIndicator} now allows us to provide a small
modification to \prettyref{alg:minimalNullspaceBasis} to produce
the rank profile of $\mathbf{F}$. The modified algorithm is given
in \prettyref{alg:minimalNullspaceBasisWithRankProfile}. Note that
the rank profile in our algorithm is represented using a list of $n$
indicators that indicate the first $r$ linearly independent rows
of $\mathbf{F}$. At this point, the rank profile of $\mathbf{F}$
still costs the same to compute as a kernel basis of $\mathbf{F}$.
In the following, we see how column basis computation can be used
to improve this.

\begin{algorithm}[t]
\caption{$\mnbrp(\mathbf{F},\vec{s})$}
\label{alg:minimalNullspaceBasisWithRankProfile}

\begin{algorithmic}[1]
\REQUIRE{$\mathbf{F}\in\mathbb{K}\left[x\right]^{m\times n}$, $\vec{s}=[s_{1},\dots,s_{n}]\in\mathbb{Z}^{n}$
with entries arranged in non-decreasing order and bounding the corresponding
column degrees of $\mathbf{F}$. }

\ENSURE{A $\vec{s}$-minimal kernel basis $\mathbf{N}$ of $\mathbf{F}$
and the row rank profile of $\mathbf{F}$ given by a list binary indicators
$\bar{e}=\left[e_{1},\dots,e_{m}\right]$, with $1$'s indicating
the columns in the rank profile.}

\STATE{$\xi:=\sum_{i=1}^{n}s_{i}$; $\rho:=\sum_{i=n-m+1}^{n}s_{i};$ $s:=\rho/m$; }

\STATE{$\left[\mathbf{P},\vec{b}\right]:=\mab\left(\mathbf{F},3s,\vec{s}\right)$,
a $\left(\mathbf{F},3s,\vec{s}\right)$-basis with the columns of
$\mathbf{P}$ and the entries of is $\vec{s}$-column degrees $\vec{b}$
arranged so that the entries of $\vec{b}$ are in non-decreasing order;}

\STATE{$\left[\mathbf{P}_{1},\mathbf{P}_{2}\right]:=\mathbf{P}$ where $\mathbf{P}_{1}$
consists of all columns $\mathbf{p}$ of $\mathbf{P}$ satisfying
$\mathbf{F}\mathbf{p}=0$;}

\IF{$m=1$}

\ifbody{\IF{$\mathbf{F}=0$ }

\ifbody{\RETURN $\mathbf{P}_{1},[0]$

\ELSE{}

\RETURN $\mathbf{P}_{1},[1]$}

\ELSE{}

\STATE{ $\vec{t}:=\deg_{\vec{s}}\mathbf{P}_{2}-\left[3s,3s,\dots,3s\right];$}

\STATE{$\mathbf{G}:=\mathbf{F}\mathbf{P}_{2}/x^{3s}$;}

\STATE{$\left[\mathbf{G}_{1}^{T},\mathbf{G}_{2}^{T}\right]^{T}:=\mathbf{G}$,
with $\mathbf{G}_{1}$ having $\left\lfloor m/2\right\rfloor $ rows
and $\mathbf{G}_{2}$ having $\left\lceil m/2\right\rceil $ rows;}

\STATE{$\mathbf{N}_{1},\bar{e}_{1}:=\mnb\left(\mathbf{G}_{1},\vec{t}\right);$ }

\STATE{$\mathbf{N}_{2},\bar{e}_{2}:=\mnb\left(\mathbf{G}_{2}\mathbf{N}_{1},\cdeg_{\vec{t}}\mathbf{N}_{1}\right);$}

\STATE{$\mathbf{Q}:=\mathbf{N}_{1}\mathbf{N}_{2}$;}

\RETURN $\left[\mathbf{P}_{1},\mathbf{P}_{2}\mathbf{Q}\right],\left[\bar{e}_{1},\bar{e}_{2}\right]$}
\end{algorithmic}
\end{algorithm}




\section{Successive Rank Profile Computation}

To compute the rank and rank profile of $\mathbf{F}$ in a rank-sensitive
way, we do a series of computations with sets of increasing number
of rows from $\mathbf{F}$.

We start with the first nonzero row of $\mathbf{F}$, which is the
first row in the rank-profile. Suppose we have found the indices
$\bar{j}=\left[j_{1},\dots,j_{k}\right]$ in the rank profile. To
find the next linearly independent rows, we work with the matrix $\mathbf{G}=\mathbf{F}_{\left[\bar{j},j_{k}+1\dots,j_{k}+k\right]}$,
the matrix consists of the $k$ linearly independent rows indexed
by $\bar{j}$ and the next $k$ rows. We compute a column basis $\mathbf{T}$
of this matrix, which has the same rank profile as $\mathbf{G}$.
Now we can use \prettyref{alg:minimalNullspaceBasisWithRankProfile}
to compute the rank profile of $\mathbf{T}$, which gives more indices
for the rank profile of $\mathbf{F}$. We repeat this procedure until
all rows of $\mathbf{F}$ are processed. This gives us \prettyref{alg:rankProfile}
for computing the rank profile of $\mathbf{F}$.

\begin{algorithm}[t]
\caption{$\rankProfile(\mathbf{F})$: Compute a rank profile of $\mathbf{F}$}
\label{alg:rankProfile}

\begin{algorithmic}
[1]\REQUIRE{$\mathbf{F}\in\mathbb{K}\left[x\right]^{m\times n}$. }

\ENSURE{A row rank profile $\bar{j}=\left[j_{1},\dots,j_{r}\right]\in\mathbb{Z}^{r}$
of $\mathbf{F}$. }

\STATE{$k:=1$; ($k$ keeps track of the current row) }

\STATE{\textbf{while }$k\le m$ \textbf{and }$\mathbf{F}_{k}\ne0$ \textbf{do
$k:=k+1$ end while}; (find the first nonzero row $\mathbf{F}_{k}$)}

\STATE{$\bar{j}=\left[k\right];$ $r:=1$; ($r$ is the current rank)}

\WHILE{$k<m$ \AND{} }

\whilebody{\STATE{$k':=\min\left(m,k+r-1\right)$; (last row in the block)}

\STATE{$\mathbf{G}:=\mathbf{F}_{\left[\bar{j},k\dots,k'\right]};$ (the $r$
linearly independent rows and the next block of rows)}

\STATE{$\mathbf{T}:=\colBasis(\mathbf{G});$}

\STATE{$\mathbf{N},\bar{e}:=\mnbrp(\mathbf{G});$ }

\FOR{$i$ \textbf{from $r+1$ to }$2r$ (convert indicator to indices and
append to $\bar{j}$) }

\forbody{\STATE{\textbf{if} $e_{i}=1$ \textbf{then} $\bar{j}:=\left[\bar{j},e_{i}-r+k-1\right]$
\textbf{end if};}}

\STATE{$k:=k+r$; }}

\RETURN $\bar{j}$;
\end{algorithmic}
\end{algorithm}



The main remaining task is to analyze the computational cost of \prettyref{alg:rankProfile}. 
\begin{thm}
The cost of \prettyref{alg:rankProfile} is $O^{\sim}\left(\xi mr^{\omega-2}\right)$
field operations for computing a rank profile of $\mathbf{F}\in\mathbb{K}\left[x\right]^{m\times n}$.\end{thm}
\begin{proof}
Let $r_{i}$ be the number of rows in the new block considered at
step $i$. Then the cost at step $i$ is $O^{\sim}\left(r_{i}^{\omega-1}\xi\right)$
field operations for the column basis and the kernel basis with rank
profile computation. The total cost is then 
\[
\sum O^{\sim}\left(r_{i}^{\omega-1}\xi\right)=O^{\sim}\left(\xi\sum r_{i}^{\omega-1}\right).
\]
 We also know that $\sum r_{i}=m$, and the maximum of $r_{i}$ is
the rank $r$ of $\mathbf{F}$. Note that 
\[
\sum r_{i}^{\omega-1}\le\frac{m}{r}r^{\omega-1}=mr^{\omega-2}.
\]
 Hence 
\[
\sum O^{\sim}\left(r_{i}^{\omega-1}\xi\right)=O^{\sim}\left(\xi\sum r_{i}^{\omega-1}\right)=O^{\sim}\left(\xi mr^{\omega-2}\right).
\]

\end{proof}

\section{Applications of Rank Profile Computation}


\subsection{\label{sub:removeDimensionAssumption}Remove the assumption $n\ge m$}

In order basis, kernel basis, and column basis computations from previous
chapters, we have assumed that the column dimension $n$ is no less
than the row dimension $m.$ We can now use the rank profile computation
to ensure that this is always the case. For order basis and kernel
basis computation, we can just determine the rank profile $\bar{j}$
of the input matrix $\mathbf{F}$, and then work with just $\mathbf{F}_{\bar{j}}$,
which consists of only $r$ linearly independent rows, as we know
the rank $r$ is always bounded by the column dimension $n$. For
column basis computation, the assumption is only required by the kernel
basis computation used. Therefore removing this assumption from the
kernel basis computation also removes this assumption from the column
basis computation.


\subsection{Rank-sensitive computation of minimal kernel bases}

With the ability to compute a rank profile efficiently, we can now
slightly improve \prettyref{cor:costOfMinimalNullspaceBasis} on the
cost of kernel basis computation with a matrix of degree $d$, by
using only the linearly independent rows from $\mathbf{F}$, hence
reducing the row dimension of the input matrix from $m$ to $r$,
after a cost of $O^{\sim}\left(\xi mr^{\omega-2}\right)=O^{\sim}\left(nmr^{\omega-2}d\right)$
to compute the rank profile. 
\begin{thm}
Given a matrix $\mathbf{F}\in\mathbb{K}\left[x\right]^{m\times n}$with
degree $d$, a minimal kernel basis of $\mathbf{F}$ can be computed
with a cost of $O^{\sim}(nmr^{\omega-2}d+n^{\omega-1}rd)$.
\end{thm}
It looks difficult to further improve this cost by removing the exponent
$\omega$ from the column dimension $n$ if a minimal kernel basis
is required. Since the minimality requires us to work with some matrix
involving all $n$ columns at the same time. 





\chapter{Conclusion\label{sec:Future-Research}}

In this thesis, we have presented efficient deterministic algorithms
for a number of polynomial matrix computation problems, including
the computation of order basis, minimal nullspace basis, matrix inverse,
column basis, unimodular completion, determinant, Hermite normal form,
rank, and rank profile. The algorithm for kernel basis computation
also immediately gives us a new way to solve linear systems. An existing
efficient deterministic method for solving linear systems was given
by \citet{GSSV2012}. The algorithm for column basis also immediately
allows us to compute matrix GCD, column reduced forms and Popov normal
forms for matrices of any dimension. 

We first gave algorithms for computing a shifted order basis of an
$m\times n$ matrix of power series over a field $\mathbb{K}$ with
$m\le n$. For a given order $\sigma$ and balanced shift $\vec{s}$
the first algorithm determines an order basis with a cost of $O^{\sim}(n^{\omega}a)$
field operations in $\mathbb{K}$, where $a=m\sigma/n.$ We then provided
a method to refine the cost to $O^{\sim}(n^{\omega-1}m\sigma)$. %Here $O^{\sim}$
%is just $O$ with log factors omitted and 
%$\MM\left(n,d\right)$ denotes
%the cost of multiplying two polynomial matrices with dimension $n$
%and degree $d$. 
\begin{comment}
This extends earlier work of Storjohann which only determines a subset
of an order basis that is within a specified degree bound $\delta$
using $O^{\sim}(n^{\omega}\delta)$ field operations for $\delta\ge\lceil m\sigma/n\rceil$.
\end{comment}
{} While the first algorithm addresses the case when the column degrees
of a complete order basis are unbalanced given a balanced input shift,
it is not efficient in the case when an unbalanced shift results in
the row degrees also becoming unbalanced. %The column degrees of a complete basis may be unbalanced, which is
%a major issue we address in the first algorithm. When the input shift
%is unbalanced, the row degrees of the basis can also be unbalanced
%in addition to the unbalanced column degrees. For this, we present
We have presented a second algorithm which balances the high degree
rows and computes an order basis also using $O^{\sim}(n^{\omega}a)$
field operations in the case that the shift is unbalanced but satisfies
the condition $\sum_{i=1}^{n}(\max(\vec{s})-\vec{s}_{i})\le m\sigma$.%
\begin{comment}
Every problem with any unbalanced shift can be in fact reduced to
a problem with a shift that satisfying this condition if the degrees
of a resulting order basis is known. 
\end{comment}
{} %
\begin{comment}
Many unbalanced shift problems can be in fact converted to problems
satisfying this condition. 
\end{comment}
{} This condition essentially allows us to locate those high degree
rows that need to be balanced. %
\begin{comment}
In more general unbalanced shift cases, this algorithm may not work
well directly since we do not know in advance which are the high degree
rows need to be balanced. But it may work efficiently if we have an
effective way of estimating the resulting row degrees. 
\end{comment}
{} %
\begin{comment}
This extends the earlier work by the authors from ISSAC'09.
\end{comment}


We then presented an algorithm for the computation of a minimal nullspace
basis of an $m\times n$ input matrix of univariate polynomials over
a field $\mathbb{K}$ with $m\le n$. This algorithm computes a minimal
nullspace basis of a degree $d$ input matrix with a cost of $O^{\sim}\left(n^{\omega-1}md\right)$
field operations in $\mathbb{K}$. %
\begin{comment}
Here the soft-$O$ notation is Big-$O$ with log factors removed while
$\omega$ is the exponent of matrix multiplication.
\end{comment}
{} The same algorithm also works in the more general situation on computing
a shifted minimal nullspace basis, with a given degree shift $\vec{s}\in\mathbb{Z}^{n}$
whose entries bound the corresponding column degrees of the input
matrix. In this case a $\vec{s}$-minimal right nullspace basis can
be computed with a cost of $O^{\sim}(n^{\omega}s)$ field operations,
where $s$ is the average of the largest $m$ entries of $\vec{s}$. 

Order basis computation and nullspace basis computation were then
applied to the remaining problems. An algorithm for computing the
inverse of an matrix in $\mathbb{K}\left[x\right]^{n\times n}$ was
then given with a cost of $O^{\sim}\left(n^{3}s\right)$ field operations,
where $s$ is the average of the column or row degrees of the input
matrix. The inverse represented alternatively by a product of $\left\lceil \log n\right\rceil $
matrices costs only $O^{\sim}\left(n^{\omega}s\right)$ to compute.
We then discussed the computation of a column basis of an input matrix
in $\mathbb{K}\left[x\right]^{m\times n}$ with a cost of $O^{\sim}\left(m^{\omega}ns\right)$,
where $s$ is again the average column degree of the input matrix.
Next, an algorithm was presented for computing an unimodular completion
of an input matrix in $\mathbb{K}\left[x\right]^{m\times n}$, $m<n$
with a cost of $O^{\sim}\left(n^{\omega}s\right)$, where $s$ is
the average of the $m$ largest column degrees of the input matrix.
Then an algorithm for computing the determinant of an input matrix
in $\mathbb{K}\left[x\right]^{n\times n}$ with a cost of $O^{\sim}\left(n^{\omega}s\right)$
was given, where $s$ is the average column or row degree of the input
matrix. Then we looked at an algorithm for computing the Hermite normal
form of a degree $d$ input matrix in $\mathbb{K}\left[x\right]^{n\times n}$
with a cost of $O^{\sim}\left(n^{\omega}d\right)$. Finally, we provided
algorithms for rank-sensitive computations of the rank and rank profile
of an input matrix in $\mathbb{K}\left[x\right]^{m\times n}$ with
a cost of $O^{\sim}\left(mr^{\omega-2}ns\right)$, where $s$ is the
average column degree of the input matrix, and then applied the rank
profile algorithm to rank-sensitive computation of minimal kernel
basis to obtain a cost of $O^{\sim}(nmr^{\omega-2}d+n^{\omega-1}rd)$.

We reduce all these problems to polynomial matrix multiplications.
The computational costs of our algorithms are then similar to the
costs of multiplying matrices, whose dimensions match the input matrix
dimensions in the original problems, and whose degrees equal the average
column degrees of the original input matrices in most cases. The use
of the average column degrees instead of the commonly used matrix
degrees, or equivalently the maximum column degrees, makes our computational
costs more precise and tighter. In addition, the shifted minimal bases
computed by our algorithms are more general than the standard minimal
base.
 

\bibliographystyle{plainnat}
\bibliography{paper}
\addcontentsline{toc}{chapter}{Bibliography}
\end{document}
